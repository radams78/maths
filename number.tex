\chapter{Natural Numbers}

\section{Successors}

\begin{definition}[Successor (Pairing, Union)]
For any set $a$, its \emph{Successor} $a^+$ is the set $a \cup \{ a \}$    
\end{definition}

\begin{theorem}[Pairing, Union]
    If $a$ is a transitive set then $\bigcup (a^+) = a$.
\end{theorem}

\begin{proof}
    \pf
    \begin{align*}
        \bigcup (a^+) & = \bigcup (a \cup \{a\}) \\
        & = \bigcup a \cup \bigcup \{a\} \\
        & = \bigcup a \cup a \\
        & = a & (\bigcup a \subseteq a) \qed
    \end{align*}
\end{proof}

\begin{theorem}
    \label{theorem:transitive_successor}
    If $A$ is a transitive set then $A^+$ is transitive.
\end{theorem}

\begin{proof}
    \pf\ If $A$ is transitive then $\bigcup (A^+) = A \subseteq A^+$. \qed
\end{proof}

\section{Inductive Sets}

\begin{definition}[Inductive (Extensionality, Empty Set, Pairing, Union)]
    A set $A$ is \emph{inductive} iff $\emptyset \in A$ and, for every $a \in A$,
    we have $a^+ \in A$.
\end{definition}

\begin{axiom}[Axiom of Infinity (Extensionality, Empty Set, Pairing, Union)]
    There exists an inductive set.
\end{axiom}

\section{Natural Numbers}

\begin{definition}[Natural Number (Extensionality, Empty Set, Pairing, Union)]
    A \emph{natural number} is a set that belongs to every inductive set.

    We write $\NN$ for the class of all natural numbers.
\end{definition}

\begin{theorem}[Extensionality, Empty Set, Pairing, Union, Infinity, Subset]
    The class of natural numbers is a set.
\end{theorem}

\begin{proof}
    \pf
    \step{1}{\pick\ an inductive set $I$.}
    \begin{proof}
        \pf\ By the Axiom of Infinity.
    \end{proof}
    \step{2}{$\NN \subseteq I$}
    \qedstep
    \begin{proof}
        \pf\ By a Subset Axiom.
    \end{proof}
    \qed
\end{proof}

\begin{theorem}[Extensionality, Empty Set, Pairing, Union, Infinity, Subset]
    The set $\NN$ is inductive.
\end{theorem}

\begin{proof}
    \pf
    \step{1}{$\emptyset \in \NN$}
    \begin{proof}
        \pf\ Since $\emptyset$ is a member of every inductive set.
    \end{proof}
    \step{2}{For all $n \in \NN$ we have $n^+ \in \NN$}
    \begin{proof}
        \pf\ If $n$ is a member of every inductive set then so is $n^+$.
    \end{proof}
    \qed
\end{proof}

\begin{theorem}[Extensionality, Empty Set, Pairing, Union, Infinity, Subset]
    The set $\NN$ is a subset of every inductive set.
\end{theorem}

\begin{proof}
    \pf\ Immediate from definition. \qed
\end{proof}

\begin{corollary}[Proof by Induction (Extensionality, Empty Set, Pairing, Union, Infinity, Subset)]
    If $A \subseteq \NN$ and $A$ is inductive then $A = \NN$.
\end{corollary}

\begin{definition}[Zero (Empty Set)]
    The natural number \emph{zero}, $0$, is defined to be $\emptyset$.
\end{definition}

\begin{theorem}[Extensionality, Empty Set, Pairing, Union, Infinity, Subset]
    Every natural number except 0 is a successor of a natural number.
\end{theorem}

\begin{proof}
    \pf\ The set $\{ x \in \NN \mid x = 0 \vee \exists y \in \NN. x = y^+ \}$ is inductive. \qed
\end{proof}

\begin{theorem}[Extensionality, Empty Set, Pairing, Union, Infinity, Subset]
    Every natural number is transitive.
\end{theorem}

\begin{proof}
    \pf\ By induction using Theorem \ref{theorem:transitive_successor}. \qed
\end{proof}

\begin{theorem}[Extensionality, Empty Set, Pairing, Union, Infinity, Subset]
    The set $\NN$ is transitive.
\end{theorem}

\begin{proof}
    \pf
    \step{1}{For every natural number $n$ and every $m \in n$ then $m$
    is a natural number.}
    \begin{proof}
        \step{a}{Every member of $\emptyset$ is a natural number.}
        \begin{proof}
            \pf\ Vacuous.
        \end{proof}
        \step{b}{If $n$ is a natural number and a set of natural numbers then $n^+$ is a
        set of natural numbers.}
        \begin{proof}
            \pf\ From the definition of $n^+$.
        \end{proof}
        \qedstep
        \begin{proof}
            \pf\ By induction.
        \end{proof}
    \end{proof}
    \qed
\end{proof}

\begin{theorem}[Extensionality, Empty Set, Pairing, Union, Infinity, Subset]
    Let $A$ be a set, $a \in A$, and $F : A \rightarrow A$. Then there exists
    a unique function $h : \NN \rightarrow A$ such that $h(0) = a$
    and, for all $n \in \NN$, we have $h(n^+) = F(h(n))$.
\end{theorem}

\begin{proof}
    \pf
    \step{1}{Call a function $v$ \emph{acceptable} iff $\dom v \subseteq \NN$,
    $\ran v \subseteq A$, and:
    \begin{enumerate}
        \item If $0 \in \dom v$ then $v(0) = a$.
        \item For all $n \in \NN$, if $n^+ \in \dom v$ then $n \in \dom v$
        and $v(n^+) = F(v(n))$.
    \end{enumerate}
    }
    \step{2}{\pflet{$\mathcal{K}$ be the set of all acceptable functions.}}
    \step{3}{\pflet{$h = \bigcup \mathcal{K}$}}
    \step{4}{$h$ is a function.}
    \begin{proof}
        \step{1}{If $(0,y) \in h$ and $(0,y') \in h$ then $y = y'$}
        \begin{proof}
            \pf\ We have $y = y' = a$.
        \end{proof}
        \step{2}{For any natural number $n$, if there is at most one $y$ such that
        $(n,y) \in h$, then there is at most one $y$ such that $(n^+,y) \in h$}
        \begin{proof}
            \step{a}{\pflet{$n$ be a natural number.}}
            \step{b}{\assume{there is at most one $y$ such that $(n,y) \in h$}}
            \step{c}{\assume{$(n^+,y)$ and $(n^+,y')$ are in $h$)}}
            \step{d}{\pick\ acceptable functions $u$ and $v$ such that $u(n^+) = y$
            and $v(n^+) = y'$}
            \step{e}{$n \in \dom u$, $n \in \dom v$ and $y = F(u(n))$, $y'=F(v(n))$}
            \step{f}{$u(n) = v(n)$}
            \begin{proof}
                \pf\ By the induction hypothesis \stepref{b}
            \end{proof}
            \step{g}{$y = y'$}
        \end{proof}
        \qedstep
        \begin{proof}
            \pf\ By induction.
        \end{proof}
    \end{proof}
    \step{5}{$h$ is acceptable.}
    \begin{proof}
        \step{a}{If $0 \in \dom h$ then $h(0) = a$}
        \step{b}{If $n^+ \in \dom h$ then $n \in \dom h$ and $h(n^+) = F(h(n))$}
        \begin{proof}
            \step{i}{\assume{$n^+ \in \dom h$}}
            \step{ii}{\pick\ an acceptable $v$ such that $n^+ \in \dom v$}
            \step{iii}{$v(n^+) = F(v(n))$}
            \step{iv}{$h(n^+) = F(h(n))$}
        \end{proof}
    \end{proof}
    \step{6}{$\dom h = \NN$}
    \begin{proof}
        \step{a}{$0 \in \dom h$}
        \begin{proof}
            \pf\ Since $\{(0,a)\}$ is an acceptable function.
        \end{proof}
        \step{b}{For all $n \in \dom h$ we have $n^+ \in \dom h$}
        \begin{proof}
            \step{i}{\assume{$n \in \dom h$}}
            \step{ii}{\pflet{$v$ be an acceptable function with $n \in \dom v$}}
            \step{iii}{\assume{without loss of generality $n^+ \notin \dom v$}}
            \step{iv}{$v \cup \{ (n^+, F(v(n))) \}$ is acceptable}
            \step{v}{$n^+ \in \dom v$}
        \end{proof}
    \end{proof}
    \step{7}{If $h' : \NN \rightarrow A$, $h'(0) = a$ and, for all $n \in \NN$,
    we have $h'(n^+) = F(h'(n))$, then $h' = h$}
    \begin{proof}
        \pf\ Prove $h(n) = h'(n)$ by induction on $n$.
    \end{proof}
    \qed
\end{proof}

\section{Peano Systems}

\begin{definition}[Peano System]
    A \emph{Peano system} consists of a set $N$, an element $z \in N$,
    and a function $S : N \rightarrow N$ such that:
    \begin{itemize}
        \item $S$ is one-to-one
        \item $z \notin \ran S$
        \item For any set $A \subseteq N$, if $z \in A$ and $S(A) \subseteq A$
        then $A = N$.
    \end{itemize}
\end{definition}

\begin{theorem}
    $\NN$ is a Peano system with zero $0$ and successor $n \mapsto n^+$.
\end{theorem}

\begin{theorem}
    For any Peano system $(N, z, S)$, there exists a unique bijection
    $h : \NN \cong N$ such that $h(0) = z$ and $S(h(n)) = h(n^+)$
    for all $n$.
\end{theorem}

\section{Arithmetic}

\begin{definition}[Addition]
    Define \emph{addition} $+ : \NN^2 \rightarrow \NN$ recursively by
    \begin{align*}
        m + 0 & = m \\
        m + n^+ & = (m + n)^+
    \end{align*}
    for any $m, n \in \NN$.
\end{definition}

\begin{theorem}
    Addition is associative.
\end{theorem}

\begin{theorem}
    Addition is commutative
\end{theorem}

\begin{definition}[Multiplication]
    Define \emph{multiplication} $\cdot : \NN^2 \rightarrow \NN$ recursively by
    \begin{align*}
        m 0 & = 0 \\
        m n^+ & = mn + m
    \end{align*}
    for any $m, n \in \NN$
\end{definition}

\begin{theorem}
    Multiplication is associative.
\end{theorem}

\begin{theorem}
    Multiplication is commutative.
\end{theorem}

\begin{theorem}
    Multiplication distributes over addition.
\end{theorem}

\begin{definition}
    For natural numbers $m$ and $n$, we write $m < n$ iff $m \in n$.
    We write $m \leq n$ iff $m < n$ or $m = n$.
\end{definition}

\begin{theorem}
    We have $m < n$ iff $m^+ < n^+$.
\end{theorem}

\begin{theorem}
    We never have $n < n$.
\end{theorem}

\begin{theorem}
    The ordering on $\NN$ satisfies trichotomy; that is, for any $m$, $n$,
    exactly one of $m < n$, $m = n$, $n < m$ holds.
\end{theorem}

\begin{theorem}
    For any natural numbers $m$ and $n$, we have $m \leq n$ iff $m \subseteq n$.
\end{theorem}

\begin{theorem}
    We have $m < n$ iff $m + p < n + p$.
\end{theorem}

\begin{corollary}
    If $m + p = n + p$ then $m = n$.
\end{corollary}

\begin{theorem}
    If $p \neq 0$ then $m < n$ iff $mp < np$.
\end{theorem}

\begin{corollary}
    If $mp = np$ and $p \neq 0$ then $m = n$.
\end{corollary}

\begin{theorem}[Well-Ordering of $\NN$]
    Any nonempty set $A \subseteq \NN$ has a least element.
\end{theorem}

\begin{corollary}
    There is no function $f : \NN \rightarrow \NN$ such that $f(n^+) < f(n)$
    for all $n$.
\end{corollary}

\begin{theorem}[Strong Induction]
    Let $A \subseteq \NN$. Suppose that, for every natural number $n$,
    if $\forall m < n. m \in A$ then $n \in A$. Then $A = \NN$.
\end{theorem}

\begin{theorem}[Pigeonhole Principle]
    No natural number is equinumerous with a proper subset of itself.
\end{theorem}

\begin{proof}
    \pf\ Prove by induction on $n$ that if $f : n \rightarrow n$ is injective
    then it is surjective. \qed
\end{proof}

\chapter{Integers}

\begin{lemma}
    Define $\sim$ on $\NN^2$ by: $(m,n) \sim (p,q)$ iff $m + q = n + p$.
    Then $\sim$ is an equivalence relation on $\NN^2$.
\end{lemma}

\begin{definition}[Integers]
    The set $\ZZ$ of \emph{integers} is $\NN^2 / \sim$.
\end{definition}

\begin{definition}
    Define \emph{addition} $+ : \ZZ^2 \rightarrow \ZZ$ by: $(m,n) + (p,q) = (m+p,n+q)$.
    
    Prove this is well-defined.
\end{definition}

\begin{theorem}
    Addition is associative and commutative.
\end{theorem}

\begin{definition}[Zero]
    The integer \emph{zero} is $0 = (0,0)$.
\end{definition}

\begin{theorem}
    For any integer $a$, we have $a + 0 = a$.
\end{theorem}

\begin{theorem}
    For any integer $a$, there exists a unique integer $b$ such that $a + b = 0$.
\end{theorem}

\begin{definition}[Multiplication]
    Define multiplication on $\ZZ$ by $(m,n)(p,q) = (mp+nq,mq+np)$.
\end{definition}

\begin{theorem}
    Multiplication is associative, commutative and distributive over addition.
\end{theorem}

\begin{definition}
    The integer \emph{one} is $1 = (1,0)$.
\end{definition}

\begin{theorem}
    For any integer $a$ we have $a1 = a$.
\end{theorem}

\begin{theorem}
    $1 \neq 0$
\end{theorem}

\begin{theorem}
    Whenever $ab = 0$ then either $a = 0$ or $b = 0$.
\end{theorem}

\begin{definition}
    Define $<$ on $\ZZ$ by: $(m,n) < (p,q)$ iff $m + q < n + p$.
\end{definition}

\begin{theorem}
    The relation $<$ is a strict linear ordering on $\ZZ$.
\end{theorem}

\begin{theorem}
    We have $a < b$ iff $< + c < b + c$.
\end{theorem}

\begin{corollary}
    If $a + c = b + c$ then $a = b$.
\end{corollary}

\begin{theorem}
    If $0 < c$ then $a < b$ iff $ac < bc$.
\end{theorem}

\begin{corollary}
    If $ac = bc$ and $c \neq 0$ then $a = b$.
\end{corollary}

\begin{definition}
    We identify any natural number $n$ with the integer $(n,0)$.
\end{definition}

\begin{theorem}
    This embedding preserves 0, 1, addition, multiplication and the ordering.
\end{theorem}

\chapter{Rational Numbers}

\begin{definition}[Rational Numbers]
    The set of \emph{rationals} $\QQ$ is $\ZZ \times (\ZZ \setminus \{0\}) / \sim$, where
    $(a,b) \sim (c,d)$ iff $ad = bc$.
\end{definition}

\begin{definition}[Addition]
    Define addition on $\QQ$ by: $(a,b) + (c,d) = (ad+bc,bd)$.
\end{definition}

\begin{theorem}
    Addition is commutative and associative
\end{theorem}

\begin{definition}
    The rational number 0 is $(0,1)$.
\end{definition}

\begin{theorem}
    For any rational $q$ we have $q + 0 = q$.
\end{theorem}

\begin{theorem}
    For any rational $q$, there exists a unique rational $r$ such that $q + r = 0$.
\end{theorem}

\begin{definition}
    Define multiplication on $\QQ$ by: $(a,b)(c,d) = (ac,bd)$.
\end{definition}

\begin{theorem}
    Multiplication is commutative, associative and distributive over addition.
\end{theorem}

\begin{definition}
    The rational number 1 is $(1,1)$.
\end{definition}

\begin{theorem}
    For every nonzero rational $r$, there exists a nonzero rational $q$ such that $rq = 1$.
\end{theorem}

\begin{corollary}
    If $qr = 0$ then either $q = 0$ or $r = 0$.
\end{corollary}

\begin{definition}
    Define $<$ on $\QQ$ by: for $b$ and $d$ positive, $(a,b) < (c,d)$ iff $ad < bc$.
\end{definition}

\begin{theorem}
    The relation $<$ is a strict linear ordering on $\QQ$.
\end{theorem}

\begin{theorem}
    We have $q < r$ iff $q + s < r + s$
\end{theorem}

\begin{corollary}
    If $q + s = r + s$ then $q = r$.
\end{corollary}

\begin{theorem}
    If $s > 0$ then we have $q < r$ iff $qs < rs$.
\end{theorem}

\begin{corollary}
    If $qs = rs$ and $s \neq 0$ then $q = r$.
\end{corollary}

\begin{definition}
    We identify an integer $n$ with the rational $(n,1)$.
\end{definition}

\begin{theorem}
    This embedding preserves zero, one, addition, multiplication and the ordering.
\end{theorem}

\chapter{Real Numbers}

\begin{definition}[Dedekind Cut]
    A \emph{Dedekind cut} is a subset $X \subseteq \QQ$ such that:
    \begin{itemize}
        \item $X$ is nonempty
        \item $X \neq \QQ$
        \item $X$ is closed downward
        \item $X$ has no largest element.
    \end{itemize}
\end{definition}

\begin{definition}[Real Numbers]
    The set of \emph{real numbers} $\RR$ is the set of all Dedekind cuts.
\end{definition}

\begin{definition}
    Define $<$ on $\RR$ by: $x < y$ iff $x$ is a proper subset of $y$.
\end{definition}

\begin{theorem}
    The relation $<$ is a strict linear ordering on $\RR$.
\end{theorem}

\begin{theorem}
    Any bounded nonempty subset of $\RR$ has a least upper bound.
\end{theorem}

\begin{definition}
    Define addition on $\RR$ by: $x + y = \{ q + r \mid q \in x, r \in y \}$.
\end{definition}

\begin{theorem}
    Addition is associative and commutative.
\end{theorem}

\begin{definition}
    The zero real $0$ is $\{ q \in \QQ \mid q < 0 \}$.
\end{definition}

\begin{theorem}
    For any $x \in \RR$ we have $x + 0 = x$.
\end{theorem}

\begin{definition}
    Given a real $x$, define $-x = \{ q \in \QQ \mid \exists r > q. -r \notin x \}$.
\end{definition}

\begin{theorem}
    For any real $x$ we have $x + (-x) = 0$.
\end{theorem}

\begin{corollary}
    If $x + z = y + z$ then $x = y$.
\end{corollary}

\begin{theorem}
    We have $x < y$ iff $x + z < y + z$.
\end{theorem}

\begin{definition}
    Define the \emph{absolute value} of a real $x$ by $|x| = x \cup -x$.
\end{definition}

\begin{theorem}
    For any real $x$ we have $0 \leq |x|$.
\end{theorem}

\begin{definition}
    Define multiplication on $\RR$ by:
    \begin{itemize}
        \item If $x$ and $y$ are nonnegative then
        \[ xy = 0 \cup \{ qr \mid 0 \leq q, 0 \leq r, q \in x, r \in y \} \]
        \item If $x$ and $y$ are both negative then $xy = |x| |y|$
        \item If one of $x$ and $y$ is negaive and the other not then $xy = -|x| |y|$.
    \end{itemize}
\end{definition}

\begin{theorem}
    Multiplication is associative, commutative and distributive over addition.
\end{theorem}

\begin{definition}
    The real number 1 is $\{ q \in \QQ \mid q < 1 \}$.
\end{definition}

\begin{theorem}
    $0 \neq 1$
\end{theorem}

\begin{theorem}
    For any real $x$ we have $x1 = x$
\end{theorem}

\begin{theorem}
    For any nonzero $x$, there exists a real $y$ with $xy = 1$.
\end{theorem}

\begin{theorem}
    If $0 < x$ then $y < z$ iff $xy < xz$.
\end{theorem}

\begin{definition}
    Identify a rational $q$ with $\{ r \in \QQ \mid r < q \}$.
\end{definition}

\begin{theorem}
    This embedding preserves zero, one, addition, multiplication and the ordering.
\end{theorem}

\section{The Cantor Set}

\begin{definition}[Cantor Set]
    \label{definition:Cantor_set}
    Define the sequence of sets $A_n \subseteq \RR$ by
    \begin{align*}
        A_0 & = [0,1] \\
        A_n & = A_{n-1} - \bigcup_{k=0}^{3^{n-1}-1} ((3k+1)/3^n, (3k+2)/3^n)
    \end{align*}
    The \emph{Cantor set} is $\bigcap_{n=0}^\infty A_n$.
\end{definition}

\begin{proposition}
    The set $A_n$ is a union of finitely many disjoint closed intervals
    of length $1 / 3^n$, and the endpoints of these intervals lie in $C$.
\end{proposition}

\begin{proof}
    \pf\ An easy induction on $n$. \qed
\end{proof}

\chapter{Finite Sets}

\begin{definition}[Finite]
    A set is \emph{finite} iff it is equinumerous with a natural number;
    otherwise it is \emph{infinite}.
\end{definition}

\begin{theorem}
    No finite set is equinumerous with a proper subset of itself.
\end{theorem}

\begin{proof}
    \pf\ From the Pigeonhole Principle.
\end{proof}

\begin{corollary}
    The set $\NN$ is infinite.
\end{corollary}

\begin{corollary}
    A finite set is equinumerous with a unique natural number.
\end{corollary}

\begin{lemma}
    If $A$ is a proper subset of a natural number $n$ then there exists
    $m < n$ such that $C \equiv m$.
\end{lemma}

\begin{corollary}
    A subset of a finite set is finite.
\end{corollary}

\begin{theorem}[Regularity]
    There is no function $f$ with domain $\NN$ such that $f(n+1) \in f(n)$ for all $n$.
\end{theorem}

\begin{proof}
    \pf
    \step{1}{\assume{for a contradiction $f$ is a function with domain $\NN$
    such that $f(n+1) \in f(n)$ for all $n$.}}
    \step{2}{\pick\ $m \in \ran f$ such that $m \cap \ran f = \emptyset$}
    \begin{proof}
        \pf\ By the Axiom of Regularity.
    \end{proof}
    \step{3}{\pick\ $n \in \NN$ such that $f(n) = m$}
    \step{4}{$f(n+1) \in m \cap \ran f$}
    \qedstep
    \begin{proof}
        \pf\ This is a contradiction.
    \end{proof}
    \qed
\end{proof}

\begin{theorem}
    A relation $R$ is well-founded if and only if there is no function $f$ with domain $\NN$
    such that, for all $n \in \NN$, we have $f(n+1) R f(n)$.
\end{theorem}

\section{The Finite Intersection Property}

\begin{definition}[Finite Intersection Property]
    Let $X$ be a set and $\AA \subseteq \pow X$. Then $\AA$ satisfies the \emph{finite intersection property} if and only if every nonempty finite subset of $\AA$
    has nonempty intersection.
\end{definition}

\begin{lemma}
    \label{lemma:finite_intersection_maximal}
    Let $X$ be a set. Let $\DD \subseteq \pow X$ be maximal with respect to the finite intersection property.
    Then any finite intersection of elements of $\DD$ is an element of $\DD$.
\end{lemma}

\begin{proof}
    \pf
    \step{1}{\pflet{$D_1, D_2 \in \DD$}}
    \step{2}{$\DD \cup \{ D_1 \cap D_2 \}$ has the finite intersection property.}
    \begin{proof}
        \pf\ Any finite intersection of members of $\DD \cup \{ D_1 \cap D_2 \}$
        is a finite intersection of members of $\DD$.
    \end{proof}
    \step{3}{$\DD = \DD \cup \{ D_1 \cap D_2 \}$}
    \begin{proof}
        \pf\ By maximality of $\DD$.
    \end{proof}
    \step{4}{$D_1 \cap D_2 \in \DD$.}
    \qed
\end{proof}

\begin{lemma}
    \label{lemma:member_maximal_finite_intersection}
    Let $X$ be a set. Let $\DD \subseteq \pow X$ be maximal with respect to the finite intersection property.
    Let $A \subseteq X$. If $A$ intersects every member of $\DD$ then $A \in \DD$.
\end{lemma}

\begin{proof}
    \pf
    \step{1}{$\DD \cup \{ A \}$ has the finite intersection property.}
    \begin{proof}
        \step{a}{\pflet{$D_1, \ldots, D_n \in \DD$} \prove{$D_1 \cap \cdots \cap D_n \cap A \neq \emptyset$}}
        \step{b}{$D_1 \cap \cdots \cap D_n \in \DD$}
        \begin{proof}
            \pf\ Lemma \ref{lemma:finite_intersection_maximal}.
        \end{proof}
        \step{c}{$D_1 \cap \cdots \cap D_n \cap A \neq \emptyset$}
        \begin{proof}
            \pf\ Since $A$ intersects every member of $\DD$.
        \end{proof}
    \end{proof}
    \qedstep
    \begin{proof}
        \pf\ By maximality of $\DD$.
    \end{proof}
    \qed
\end{proof}

\begin{proposition}
    Let $X$ be a set. Let $\DD \subseteq \pow X$ be maximal with respect to the
    finite intersection property. Let $A, D \in \pow X$.
    If $D \in \DD$ and $D \subseteq A$ then $A \in \DD$.
\end{proposition}

\begin{proof}
    \pf
    \step{1}{$\DD \cup \{A\}$ satisfies the finite intersection property.}
    \begin{proof}
        \step{a}{\pflet{$D_1, \ldots, D_n \in \DD$}}
        \step{b}{$D_1 \cap \cdots \cap D_n \cap D \neq \emptyset$}
        \begin{proof}
            \pf\ Since $\DD$ satisfies the finite intersection property.
        \end{proof}
        \step{c}{$D_1 \cap \cdots \cap D_n \cap A \neq \emptyset$}
    \end{proof}
    \step{2}{$\DD = \DD \cup \{A\}$}
    \begin{proof}
        \pf\ By the maximality of $\DD$.
    \end{proof}
    \step{3}{$A \in \DD$}
    \qed
\end{proof}