\chapter{Metric Spaces}

\section{The Metric Topology}

\begin{definition}[Metric]
    Let $X$ be a set. A \emph{metric} on $X$ is a function $d : X^2 \rightarrow \RR$ such that:
    \begin{enumerate}
        \item For all $x,y \in X$, $d(x,y) \geq 0$
        \item For all $x,y \in X$, $d(x,y) = 0$ if and only if $x = y$
        \item For all $x,y \in X$, $d(x,y) = d(y,x)$
        \item (\emph{Triangle Inequality}) For all $x, y, z \in X$, $d(x,z) \leq d(x,y) + d(y,z)$
    \end{enumerate}
    We call $d(x,y)$ the \emph{distance} between $x$ and $y$.
\end{definition}

\begin{definition}[Open Ball]
    Let $X$ be a metric space. Let $a \in X$ and $\epsilon > 0$. The \emph{open ball} with \emph{centre} $a$ and \emph{radius} $\epsilon$
    is
    \[ B(a, \epsilon) = \{ x \in X \mid d(a,x) < \epsilon \} \enspace . \]
\end{definition}

\begin{definition}[Metric Topology]
    Let $X$ be a metric space. The \emph{metric topology} on $X$ is the topology generated by the basis consisting of all the open balls.
\end{definition}

We prove this is a basis for a topology.

\begin{proof}
    \pf
    \step{1}{For every point $a$, there exists a ball $B$ such that $a \in B$}
    \begin{proof}
        \pf\ We have $a \in B(a,1)$.
    \end{proof}
    \step{2}{For any balls $B_1$, $B_2$ and point $a \in B_1 \cap B_2$, there exists a ball $B_3$ such that $a \in B_3 \subseteq B_1 \cap B_2$}
    \begin{proof}
        \step{a}{\pflet{$B_1 = B(c_1, \epsilon_1)$ and $B_2 = B(c_2, \epsilon_2)$}}
        \step{b}{\pflet{$\delta = \min(\epsilon_1 - d(c_1,a), \epsilon_2 - d(c_2,a))$} \prove{$B(a,\delta) \subseteq B_1 \cap B_2$}}
        \step{c}{\pflet{$x \in B(a, \delta)$}}
        \step{d}{$x \in B_1$}
        \begin{proof}
            \pf
            \begin{align*}
                d(x,c_1) & = d(x,a) + d(a,c_1) \\
                & < \delta + d(a,c_1) \\
                & \leq \epsilon_1
            \end{align*}
        \end{proof}
        \step{e}{$x \in B_2$}
        \begin{proof}
            \pf\ Similar.
        \end{proof}
    \end{proof}
    \qed
\end{proof}

\begin{proposition}
    \label{proposition:open_in_metric_space}
    Let $X$ be a metric space and $U \subseteq X$. Then $U$ is open if and only if, for every $x \in U$, there exists $\epsilon > 0$ such that $B(x, \epsilon)
    \subseteq U$.
\end{proposition}

\begin{proof}
    \pf
    \step{1}{If $U$ is open then, for all $x \in U$, there exists $\epsilon > 0$ such that $B(x, \epsilon) \subseteq U$.}
    \begin{proof}
        \step{a}{\assume{$U$ is open.}}
        \step{b}{\pflet{$x \in U$}}
        \step{c}{\pick\ $a \in X$ and $\delta > 0$ such that $x \in B(a, \delta) \subseteq U$}
        \step{d}{\pflet{$\epsilon = \delta - d(a,x)$} \prove{$B(x, \epsilon) \subseteq U$}}
        \step{e}{\pflet{$y \in B(x, \epsilon)$}}
        \step{f}{$d(y,a) < \delta$}
        \begin{proof}
            \pf
            \begin{align*}
                d(y,a) & \leq d(a,x) + d(x,y) \\
                & < \delta + d(x,y) \\
                & = \epsilon
            \end{align*}
        \end{proof}
        \step{g}{$y \in U$}
    \end{proof}
    \step{2}{If, for all $x \in U$, there exists $\epsilon > 0$ such that $B(x, \epsilon) \subseteq U$, then $U$ is open.}
    \begin{proof}
        \pf\ Immediate from definitions.
    \end{proof}
    \qed
\end{proof}

\begin{definition}[Discrete Metric]
    Let $X$ be a set. The \emph{discrete metric} on $X$ is defined by
    \[ d(x,y) = \begin{cases}
        0 & \text{if } x = y \\
        1 & \text{if } x \neq y
    \end{cases} \]
\end{definition}

\begin{proposition}
    The discrete metric induces the discrete topology.
\end{proposition}

\begin{proof}
    \pf\ For any (open) set $U$ and point $a \in U$, we have $a \in B(a,1) \subseteq U$. \qed
\end{proof}

\begin{definition}[Standard Metric on $\RR$]
    The \emph{standard metric} on $\RR$ is defined by $d(x,y) = |x-y|$.
\end{definition}

\begin{proposition}
    The standard metric on $\RR$ induces the standard topology on $\RR$.
\end{proposition}

\begin{proof}
    \pf
    \step{1}{Every open ball is open in the standard topology on $\RR$.}
    \begin{proof}
        \pf\ $B(a, \epsilon) = (a - \epsilon, a + \epsilon)$
    \end{proof}
    \step{2}{For every open set $U$ and point $a \in U$, there exists $\epsilon > 0$ such that $B(a, \epsilon) \subseteq U$}
    \begin{proof}
        \step{a}{\pflet{$U$ be an open set and $a \in U$}}
        \step{b}{\pick\ an open interval $b$, $c$ such that $a \in (b,c) \subseteq U$}
        \step{c}{\pflet{$\epsilon = \min(a-b,c-a)$}}
        \step{d}{$B(a, \epsilon) \subseteq U$}
    \end{proof}
    \qed
\end{proof}

\begin{definition}[Metrizable]
    A topological space $X$ is \emph{metrizable} if and only if there exists a metric on $X$ that induces the topology.
\end{definition}

\begin{definition}[Bounded]
    Let $X$ be a metric space and $A \subseteq X$. Then $A$ is \emph{bounded} if and only if there exists $M$ such that,
    for all $x, y \in A$, we have $d(x,y) \leq M$.
\end{definition}

\begin{definition}[Diameter]
    Let $X$ be a metric space and $A \subseteq X$. The \emph{diameter} of $A$ is
    \[ \diam A = \sup_{x,y \in A} d(x,y) \enspace . \]
\end{definition}

\begin{definition}[Standard Bounded Metric]
    Let $d$ be a metric on $X$. The \emph{standard bounded metric} corresponding to $d$ is the metric $\overline{d}$ defined by
    \[ \overline{d}(x,y) = \min(d(x,y),1) \enspace . \]
\end{definition}

We prove this is a metric.

\begin{proof}
    \pf
    \step{1}{$\overline{d}(x,y) \geq 0$}
    \begin{proof}
        \pf\ Since $d(x,y) \geq 0$
    \end{proof}
    \step{2}{$\overline{d}(x,y) = 0$ if and only if $x = y$}
    \begin{proof}
        \pf\ $\overline{d}(x,y) = 0$ if and only if $d(x,y) = 0$ if and only if $x = y$
    \end{proof}
    \step{3}{$\overline{d}(x,y) = \overline{d}(y,x)$}
    \begin{proof}
        \pf\ Since $d(x,y) = d(y,x)$
    \end{proof}
    \step{4}{$\overline{d}(x,z) \leq \overline{d}(x,y) + \overline{d}(y,z)$}
    \begin{proof}
        \pf
        \begin{align*}
            \overline{d}(x,y) + \overline{d}(y,z) & = \min(d(x,y),1) + \min(d(y,z),1) \\
            & = \min(d(x,y) + d(y,z), d(x,y) + 1, d(y,z) + 1, 2) \\
            & \geq \min(d(x,z), 1) \\
            & = \overline{d}(x,z)
        \end{align*}
    \end{proof}
    \qed
\end{proof}

\begin{lemma}
    \label{lemma:basis_radius_less_than_one}
    In any metric space $X$, the set $\BB = \{ B(a, \epsilon) \mid a \in X, \epsilon < 1 \}$ is a basis for the metric topology.
\end{lemma}

\begin{proof}
    \pf
    \step{1}{Every element of $\BB$ is open.}
    \begin{proof}
        \pf\ From Lemma \ref{lemma:basis_unions}.
    \end{proof}
    \step{2}{For every open set $U$ and point $a \in U$, there exists $B \in \BB$ such that $a \in B \subseteq U$}
    \begin{proof}
        \step{a}{\pflet{$U$ be an open set and $a \in U$}}
        \step{b}{\pick $\epsilon > 0$ such that $B(a, \epsilon) \subseteq U$}
        \step{c}{$B(a,\min(\epsilon,1/2)) \subseteq U$}
    \end{proof}
    \qedstep
    \begin{proof}
        \pf\ Lemma \ref{lemma:basis}.
    \end{proof}
    \qed
\end{proof}

\section{Metrically Equivalent}

\begin{definition}[Metrically Equivalent]
    Let $d$ and $d'$ be metrics on the same set $X$. Then $d$ and $d'$ are \emph{metrically equivalent} if and only if the identity map
    is uniformly continuous as a map $(X,d) \rightarrow (X,d')$ and as a map $(X,d') \rightarrow (X,d)$
\end{definition}

\begin{proposition}
    For any metric $d$ on a set $X$, the standard bounded metric $\overline{d}$ is metrically equivalent to $d$.
\end{proposition}

\begin{proof}
    \pf
    \step{1}{$i:(X,d) \rightarrow (X,d')$ is unifomly continuous.}
    \begin{proof}
        \pf\ For $\epsilon > 0$ and $x,y \in X$, if $d(x,y) < \epsilon$ then $\overline{d}[x,y) < \epsilon$.
    \end{proof}
    \step{2}{$i:(X,d') \rightarrow (X,d)$ is unifomly continuous.}
    \begin{proof}
        \pf\ For $\epsilon > 0$ and $x,y \in X$, if $\overline{d}(x,y) < \min(\epsilon,1/2)$ then $d[x,y) < \epsilon$.
    \end{proof}
    \qed
\end{proof}

\begin{proposition}
    If $d$ and $d'$ are metrically equivalent metrics on the same set $X$, then they induce the same topology.
\end{proposition}

\begin{proof}
    \pf
    \step{1}{Every open set under $d$ is open under $d'$.}
    \begin{proof}
        \step{a}{\pflet{$x \in X$ and $\epsilon > 0$}}
        \step{b}{\pick\ $\delta > 0$ such that, for all $y, z \in X$, if $d'(y,z) < \delta$ then $d(y,z) < \epsilon$}
        \step{c}{$B_{d'}(x,\delta) \subseteq B_d(x,\epsilon)$}
    \end{proof}
    \step{1}{Every open set under $d'$ is open under $d$.}
    \begin{proof}
        \pf\ Similar.
    \end{proof}
    \qed
\end{proof}

\begin{proposition}
    $\RR^2$ under the dictionary order topology is metrizable.
\end{proposition}

\begin{proof}
    \pf\ Define $d : \RR^2 \rightarrow \RR$ by
    \begin{align*}
        d((x,y),(x,z)) & = \max(|y-z|,1) \\
        d((x,y),(x',y')) & = 1 & \text{if } x \neq x' \qed
    \end{align*}        \step{c}{$x \in \bigcap_{i=1}^N \inv{\pi_i}() \subseteq B_D(a, \epsilon)$}
\end{proof}

\begin{proposition}
    \label{proposition:continuous_distance}
    Let $d : X^2 \rightarrow \RR$ be a metric on $X$. Then the metric topology on $X$ is the coarsest topology such that $d$ is continuous.
\end{proposition}

\begin{proof}
    \pf
    \step{1}{$d$ is continuous.}
    \begin{proof}
        \step{a}{\pflet{$a, b \in X$}}
        \step{b}{\pflet{$\epsilon > 0$}}
        \step{c}{\pflet{$\delta = \epsilon / 2$}}
        \step{d}{\pflet{$x, y \in X$}}
        \step{e}{\assume{$\rho((a,b),(x,y)) < \delta$}}
        \step{f}{$|d(a,b) - d(x,y)| < \epsilon$}
        \begin{proof}
            \step{i}{$d(a,b) - d(x,y) < \epsilon$}
            \begin{proof}
                \pf
                \begin{align*}
                    d(a,b) & \leq d(a,x) + d(x,y) + d(y,b) \\
                    & \leq d(x,y) + 2 \rho((a,b),(x,y)) \\
                    & < d(x,y) + 2 \delta \\
                    & = d(x,y) + \epsilon
                \end{align*}
            \end{proof}
            \step{ii}{$d(a.b) - d(x,y) > - \epsilon$}
            \begin{proof}
                \pf\ Similar.
            \end{proof}
        \end{proof}
        \qedstep
    \end{proof}
    \step{2}{If $\TT$ is any topology under which $d$ is continuous then $\TT$ is finer than the metric topology.}
    \begin{proof}
        \pf\ Since $B(a, \epsilon) = \inv{d_a}((-\infty, \epsilon))$
    \end{proof}
    \qed    
\end{proof}

\begin{proposition}
    Let $X$ be a metric space with metric $d$ and $A \subseteq X$. The restriction of $d$ to $A$ is a metric on $A$ that induces the subspace topology.
\end{proposition}

\begin{proof}
    \pf
    \step{1}{The restriction of $d$ to $A$ is a metric on $A$.}
    \step{2}{Every open ball under $d \restriction A$ is open under the subspace topology.}
    \begin{proof}
        \pf\ $B_{d \restriction A}(a, \epsilon) = B_d(a, \epsilon) \cap A$.
    \end{proof}
    \step{3}{If $U$ is open in the subspace topology and $x \in U$, then there exists a $d \restriction A$-ball $B$ such that $x \in B \subseteq U$.}
    \begin{proof}
        \step{a}{\pick\ $V$ open in $X$ such that $U = V \cap A$}
        \step{b}{\pick\ $\epsilon > 0$ such that $B_d(x, \epsilon) \subseteq V$}
        \step{c}{Take $B = B_{d \restriction A}(x, \epsilon)$}
    \end{proof}
    \qed
\end{proof}

\begin{corollary}
    A subspace of a metrizable space is metrizable.
\end{corollary}

\begin{proposition}
    \label{proposition:metrizable_Hausdorff}
    Every metrizable space is Hausdorff.
\end{proposition}

\begin{proof}
    \pf
    \step{1}{\pflet{$X$ be a metric space}}
    \step{2}{\pflet{$a, b \in X$ with $a \neq b$}}
    \step{3}{\pflet{$\epsilon = d(a,b) / 2$}}
    \step{4}{\pflet{$U = B(a, \epsilon)$ and $V = B(b, \epsilon)$}}
    \step{5}{$U$ and $V$ are disjoint neighbourhoods of $a$ and $b$ respectively.}
    \qed
\end{proof}

\begin{corollary}
    \label{corollary:metrizable_T1}
    Every metrizable space is $T_1$.
\end{corollary}

\begin{proposition}
    \label{proposition:metrizable_normal}
    Every metrizable space is normal.
\end{proposition}

\begin{proof}
    \pf
    \step{1}{\pflet{$X$ be a metric space.}}
    \step{2}{$X$ is $T_1$.}
    \begin{proof}
        \pf\ Corollary \ref{corollary:metrizable_T1}.
    \end{proof}
    \step{3}{\pflet{$A$ and $B$ be disjoint closed sets in $X$.}}
    \step{4}{\pflet{$U = \bigcup \{ B(a, \epsilon / 2) \mid
    a \in A, B(a, \epsilon) \cap B = \emptyset \}$}}
    \step{5}{\pflet{$V = \bigcup \{ B(b, \epsilon / 2) \mid
    b \in B, B(b, \epsilon) \cap A = \emptyset \}$}}
    \step{6}{$U$ and $V$ are disjoint open neighbourhoods of $A$ and
    $B$ respectively.}
    \qed
\end{proof}

\begin{corollary}
    Every metrizable space is completely normal.
\end{corollary}

\begin{proof}
    \pf\ Since a subspace of a metrizable space is metrizable. \qed
\end{proof}

\begin{proposition}[CC]
    \label{proposition:countable_product_metrizable}
    The product of a countable family of metrizable spaces is metrizable.
\end{proposition}

\begin{proof}
    \pf
    \step{0}{\pflet{$(X_n, d_n)$ be a sequence of metric spaces.}}
    \step{0a}{\assume{w.l.o.g.~each $d_n$ is bounded above by 1.}}
    \begin{proof}
        \pf\ By Proposition \ref{proposition:standard_bounded_metric}.
    \end{proof}
    \step{1}{\pflet{$D$ be the metric on $\RR^\omega$ defined by $D(x,y) = \sup_i (d_i(x_i, y_i) / i)$.}}
    \begin{proof}
        \step{a}{$D(x,y) \geq 0$}
        \step{b}{$D(x,y) = 0$ if and only if $x = y$}
        \step{c}{$D(x,y) = D(y,x)$}
        \step{d}{$D(x,z) \leq D(x,y) + D(y,z)$}
        \begin{proof}
            \pf
            \begin{align*}
                D(x,z) & = \sup_i \frac{d_i(x_i, z_i)}{i} \\
                & \leq \sup_i \frac{d_i(x_i, y_i) + d_i(y_i, z_i)}{i} \\
                & \leq \sup_i \frac{d_i(x_i, y_i)}{i} + \sup_i \frac{d_i(y_i, z_i)}{i} \\
                & = D(x,y) + D(y,z)
            \end{align*}
        \end{proof}
    \end{proof}
    \step{2}{Every open ball $B_D(a, \epsilon)$ is open in the product topology.}
    \begin{proof}
        \step{b}{\pick\ $N$ such that $1 / \epsilon < N$}
        \step{c}{$B_D(a,\epsilon) = \prod_{i=1}^\infty U_i$ where $U_i = B(a_i, i \epsilon)$ if $i \leq N$, and $U_i = X_i$ if $i > N$}
    \end{proof}
    \step{3}{For any open set $U$ and $a \in U$, there exists $\epsilon > 0$ such that $B_D(a, \epsilon) \subseteq U$.}
    \begin{proof}
        \step{a}{\pflet{$n \geq 1$, $V$ be an open set in $\RR$ and $a \in \inv{\pi_n}(V)$}}
        \step{b}{\pick\ $\epsilon > 0$ such that $B_{d_n}(a, \epsilon) \subseteq V$}
        \step{c}{$B_D(a, \epsilon / n) \subseteq \inv{\pi_n}(V)$}
    \end{proof}
    \qed
\end{proof}

\begin{corollary}
    The space $\RR^\omega$ is metrizable.
\end{corollary}

\begin{theorem}
    \label{theorem:continuous_metric}
    Let $X$ and $Y$ be metric spaces and $f : X \rightarrow Y$. Then $f$ is continuous if and only if, for all $x \in X$ and $\epsilon > 0$,
    there exists $\delta > 0$ such that, for all $y \in X$, if $d(x,y) < \delta$ then $d(f(x), f(y)) < \epsilon$.
\end{theorem}

\begin{proof}
    \pf
    \step{1}{If $f$ is continuous then, for all $x \in X$ and $\epsilon > 0$,
    there exists $\delta > 0$ such that, for all $y \in X$, if $d(x,y) < \delta$ then $d(f(x), f(y)) < \epsilon$}
    \begin{proof}
        \step{a}{\assume{$f$ is continuous.}}
        \step{b}{\pflet{$x \in X$ and $\epsilon > 0$}}
        \step{c}{\pick\ a neighbourhood $U$ of $x$ such that $f(U) \subseteq B(f(x), \epsilon)$}
        \begin{proof}
            \pf\ Theorem \ref{theorem:continuous}.
        \end{proof}
        \step{d}{\pick\ $\delta > 0$ such that $B(x, \delta) \subseteq U$}
        \begin{proof}
            \pf\ Proposition \ref{proposition:open_in_metric_space}.
        \end{proof}
        \step{e}{For all $y \in X$, if $d(x,y) < \delta$ then $d(f(x), f(y)) < \epsilon$}
    \end{proof}
    \step{2}{If for all $x \in X$ and $\epsilon > 0$,
    there exists $\delta > 0$ such that, for all $y \in X$, if $d(x,y) < \delta$ then $d(f(x), f(y)) < \epsilon$, then $f$ is continuous.}
    \begin{proof}
        \step{a}{\assume{for all $x \in X$ and $\epsilon > 0$, there exists
        $\delta > 0$ such that, for all $y \in X$, if $d(x,y) < \delta$
        then $d(f(x),f(y)) < \epsilon$}}
        \step{b}{\pflet{$x \in X$ and $V$ be a neighbourhood of $f(x)$}}
        \step{c}{\pick\ $\epsilon > 0$ such that $B(f(x),\epsilon) \subseteq V$}
        \begin{proof}
            \pf\ Proposition \ref{proposition:open_in_metric_space}.
        \end{proof}
        \step{d}{\pick\ $\delta > 0$ such that, for all $y \in X$, if $d(x,y) < \delta$
        then $d(f(x),f(y)) < \epsilon$}
        \begin{proof}
            \pf\ By \stepref{a}
        \end{proof}
        \step{e}{\pflet{$U = B(x,\delta)$}}
        \step{f}{$U$ is a neighbourhood of $x$ with $f(U) \subseteq V$}
        \qedstep
        \begin{proof}
            \pf\ Theorem \ref{theorem:continuous}.
        \end{proof}
    \end{proof}
    \qed
\end{proof}

\begin{proposition}
    \label{proposition:convergence_metric}
    Let $X$ be a metric space. Let $(a_n)$ be a sequence in $X$ and $l \in X$.
    Then $a_n \rightarrow l$ as $n \rightarrow \infty$ if and only if,
    for all $\epsilon > 0$, there exists $N$ such that, for all $n \geq N$,
    we have $d(a_n, l) < \epsilon$.
\end{proposition}

\begin{proof}
    \pf\ From Proposition \ref{proposition:convergence_basis}. \qed
\end{proof}

\begin{proposition}
    \label{proposition:metrizable_first_countable}
    Every metrizable space is first countable.
\end{proposition}

\begin{proof}
    \pf\ In any metric space $X$, the open balls $B(a,1/n)$ for $n \geq 1$ form a local basis at $a$.
\end{proof}

\begin{corollary}
    The space $\RR^I$ is not metrizable.
\end{corollary}

\begin{example}
    $\RR^\omega$ under the box topology is not metrizable.
\end{example}

\begin{example}
    If $J$ is uncountable then $\RR^J$ under the product topology is not metrizable.
\end{example}

\begin{example}
    The space $\overline{S_\Omega}$ is not metrizable by Example \ref{example:S_Omega_bar_not_first_countable}.
\end{example}

\begin{example}
    The space $S_\Omega \times \overline{S_\Omega}$ is not metrizable because it is not first countable.
\end{example}

\begin{proposition}[Choice]
    \label{proposition:compact_bounded}
    Let $X$ be a metrizable space. The following are equivalent:
    \begin{enumerate}
        \item $X$ is bounded under every metric that induces the topology of $X$.
        \item Every continuous function $X \rightarrow \RR$ is bounded.
        \item $X$ is compact.
    \end{enumerate}
\end{proposition}

\begin{proof}
    \pf
    \step{1}{$1 \Rightarrow 2$}
    \begin{proof}
        \step{a}{\assume{$X$ is bounded under every metric that induces the topology of $X$.}}
        \step{b}{\pflet{$\phi : X \rightarrow \RR$ be continuous.}}
        \step{c}{Define $F : X \rightarrow X \times \RR$ by $F(x) = (x,\phi(x))$}
        \step{d}{$F$ is an imbedding.}
        \step{e}{\pick\ a metric $d$ on $X$ that induces its topology.}
        \step{f}{Define $d' : X^2 \rightarrow \RR$ by: $d'(x,y) = d(x,y) + |\phi(x) - \phi(y)|$}
        \step{g}{$d'$ is a metric that induces the topology on $X$.}
        \begin{proof}
            \pf\ It induces the product topology on $X \times \RR$ hence on $F(X)$.
        \end{proof}
        \step{h}{$d'$ is bounded.}
        \step{i}{\pick\ $B$ such that $d'(x,y) < B$ for all $x,y$}
        \step{j}{$|\phi(x)-\phi(y)| < B$ for all $x,y \in X$}
    \end{proof}
    \step{2}{$2 \Rightarrow 3$}
    \begin{proof}
        \step{a}{\assume{Every continuous function $X \rightarrow \RR$
        is bounded.}}
        \step{b}{\assume{For a contradiction $A \subseteq X$ is infinite and has no limit point.}}
        \step{c}{\pick\ a surjection $\phi : A \twoheadrightarrow \ZZ^+$}
        \step{d}{$\phi$ is continuous.}
        \begin{proof}
            \pf\ Since $A$ is discrete.
        \end{proof}
        \step{e}{$\phi$ is bounded.}
        \qedstep
        \begin{proof}
            \pf\ This contradicts the fact that $\phi$ is surjective.
        \end{proof}
    \end{proof}
    \step{3}{$3 \rightarrow 1$}
    \begin{proof}
        \step{1}{\assume{$X$ is compact.}}
        \step{1a}{\pflet{$d$ be any metric that induces the topology of $X$.}}
        \step{2}{\pick\ $a \in X$}
        \step{3}{$\{ B(a,n) \mid n \in \ZZ^+ \}$ covers $X$}
        \step{4}{\pick\ a finite subcover $\{ B(a,n_1), \ldots, B(a,n_k) \}$}
        \step{5}{\pflet{$N = \max(n_1, \ldots, n_k)$}}
        \step{6}{For all $x, y \in X$ we have $d(x,y) < 2N$}
        \begin{proof}
            \pf
            \begin{align*}
                d(x,y) & \leq d(x,a) + d(a,y) \\
                & < N + N
            \end{align*}
        \end{proof}
    \end{proof}
    \qed
\end{proof}

\begin{proposition}
    \label{proposition:bounded_compact}
    A compact subspace of a metric space is bounded.
\end{proposition}

This example shows the converse does not hold:

\begin{example}
    The space $\RR$ under the standard bounded metric is bounded but not compact.
\end{example}

\begin{proposition}
    A connected metric space with more than one point is uncountable.
\end{proposition}

\begin{proof}
    \pf
    \step{1}{\pflet{$X$ be a connected metric space with more than one point.}}
    \step{2}{\pick\ $a \in X$}
    \step{3}{$d(a,-) : X \rightarrow \RR$ is continuous.}
    \begin{proof}
        \pf\ Proposition \ref{proposition:continuous_distance}.
    \end{proof}
    \step{4}{$\{ d(a,x) \mid x \in X \}$ is a connected subspace of $\RR$ that includes 0.}
    \begin{proof}
        \pf\ Theorem \ref{theorem:connected_continuous_image}.
    \end{proof}
    \step{5}{$\{ d(a,x) \mid x \in X \} \neq \{0\}$}
    \begin{proof}
        \pf\ Since $X$ has more than one point.
    \end{proof}
    \step{6}{$\{ d(a,x) \mid x \in X \}$ is uncountable.}
    \begin{proof}
        \pf\ Since it includes a closed interval (Corollary \ref{corollary:closed_interval_uncountable}).
    \end{proof}
    \qed
\end{proof}

\begin{corollary}
    Every second countable locally compact Hausdorff space is metrizable.
\end{corollary}

\begin{example}
    Not every second countable Hausdorff space is metrizable.

    The space $\RR_K$ is second countable and Hausdorff but not metrizable.
\end{example}

\begin{example}
    There exists a space that is perfectly normal, first countable, Lindel\"{o}f,
    and separable, but not metrizable.

    The space $\RR_l$ is such a space.
\end{example}

\begin{proposition}
    Let $X$ be a compact Hausdorff space that is the union of the closed
    subspaces $X_1$ and $X_2$. If $X_1$ and $X_2$ are metrizable then $X$
    is metrizable.
\end{proposition}

\begin{proof}
    \pf
    \step{1}{\pick\ countable bases $\BB_1$ and $\BB_2$ for $X_1$ and $X_2$
    respectively.}
    \step{2}{For $B \in \BB_1$, \pick\ an open set
    $U_B$ in $X$ such that $U_B \cap X_1 = B$}
    \step{3}{For $B \in \BB_2$, \pick\ an open set
    $V_B$ in $X$ such that $V_B \cap X_2 = B$}
    \step{4}{\pflet{$\AA = \{ U_B \mid B \in \BB_1 \}
    \cup \{ V_B \mid B \in \BB_2 \} \cup \{ X_1 - X_2, X_2 - X_1 \}$}}
    \step{5}{\pflet{$\BB$ be the set of all finite intersections of
    elements of $\AA$.} \prove{$\BB$ is a basis for $X$.}}
    \step{6}{\pflet{$x \in X$}}
    \step{7}{\pflet{$U$ be an open neighbourhood of $x$}}
    \step{8}{\case{$x \in X_1 - X_2$}}
    \begin{proof}
        \step{a}{\pick\ $B \in \BB_1$ such that $x \in B \subseteq U \cap X_1$}
        \step{b}{$x \in U_B \cap (X_1 - X_2) \subseteq U$}
    \end{proof}
    \step{9}{\case{$x \in X_2 - X_1$}}
    \begin{proof}
        \pf\ Similar.
    \end{proof}
    \step{10}{\case{$x \in X_1 \cap X_2$}}
    \begin{proof}
        \step{a}{\pick\ $B \in \BB_1$ such that $x \in B \subseteq U \cap X_1$}
        \step{b}{\pick\ $B' \in \BB_1$ such that $x \in B' \subseteq U \cap X_2$}
        \step{c}{$x \in U_B \cap V_{B'} \subseteq U$}
    \end{proof}
    \qed
\end{proof}

\begin{example}
    The ordered square is not metrizable, because it is Lindel\"{o}f but not second countable.
\end{example}

\begin{proposition}
    The continuous image of a metrizable space is not necessarily metrizable.
\end{proposition}

\begin{proof}
    \pf\ Take the identity function from a set under the discrete topology
    to the same set under the indiscrete topology. \qed
\end{proof}

\begin{proposition}
    Let $X$ be a metrizable space. Then $X$ has a metrizable compactification if and only if $X$ is second countable.
\end{proposition}

\begin{proof}
    \pf
    \step{1}{If $X$ has a metrizable compactification then $X$ is second countable.}
    \begin{proof}
        \step{a}{\assume{$X$ has a metrizable compactification.}}
        \step{b}{\pick\ a metrizable compactification $Y$ of $X$.}
        \step{c}{$Y$ is second countable.}
        \begin{proof}
            \pf\ Proposition \ref{proposition:Lindelof_metrizable_second_countable}.
        \end{proof}
        \step{d}{$X$ is second countable.}
        \begin{proof}
            \pf\ Proposition \ref{proposition:second_countable_subspace}.
        \end{proof}
    \end{proof}
    \step{2}{If $X$ is second countable then it has a metrizable compactification.}
    \begin{proof}
        \step{a}{\assume{$X$ is second countable.}}
        \step{b}{\pick\ an embedding $e : X \rightarrow [0,1]^\omega$}
        \begin{proof}
            \pf\ By the Urysohn Metrization Theorem.
        \end{proof}
        \step{c}{$e : X \rightarrow \overline{e(X)}$ is a metrizable compactification of $X$.}
    \end{proof}
\end{proof}

\begin{proposition}
    Let $X$ be a completely regular space. If $\beta(X)$ is metrizable then $X$ is compact.
\end{proposition}

\begin{proof}
    \pf
    \step{1}{\assume{$\beta(X)$ is metrizable}}
    \step{2}{$X$ is metrizable.}
    \step{3}{$X$ is normal.}
    \step{4}{$\beta(X) = X$}
    \begin{proof}
        \step{4}{\pflet{$y \in \beta(X)$}}
        \step{5}{There exists a sequence in $X$ that converges to $y$.}
        \begin{proof}
            \pf\ By the Sequence Lemma since $\beta(X)$ is first countable.
        \end{proof}
        \step{a}{$y \in X$}
        \begin{proof}
            \pf\ Proposition \ref{proposition:stone_cech_sequence}.
        \end{proof}
    \end{proof}
    \qed
\end{proof}

\begin{proposition}
    Every separable metrizable space is second countable.
\end{proposition}

\begin{proof}
    \pf
    \step{1}{\pflet{$X$ be a separable metrizable space.}}
    \step{2}{\pick\ a countable dense subset $D$. \prove{$\{ B(x,1/n) \mid x \in D,
    n \in \ZZ^+ \}$ is a basis for $X$}}
    \step{3}{\pflet{$x \in X$}}
    \step{4}{\pflet{$U$ be a neighbourhood of $x$.}}
    \step{5}{\pick\ $n$ such that $B(x,1/n) \subseteq U$.}
    \step{6}{\pick\ $d \in D \cap B(x,1/2n)$}
    \step{7}{$x \in B(d,1/2n) \subseteq U$.}
    \qed
\end{proof}

\section{Real Linear Algebra}

\begin{definition}[Square Metric]
    The \emph{square metric} $\rho$ on $\RR^n$ is defined by
    \[ \rho(\vec{x}, \vec{y}) = \max(|x_1 - y_1|, \ldots, |x_n - y_n|) \]
\end{definition}

We prove this is a metric.

\begin{proof}
    \pf
    \step{1}{$\rho(\vec{x}, \vec{y}) \geq 0$}
    \begin{proof}
        \pf\ Immediate from definition.
    \end{proof}
    \step{2}{$\rho(\vec{x}, \vec{y}) = 0$ if and only if $\vec{x} = \vec{y}$}
    \begin{proof}
        \pf\ Immediate from definition.
    \end{proof}
    \step{3}{$\rho(\vec{x}, \vec{y}) = \rho(\vec{y}, \vec{x})$}
    \begin{proof}
        \pf\ Immediate from definition.
    \end{proof}
    \step{4}{$\rho(\vec{x}, \vec{z}) \leq \rho(\vec{x}, \vec{y}) + \rho(\vec{y}, \vec{z})$}
    \begin{proof}
        \pf\ Since $|x_i - z_i| \leq |x_i - y_i| + |y_i - z_i|$.
    \end{proof}
    \qed
\end{proof}

\begin{proposition}
    The square metric induces the standard topology on $\RR^n$.
\end{proposition}

\begin{proof}
    \pf
    \step{1}{For every $a \in X$ and $\epsilon > 0$, we have $B_\rho(a, \epsilon)$ is open in the standard product topology.}
    \begin{proof}
        \pf
        \[ B_\rho(a, \epsilon) = (a_1 - \epsilon, a_1 + \epsilon) \times \cdots \times (a_n - \epsilon, a_n + \epsilon) \]
    \end{proof}
    \step{2}{For any open sets $U_1$, \ldots, $U_n$ in $\RR$, we have $U_1 \times \cdots \times U_n$ is open in the square metric topology.}
    \begin{proof}
        \step{a}{\pflet{$\vec{a} \in U_1 \times \cdots \times U_n$}}
        \step{b}{For $i = 1, \ldots, n$, \pick\ $\epsilon_i > 0$ such that $(a_i - \epsilon_i, a_i + \epsilon_i) \subseteq U_i$}
        \step{c}{\pflet{$\epsilon = \min(\epsilon_1, \ldots, \epsilon_n)$}}
        \step{d}{$B_\rho(\vec{a}, \epsilon) \subseteq U$}
    \end{proof}
    \qed
\end{proof}

\begin{definition}
    Given $\vec{x}, \vec{y} \in \RR^n$, define the \emph{sum} $\vec{x} + \vec{y}$ by
    \[ (x_1, \ldots, x_n) + (y_1, \ldots, y_n) = (x_1 + y_1, \ldots, x_n + y_n) \enspace . \]
\end{definition}

\begin{definition}
    Given $\lambda \in \RR$ and $\vec{x} \in \RR^n$, define the \emph{scalar product} $\lambda \vec{x} \in \RR^n$ by
    \[ \lambda (x_1, \ldots, x_n) = (\lambda x_1, \ldots, \lambda x_n) \]
\end{definition}

\begin{definition}[Inner Product]
    Given $\vec{x}, \vec{y} \in \RR^n$, define the \emph{inner product} $\vec{x} \cdot \vec{y} \in \RR$ by
    \[ (x_1, \ldots, x_n) \cdot (y_1, \ldots, y_n) = x_1 y_1 + \cdots + x_n y_n \enspace . \]
    We write $\vec{x}^2$ for $\vec{x} \cdot \vec{x}$.
\end{definition}

\begin{definition}[Norm]
    Let $n \geq 1$. The \emph{norm} on $\RR^n$ is the function $\| \ \| : \RR^n \rightarrow \RR$ defined by
    \[ \| (x_1, \ldots, x_n) \| = \sqrt{x_1^2 + \cdots + x_n^2} \]
\end{definition}

\begin{lemma}
    \[ \| \vec{x} \|^2 = \vec{x}^2 \]
\end{lemma}

\begin{proof}
    \pf\ Immediate from definitions. \qed
\end{proof}

\begin{lemma}
    \[ \vec{x} \cdot (\vec{y} + \vec{z}) = \vec{x} \cdot \vec{y} + \vec{x} \cdot \vec{z} \]
\end{lemma}

\begin{proof}
    \pf\ Each is equal to $(x_1 y_1 + x_1 z_1, \ldots, x_n y_n + x_n z_n)$. \qed
\end{proof}

\begin{lemma}
    \label{lemma:Cauchy-Schwarz}
    \[ |\vec{x} \cdot \vec{y}| \leq \| \vec{x} \| \| \vec{y} \| \]
\end{lemma}

\begin{proof}
    \pf
    \step{1}{\assume{$\vec{x} \neq \vec{0} \neq \vec{y}$}}
    \begin{proof}
        \pf\ Otherwise both sides are 0.
    \end{proof}
    \step{2}{\pflet{$a = 1 / \| \vec{x} \|$}}
    \step{3}{\pflet{$b = 1 / \| \vec{y} \|$}}
    \step{5}{$(a \vec{x} + b \vec{y})^2 \geq 0$ and $(a \vec{x} - b \vec{y})^2 \geq 0$}
    \step{6}{$a^2 \| \vec{x} \|^2 + 2 a b \vec{x} \cdot \vec{y} + b^2 \| \vec{y} \|^2 \geq 0$ and $a^2 \| \vec{x} \|^2 - 2 a b \vec{x} \cdot \vec{y} + b^2 \| \vec{y} \|^2 \geq 0$}
    \step{7}{$2ab \vec{x} \cdot \vec{y} + 2 \geq 0$ and $-2ab \vec{x} \cdot \vec{y} + 2 \geq 0$}
    \step{8}{$\vec{x} \cdot \vec{y} \geq - 1/ab$ and $\vec{x} \cdot \vec{y} \leq 1/ab$}
    \step{9}{$\vec{x} \cdot \vec{y} \geq - \| \vec{x} \| \| \vec{y} \|$ and $\vec{x} \cdot \vec{y} \leq \| \vec{x} \| \| \vec{y} \|$}
    \qed
\end{proof}

\begin{lemma}[Triangle Inequality]
    \label{lemma:triangle_inequality}
    \[ \| \vec{x} + \vec{y} \| \leq \| \vec{x} \| + \| \vec{y} \| \]
\end{lemma}

\begin{proof}
    \pf
    \begin{align*}
        \| \vec{x} + \vec{y} \|^2 & = \| \vec{x} \|^2 + 2 \vec{x} \cdot \vec{y} + \| \vec{y} \|^2 \\
        & \leq \| \vec{x} \|^2 + 2 \| \vec{x} \| \| \vec{y} \| + \| \vec{y} \|^2 & (\text{Lemma \ref{lemma:Cauchy-Schwarz}}) \\
        & = (\| \vec{x} \| + \| \vec{y} \|)^2 & \qed
    \end{align*}
\end{proof}

\begin{definition}[Euclidean Metric]
    Let $n \geq 1$. The \emph{Euclidean metric} on $\RR^n$ is defined by
    \[ d(\vec{x}, \vec{y}) = \| \vec{x} - \vec{y} \| \enspace . \]
\end{definition}

We prove this is a metric.

\begin{proof}
    \step{1}{$d(\vec{x}, \vec{y}) \geq 0$}
    \begin{proof}
        \pf\ Immediate from definition.
    \end{proof}
    \step{2}{$d(\vec{x}, \vec{y}) = 0$ if and only if $\vec{x} = \vec{y}$}
    \begin{proof}
        \pf\ $d(\vec{x}, \vec{y}) = 0$ if and only if $\vec{x} - \vec{y} = \vec{0}$.
    \end{proof}
    \step{3}{$d(\vec{x}, \vec{y}) = d(\vec{y}, \vec{x})$}
    \begin{proof}
        \pf\ Immediate from definition.
    \end{proof}
    \step{4}{$d(\vec{x}, \vec{z}) \leq d(\vec{x}, \vec{y}) + d(\vec{y}, \vec{z})$}
    \begin{proof}
        \pf
        \begin{align*}
            \| \vec{x} - \vec{z} \| & = \| (\vec{x} - \vec{y}) + (\vec{y} - \vec{z}) \| \\
            & \leq \| \vec{x} - \vec{y} \| + \| \vec{y} - \vec{z} \| & (\text{Lemma \ref{lemma:triangle_inequality}})
        \end{align*}
    \end{proof}
    \qed
\end{proof}

\begin{proposition}
    The Euclidean metric induces the standard topology on $\RR^n$.
\end{proposition}

\begin{proof}
    \pf
    \step{1}{\pflet{$\rho$ be the square metric.}}
    \step{2}{For all $\vec{a} \in \RR^n$ and $\epsilon > 0$, we have $B_d(\vec{a}, \epsilon) \subseteq B_\rho(\vec{a}, \epsilon)$}
    \begin{proof}
        \step{a}{\pflet{$\vec{x} \in B_d(\vec{a}, \epsilon)$}}
        \step{b}{$\sqrt{(x_1 - a_1)^2 + \cdots + (x_n - a_n)^2} < \epsilon$}
        \step{c}{$(x_1 - a_1)^2 + \cdots + (x_n - a_n)^2 < \epsilon^2$}
        \step{d}{For all $i$ we have $(x_i - a_i)^2 < \epsilon^2$}
        \step{e}{For all $i$ we have $|x_i - a_i| < \epsilon$}
        \step{f}{$\rho(\vec{x}, \vec{a}) < \epsilon$}
    \end{proof}
    \step{3}{For all $\vec{a} \in \RR^n$ and $\epsilon > 0$, we have $B_\rho(\vec{a}, \epsilon / \sqrt{n}) \subseteq B_d(\vec{a}, \epsilon)$}
    \begin{proof}
        \step{a}{\pflet{$\vec{x} \in B_\rho(\vec{a}, \epsilon / \sqrt{n})$}}
        \step{b}{$\rho(\vec{x}, \vec{a}) < \epsilon / \sqrt{n}$}
        \step{c}{For all $i$ we have $|x_i - x_a| < \epsilon / \sqrt{n}$}
        \step{d}{For all $i$ we have $(x_i - x_a)^2 < \epsilon^2 / n$}
        \step{e}{$d(\vec{x}, \vec{a}) < \epsilon$}
    \end{proof}
    \qedstep
    \begin{proof}
        \pf\ By Lemma \ref{lemma:metrics_same_topology}.
    \end{proof}
    \qed
\end{proof}

\begin{proposition}
    Let $n \geq 0$. For all $c \in \RR^n$ and $\epsilon > 0$, the open ball $B(c,\epsilon)$ is path connected.
\end{proposition}

\begin{proof}
    \pf
    \step{1}{\pflet{$a, b \in B(c,\epsilon)$}}
    \step{2}{\pflet{$p : [0,1] \rightarrow B(c,\epsilon)$ be the function $p(t) = (1-t)a + tb$}}
    \begin{proof}
        \pf\ We have $p(t) \in B(c,\epsilon)$ for all $t$ because
        \begin{align*}
            d(p(t),c) & = \| (1-t)a + tb - c \| \\
        & = \| (1-t)(a-c) + t(b-c) \| \\
        & \leq (1-t) \| a-c \| + t \| b-c \| \\
        & < (1-t) \epsilon + t \epsilon \\
        & = \epsilon
        \end{align*}
    \end{proof}
    \step{3}{$p$ is a path from $a$ to $b$.}
    \qed
\end{proof}

\begin{proposition}
    Let $n \geq 0$. For all $c \in \RR^n$ and $\epsilon > 0$, the closed ball $\overline{B(c,\epsilon)}$ is path connected.
\end{proposition}

\begin{proof}
    \pf
    \step{1}{\pflet{$a, b \in \overline{B(c,\epsilon)}$}}
    \step{2}{\pflet{$p : [0,1] \rightarrow \overline{B(c,\epsilon)}$ be the function $p(t) = (1-t)a + tb$}}
    \begin{proof}
        \pf\ We have $p(t) \in \overline{B(c,\epsilon)}$ for all $t$ because
        \begin{align*}
            d(p(t),c) & = \| (1-t)a + tb - c \| \\
        & = \| (1-t)(a-c) + t(b-c) \| \\
        & \leq (1-t) \| a-c \| + t \| b-c \| \\
        & \leq (1-t) \epsilon + t \epsilon \\
        & = \epsilon
        \end{align*}
    \end{proof}
    \step{3}{$p$ is a path from $a$ to $b$.}
    \qed
\end{proof}

\begin{lemma}
    If $\sum_{i=0}^\infty x_i^2$ and $\sum_{i=0}^\infty y_i^2$ converge then $\sum_{i=0}^\infty |x_i y_i|$ converges.
\end{lemma}

\begin{proof}
    \pf
    \step{1}{For all $N \geq 0$ we have
        $\sum_{i=0}^N |x_i y_i| \leq \sqrt{\sum_{i=0}^N |x_i|^2} \sqrt{\sum_{i=0}^N |y_i|^2}$}
    \begin{proof}
        \pf\ By the Cauchy-Schwarz inequality
    \end{proof}
    \qedstep
    \begin{proof}
        \pf\ Since $\sum_{i=0}^N |x_i y_i|$ is an increasing sequence bounded above by \\ $(\sum_{i=0}^\infty x_i^2) (\sum_{i=0}^\infty y_i^2)$.
    \end{proof}
    \qed
\end{proof}

\begin{corollary}
    \label{corollary:l2_sum_converge}
    If $\sum_{i=0}^\infty x_i^2$ and $\sum_{i=0}^\infty y_i^2$ converge then $\sum_{i=0}^\infty (x_i + y_i)^2$ converges.
\end{corollary}

\begin{proof}
    \pf\ Since $\sum_{i=0}^\infty x_i^ 2$, $\sum_{i=0}^\infty y_i^2$ and $2 \sum_{i=0}^\infty x_i y_i$ all converge.
\end{proof}

\begin{definition}[$l^2$-metric]
    The \emph{$l^2$-metric} on 
    \[ \left\{ (x_n) \in \RR^\omega \mid \sum_{n=0}^\infty x_n^2 \text{ converges} \right\} \] is defined by
    \[ d(x,y) = \left( \sum_{n=0}^\infty (x_n - y_n)^2 \right)^{1/2} \]
\end{definition}

We prove this is a metric.

\begin{proof}
    \pf
    \step{1}{$d$ is well-defined.}
    \begin{proof}
        \pf\ By Corollary \ref{corollary:l2_sum_converge}.
    \end{proof}
    \step{2}{$d(x,y) \geq 0$}
    \step{3}{$d(x,y) = 0$ if and only if $x = y$}
    \step{4}{$d(x,y) = d(y,x)$}
    \step{5}{$d(x,z) \leq d(x,y) + d(y,z)$}
    \begin{proof}
        \pf\ By Lemma \ref{lemma:triangle_inequality}.
    \end{proof}
    \qed
\end{proof}

\begin{theorem}
    Addition is a continuous function $\RR^2 \rightarrow \RR$.
\end{theorem}

\begin{proof}
    \pf
    \step{1}{\pflet{$a, b \in \RR$}}
    \step{2}{\pflet{$\epsilon > 0$}}
    \step{3}{\pflet{$\delta = \epsilon / 2$}}
    \step{4}{\pflet{$x, y \in \RR$}}
    \step{5}{\assume{$\rho((a,b),(x,y)) < \delta$}}
    \step{6}{$|(a+b)-(x+y)| < \epsilon$}
    \begin{proof}
        \pf
        \begin{align*}
            |(a+b)-(x+y)| & = |a-x| + |b-y| \\
            & \leq 2 \rho((a,b),(x,y)) \\
            & < 2 \delta \\
            & = \epsilon
        \end{align*}
    \end{proof}
    \qedstep
    \begin{proof}
        \pf\ Theorem \ref{theorem:continuous_metric}
    \end{proof}
    \qed
\end{proof}

\begin{theorem}
    Multiplication is a continuous function $\RR^2 \rightarrow \RR$.
\end{theorem}

\begin{proof}
    \pf
    \step{1}{\pflet{$a, b \in \RR$}}
    \step{2}{\pflet{$\epsilon > 0$}}
    \step{3}{\pflet{$\delta = \min(\epsilon / ( |a| + |b| + 1), 1)$}}
    \step{4}{\pflet{$x, y \in \RR$}}
    \step{5}{\assume{$\rho((a,b),(x,y)) < \delta$}}
    \step{6}{$|ab - xy| < \epsilon$}
    \begin{proof}
        \pf
        \begin{align*}
            |ab - xy| & = |a(b-y) + (a-x)b - (a-x)(b-y)| \\
            & \leq |a| |b-y| + |b| |a-x| + |a-x| |b-y| \\
            & < |a| \delta + |b| \delta + \delta^2 & (\text{\stepref{5}})\\
            & \leq |a| \delta + |b| \delta + \delta & (\text{\stepref{3}})\\
            & \leq \epsilon & (\text{\stepref{3}})
        \end{align*}
    \end{proof}
    \qedstep
    \begin{proof}
        \pf\ Theorem \ref{theorem:continuous_metric}
    \end{proof}
    \qed
\end{proof}

\begin{theorem}
    The function $f : \RR \setminus \{ 0 \} \rightarrow \RR$ defined by $f(x) = \inv{x}$ is continuous.
\end{theorem}

\begin{proof}
    \pf
    \step{1}{For all $a \in \RR$ we have $\inv{f}((a,+\infty))$ is open.}
    \begin{proof}
        \pf\ The set is
        \begin{align*}
            (\inv{a}, + \infty) & \text{if } a > 0 \\
            (0, + \infty) & \text{if } a = 0 \\
            (- \infty, \inv{a}) \cup (0, + \infty) & \text{if } a < 0
        \end{align*}
    \end{proof}
    \step{2}{For all $a \in \RR$ we have $\inv{f}((-\infty,a))$ is open.}
    \begin{proof}
        \pf\ Similar.
    \end{proof}
    \qedstep
    \begin{proof}
        \pf\ By Proposition \ref{proposition:continuous_subbasis} and Lemma
        \ref{lemma:open_rays_subbasis}.
    \end{proof}
    \qed
\end{proof}

\begin{definition}
    For $n \geq 0$, the \emph{unit ball} $B^n$ is the space $\{ x \in \RR^n \mid \| x \| \leq 1 \}$.
\end{definition}

\begin{proposition}
    For all $n \geq 0$, the unit ball $B^n$ is path connected.
\end{proposition}

\begin{proof}
    \pf
    \step{1}{\pflet{$a, b \in B^n$}}
    \step{2}{\pflet{$p : [0,1] \rightarrow B^n$ be the function $p(t) = (1-t)a + tb$}}
    \begin{proof}
        \pf\ We have $p(t) \in B^n$ for all $t$ because
        \begin{align*}
            \| (1-t)a + tb \| & \leq (1-t) \| a \| + t \| b \| \\
            & \leq (1-t) + t \\
            & = 1
        \end{align*}
    \end{proof}
    \step{3}{$p$ is a path from $a$ to $b$.}
    \qed
\end{proof}

\begin{definition}[Punctured Euclidean Space]
    For $n \geq 0$, defined \emph{punctured Euclidean space} to be $\RR^n \setminus \{ 0 \}$.
\end{definition}

\begin{proposition}
    For $n > 1$, punctured Euclidean space $\RR^n \setminus \{ 0 \}$ is path connected.
\end{proposition}

\begin{proof}
    \pf
    \step{1}{\pflet{$a, b \in \RR^n \setminus \{0\}$}}
    \step{2}{\case{0 is on the line from $a$ to $b$}}
    \begin{proof}
        \step{a}{\pick\ a point $c$ not on the line from $a$ to $b$}
        \step{b}{The path consisting of a straight line from $a$ to $c$ followed by a straight line from $c$ to $b$ is a path from $a$ to $b$.}
    \end{proof}
    \step{3}{\case{0 is not on the line from $a$ to $b$}}
    \begin{proof}
        \pf\ The straight line from $a$ to $b$ is a path from $a$ to $b$.
    \end{proof}
\end{proof}

\begin{corollary}
    For $n > 1$, the spaces $\RR$ and $\RR^n$ are not homeomorphic.
\end{corollary}

\begin{proof}
    \pf\ For any point $a$, the space $\RR \setminus \{a\}$ is disconnected.
\end{proof}

\begin{definition}[Unit Sphere]
    For $n \geq 1$, the \emph{unit sphere} $S^{n-1}$ is the space
    \[ S^{n-1} = \{ x \in \RR^n \mid \| x \| = 1 \} \enspace . \]
\end{definition}

\begin{proposition}
    For $n > 1$, the unit sphere $S^{n-1}$ is path connected.
\end{proposition}

\begin{proof}
    \pf\ The map $g : \RR^n \setminus \{ 0 \} \rightarrow S^{n-1}$ defined by $g(x) = x / \| x \|$ is continuous and surjective. The result follows by Proposition
    \ref{proposition:path_connected_continuous_image}. \qed
\end{proof}

\begin{proposition}
    Let $f : S^1 \rightarrow \RR$ be continuous. Then there exists $x \in S^1$ such that $f(x) = f(-x)$.
\end{proposition}

\begin{proof}
    \pf
    \step{1}{\pflet{$g : S^1 \rightarrow \RR$ be the function $g(x) = f(x) - f(-x)$} \prove{There exists $x \in S^1$ such that $g(x) = 0$}}
    \step{2}{\assume{without loss of generality $g((1,0)) > 0$}}
    \step{3}{$g((-1,0)) < 0$}
    \step{4}{There exists $x$ such that $g(x) = 0$}
    \begin{proof}
        \pf\ By the Intermediate Value Theorem.
    \end{proof}
    \qed
\end{proof}

\begin{definition}[Topologist's Sine Curve]
    Let $S = \{ (x, \sin 1/x) \mid 0 < x \leq 1 \}$. The \emph{topologist's sine curve} is the closure $\overline{S}$ of $S$.
\end{definition}

\begin{proposition}
    \[ \overline{S} = S \cup (\{0\} \times [-1,1]) \]
\end{proposition}
\begin{proposition}
    The topologist's sine curve is connected.
\end{proposition}

\begin{proof}
    \pf
    \step{1}{\pflet{$S = \{ (x, \sin 1/x) \mid 0 < x \leq 1 \}$}}
    \step{2}{$S$ is connected.}
    \begin{proof}
        \pf\ Theorem \ref{theorem:connected_continuous_image}.
    \end{proof}
    \step{3}{$\overline{S}$ is connected.}
    \begin{proof}
        \pf\ Theorem \ref{theorem:connected_closure}.
    \end{proof}
    \qed
\end{proof}

\begin{proposition}[CC]
    The topologist's sine curve is not path connected.
\end{proposition}

\begin{proof}
    \pf
    \step{1}{\assume{for a contradiction $p : [0,1] \rightarrow \overline{S}$ is a path from $(0,0)$ to $(1, \sin 1)$.}}
    \step{2}{$\inv{p}(\{0\} \times [0,1])$ is closed.}
    \step{3}{\pflet{$b$ be the greatest element of $\inv{p}(\{0\} \times [0,1])$.}}
    \step{3a}{$b < 1$}
    \begin{proof}
        \pf\ Since $p(1) = (1, \sin 1)$.
    \end{proof}
    \step{4}{\pick\ a sequence $(t_n)_{n \geq 1}$ in $(b,1]$ such that $t_n \rightarrow b$ and $\pi_2(p(t_n)) = (-1)^n$}
    \begin{proof}
        \step{a}{\pflet{$n \geq 1$}}
        \step{b}{\pick\ $u$ with $0 < u < \pi_1(p(1/n))$ such that $\sin (1/u) = (-1)^n$}
        \step{c}{\pick\ $t_n$ such that $0 < t_n < 1/n$ and $\pi_1(p(t_n)) = u$}
        \begin{proof}
            \pf\ One exists by the Intermediate Value Theorem.
        \end{proof}
    \end{proof}
    \qedstep
    \begin{proof}
        \pf\ This contradicts \ref{proposition:converge_continuous}.
    \end{proof}
    \qed
\end{proof}

\begin{theorem}
    Let $A$ be a subspace of $\RR^n$. Then the following are equivalent:
    \begin{enumerate}
        \item $A$ is compact.
        \item $A$ is closed and bounded under the Euclidean metric.
        \item $A$ is closed and bounded under the square metric.
    \end{enumerate}
\end{theorem}

\begin{proof}
    \pf
    \step{1}{$1 \Rightarrow 2$}
    \begin{proof}
        \pf\ By Corollary \ref{corollary:closed_compact} and Proposition \ref{proposition:bounded_compact}.
    \end{proof}
    \step{2}{$2 \Rightarrow 3$}
    \begin{proof}
        \pf\ If $d(x,y) \leq M$ for all $x,y \in A$ then $\rho(x,y) \leq M / \sqrt{2}$.
    \end{proof}
    \step{3}{$3 \Rightarrow 1$}
    \begin{proof}
        \step{a}{\assume{$A$ is closed and $\rho(x,y) \leq M$ for all $x,y \in A$}}
        \step{b}{\pick\ $a \in A$}
        \begin{proof}
            \pf\ We may assume w.l.o.g. $A$ is nonempty since the empty space is compact.
        \end{proof}
        \step{c}{$A$ is a closed subspace of $[a_1 - M, a_1 + M] \times \cdots \times [a_n - M, a_n + M]$}
        \step{d}{$A$ is compact}
        \begin{proof}
            \pf\ Proposition \ref{proposition:closed_subspace_compact}.
        \end{proof}
    \end{proof}
    \qed
\end{proof}

\begin{corollary}
    The unit sphere $S^{n-1}$ and the closed unit ball $B^n$ are compact for any $n$.
\end{corollary}

\section{The Uniform Topology}

\begin{definition}[Uniform Metric]
    Let $Y$ be a metric space.
    Let $J$ be a set. The \emph{uniform metric} $\overline{\rho}$ on $Y^J$ is defined by
    \[ \overline{\rho}(a,b) = \sup_{j \in J} \overline{d}(a_j, b_j) \]
    where $\overline{d}$ is the standard bounded metric corresponding to the metric on $Y$.

    The \emph{uniform topology} on $Y^J$ is the topology induced by the uniform metric.
\end{definition}

We prove this is a metric.

\begin{proof}
    \pf
    \step{1}{$\overline{\rho}(a,b) \geq 0$}
    \begin{proof}
        \pf\ Immediate from definitions.
    \end{proof}
    \step{2}{$\overline{\rho}(a,b) = 0$ if and only if $a = b$}
    \begin{proof}
        \pf\ Immediate from definitions.
    \end{proof}
    \step{3}{$\overline{\rho}(a,b) = \overline{\rho}(b,a)$}
    \begin{proof}
        \pf\ Immediate from definitions.
    \end{proof}
    \step{4}{$\overline{\rho}(a,c) \leq \overline{\rho}(a,b) + \overline{\rho}(b,c)$}
    \begin{proof}
        \pf
        \begin{align*}
            \overline{\rho}(a,c) & = \sup_{j \in J} \overline{d}(a_j, c_j) \\
            & \leq \sup_{j \in J} (\overline{d}(a_j, b_j) + \overline{d}(b_j, c_j)) \\ 
            & \leq \sup_{j \in J} \overline{d}(a_j, b_j) + \sup_{j \in J} \overline{d}(b_j, c_j) \\
            & = \overline{\rho}(a,b) + \overline{\rho}(b,c)
        \end{align*}
    \end{proof}
    \qed
\end{proof}

\begin{proposition}
    The uniform topology on $Y^J$ is finer than the product topology.
\end{proposition}

\begin{proof}
    \pf
    \step{1}{\pflet{$j \in J$ and $U$ be open in $Y$} \prove{$\inv{\pi_j}(U)$ is open in the uniform topology.}}
    \step{2}{\pflet{$a \in \inv{\pi_j}(U)$}}
    \step{3}{\pick\ $\epsilon > 0$ such that $B(a_j,\epsilon) \subseteq U$}
    \step{4}{$B_{\overline{\rho}}(a, \epsilon) \subseteq \inv{\pi_j}(U)$}
    \qed
\end{proof}

\begin{proposition}
    The uniform topology on $Y^J$ is coarser than the box topology.
\end{proposition}

\begin{proof}
    \pf
    \step{1}{\pflet{$a \in Y^J$ and $\epsilon > 0$} \prove{$B(a, \epsilon)$ is open in the box topology.}}
    \step{2}{\pflet{$b \in B(a, \epsilon)$}}
    \step{3}{For $j \in J$ we have $\overline{d}(a_j,b_j) < \epsilon$}
    \step{4}{For $j \in J$, \pflet{$\delta_j = (\epsilon - \overline{d}(a_j,b_j)) / 2$}}
    \step{5}{$\prod_{j \in J} B(b_j,\delta_j) \subseteq B(a, \epsilon)$}
    \qed
\end{proof}

\begin{proposition}
    The uniform topology on $\RR^J$ is strictly finer than the product topology if and only if $J$ is infinite.
\end{proposition}

\begin{proof}
    \pf
    \step{1}{If $J$ is finite then the uniform and product topologies coincide.}
    \begin{proof}
        \pf\ The uniform, box and product topologies are all the same.
    \end{proof}
    \step{2}{If $J$ is infinite then the uniform and product topologies are different.}
    \begin{proof}
        \pf\ The set $B(\vec{0}, 1)$ is open in the uniform topology but not the product topology.
    \end{proof}
    \qed
\end{proof}

\begin{proposition}[DC]
    The uniform topology on $\RR^J$ is strictly coarser than the box topology if and only if $J$ is infinite.
\end{proposition}

\begin{proof}
    \pf
    \step{1}{If $J$ is finite then the uniform and box topologies coincide.}
    \begin{proof}
        \pf\ The uniform, box and product topologies are all the same.
    \end{proof}
    \step{2}{If $J$ is infinite then the uniform and box topologies are different.}
    \begin{proof}
        \pf\ Pick an $\omega$-sequence $(j_1, j_2, \ldots)$ in $J$. Let $U = \prod_{j \in J} U_j$ where $U_{j_i} = (-1/i, 1/i)$ and $U_j = (-1,1)$ for all other $j$.
        Then $\vec{0} \in U$ but there is no $\epsilon > 0$ such that $B(\vec{0}, \epsilon) \subseteq U$.
    \end{proof}
    \qed
\end{proof}

\begin{proposition}
    The closure of $\RR^\infty$ in $\RR^\omega$ under the uniform topology is $\RR^\omega$.
\end{proposition}

\begin{proof}
    \pf\ Given any open ball $B(a, \epsilon)$, pick an integer $N$ such that $1 / \epsilon < N$. Then $B(a, \epsilon)$ includes sequences whose $n$th entry is 0
    for all $n \geq N$. \qed
\end{proof}

\begin{example}
    The space $\RR^\omega$ is disconnected under the uniform topology. The set of bounded sequences and the set of unbounded sequences form a separation.
\end{example}

\begin{corollary}
    The space $\RR^\omega$ under the uniform topology is not
    path connected.
\end{corollary}

\begin{proposition}
    Give $\RR^\omega$ the uniform topology. Let $x, y \in \RR^\omega$. Then $x$ and $y$ are in the same component if and only if $x-y$ is bounded.
\end{proposition}

\begin{proof}
    \pf
    \step{1}{The component containing 0 is the set of bounded sequences.}
    \begin{proof}
        \step{a}{\pflet{$B$ be the set of bounded sequences.}}
        \step{aa}{$B$ is path-connected.}
        \begin{proof}
            \step{i}{\pflet{$x. y \in B$}}
            \step{o}{\pick\ $b > 0$ such that $|x_j|,|y_j| \leq b$ for all $j$}
            \step{ii}{\pflet{$p : [0,1] \rightarrow B$ be the function $p(t) = (1-t)x + ty$} \prove{$p$ is continuous.}}
            \step{iii}{\pflet{$t \in [0,1]$ and $\epsilon > 0$}}
            \step{iv}{\pflet{$\delta = \epsilon / 2 b$}}
            \step{v}{\pflet{$s \in [0,1]$ with $|s-t| < \delta$}}
            \step{vi}{$\overline{\rho}(p(s),p(t)) < \epsilon$}
            \begin{proof}
                \pf
                \begin{align*}
                    \overline{\rho}(p(s),p(t)) & = \sup_j \overline{d}((1-s)x_j + sy_j, (1-t)x_j + ty_j) \\
                    & \leq |(s-t)x_j + (t-s)y_j| \\
                    & \leq |s-t| |x_j - y_j| \\
                    & < 2b \delta \\
                    & = \epsilon
                \end{align*}
            \end{proof}
        \end{proof}
        \step{a}{$B$ is connected.}
        \begin{proof}
            \pf\ Proposition \ref{proposition:connected_path_connected}.
        \end{proof}
        \step{b}{If $C$ is connected and $B \subseteq C$ then $B = C$.}
        \begin{proof}
            \pf\ Otherwise $B \cap C$ and $C \setminus B$ form a separation of $C$.
        \end{proof}
    \end{proof}
    \qedstep
    \begin{proof}
        \pf\ Since $\lambda x. x-y$ is a Homeomorphism of $\RR^\omega$ with itself.
    \end{proof}
    \qed
\end{proof}

\begin{example}
    The space $[0,1]^\omega$ under the uniform topology is not locally compact.

    It is not compact because the set $\{ 0,1 \}^\omega$ has no limit point.

    Now, assume for a contradiction $[0,1]^\omega$ is locally compact. Pick $\epsilon > 0$
    such that $B(0,\epsilon)$ is included in a compact subspace. Then $\overline{B(0,\epsilon)}$ is compact.
    But $\overline{B(0,\epsilon)} = [0,1]^\omega$ if $\epsilon \geq 1$, or $[0,\epsilon]^\omega$
    if $\epsilon < 1$. In either case $\overline{B(0,\epsilon)} \cong [0,1]^\epsilon$ which is not
    compact.
\end{example}

\begin{corollary}
    The space $\RR^\omega$ under the uniform topology is not locally compact.
\end{corollary}

\begin{example}
    The space $\RR^\omega$ under the uniform topology is not second countable.

    \begin{proof}
        \pf\ The set $\{0,1\}^\omega$ is an uncountable discrete subspace. \qed
    \end{proof}
\end{example}

\begin{corollary}
    The space $\RR^\omega$ under the box topology is not Lindel\"{o}f.
\end{corollary}

\begin{corollary}
    The space $\RR^\omega$ under the box topology is not separable.
\end{corollary}

\begin{proposition}[Choice]
    \label{proposition:Lindelof_metrizable_second_countable}
    Every Lindel\"{o}f metrizable space is second countable.
\end{proposition}

\begin{proof}
    \pf
    \step{1}{\pflet{$X$ be a Lindel\"{o}f metrizable space.}}
    \step{2}{For $n \in \ZZ^+$, \pick\ a countable set $\AA_n$ of open balls of
    radius $1/n$ that covers $X$.}
    \step{3}{\pflet{$\BB = \bigcup_{n=1}^\infty \AA_n$} \prove{$\BB$ is a basis for $X$.}}
    \step{4}{\pflet{$x \in X$}}
    \step{5}{\pflet{$U$ be a neighbourhood of $x$.}}
    \step{6}{\pick\ $n$ such that $B(x,1/n) \subseteq U$}
    \step{7}{\pick\ $B \in \AA_{2n}$ such that $x \in B$}
    \step{8}{$B \subseteq U$}
    \begin{proof}
        \pf\ Since $\diam B \leq n$.
    \end{proof}
    \qed
\end{proof}

\begin{corollary}
    A compact Hausdorff space is metrizable if and only if it is second countable.
\end{corollary}

\begin{proof}
    \pf\ By this Proposition and the Urysohn Metrization Theorem. \qed
\end{proof}

\begin{example}
    The space $\RR_l$ is not metrizable, because it is Lindel\"{o}f but not
    second countable.
\end{example}

\begin{corollary}
    The Sorgenfrey plane is not metrizable, because it has a subspace homeomorphic to $\RR_l$.
\end{corollary}

\begin{example}
    The ordered square is not metrizable, because it is compact but not separable.
\end{example}

\begin{proposition}
    For any set $J$, the space $\RR^J$ under the box topology is completely regular.
\end{proposition}

\begin{proof}
    \pf
    \step{1}{\pflet{$J$ be any set.}}
    \step{2}{For any closed set $A \subseteq \RR^J$ disjoint from $(-1,1)^J$,
    $A$ and $\{0\}$ can be separated by a continuous function.}
    \begin{proof}
        \step{a}{\pflet{$A$ be a closed set disjoint from $(-1,1)^J$}}
        \step{b}{\pick\ a function $f : \RR^J \rightarrow [0,1]$
        continuous with respect to the uniform topology that
        separates $\{0\}$ from $A$.}
        \step{c}{$f$ is continuous with respect to the box topology.}
    \end{proof}
    \step{3}{\pflet{$A$ be any closed set.}}
    \step{4}{\pflet{$a$ be any point not in $A$.}}
    \step{5}{\pick\ a basic open neighbourhood $U$ of $a$ disjoint from $A$.}
    \step{6}{\pick\ a homemorphism $\phi : \RR^J \cong \RR^J$ that maps $a$ to $0$ and
    $U$ to $(-1,1)^J$}
    \step{7}{\pick\ a continuous function $f : \RR^J \rightarrow [0,1]$
    such that $f(0) = 0$ and $f(\phi(A)) = \{1\}$}
    \step{8}{\pflet{$g = f \circ \phi$}}
    \step{9}{$g(a) = 0$ and $g(A) = \{1\}$}
\end{proof}

\begin{proposition}
    The space $\RR^\omega$ under the uniform topology is locally path connected.
\end{proposition}

\begin{proof}
    \pf
    \step{1}{\pflet{$a \in \RR^\omega$ and $0 < \epsilon < 1/2$} \prove{$B(a, \epsilon)$ is path connected.}}
    \step{2}{\pflet{$b, c \in B(a, \epsilon)$}}
    \step{3}{Define $p : I \rightarrow \RR^\omega$ by $p(t) = a(1-t)+bt$}
    \step{4}{$p$ is continuous.}
    \begin{proof}
        \step{a}{\pflet{$\delta > 0$}}
        \step{b}{\pflet{$t \in I$}}
        \step{c}{\pflet{$\gamma = \delta / 3 \epsilon$}}
        \step{d}{\pflet{$s \in I$ with $|s-t|<\gamma$}}
        \step{e}{$\overline{\rho}(p(s),p(t)) < \delta$}
        \begin{proof}
            \pf
            \begin{align*}
                \overline{\rho}(p(s),p(t))
                & = \sup_n \overline{d}(a_n(1-s)+b_ns,a_n(1-t)+b_nt) \\
                & \leq \sup_n |a_n(t-s) + b_n(s-t)| \\
                & = \sup_n |a_n - b_n||t-s| \\
                & \leq 2 \epsilon \gamma \\
                & < \delta
            \end{align*}
        \end{proof}
    \end{proof}
    \qed
\end{proof}

\begin{corollary}
    The space $\RR^\omega$ under the uniform topology is locally connected.
\end{corollary}

\begin{example}
    The space $[0,1]^\omega$ under the uniform topology is not limit point compact.

    The infinite set $\{0,1\}^\omega$ has no limit point.
\end{example}

\begin{corollary}
    The space $\RR^\omega$ under the uniform topology is not limit point compact.
\end{corollary}

\begin{corollary}
    The space $\RR^\omega$ under the uniform topology is not compact.
\end{corollary}

\begin{theorem}
    Let $X$ be a topological space and $Y$ a metric space. The set $\mathcal{B}(X,Y)$ of bounded functions from $X$ to $Y$
    is closed in $Y^X$ under the uniform topology.
\end{theorem}

\begin{proof}
    \pf
    \step{1}{\pflet{$(f_n)$ be a sequence of bounded functions.}}
    \step{2}{\pflet{$f_n \rightarrow f$ as $n \rightarrow \infty$} \prove{$f$ is bounded.}}
    \step{3}{\pick\ $N$ such that $\forall n \geq N. \overline{\rho}(f_n,f) < 1/2$}
    \step{4}{$\forall x \in X. d(f_N(x),f(x)) < 1/2$}
    \step{5}{$\diam f(X) \leq \diam f_N(X) + 1$}
    \qed
\end{proof}

\section{Uniform Convergence}

\begin{definition}[Uniform Convergence]
    Let $X$ be a set and $Y$ a metric space. Let $(f_n : X \rightarrow Y)$ be a sequence of functions and $f : X \rightarrow Y$ be a function.
    Then $f_n$ \emph{converges uniformly} to $f$ as $n \rightarrow \infty$ if and only if, for all $\epsilon > 0$, there exists $N$ such that, for all $n \geq N$ and $x \in X$,
    we have $d(f_n(x),f(x)) < \epsilon$.
\end{definition}

\begin{example}
    Define $f_n : [0,1] \rightarrow \RR$ by $f_n(x) = x^n$ for $n \geq 1$, and $f : [0,1] \rightarrow \RR$ by $f(x) = 0$ if $x < 1$, $f(1) = 1$.  Then $f_n$ converges to $f$
    pointwise but not uniformly.
\end{example}

\begin{theorem}[Uniform Limit Theorem]
    Let $X$ be a topological space and $Y$ a metric space.  Let $(f_n : X \rightarrow Y)$ be a sequence of continuous functions and $f : X \rightarrow Y$ be a function.
    If $f_n$ converges uniformly to $f$ as $n \rightarrow \infty$, then $f$ is continuous.
\end{theorem}

\begin{proof}
    \pf
    \step{1}{\pflet{$x \in X$ and $\epsilon > 0$}}
    \step{2}{\pick\ $N$ such that, for all $n \geq N$ and $y \in X$, we have $d(f_n(y), f(y)) < \epsilon / 3$}
    \step{3}{\pick\ a neighbourhood $U$ of $x$ such that $f_N(U) \subseteq B(f_N(x), \epsilon / 3)$ \prove{$f(U) \subseteq B(f(x), \epsilon)$}}
    \step{4}{\pflet{$y \in U$}}
    \step{5}{$d(f(y),f(x)) < \epsilon$}
    \begin{proof}
        \pf
        \begin{align*}
            d(f(y),f(x)) & \leq d(f(y),f_N(y)) + d(f_N(y),f_N(x)) + d(f_N(x),f(x)) & (\text{Triangle Inequality}) \\
            & < \epsilon / 3 + \epsilon / 3 + \epsilon / 3 & (\text{\stepref{2}, \stepref{3}})\\
            & = \epsilon
        \end{align*}
    \end{proof}
    \qed
\end{proof}

\begin{proposition}
    Let $X$ be a topological space and $Y$ a metric space.  Let $(f_n : X \rightarrow Y)$ be a sequence of continuous functions and $f : X \rightarrow Y$ be a function.
    Let $(a_n)$ be a sequence of points in $X$ and $a \in X$. If $f_n$ converges uniformly to $f$ and $a_n$ converges to $a$ in $X$ then $f_n(a_n)$ converges to $f(a)$
    uniformly in $Y$.
\end{proposition}

\begin{proof}
    \pf
    \step{1}{\pflet{$\epsilon > 0$}}
    \step{2}{\pick\ $N_1$ such that, for all $n \geq N_1$ and $x \in X$, we have $d(f_n(x),f(x)) < \epsilon / 2$}
    \step{3}{\pick\ $N_2$ such that, for all $n \geq N_2$, we have $a_n \in \inv{f}(B(a,\epsilon/2))$}
    \begin{proof}
        \pf\ Using the fact that $f$ is continuous from the Uniform Limit Theorem.
    \end{proof}
    \step{4}{\pflet{$N = \max(N_1,N_2)$}}
    \step{5}{\pflet{$n \geq N$}}
    \step{6}{$d(f_n(a_n),f(a)) < \epsilon$}
    \begin{proof}
        \pf
        \begin{align*}
            d(f_n(a_n),f(a)) & \leq d(f_n(a_n),f(a_n)) + d(f(a_n),f(a)) & (\text{Triangle Inequality}) \\
            & < \epsilon / 2 + \epsilon / 2 & (\text{\stepref{2}, \stepref{3}}) \\
            & = \epsilon
        \end{align*}
    \end{proof}
    \qed
\end{proof}

\begin{proposition}
    Let $X$ be a set. Let $(f_n : X \rightarrow \RR)$ be a sequence of functions and $f : X \rightarrow \RR$ be a function.
    Then $f_n$ converges unifomly to $f$ as $n \rightarrow \infty$ if and only if $f_n \rightarrow f$ as $n \rightarrow \infty$
    in $\RR^X$ under the uniform topology.
\end{proposition}

\begin{proof}
    \pf
    \step{1}{If $f_n$ converges uniformly to $f$ then $f_n$ converges to $f$ under the uniform topology.}
    \begin{proof}
        \step{a}{\assume{$f_n$ converges uniformly to $f$}}
        \step{b}{\pflet{$\epsilon > 0$}}
        \step{c}{\pick\ $N$ such that, for all $n \geq N$ and $x \in X$, we have $d(f_n(x),f(x)) < \epsilon / 2$}
        \step{d}{For all $n \geq N$ we have $\overline{\rho}(f_n,f) \leq \epsilon / 2$}
        \step{e}{For all $n \geq N$ we have $\overline{\rho}(f_n,f) < \epsilon$}
    \end{proof}
    \step{2}{If $f_n$ converges to $f$ under the uniform topology then $f_n$ converges uniformly to $f$.}
    \begin{proof}
        \step{a}{\assume{$f_n$ converges to $f$ under the uniform topology.}}
        \step{b}{\pflet{$\epsilon > 0$}}
        \step{c}{\pick\ $N$ such that, for all $n \geq N$, we have $\overline{\rho}(f_n,f) < \min(\epsilon,1/2)$}
        \step{d}{\pflet{$n \geq N$}}
        \step{e}{\pflet{$x \in X$}}
        \step{f}{$\overline{\rho}(f_n,f) < \min(\epsilon,1/2)$}
        \begin{proof}
            \pf\ From \stepref{c}.
        \end{proof}
        \step{f}{$d(f_n(x),f(x)) < \min(\epsilon,1/2)$}
        \step{g}{$d(f_n(x),f(x)) < \epsilon$}
    \end{proof}
    \qed
\end{proof}

\begin{corollary}
    Let $X$ be a topological space and $Y$ a metric space. The set $\mathcal{C}(X,Y)$ of all continuous functions from $X$ to $Y$
    is closed in $Y^X$ under the uniform topology.
\end{corollary}

\begin{proof}
    \pf\ By the Uniform Limit Theorem. \qed
\end{proof}

\begin{proposition}
    \label{proposition:evaluation_continuous}
    Let $X$ be a topological space and $Y$ a metric space. Give $\mathcal{C}(X,Y)$ the uniform topology. Then the evaluation map
    $e : X \times \mathcal{C}(X,Y) \rightarrow Y$ is continuous.
\end{proposition}

\begin{proof}
    \pf
    \step{1}{\pflet{$x \in X$}}
    \step{2}{\pflet{$f \in \mathcal{C}(X,Y)$}}
    \step{3}{\pflet{$\epsilon > 0$}}
    \step{4}{\pick\ an open neighbourhood $U$ of $x$ such that $\forall y \in U. d(f(x),f(y)) < \epsilon / 2$}
    \step{5}{\pflet{$y \in U$}}
    \step{6}{\pflet{$g \in \mathcal{C}(X,Y)$}}
    \step{7}{\assume{$\overline{\rho}(f,g) < \min(\epsilon / 2, 1/2)$}}
    \step{8}{$d(f(x),g(y)) < \epsilon$}
    \begin{proof}
        \pf
        \begin{align*}
            d(f(x),g(y)) & \leq d(f(x),f(y)) + d(f(y),g(y)) \\
            & < \epsilon
        \end{align*}
    \end{proof}
    \qed
\end{proof}

\section{Isometric Imbeddings}

\begin{definition}
    Let $X$ and $Y$ be metric spaces. An \emph{isometric imbedding} $f : X \rightarrow Y$ is a function such that, for all $x, y \in X$, we have $d(f(x),f(y)) = d(x,y)$.
\end{definition}

\begin{proposition}
    \label{proposition:isometric_imbedding}
    Every isometric imbedding is an imbedding.
\end{proposition}

\begin{proof}
    \pf
    \step{1}{\pflet{$f : X \rightarrow Y$ be an isometric imbedding.}}
    \step{2}{$f$ is injective.}
    \begin{proof}
        \pf\ If $f(x) = f(y)$ then $d(f(x),f(y)) = 0$ hence $d(x,y) = 0$ hence $x = y$.
    \end{proof}
    \step{3}{$f$ is continuous.}
    \begin{proof}
        \pf\ For all $\epsilon > 0$, if $d(x,y) < \epsilon$ then $d(f(x),f(y)) < \epsilon$.
    \end{proof}
    \step{4}{$f : X \rightarrow f(X)$ is an open map.}
    \begin{proof}
        \pf\ $f(B(a,\epsilon)) = B(f(a),\epsilon) \cap f(X)$.
    \end{proof}
    \qed
\end{proof}

\section{Distance to a Set}

\begin{definition}
    Let $X$ be a metric space, $x \in X$ and $A \subseteq X$ be nonempty. The \emph{distance} from $x$ to $A$ is defined as
    \[ d(x,A) = \inf_{a \in A} d(x,a) \enspace . \]
\end{definition}

\begin{proposition}
    \label{proposition:distance_continuous}
    Let $X$ be a metric space and $A \subseteq X$ be nonempty. Then the function
    $d(-,A) : X \rightarrow \RR$ is continuous.
\end{proposition}

\begin{proof}
    \pf
    \step{1}{\pflet{$X$ be a metric space.}}
    \step{2}{\pflet{$A \subseteq X$ be nonempty.}}
    \step{3}{\pflet{$x \in X$ and $\epsilon > 0$}}
    \step{4}{\pflet{$\delta = \epsilon$}}
    \step{5}{\pflet{$y \in B(x, \delta)$}}
    \step{6}{$|d(x,A) - d(y,A)| < \epsilon$}
    \begin{proof}
        \step{a}{$d(x,A) - d(y,A) < \epsilon$}
        \begin{proof}
            \pf
            \step{i}{For all $a \in A$ we have $d(x,A) \leq d(x,y) + d(y,a)$}
            \begin{proof}
                \pf
                \begin{align*}
                    d(x,A) & \leq d(x,a) & (\text{definition of $d(x,A)$})\\
                    & \leq d(x,y) + d(y,a) & (\text{Triangle Inequality})
                \end{align*}
            \end{proof}
            \step{ii}{$d(x,A) - d(x,y) \leq d(y,A)$}
        \end{proof}
        \step{b}{$d(y,A) - d(x,A) < \epsilon$}
        \begin{proof}
            \pf\ Similar.
        \end{proof}
    \end{proof}
    \qedstep
    \begin{proof}
        \pf\ Theorem \ref{theorem:continuous_metric}.
    \end{proof}
    \qed
\end{proof}

\begin{theorem}
    Let $X$ be a metric space, $A \subseteq X$ be nonempty, and $x \in X$.
    Then $d(x, A) = 0$ if and only if $x \in \overline{A}$.
\end{theorem}

\begin{proof}
    \pf
    \step{1}{\pflet{$X$ be a metric space.}}
    \step{2}{\pflet{$A \subseteq X$ be nonempty.}}
    \step{3}{\pflet{$x \in X$}}
    \step{4}{If $d(x,A) = 0$ then $x \in \overline{A}$}
    \begin{proof}
        \step{a}{\assume{$d(x,A) = 0$}}
        \step{b}{\pflet{$U$ be any neighbourhood of $x$.}}
        \step{c}{\pick\ $\epsilon > 0$ such that $B(x,\epsilon) \subseteq U$}
        \begin{proof}
            \pf\ Proposition \ref{proposition:open_in_metric_space}, \stepref{1},
            \stepref{b}.
        \end{proof}
        \step{d}{\pick\ $a \in A$ such that $d(x,a) < \epsilon$}
        \begin{proof}
            \pf\ From \stepref{a}, \stepref{c}.
        \end{proof}
        \step{e}{$a \in A \cap U$}
        \begin{proof}
            \pf\ From \stepref{c}, \stepref{d}.
        \end{proof}
        \qedstep
        \begin{proof}
            \pf\ Theorem \ref{theorem:closure_neighbourhood}.
        \end{proof}
    \end{proof}
    \step{5}{If $x \in \overline{A}$ then $d(x,A) = 0$}
    \qed
\end{proof}

\begin{theorem}
    Let $X$ be a metric space. Let $A \subseteq X$ be nonempty and compact.
    Let $x \in X$. Then there exists $a \in A$ such that $d(x,A) = d(x,a)$.
\end{theorem}

\begin{proof}
    \pf\ By the Extreme Value Theorem, the function $d(x,-) : A \rightarrow \RR$ attains its minimum. \qed
\end{proof}

\section{Lebesgue Numbers}

\begin{definition}[Lebesgue Number]
    Let $X$ be a metric space. Let $\UU$ be an open covering of $X$. A \emph{Lebesgue number}
    for $\UU$ is a real number $\delta > 0$ such that, for every subset $A \subseteq X$ with diameter
    diameter $< \delta$, there exists $U \in \UU$ such that $A \subseteq U$.
\end{definition}

\begin{theorem}[Lebesgue Number Lemma]
    Every open covering of a compact metric space has a Lebesgue number.
\end{theorem}

\begin{proof}
    \pf
    \step{1}{\pflet{$X$ be a compact metric space.}}
    \step{2}{\pflet{$\UU$ be an open covering of $X$.}}
    \step{4}{\pick\ a finite subset $\{ U_1, \ldots, U_n \}$ of $\UU$ that covers $X$.}
    \step{5}{For $i = 1, \ldots, n$, \pflet{$C_i = X - U_i$}}
    \step{6}{\pflet{$f : X \rightarrow \RR$, $$ f(x) = 1/n \sum_{i=1}^n d(x, C_i)$$}}
    \step{7}{For all $x \in X$ we have $f(x) > 0$}
    \begin{proof}
        \step{a}{\pflet{$x \in X$}}
        \step{b}{\pick\ $i$ such that $x \in U_i$}
        \begin{proof}
            \pf\ From \stepref{4}.
        \end{proof}
        \step{c}{\pick\ $\epsilon > 0$ such that $B(x, \epsilon) \subseteq U_i$}
        \begin{proof}
            \pf\ Proposition \ref{proposition:open_in_metric_space}.
        \end{proof}
        \step{d}{$d(x, C_i) \geq \epsilon$}
        \step{e}{$f(x) \geq \epsilon / n$}
    \end{proof}
    \step{8}{$f$ is continuous.}
    \begin{proof}
        \pf\ Proposition \ref{proposition:distance_continuous}.
    \end{proof}
    \step{9}{\pflet{$\delta$ be the minimum value of $f(X)$}}
    \begin{proof}
        \pf\ By the Extreme Value Theorem
    \end{proof}
    \step{10}{$\delta > 0$}
    \begin{proof}
        \pf\ From \stepref{7}
    \end{proof}
    \step{11}{For every subset $A \subseteq X$ with diameter $< \delta$, there exists
    $U \in \UU$ such that $A \subseteq U$}
    \begin{proof}
        \step{a}{\pflet{$A \subseteq X$ with $\diam A < \delta$}}
        \step{b}{\pick\ $x_0 \in A$}
        \step{c}{$A \subseteq B(x_0, \delta)$}
        \step{d}{$f(x_0) \geq \delta$}
        \step{e}{\pick\ $m$ such that $d(x_0,C_m)$ is the largest out of
        $d(x_0, C_1)$, \ldots, $d(x_0, C_n)$}
        \step{f}{$d(x_0, C_m) \geq f(x_0)$}
        \step{g}{$B(x_0, \delta) \subseteq U_m$}
        \step{h}{$A \subseteq U_m$}
    \end{proof}
    \step{12}{$\delta$ is a Lebesgue number for $\UU$}
    \qed
\end{proof}

\begin{theorem}[AC]
    Every sequentially compact metric space is compact.
\end{theorem}

\begin{proof}
    \pf
    \step{1}{\pflet{$X$ be a sequentially comapct metric space.}}
    \step{2}{Every open covering of $X$ has a Lebesgue number.}
    \begin{proof}
        \step{a}{\pflet{$\AA$ be an open covering of $X$.}}
        \step{b}{\assume{for a contradiction $\AA$ has no Lebesgue number.}}
        \step{c}{For $n \geq 1$, \pick\ a set $C_n$ with diameter $< 1 / n$
        that is not included in any member of $\AA$.}
        \step{d}{For $n \geq 1$, \pick\ $x_n \in C_n$.}
        \step{e}{\pick\ a convergent subsequence $(C_{n_r})$ of $(C_n)$
        with limit $a$.}
        \step{f}{\pick\ $A \in \AA$ such that $a \in A$}
        \step{g}{\pick\ $\epsilon > 0$ such that $B(a, \epsilon) \subseteq A$.}
        \step{h}{\pick\ $r$ such that $1/n_r < \epsilon / 2$ and $d(x_{n_r},a) < \epsilon / 2$}
        \step{i}{$C_{n_r} \subseteq B(a, \epsilon)$}
        \step{j}{$C_{n_r} \subseteq A$}
        \qedstep
        \begin{proof}
            \pf\ This contradicts \stepref{c}.
        \end{proof}
    \end{proof}
    \step{3}{For every $\epsilon > 0$, there exists a finite covering of $X$
    by $\epsilon$-balls.}
    \begin{proof}
        \step{a}{\assume{for a contradiction that there exists $\epsilon > 0$ such that
        $X$ cannot be finitely covered by $\epsilon$-balls.}}
        \step{b}{\pick\ a sequence of points $(x_n)$ such that $x_n \in X - (B(x_1, \epsilon) \cup \cdots \cup B(x_{n-1},\epsilon))$}
        \step{c}{$d(x_m,x_n) \geq \epsilon$ for all $m,n$ distinct}
        \step{d}{$(x_n)$ has no convergent subsequence}
        \qedstep
        \begin{proof}
            \pf\ This contradicts \stepref{1}.
        \end{proof}
    \end{proof}
    \step{4}{\pflet{$\AA$ be an open covering of $X$.}}
    \step{5}{\pick\ a Lebesgue number $\delta$ for $\AA$.}
    \begin{proof}
        \pf\ By \stepref{2}.
    \end{proof}
    \step{6}{\pflet{$\epsilon = \delta / 3$}}
    \step{7}{\pick\ a finite covering $\{ B_1, \ldots, B_n \}$ of $X$ be $\epsilon$-balls.}
    \begin{proof}
        \pf\ By \stepref{3}.
    \end{proof}
    \step{8}{For $i = 1, \ldots, n$, \pick\ $U_i \in \AA$ such that $B_i \subseteq A_i$}
    \begin{proof}
        \pf\ By \stepref{5} since $\diam B_i = 2 \epsilon < \delta$.
    \end{proof}
    \step{9}{$\{ U_1, \ldots, U_n \}$ covers $X$.}
    \qed
\end{proof}

\begin{corollary}
    The space $\RR^\omega$ is not sequentially compact.
\end{corollary}

\begin{corollary}
    The space $\RR^\omega$ is not limit point compact.
\end{corollary}

\begin{example}
    The space $S_\Omega$ is not metrizable, because it is sequentially compact but not compact.
\end{example}

\section{Uniform Continuity}

\begin{definition}[Uniformly Continuous]
    Let $X$ and $Y$ be metric spaces. Let $f : X \rightarrow Y$.
    Then $f$ is \emph{uniformly continuous} if and only if,
    for every $\epsilon > 0$, there exists $\delta > 0$ such that,
    for all $x, y \in X$, if $d(x,y) < \delta$ then
    $d(f(x),f(y)) < \epsilon$.
\end{definition}

\begin{theorem}[Uniform Continuity Theorem]
    Every continuous function from a compact metric space to a
    metric space is uniformly continuous.
\end{theorem}

\begin{proof}
    \pf
    \step{1}{\pflet{$X$ be a compact metric space.}}
    \step{2}{\pflet{$Y$ be a metric space.}}
    \step{3}{\pflet{$f : X \rightarrow Y$ be a continuous function.}}
    \step{4}{\pflet{$\epsilon > 0$}}
    \step{5}{\pflet{$\UU = \{ \inv{f}(B(y, \epsilon / 2)) \mid y \in Y \}$}}
    \step{6}{\pick\ a Lebesgue number $\delta > 0$ for $\UU$.}
    \begin{proof}
        \pf\ By the Lebesgue Number Lemma.
    \end{proof}
    \step{7}{\pflet{$x, x' \in X$}}
    \step{8}{\assume{$d(x,x') < \delta$}}
    \step{9}{\pick\ $y \in Y$ such that $\{ x, x' \} \subseteq \inv{f}(B(y, \epsilon / 2))$}
    \begin{proof}
        \pf\ Since $\diam \{x, x'\} < \delta$.
    \end{proof}
    \step{10}{$d(f(x),f(x')) < \epsilon$}
    \begin{proof}
        \pf
        \begin{align*}
            d(f(x),f(x')) & \leq d(f(x),y) + d(y,f(x')) & (\text{Triangle Inequality}) \\
            & < \epsilon / 2 + \epsilon / 2 & (\text{\stepref{9}}) \\
            & = \epsilon
        \end{align*}
    \end{proof}
    \qed
\end{proof}

\begin{example}
    The space $\mathcal{C}(I, \RR)$ of continuous functions $I \rightarrow \RR$,
    as a subspace of $\RR^I$ under the uniform topology, is second countable.

    \begin{proof}
        \pf
        \step{1}{\pflet{$D$ be the set of continuous functions whose graphs consist
        of finitely many line segments with rational end points.} \prove{$D$ is dense.}}
        \step{2}{\pflet{$f \in \mathcal{C}(I, \RR)$ and $\epsilon > 0$}
        \prove{$B(f, \epsilon)$ intersects $D$.}}
        \step{3}{$f$ is uniformly continuous.}
        \begin{proof}
            \pf\ Uniform Continuity Theorem.
        \end{proof}
        \step{4}{\pick\ $\delta > 0$ such that, for all $x, y \in I$, 
        if $|x-y| < \delta$ then $|f(x)-f(y)| < \epsilon / 4$}
        \step{4a}{\pick\ $N \in \ZZ^+$ such that $1/N < \delta$}
        \step{5}{For $0 \leq k \leq N$, \pick\ a rational number $q_k$
        such that $|q_k - f(k/N)| < \epsilon / 6$}
        \step{6}{\pflet{$q \in D$ be the function whose graph consists of the line
        segments with endpoints $(0,q_0)$, $(1/N,q_1)$, \ldots, $(1,q_N)$}
        \prove{$q \in B(f, \epsilon)$}}
        \step{7}{For $0 \leq k < N$, we have $|q_{k+1} - q_k| < 7 \epsilon / 12$}
        \begin{proof}
            \pf
            \begin{align*}
                |q_{k+1} - q_k| & \leq |q_{k+1} - f((k+1)/N)| + |f((k+1)/N) - f(k/N)|
                + |f(k/N)-q_k| \\
                & < \epsilon / 6 + \epsilon / 4 + \epsilon / 6 \\
                & = 7 \epsilon / 12
            \end{align*}
        \end{proof}
        \step{8}{\pflet{$x \in I$}}
        \step{9}{\pick\ $k$ such that $k/N \leq x \leq (k+1)/N$}
        \step{10}{$|f(x) - q(x)| < \epsilon$}
        \begin{proof}
            \pf
            \begin{align*}
                |f(x) - q(x)| & \leq |f(x) - f(k/N)| + |f(k/N) - q_k| + |q_k - q(x)| \\
                & \leq |f(x) - f(k/N)| + |f(k/N) - q_k| + |q_k - q_{k+1}| \\
                & < \epsilon / 4 + \epsilon / 6 + 7 \epsilon / 12 \\
                & = \epsilon
            \end{align*}
        \end{proof}
        \qed
    \end{proof}
\end{example}

\section{Epsilon-neighbourhoods}

\begin{definition}[$\epsilon$-neighbourhood]
    Let $X$ be a metric space. Let $A \subseteq X$ be nonempty. Let $\epsilon > 0$.
    Then the \emph{$\epsilon$-neighbourhood} of $A$, $U(A,\epsilon)$, is the set
    \[ U(A, \epsilon) = \{ x \in X \mid d(x,A) < \epsilon \} \enspace . \]
\end{definition}

\begin{proposition}
    \label{proposition:epsilon_neighbourhood}
    Let $X$ be a metric space. Let $A \subseteq X$ be nonempty. Let $\epsilon > 0$.
    Then $U(A, \epsilon) = \bigcup_{a \in A} B(a, \epsilon)$.    
\end{proposition}

\begin{proof}
    \pf
    \step{1}{\pflet{$X$ be a metric space.}}
    \step{2}{\pflet{$A \subseteq X$ be nonempty.}}
    \step{3}{\pflet{$\epsilon > 0$}}
    \step{4}{$U(A, \epsilon) \subseteq \bigcup_{a \in A} B(a, \epsilon)$}
    \begin{proof}
        \step{a}{\pflet{$x \in U(A, \epsilon)$}}
        \step{b}{$d(x,A) < \epsilon$}
        \step{c}{$\epsilon$ is not a lower bound for $\{ d(x,a) \mid a \in A \}$}
        \step{d}{\pick\ $a \in A$ such that $d(x,a) < \epsilon$}
        \step{e}{$x \in B(a,\epsilon)$}
    \end{proof}
    \step{5}{$\bigcup_{a \in A} B(a, \epsilon) \subseteq U(A, \epsilon)$}
    \begin{proof}
        \step{a}{\pflet{$a \in A$ and $x \in B(a, \epsilon)$}}
        \step{b}{$d(x,A) \leq d(x,a)$}
        \step{c}{$d(x,A) < \epsilon$}
        \step{d}{$x \in U(A, \epsilon)$}
    \end{proof}
    \qed
\end{proof}

\begin{proposition}
    Let $X$ be a metric space. Let $A \subseteq X$ be nonempty and compact.
    Let $U$ be an open set such that $A \subseteq U$. Then there exists $\epsilon > 0$
    such that $U(A,\epsilon) \subseteq U$.
\end{proposition}

\begin{proof}
    \pf
    \step{1}{\pflet{$X$ be a metric space.}}
    \step{2}{\pflet{$A \subseteq X$ be nonempty and compact.}}
    \step{3}{\pflet{$U$ be an open set such that $A \subseteq U$}}
    \step{4}{$\{ B(a, \epsilon) \mid a \in A, \epsilon > 0, B(a, 2\epsilon) \subseteq U \}$ covers $A$.}
    \begin{proof}
        \pf\ By Proposition \ref{proposition:open_in_metric_space}.
    \end{proof}
    \step{5}{\pick\ a finite subcover $\{ B(a_1, \epsilon_1), \ldots, B(a_n, \epsilon_n) \}$}
    \begin{proof}
        \pf\ Since $A$ is compact (\stepref{2}).
    \end{proof}
    \step{6}{\pflet{$\epsilon = \min(\epsilon_1, \ldots, \epsilon_n)$} \prove{$U(A, \epsilon) \subseteq U$}}
    \step{7}{\pflet{$x \in U(A, \epsilon)$}}
    \step{8}{\pick\ $a \in A$ such that $d(x,a) < \epsilon$}
    \begin{proof}
        \pf\ Proposition \ref{proposition:epsilon_neighbourhood}.
    \end{proof}
    \step{9}{\pick\ $i$ such that $a \in B(a_i, \epsilon_i)$}
    \begin{proof}
        \pf\ By \stepref{5}.
    \end{proof}
    \step{10}{$d(x,a_i) < 2 \epsilon$}
    \begin{proof}
        \pf\ By the Triangle Inequality.
    \end{proof}
    \step{11}{$x \in U$}
    \begin{proof}
        \pf\ From \stepref{4}.
    \end{proof}
    \qed
\end{proof}

This example shows that we cannot weaken the hypothesis that $A$ is compact
to $A$ being closed:

\begin{example}
    Let $X = \RR^2$. Let $A = \{ (x, 1/x) \mid x > 0 \}$.
    Let $U = \{ (x,y) \mid x > 0, y > 0 \}$. Then $A$ is nonempty
    and closed (Proposition \ref{proposition:graph_closed}).
    The set $U$ is open and $A \subseteq U$. But there is no $\epsilon > 0$
    such that $U(A, \epsilon) \subseteq U$.

    \begin{proof}
        \pf
        \step{1}{\pflet{$\epsilon > 0$}}
        \step{2}{$(2/\epsilon, \epsilon / 2) \in A$}
        \step{3}{$(2/\epsilon, 0) \in U(A, \epsilon)$}
        \step{4}{$(2/\epsilon, 0) \notin U$}
        \qed
    \end{proof}
\end{example}

\begin{proposition}
    Every metrizable space is perfectly normal.
\end{proposition}

\begin{proof}
    \pf
    \step{1}{\pflet{$X$ be a metric space.}}
    \step{2}{$X$ is normal.}
    \begin{proof}
        \pf\ Proposition \ref{proposition:metrizable_normal}.
    \end{proof}
    \step{3}{Every closed set is $G_\delta$.}
    \begin{proof}
        \pf\ For any closed set $A$ we have $A = \bigcap_{n \in \ZZ^+}
        B(A,1/n)$.
    \end{proof}
    \qed
\end{proof}

\begin{corollary}
    The space $\RR_K$ is not metrizable.
\end{corollary}

\begin{proposition}[Choice]
    \label{proposition:countably_locally_discrete_refinement}
    Let $X$ be a metrizable space. Let $\AA$ be an open covering of $X$.
    Then there exists a countably locally discrete open covering of $X$ that refines $\AA$.
\end{proposition}

\begin{proof}
    \pf
    \step{1}{\pick\ a well-ordering $<$ on $\AA$.}
    \step{2}{\pick\ a metric $d$ on $X$ that induces its topology.}
    \step{3}{For $n \in \ZZ^+$ and $U \in \AA$, \pflet{$S_n(U) = \{ x \in X \mid B(x, 1/n) \subseteq U \}$}}
    \step{3}{For $n \in \ZZ^+$ and $U \in \AA$, \pflet{$T_n(U) = S_n(U) - \bigcup_{V < U} V$}}
    \step{4}{For any distinct elements $V, W \in \AA$ and $n \in \ZZ^+$, if $x \in T_n(V)$ and $y \in T_n(W)$ then $d(x,y) \geq 1/n$}
    \begin{proof}
        \step{a}{\assume{w.l.o.g.  $V < W$}}
        \step{b}{$B(x, 1/n) \subseteq V$}
        \step{c}{$y \notin V$}
    \end{proof}
    \step{5}{For $n \in \ZZ^+$ and $U \in \AA$, \pflet{$E_n(U) = B(T_n(U),\epsilon / 3)$}}
    \step{4}{For any distinct elements $V, W \in \AA$ and $n \in \ZZ^+$, if $x \in E_n(V)$ and $y \in E_n(W)$ then $d(x,y) \geq 1/3n$}
    \step{6}{For $n \in \ZZ^+$ and $U \in \AA$, we have $E_n(U) \subseteq U$}
    \step{7}{For $n \in \ZZ^+$, \pflet{$\mathcal{E}_n = \{ E_n(U) \mid U \in \AA \}$}}
    \step{8}{For $n \in \ZZ^+$, we have $\mathcal{E}_n$ is locally discrete.}
    \begin{proof}
        \pf\ For $x \in X$, we have $B(x,1/6n)$ intersects at most one element of $\mathcal{E}_n$.
    \end{proof}
    \step{8}{For $n \in \ZZ^+$, we have $\mathcal{E}_n$ refines $\AA$.}
    \step{9}{\pflet{$\mathcal{E} = \bigcup_{n \in \ZZ^+} \mathcal{E}_n$}}
    \step{10}{$\mathcal{E}$ covers $X$.}
    \begin{proof}
        \step{a}{\pflet{$x \in X$}}
        \step{b}{\pick\ $U \in \AA$ such that $x \in U$}
        \step{c}{\pick\ $n$ such that $B(x,1/n) \subseteq U$}
        \step{d}{$x \in E_n(U) \in \mathcal{E}$}
    \end{proof}
    \qed
\end{proof}

\begin{corollary}
    \label{corollary:metrizable_paracompact}
    Every metrizable space is paracompact.
\end{corollary}

\begin{proof}
    \pf\ By Lemma \ref{lemma:paracompactness}. \qed
\end{proof}

\begin{theorem}[Nagata-Smirnov Metrization Theorem, Bing Metrization Theorem (Choice)]
    Let $X$ be a topological space. Then the following are equivalent.
    \begin{enumerate}
        \item $X$ is metrizable.
        \item $X$ is regular and has a countably locally finite basis.
        \item $X$ is regular and has a countably locally discrete basis.
    \end{enumerate}
\end{theorem}

\begin{proof}
    \pf
    \step{1}{Every metrizable space is regular.}
    \begin{proof}
        \pf\ Proposition \ref{proposition:metrizable_normal}.
    \end{proof}
    \step{2}{Every metrizable space has a countably locally discrete basis.}
    \begin{proof}
        \step{a}{\pflet{$X$ be a metrizable space.}}
        \step{b}{\pick\ a metric $d$ that induces the topology on $X$.}
        \step{c}{For $m \in \ZZ^+$, \pflet{$\AA_m$ be the set of all open balls of radius $1/m$.}}
        \step{d}{For $m \in \ZZ^+$, \pick\ a countably locally discrete open refinement $\BB_m$ of $\AA_m$ that covers $X$.}
        \begin{proof}
            \pf\ Proposition \ref{proposition:countably_locally_discrete_refinement}.
        \end{proof}
        \step{e}{For $m \in \ZZ^+$, every element of $\BB_m$ has diameter at most $2 / m$.}
        \step{f}{\pflet{$\BB = \bigcup_m \BB_m$}}
        \step{g}{$\BB$ is countably locally discrete.}
        \step{h}{$\BB$ is a basis for $X$.}
        \begin{proof}
            \step{i}{\pflet{$x \in X$}}
            \step{ii}{\pflet{$\epsilon > 0$}}
            \step{iii}{\pick\ $m$ such that $1 / m < \epsilon / 2$}
            \step{iv}{\pick\ $B \in \BB_m$ such that $x \in B$}
            \begin{proof}
                \pf\ Since $\BB_m$ covers $X$.
            \end{proof}
            \step{v}{$B \subseteq B(x,\epsilon)$}
        \end{proof}
    \end{proof}
    \step{3}{Every regular space with a countably locally finite basis is metrizable.}
    \begin{proof}
        \step{a}{\pflet{$X$ be a regular space with a countably locally finite basis.}}
        \step{b}{\pick\ a countably locally basis $\BB$ for $X$.}
        \step{c}{$X$ is perfectly normal.}
        \begin{proof}
            \pf\ Lemma \ref{lemma:regular_countably_locally_finite_basis_perfectly_normal}.
        \end{proof}
        \step{d}{\pick\ a sequence $(\BB_n)$ of locally finite sets such that $\BB = \bigcup_n \BB_n$}
        \step{e}{For $n \in \ZZ^+$ and $B \in \BB_n$, \pick\ a continuous $f_{nB} : X \rightarrow [0,1/n]$
        such that $\inv{f_{nB}}(0) = X - B$}
        \begin{proof}
            \pf\ Theorem \ref{theorem:vanish_precisely_closed_gdelta}.
        \end{proof}
        \step{f}{$\{ f_{nB} \mid n \in \ZZ^+, B \in \BB_n \}$ separates points from closed sets.}
        \begin{proof}
            \step{i}{\pflet{$x_0 \in X$}}
            \step{ii}{\pflet{$U$ be an open neightbourhood of $x_0$.}}
            \step{iii}{\pick\ $B \in \BB$ such that $x_0 \in B \subseteq U$.}
            \step{iv}{\pick\ $n$ such that $B \in \BB_n$.}
            \step{v}{$f_{nB}(x_0) > 0$}
            \step{vi}{$f_{nB}(x_0)$ vanishes outside $U$.}
        \end{proof}
        \step{g}{\pflet{$J = \Sigma_{n \in \ZZ^+} \BB_n$} \prove{$X$ is embeddable in $[0,1]^J$ under the uniform topology.}}
        \step{h}{Define $F : X \rightarrow [0,1]^J$ by: $F(x) = (f_{nB}(x))_{(n,B) \in J}$}
        \step{j}{$F$ is an embedding with respect to the product topology.}
        \begin{proof}
            \pf\ By the Embedding Theorem.
        \end{proof}
        \step{k}{$F$ is an embedding with respect to the uniform topology.}
        \begin{proof}
            \step{i}{$F$ is injective.}
            \begin{proof}
                \pf\ From \stepref{j}.
            \end{proof}
            \step{ii}{$F$ is continuous with respect to the product topology.}
            \begin{proof}
                \step{one}{\pflet{$\rho$ be the uniform metric on $[0,1]^J$}}
                \step{two}{For $x, y \in X$ we have $\rho(x,y) = \sup_{\alpha \in J} |x_\alpha - y_\alpha|$}
                \step{three}{\pflet{$x_0 \in X$}}
                \step{four}{\pflet{$\epsilon > 0$}}
                \step{five}{For $n \in \ZZ^+$, \pick\ an open neighbourhood $U_n$ of $x_0$
                that intersects only finitely many elements of $\BB_n$, say $B_{n1}$, \ldots, $B_{nk_n}$}
                \step{six}{For $n \in \ZZ^+$, \pick\ an open neighbourhood $V_n$ of $x_0$
                such that, for $1 \leq i \leq k_n$, $f_{nB_{ni}}(V_n) \subseteq B(f_{nB_{ni}}(x_0),\epsilon/2)$}
                \step{seven}{\pick\ $N$ such that $1/N \leq \epsilon / 2$}
                \step{eight}{\pflet{$W = V_1 \cap \cdots \cap V_N$}}
                \step{nine}{$W$ is an open neighbourhood of $x_0$}
                \step{ten}{For all $x \in W$ we have $\rho(F(x),F(x_0)) < \epsilon$}
                \begin{proof}
                    \step{A}{\pflet{$x \in W$}}
                    \step{B}{For $n \leq N$ and $B \in \BB_n$ we have $|f_{nB}(x) - f_{nB}(x_0)| \leq \epsilon / 2$}
                    \step{C}{For $n > N$ and $B \in \BB_n$ we have $|f_{nB}(x) - f_{nB}(x_0)| < \epsilon / 2$}
                    \begin{proof}
                        \pf\ Since $f_{nB}(X) \subseteq [0,1/n]$ and $1/n < \epsilon / 2$.
                    \end{proof}
                    \step{D}{$\rho(F(x),F(x_0)) \leq \epsilon / 2$}
                \end{proof}
            \end{proof}
            \step{iii}{$F$ maps open sets to open subsets of $F(X)$.}
            \begin{proof}
                \pf\ From \stepref{j} since the product topology is coarser.
            \end{proof}
        \end{proof}
    \end{proof}
    \qed
\end{proof}

\begin{corollary}[Urysohn Metrization Theorem (Choice)]
    Every second countable regular space is metrizable.
\end{corollary}

\section{Isometry}

\begin{definition}[Isometry]
    Let $X$ be a metric space. An \emph{isometry} of $X$ is a function
    $f : X \rightarrow X$ such that, for all $x, y \in X$, we have
    $d(x,y) = d(f(x),f(y))$.    
\end{definition}

\begin{proposition}
    An isometry on a compact metric space is a homeomorphism.
\end{proposition}

\begin{proof}
    \pf
    \step{1}{\pflet{$X$ be a compact metric space.}}
    \step{2}{\pflet{$f : X \rightarrow X$ be an isometry.}}
    \step{3}{$f$ is an imbedding}
    \begin{proof}
        \pf\ Proposition \ref{proposition:isometric_imbedding}.
    \end{proof}
    \step{4}{$f$ is surjective.}
    \begin{proof}
        \step{a}{\assume{for a contradiction $a \notin f(X)$}}
        \step{b}{$f(X)$ is closed}
        \begin{proof}
            \pf\ Proposition \ref{proposition:closed_map_compact_Hausdorff}.
        \end{proof}
        \step{c}{\pick\ $\epsilon > 0$ such that $B(a,\epsilon) \cap f(X) = \emptyset$}
        \step{d}{For $m, n \in \NN$ with $m \neq n$, we have $d(f^m(a),f^n(a)) \geq \epsilon$}
        \begin{proof}
            \step{i}{\assume{without loss of generality $m < n$}}
            \step{ii}{$d(a, f^{n-m}(a)) \geq \epsilon$}
            \begin{proof}
                \pf\ \stepref{c}
            \end{proof}
            \step{iii}{$d(f^{m}(a), f^n(a)) \geq \epsilon$}
            \begin{proof}
                \pf\ \stepref{2}
            \end{proof}
        \end{proof}
        \step{e}{The sequence $(f^n(a))$ has a convergent subsequence.}
        \begin{proof}
            \pf\ Corollary \ref{corollary:compact_sequentially_compact},
            \stepref{1}, Corollary \ref{corollary:metrizable_T1}.
        \end{proof}
        \qedstep
        \begin{proof}
            \pf\ \stepref{d} and \stepref{e} form a contradiction.
        \end{proof}
        \qed
    \end{proof}
    \qed
\end{proof}

\section{Shrinking Maps}

\begin{definition}[Shrinking Map]
    Let $X$ be a metric space. Let $f : X \rightarrow X$. Then $f$ is a
    \emph{shrinking map} if and only if, for all $x, y \in X$ with $x \neq y$,
    we have $d(f(x),f(y)) < d(x,y)$.
\end{definition}

\begin{proposition}
    Let $X$ be a compact metric space. Let $f : X \rightarrow X$ be a shrinking map.
    Then $f$ has a unique fixed point.
\end{proposition}

\begin{proof}
    \pf
    \step{1}{\pflet{$A_n = f^n(X)$ for $n \geq 1$}}
    \step{2}{For all $n \geq 1$ we have $A_n$ is closed.}
    \begin{proof}
        \pf\ Proposition \ref{proposition:closed_map_compact_Hausdorff}.
    \end{proof}
    \step{3}{\pflet{$A = \bigcap_{n=1}^\infty A_n$}}
    \step{4}{\pick\ $a \in A$}
    \begin{proof}
        \pf\ Proposition \ref{proposition:nested_sequence_nonempty_intersection}.
    \end{proof}
    \step{5}{$f(A) = A$}
    \begin{proof}
        \step{a}{$f(A) \subseteq A$}
        \step{b}{$A \subseteq f(A)$}
        \begin{proof}
            \step{i}{\pflet{$x \in A$}}
            \step{ii}{For $n \geq 1$, \pick\ $x_n$ such that $x = f^n(x_n)$}
            \step{iii}{\pick\ a convergent subsequence $(f^{n_r-1}(x_{n_r}))$ of
            $(f^{n-1}(x_n))$ with limit $l$}
            \begin{proof}
                \pf\ Corollary \ref{corollary:compact_sequentially_compact}.
            \end{proof}
            \step{iv}{$f(l) = x$}
            \begin{proof}
                \pf\ Both are the limit of $f(f^{n_r-1}(a_{n_r})) = f^{n_r}(a_{n_r})$.
            \end{proof}
            \step{v}{$l \in A$}
            \begin{proof}
                \step{i}{\assume{for a contradiction $l \notin A$}}
                \step{ii}{\pick\ $N$ such that $l \notin A_N$}
                \step{iii}{\pick\ $R$ such that $n_R > N$}
                \step{iv}{For $r \geq R$ we have $f^{n_r-1}(a_{n_r}) \in A_N$}
                \qedstep
                \begin{proof}
                    \pf\ This is a contradiction.
                \end{proof}
            \end{proof}
        \end{proof}
    \end{proof}
    \step{6}{$\diam A = A$}
    \begin{proof}
        \step{a}{\pick\ $x, y \in A$ such that $d(x,y) = \diam A$}
        \begin{proof}
            \pf\ By the Extreme Value Theorem.
        \end{proof}
        \step{b}{\pick\ $x', y' \in A$ such that $x = f(x')$ and $y = f(y')$}
        \begin{proof}
            \pf\ By \stepref{5}.
        \end{proof}
        \step{d}{$x' = y'$}
        \begin{proof}
            \pf\ If $x' \neq y'$ then $\diam A = d(x,y) < d(x',y')$ which is
            a contradiction.
        \end{proof}
        \step{e}{$x = y$}
    \end{proof}
    \step{h}{$f(a) = a$}
    \begin{proof}
        \pf\ Since $a, f(a) \in A$
    \end{proof}
    \step{i}{If $f(b) = b$ then $b = a$}
    \begin{proof}
        \pf\ If $f(b) = b$ then $b \in A$.
    \end{proof}
    \qed
\end{proof}

The following example shows that we cannot weaken the hypothesis from '$X$ is
a compact metric space' to '$X$ is a complete metric space'.

\begin{example}
    The function $f : \RR \rightarrow \RR$ defined by
    $f(x) = [x + (x^2 + 1)^{1/2}] / 2$ is a shrinking map with no fixed point.
\end{example}

\section{Contractions}

\begin{definition}[Contraction]
    Let $X$ be a metric space. Let $f : X \rightarrow X$. Then $f$ is a
    \emph{contraction} if and only if there exists $\alpha < 1$ such that,
    for all $x, y \in X$, we have $d(f(x),f(y)) \leq \alpha d(x,y)$.
\end{definition}

\section{Locally Metrizable Spaces}

\begin{definition}[Locally Metrizable]
    A topological space is \emph{locally metrizable} if and only if
    every point has a metrizable open neighbourhood.
\end{definition}

\begin{example}
    The space $S_\Omega$ is locally metrizable because, for any countable ordinal $\alpha$,
    the open neighbourhood $[0,\alpha + 1)$ is embeddable in $\RR$.
\end{example}

\begin{proposition}
    Every locally metrizable regular Lindel\"{o}f space is metrizable.
\end{proposition}

\begin{proof}
    \pf
    \step{1}{\pflet{$X$ be a locally metrizable regular Lindel\"{o}f space.}}
    \step{2}{Every point in $X$ has a second countable open neightbourhood.}
    \begin{proof}
        \step{a}{\pflet{$x \in X$}}
        \step{b}{\pick\ a metrizable open neighbourhood $U$ of $x$.}
        \step{c}{\pick\ an open neighbourhood $V$ of $x$ such that
        $\overline{V} \subseteq U$.}
        \step{d}{$V$ is second countable.}
        \begin{proof}
            \pf\ It is Lindel\"{o}f and metrizable.
        \end{proof}
    \end{proof}
    \step{3}{There exists a finite open cover of $X$ by
    second countable subspaces.}
    \step{4}{$X$ is second countable.}
    \step{5}{$X$ is metrizable.}
    \begin{proof}
        \pf\ By the Urysohn Metrizable Theorem.
    \end{proof}
    \qed
\end{proof}

\begin{corollary}
    Every locally metrizable compact Hausdorff space is metrizable.
\end{corollary}

\begin{example}
    The space $\overline{S_\Omega}$ is not locally metrizable, because it is compact Hausdorff but not metrizable.
\end{example}

\begin{example}
    The space $\RR_l$ is not locally metrizable, because it is regular and Lindel\"{o}f but not metrizable.
\end{example}

\begin{proposition}
    Every subspace of a locally metrizable space is locally metrizable.
\end{proposition}

\begin{proof}
    \pf
    \step{1}{\pflet{$X$ be locally metrizable.}}
    \step{2}{\pflet{$Y \subseteq X$}}
    \step{3}{\pflet{$y \in Y$}}
    \step{4}{\pick\ a metrizable open neighbourhood $U$ of $y$ in $X$.}
    \step{5}{$U \cap Y$ is a metrizable open neighbourhood of $y$ in $Y$.}
    \qed
\end{proof}

\begin{corollary}
    The Sorgenfrey plane is not locally metrizable, because it has a subspace homeomorphic to $\RR_l$.
\end{corollary}

\begin{example}
    The space $S_\Omega \times \overline{S_\Omega}$ is not locally metrizable, because it has a subspace homemorphic to
    $\overline{S_\Omega}$.    
\end{example}

\begin{example}
    The ordered square is not locally metrizable, because it is compact Hausdorff but not metrizable.
\end{example}

\begin{proposition}
    Every locally metrizable space is first countable.
\end{proposition}

\begin{proof}
    \pf
    \step{1}{\pflet{$X$ be a locally metrizable space.}}
    \step{2}{\pflet{$x \in X$}}
    \step{3}{\pick\ a metrizable open neighbourhood $U$ of $x$.}
    \step{4}{\pick\ a countable local basis $\BB$ at $x$ in $U$.}
    \begin{proof}
        \pf\ Proposition \ref{proposition:metrizable_first_countable}.
    \end{proof}
    \step{5}{Every element of $\BB$ is open in $X$.}
    \begin{proof}
        \pf\ Lemma \ref{lemma:subspace_open}.
    \end{proof}
    \step{6}{Every neighbourhood of $x$ in $X$ includes an element of $\BB$.}
    \begin{proof}
        \pf\ For every neighbourhood $V$ of $x$ in $X$, there exists $B \in \BB$
        such that $x \in B \subseteq U \cap V$.
    \end{proof}
    \qed
\end{proof}

\begin{corollary}
    The space $\RR^\omega$ under the box topology is not locally metrizable.
\end{corollary}

\begin{corollary}
    The space $\RR^I$ is not locally metrizable.
\end{corollary}

\begin{proposition}
    The space $\RR_K$ is locally metrizable.
\end{proposition}

\begin{proof}
    \pf\ For any non-zero point, an open interval that does not contain 0 is a metrizable
    open neighbourhood. For 0, the set $(-1,1) - K$ is a metrizable open neighbourhood. \qed
\end{proof}

\begin{proposition}
    The product of two locally metrizable spaces is locally metrizable.
\end{proposition}

\begin{proof}
    \pf
    \step{1}{\pflet{$X$ and $Y$ be locally metrizable spaces.}}
    \step{2}{\pflet{$(x,y) \in X \times Y$}}
    \step{3}{\pflet{$U$ be a neighbourhood of $(x,y)$}}
    \step{4}{\pick\ neighbourhoods $V$ of $x$ and $W$ of $y$
    such that $V \times W \subseteq U$}
    \step{5}{\pick\ metrizable neighbourhoods $V'$ of $x$ and
    $W'$ of $y$ with $V' \subseteq V$ and $W' \subseteq W$}
    \step{6}{$V' \times W'$ is connected.}
    \step{7}{$(x,y) \in V' \times W' \subseteq U$}
    \qed
\end{proof}

\begin{proposition}
    The product of a countable family of locally metrizable spaces is
    not necessarily locally metrizable.
\end{proposition}

\begin{proof}
    \pf
    \step{1}{\pick\ a space $X$ that is locally metrizable but not metrizable. \prove{$X^\omega$ is not locally metrizable.}}
    \step{2}{\pflet{$x \in X^\omega$}}
    \step{3}{\assume{for a contradiction $x$ has a metrizable neighbourhood $U$}}
    \step{4}{\pick\ a basic open neighbourhood $\prod_n U_n$ with $\prod_n U_n \subseteq U$
    and $U_n = X$ for all but finitely many $n$}
    \step{5}{$X$ is metrizable}
    \begin{proof}
        \pf\ It is homeomorphic to a subspace of $U$.
    \end{proof}
    \qedstep
    \begin{proof}
        \pf\ This is a contradiction.
    \end{proof}
    \qed
\end{proof}

\begin{proposition}
    The continuous image of a locally metrizable space is not necessarily locally metrizable.
\end{proposition}

\begin{proof}
    \pf\ Take the identity function from a set under the discrete topology
    to the same set under the indiscrete topology. \qed
\end{proof}

\begin{theorem}[Smirnov Metrization Theorem (Choice)]
    A topological space is metrizable if and only if it is paracompact, Hausdorff, and locally metrizable.
\end{theorem}

\begin{proof}
    \pf
    \step{1}{Every metrizable space is paracompact.}
    \begin{proof}
        \pf\ Corollary \ref{corollary:metrizable_paracompact}.        
    \end{proof}
    \step{2}{Every metrizable space is Hausdorff.}
    \begin{proof}
        \pf\ Proposition \ref{proposition:metrizable_Hausdorff}.
    \end{proof}
    \step{3}{Every metrizable space is locally metrizable.}
    \begin{proof}
        \pf\ Trivial.
    \end{proof}
    \step{4}{Every paracompact Hausdorff locally metrizable space is metrizable.}
    \begin{proof}
        \step{a}{\pflet{$X$ be a paracompact Hausdorff locally metrizable space.}}
        \step{b}{$X$ has a countably locally finite basis.}
        \begin{proof}
            \step{i}{\pick\ a locally finite set $\mathcal{C}$ of metrizable open subsets of $X$ that cover $X$.}
            \step{ii}{For $C \in \mathcal{C}$, \pick\ a metric $d_C : C^2 \rightarrow \RR$ that induces the subspace topology.}
            \step{iii}{For all $C \in \mathcal{C}$, $x \in C$ and $\epsilon > 0$, we have $B_C(x,\epsilon)$ is open in $X$.}
            \step{iv}{For $m \in \ZZ^+$, \pflet{$\AA_m = \{ B_C(x, 1/m) \mid C \in \mathcal{C}, x \in C \}$}}
            \step{v}{For $m \in \ZZ^+$, \pick\ a locally finite open refinement $\DD_m$ of $\AA_m$ that covers $X$.}
            \step{vi}{\pflet{$\DD = \bigcup_m \DD_m$} \prove{$\DD$ is a basis for $X$.}}
            \step{vii}{\pflet{$x \in X$}}
            \step{viii}{\pflet{$U$ be an open neighbourhood of $x$.}}
            \step{ix}{\pflet{$C_1$, \ldots, $C_k$ be the elements of $\mathcal{C}$ such that $x \in C_i$}}
            \step{x}{For $1 \leq i \leq k$, \pick\ $\epsilon_i > 0$ such that $B_{C_i}(x,\epsilon_i) \subseteq U$}
            \step{xi}{\pick\ $m$ such that $2 / m \leq \min(\epsilon_1, \ldots, \epsilon_k)$}
            \step{xii}{\pick\ $D \in \DD_m$ such that $x \in D$}
            \step{xiii}{\pick\ $C \in \mathcal{C}$ and $y \in C$ such that $D \subseteq B_C(y,1/m)$}
            \step{xiv}{$x \in B_C(y, 1/m)$}
            \step{xv}{\pick\ $i$ such that $C = C_i$}
            \step{xvi}{$D \subseteq U$}
            \begin{proof}
                \pf
                \begin{align*}
                    D & \subseteq B_{C_i}(y,1/m) & (\text{\stepref{xiii}, \stepref{xv}})\\
                    & \subseteq B_{C_i}(x,\epsilon_i) & (\diam B_{C_i}(y,1/m) \leq 2/m \leq \epsilon_i) \\
                    & \subseteq U & (\text{\stepref{x}})
                \end{align*}
            \end{proof}
        \end{proof}
        \qedstep
        \begin{proof}
            \pf\ By the Nagata-Smirnov Metrization Theorem.
        \end{proof}
    \end{proof}
    \qed
\end{proof}

\section{Cauchy Sequences}

\begin{definition}[Cauchy Sequence]
    Let $X$ be a metric space. A sequence of points $(x_n)$ in $X$ is a \emph{Cauchy} sequence if and only if
    $\forall \epsilon > 0. \exists N. \forall m,n \geq N. d(x_m,x_n) < \epsilon$.
\end{definition}

\begin{proposition}
    Every convergent sequence is Cauchy.
\end{proposition}

\begin{proof}
    \pf
    \step{1}{\pflet{$X$ be a metric space.}}
    \step{2}{\pflet{$(x_n)$ be a convergent sequence in $X$.}}
    \step{3}{\pflet{$l = \lim_{n \rightarrow \infty} x_n$}}
    \step{4}{\pflet{$\epsilon > 0$}}
    \step{5}{\pick\ $N$ such that $\forall n \geq N. d(x_n,l) < \epsilon / 2$}
    \step{6}{$\forall m,n \geq N. d(x_m,x_n) < \epsilon$}
    \qed
\end{proof}

\begin{proposition}
    \label{proposition:Cauchy_standard_bounded_metric}
    Let $d$ be a metric on a set $X$. Let $\overline{d}$ be the corresponding standard bounded metric.
    Let $(x_n)$ be a sequence of points in $X$. Then $(x_n)$ is Cauchy under $d$ if and only if
    it is Cauchy under $\overline{d}$.
\end{proposition}

\begin{proof}
    \pf
    \step{1}{If $(x_n)$ is Cauchy under $d$ then $(x_n)$ is Cauchy under $\overline{d}$.}
    \begin{proof}
        \step{a}{\assume{$(x_n)$ is Cauchy under $d$}}
        \step{b}{\pflet{$\epsilon > 0$}}
        \step{c}{\pick\ $N$ such that $\forall m,n \geq N. d(x_m,x_n) < \epsilon$}
        \step{d}{$\forall m,n \geq N. \overline{d}(x_m,x_n) < \epsilon$}
    \end{proof}
    \step{2}{If $(x_n)$ is Cauchy under $\overline{d}$ then $(x_n)$ is Cauchy under $d$.}
    \begin{proof}
        \step{a}{\assume{$(x_n)$ is Cauchy under $\overline{d}$}}
        \step{b}{\pflet{$\epsilon > 0$}}
        \step{c}{\pick\ $N$ such that $\forall m,n \geq N. \overline{d}(x_m,x_n) < \min(\epsilon,1/2)$}
        \step{d}{$\forall m,n \geq N. d(x_m,x_n) < \epsilon$}
    \end{proof}
    \qed
\end{proof}

\section{Complete Metric Spaces}

\begin{definition}[Complete Metric Space]
    A metric space is \emph{complete} if and only if every Cauchy sequence converges.
\end{definition}

\begin{proposition}
    \label{proposition:closed_subspace_complete}
    A closed subspace of a complete metric space is complete.
\end{proposition}

\begin{proof}
    \pf
    \step{1}{\pflet{$X$ be a complete metric space.}}
    \step{2}{\pflet{$Y \subseteq X$ be closed.}}
    \step{3}{\pflet{$(y_n)$ be a Cauchy sequence in $Y$.}}
    \step{4}{$(y_n)$ is a Cauchy sequence in $X$.}
    \step{5}{\pflet{$y_n \rightarrow l$ as $n \rightarrow \infty$ in $X$.}}
    \step{6}{$l \in Y$}
    \step{7}{$y_n \rightarrow l$ as $n \rightarrow \infty$ in $Y$.}
    \qed
\end{proof}

\begin{proposition}
    Let $X$ be a topological space and $Y$ a complete metric space. The space $\mathcal{C}(X,Y)$ of all continuous functions
    $X \rightarrow Y$ under the uniform metric is complete.
\end{proposition}

\begin{corollary}
    Let $X$ be a set and $Y$ a complete metric space. The space $\mathcal{B}(X,Y)$ of all bounded functions $X \rightarrow Y$
    under the uniform metric is complete.
\end{corollary}

\begin{proposition}
    Let $d$ and $d'$ be metrically equivalent metrics on a set $X$. Then $(X,d)$ is complete if and only if $(X,d')$
    is complete.
\end{proposition}

\begin{proof}
    \pf
    \step{1}{If $(X,d)$ is complete then $(X,d')$ is complete.}
    \begin{proof}
        \step{a}{\assume{$(X,d)$ is complete.}}
        \step{b}{\pflet{$(x_n)$ be a Cauchy sequence under $d'$.}}
        \step{c}{$(x_n)$ is a Cauchy sequence under $d$.}
        \begin{proof}
            \step{i}{\pflet{$\epsilon > 0$}}
            \step{ii}{\pick\ $\delta > 0$ such that, for all $x,y \in X$, if $d'(x,y) < \delta$ then $d(x,y) < \epsilon$}
            \step{iii}{\pick\ $N$ such that $\forall m,n \geq N. d'(x_m,x_n) < \delta$}
            \step{iv}{$\forall m,n \geq N. d(x_m,x_n) < \epsilon$}
        \end{proof}
        \step{d}{\pflet{$l$ be the limit of $(x_n)$ under $d$.}}
        \step{e}{$(x_n)$ converges to $l$ under $d'$.}
        \begin{proof}
            \step{i}{\pflet{$\epsilon > 0$}}
            \step{ii}{\pick\ $\delta > 0$ such that, for all $x,y \in X$, if $d(x,y) < \delta$ then $d'(x,y) < \epsilon$}
            \step{iii}{\pick\ $N$ such that $\forall n \geq N. d(x_n,l) < \delta$}
            \step{iv}{$\forall n \geq N, d'(x_n,l) < \epsilon$}
        \end{proof}
    \end{proof}
    \step{2}{If $(X,d')$ is complete then $(X,d)$ is complete.}
    \begin{proof}
        \pf\ Similar.
    \end{proof}
    \qed
\end{proof}

\begin{corollary}
    Let $X$ be a set. Let $d$ be a metric on $X$. Let $\overline{d}$ be the corresponding
    standard bounded metric. Then the metric space $(X,d)$ is complete if and only if $(X,\overline{d})$
    is complete.
\end{corollary}

\begin{lemma}
    A metric space is complete if and only if every Cauchy sequence has a convergent subsequence.
\end{lemma}

\begin{proof}
    \pf
    \step{1}{\pflet{$X$ be a metric space.}}
    \step{2}{If $X$ is complete then every Cauchy sequence has a convergent subsequence.}
    \begin{proof}
        \pf\ If $X$ is complete then, for every Cauchy sequence $(x_n)$, we have $(x_n)$ is a convergent subsequence of itself.
    \end{proof}
    \step{3}{If every Cauchy sequence in $X$ has a convergent subsequence then $X$ is complete.}
    \begin{proof}
        \step{a}{\assume{Every Cauchy sequence in $X$ has a convergent subsequence.}}
        \step{b}{\pflet{$(x_n)$ be a Cauchy sequence in $X$.}}
        \step{c}{\pick\ a convergent subsequence $(x_{n_r})$.}
        \step{d}{\pflet{$l = \lim_{r \rightarrow \infty} x_{n_r}$} \prove{$x_n \rightarrow l$ as $n \rightarrow \infty$}}
        \step{e}{\pflet{$\epsilon > 0$}}
        \step{f}{\pick\ $N$ such that $\forall m,n \geq N. d(x_m,x_n) < \epsilon / 2$ and
        $\forall r (n_r \geq N \Rightarrow d(x_{n_r},l) < \epsilon / 2)$}
        \step{g}{$\forall n \geq N. d(x_n,l) < \epsilon$}
        \begin{proof}
            \step{i}{\pflet{$n \geq N$}}
            \step{ii}{\pick\ $r$ such that $n_r \geq N$}
            \step{iii}{$d(x_n,l) < \epsilon$}
            \begin{proof}
                \pf
                \begin{align*}
                    d(x_n,l) & \leq d(x_n,x_{n_r}) + d(x_{n_r},l) \\
                    & < \epsilon / 2 + \epsilon / 2 \\
                    & = \epsilon
                \end{align*}
            \end{proof}
        \end{proof}
    \end{proof}
    \qed
\end{proof}

\begin{corollary}
    \label{corollary:compact_complete}
    Every compact metric space is complete.
\end{corollary}

\begin{proposition}
    For $k \geq 1$, the Euclidean space $\RR^k$ is complete under the square metric.
\end{proposition}

\begin{proof}
    \pf\ A Cauchy sequence is bounded hence a Cauchy sequence in the compact subspace $[-B,B]^k$
    for some $B$. \qed
\end{proof}

\begin{proposition}
    For $k \geq 1$, the Euclidean space $\RR^k$ is complete under the Euclidean metric.
\end{proposition}

\begin{proof}
    \pf\ A sequence is Cauchy under the Euclidean metric if and only if it is Cauchy under the
    square metric, and converges under the Euclidean metric if and only if it converges under
    the square metric. \qed
\end{proof}

\begin{proposition}
    There exists a metric on $\RR^\omega$ which induces the product topology under which $\RR^\omega$
    is complete.
\end{proposition}

\begin{proof}
    \pf
    \step{1}{\pflet{$\overline{d}$ be the standard bounded metric on $\RR$.}}
    \step{2}{Define $D : (\RR^\omega)^2 \rightarrow \RR$ by $D(x.y) = \sup_n \overline{d}(x_n,y_n) / n$}
    \step{3}{$D$ induces the product topology.}
    \step{4}{$\RR^\omega$ is complete under $D$.}
    \begin{proof}
        \step{a}{\pflet{$(x_n)$ be a Cauchy sequence under $D$.}}
        \step{b}{For all $i$, $(\pi_i(x_n))_n$ is a Cauchy sequence in $\RR$.}
        \step{c}{For all $i$, \pflet{$l_i = \lim_{n \rightarrow \infty} \pi_i(x_n)$}}
        \step{d}{$x_n \rightarrow (l_i)_i$ as $n \rightarrow \infty$}
    \end{proof}
    \qed
\end{proof}

\begin{proposition}
    The space $\QQ$ is not complete.
\end{proposition}

\begin{proof}
    \pf\ The sequence $(1,1.4,1.41,\ldots)$ of decimal approximations to $\sqrt{2}$ is Cauchy but does not converge. \qed
\end{proof}

\begin{proposition}
    The space $(0,1)$ is not complete.
\end{proposition}

\begin{proof}
    \pf\ The sequence $(1/n)$ is Cauchy but does not converge. \qed
\end{proof}

\begin{corollary}
    There exists a metrizable space that is complete under one metric but not under another metric that induces the same topology.
\end{corollary}

\begin{proof}
    \pf\ The space $\RR$ is complete under its usual metric but homeomorphic to the incomplete space $(0,1)$. \qed
\end{proof}

\begin{theorem}
    Let $Y$ be a complete metric space and $J$ a set. Then $Y^J$ is complete under the uniform metric.
\end{theorem}

\begin{proof}
    \pf
    \step{1}{\pflet{$(f_n)$ be a Cauchy sequence in $Y^J$.}}
    \step{2}{For all $\alpha \in J$, the sequence $(f_n(\alpha))$ is Cauchy in $Y$.}
    \step{3}{For all $\alpha \in J$, \pflet{$y_\alpha = \lim_{n \rightarrow \infty} f_n(\alpha)$}}
    \step{4}{Define $y : J \rightarrow Y$ by $y(\alpha) = y_\alpha$. \prove{$f_n \rightarrow y$ as $n \rightarrow \infty$}}
    \step{5}{\pflet{$\epsilon > 0$}}
    \step{6}{\pick\ $N$ such that $\forall m,n \geq N. \overline{\rho}(f_m,f_n) < \epsilon / 2$}
    \step{7}{$\forall m,n \geq N. \forall \alpha \in J. \overline{d}(f_m(\alpha),f_n(\alpha)) < \epsilon / 2$}
    \step{8}{$\forall n \geq N, \forall \alpha \in J. \overline{d}(f_n(\alpha),y_\alpha) < \epsilon / 2$}
    \step{9}{$\forall n \geq N. \overline{\rho}(f_n,y) \leq \epsilon / 2$}
    \step{10}{$\forall n \geq N. \overline{\rho}(f_n,y) < \epsilon$}
    \qed
\end{proof}

\begin{proposition}
    Let $X$ be a metric space. Suppose that there exists $\epsilon > 0$ such that every closed ball of radius $\epsilon$
    is compact. Then $X$ is complete.
\end{proposition}

\begin{proof}
    \pf
    \step{1}{\pflet{$(x_n)$ be a Cauchy sequence in $X$.}}
    \step{2}{\pick\ $N$ such that $\forall m,n \geq N. d(x_m,x_n) < \epsilon$}
    \step{3}{$(x_n)_{n \geq N}$ is a Cauchy sequence in $\overline{B(x_N,\epsilon)}$}
    \step{4}{$(x_n)$ converges.}
    \qed
\end{proof}

\begin{proposition}
    Let $X$ and $Y$ be metric spaces. Suppose $Y$ is complete. Let $A \subseteq X$. Let $f : A \rightarrow Y$
    be uniformly continuous. Then there exists a unique continuous extension $g : \overline{A} \rightarrow Y$
    of $f$, and this extension $g$ is uniformly continuous.
\end{proposition}

\begin{proof}
    \pf
    \step{1}{Define $g : \overline{A} \rightarrow Y$ as follows. Given $x \in \overline{A}$, pick a sequence $(a_n)$
    in $A$ that converges to $x$. Then $g(x) = \lim_{n \rightarrow \infty} f(a_n)$.}
    \step{2}{$g$ is uniformly continuous.}
    \begin{proof}
        \step{a}{\pflet{$\epsilon > 0$}}
        \step{b}{Pick $\delta$ such that, for all $x, y \in X$, if $d(x,y) < \delta$ then $d(f(x),f(y)) < \epsilon / 2$}
        \step{c}{\pflet{$x,y \in X$ with $d(x,y) < \delta/3$}}
        \step{d}{\pick\ sequences $(a_n)$, $(b_n)$ in $A$ that converge to $x$ and $y$ respectively.}
        \step{e}{There exists $N$ such that $\forall n \geq N. d(a_n,b_n) < \delta$}
        \step{e}{$d(g(x),g(y)) < \epsilon$}
        \begin{proof}
            \pf
            \begin{align*}
                d(g(x),g(y)) & = d(\lim_{n \rightarrow \infty} f(a_n), \lim_{n \rightarrow \infty} f(b_n)) \\
                & = \lim_{n \rightarrow \infty} d(f(a_n),f(b_n)) \\
                & \leq \epsilon / 2
            \end{align*}
        \end{proof}
    \end{proof}
    \step{22}{$g$ extends $f$.}
    \step{3}{If $h : \overline{A} \rightarrow Y$ is a continuous extension of $f$ then $h = g$.}
\end{proof}

\begin{proposition}[Choice]
    Let $X$ be a metric space. Then $X$ is complete if and only if, for every nested sequence $A_1 \supseteq A_2 \supseteq \cdots$
    of nonempty closed sets such that $\diam A_n \rightarrow 0$, then $\bigcap_n A_n$ is nonempty.
\end{proposition}

\begin{proof}
    \pf
    \step{1}{$\Rightarrow$}
    \begin{proof}
        \step{a}{\assume{$X$ is complete.}}
        \step{b}{\pflet{$(A_n)$ be a nested sequence of nonempty closed sets such that $\diam A_n \rightarrow 0$}}
        \step{c}{For all $n$, \pick\ $a_n \in A_n$}
        \step{d}{$(a_n)$ is Cauchy.}
        \begin{proof}
            \step{i}{\pflet{$\epsilon > 0$}}
            \step{ii}{\pick\ $N$ such that $\diam A_N < \epsilon$}
            \step{iii}{$\forall m,n \geq N. d(a_m,a_n) < \epsilon$}
            \begin{proof}
                \pf\ Since $a_m,a_n \in A_N$.
            \end{proof}
        \end{proof}
        \step{e}{\pflet{$l = \lim_n a_n$} \prove{$l \in \bigcap_n A_n$}}
        \step{f}{\pflet{$n \in \ZZ^+$} \prove{$l \in A_n$}}
        \step{g}{\pflet{$\epsilon > 0$} \prove{$B(l,\epsilon)$ intersects $A_n$}}
        \step{h}{\pick\ $m$ such that $m \geq n$ and $d(a_m,l) < \epsilon$}
        \step{i}{$a_m \in B(l,\epsilon) \cap A_n$}
    \end{proof}
    \step{2}{$\Leftarrow$}
    \begin{proof}
        \step{a}{\assume{Every nested sequence of nonempty closed sets with diameters converging to 0 has nonempty intersection.}}
        \step{b}{\pflet{$(x_n)$ be a Cauchy sequence in $X$.}}
        \step{c}{For all $i$, \pick\ $N_i$ such that $\forall m,n \geq N_i. d(x_m,x_n) < 1/2^i$}
        \step{d}{$(\overline{B(x_{N_i},1/2^{i-1})})$ is a nested sequence of nonempty closed sets with diameters converging to 0.}
        \begin{proof}
            \step{i}{\pflet{$y \in \overline{B(x_{N_{i+1}},1/2^i)}$} \prove{$y \in \overline{B(x_{N_i},1/2^{i-1})}$}}
            \step{ii}{$d(x_{N_i},y) < 1/i$}
            \begin{proof}
                \pf
                \begin{align*}
                    d(x_{N_i},y) & \leq d(x_{N_i},x_{N_{i+1}}) + d(x_{N_{i+1}},y) \\
                    & < 1/2^i + 1/2^i \\
                    & = 1/2^{i-1}
                \end{align*}
            \end{proof}
        \end{proof}
        \step{e}{\pick\ $l \in \bigcap_i \overline{B(x_{N_i},1/2^i)}$}
        \step{f}{$x_n \rightarrow l$ as $n \rightarrow \infty$}
    \end{proof}
    \qed
\end{proof}

\begin{proposition}
    Let $X$ be a complete metric space. Let $f : X \rightarrow X$ be a contraction.
    Then $f$ has a unique fixed point.
\end{proposition}

\begin{proof}
    \pf
    \step{1}{\pick\ $\alpha < 1$ such that $\forall x,y \in X. d(f(x),f(y)) \leq \alpha d(x,y)$.}
    \step{2}{\pick\ $x_0 \in X$}
    \step{3}{$(f^n(x_0))$ is Cauchy.}
    \begin{proof}
        \step{a}{\pflet{$\epsilon > 0$}}
        \step{b}{Pick $N$ such that $\alpha^N / (1-\alpha) d(x_0,f(x_0)) < \epsilon$}
        \step{c}{\pflet{$m,n \geq N$}}
        \step{d}{\assume{w.l.o.g. $m \leq n$}}
        \step{d}{$d(f^m(x_0),f^n(x_0)) < \epsilon$}
        \begin{proof}
            \pf
            \begin{align*}
                d(f^m(x_0),f^n(x_0)) & \leq \alpha^m d(x_0,f^{n-m}(x_0)) \\
                & = \alpha^m (d(x_0,f(x_0)) + d(f(x_0),f^2(x_0)) + \cdots d(f^{n-m-1}(x_0),d^{n-m}(x_0))) \\
                & \leq \alpha^m (1 + \alpha + \cdots + \alpha^{n-m-1}) d(x_0,f(x_0)) \\
                & = \alpha^m (1 - \alpha^{n-m})/(1 - \alpha) d(x_0,f(x_0))  \\
                & \leq \alpha^N / (1 - \alpha) d(x_0, f(x_0)) \\ 
                & < \epsilon
            \end{align*}
        \end{proof}
    \end{proof}
    \step{4}{\pflet{$l = \lim_{n \rightarrow \infty} f^n(x_0)$}}
    \step{5}{$f(l) = l$}
    \step{6}{If $f(m) = m$ then $m = l$}
    \begin{proof}
        \pf\ Since $d(m,l) = d(f(m),f(l)) \leq \alpha d(m,l)$.
    \end{proof}
    \qed
\end{proof}

\begin{proposition}
    The space \[ L^2 = \left\{ (x_n) \in \RR^\omega \mid \sum_{n=0}^\infty x_n^2 \text{ converges} \right\} \] under the
    $l^2$ metric is complete.
\end{proposition}

\begin{proof}
    \pf
    \step{1}{\pflet{$((x_{mn})_n)_m$ be a Cauchy sequence in $L^2$.}}
    \step{2}{For all $n$, the sequence $(x_{mn})_m$ is Cauchy in $\RR$}
    \begin{proof}
        \step{o}{\pflet{$n \in \ZZ^+$}}
        \step{a}{\pflet{$\epsilon > 0$}}
        \step{b}{Pick $M$ such that $\forall m_1,m_2 \geq M. l^2(x_{m_1},x_{m_2}) < \epsilon$}
        \step{c}{\pflet{$m_1,m_2 \geq M$}}
        \step{d}{$|x_{m_1n} - x_{m_2n}| < \epsilon$}
        \begin{proof}
            \pf\ Since $|x_{m_1n} - x_{m_2n}| \leq l^2((x_{m_1},x_{m_2}))$.
        \end{proof}
    \end{proof}
    \step{3}{For all $\epsilon > 0$, there exists $M$ such that $\forall m \geq M. \| x_m - l \|^2 \leq \epsilon^2$}
    \begin{proof}
        \step{a}{\pflet{$\epsilon > 0$}}
        \step{b}{\pick\ $M$ such thath $\forall m_1,m_2 \geq M. l^2(x_{m_1},x_{m_2}) < \epsilon$}
        \step{c}{$\forall N. \forall m_1, m_2 \geq M. \sum_{i=1}^N |x_{m_1i} - x_{m_2i}| < \epsilon^2$}
        \step{d}{$\forall N. \forall m \geq M. \sum_{i=1}^N |x_{mi} - l_i| \leq \epsilon^2$}
        \step{e}{$\forall m \geq M. l^2(x_m,l) \leq \epsilon$}
    \end{proof}
    \step{4}{$l \in L^2$}
    \begin{proof}
        \pf\ We have $x_m - l \in L^2$ by \stepref{3} so $l = x^m - (x^m - l) \in L^2$.
    \end{proof}
    \step{5}{$x_m \rightarrow l$ as $m \rightarrow \infty$}
    \begin{proof}
        \pf\ From \stepref{3}.
    \end{proof}
    \qed
\end{proof}

\section{Sup Metric}

\begin{definition}[Sup Metric]
    Let $X$ be a set and $Y$ a metric space. 
    
    The \emph{sup metric} $\rho$ on $\mathcal{B}(X,Y)$ is defined by
    \[ \rho(f,g) = \sup_{x \in X} d(f(x),g(x)) \]
    It is easy to prove that this is a metric.
\end{definition}

\begin{proposition}
    The uniform metric $\overline{\rho}$ is the standard bounded metric associated with the sup metric.
\end{proposition}

\begin{corollary}
    The uniform metric and the sup metric induce the same topology on $\mathcal{B}(X,Y)$, namely the uniform topology.
\end{corollary}

\begin{corollary}
    If $Y$ is complete then $\mathcal{B}(X,Y)$ is complete under the sup metric.
\end{corollary}

\begin{proposition}
    Let $X$ be a compact space and $Y$ a metric space. The $\mathcal{C}(X,Y) \subseteq \mathcal{B}(X,Y)$.
\end{proposition}

\begin{corollary}
    If $X$ is a compact space and $Y$ a complete metric space then $\mathcal{C}(X,Y)$ is complete under the sup metric.
\end{corollary}

\begin{theorem}
    \label{theorem:completion}
    Every metric space can be isometrically embedded in a complete metric space. Specifically, any metric space $X$ can be
    isometrically embedded in $\mathcal{B}(X,\RR)$ under the sup metric.
\end{theorem}

\begin{proof}
    \pf
    \step{1}{\pflet{$X$ be a metric space.}}
    \step{2}{\pick\ $x_0 \in X$}
    \begin{proof}
        \pf\ If $X$ is empty then the unique function $X \rightarrow \mathcal{B}(X, \RR)$ is an isometric embedding.
    \end{proof}
    \step{2}{Define $\Phi : X \rightarrow \mathcal{B}(X, \RR)$ by $\Phi(a)(b) = d(a,b) - d(b,x_0)$}
    \begin{proof}
        \pf\ For $a \in X$, we have $\Phi(a)$ is bounded since $|\Phi(a)(b)| \leq d(a,x_0)$ for all $b \in X$.
    \end{proof}
    \step{3}{$\Phi$ is an isometric embedding.}
    \begin{proof}
        \step{a}{\pflet{$x, y \in X$}}
        \step{b}{$\rho(\Phi(x),\Phi(y)) = d(x,y)$}
        \begin{proof}
            \pf
            \begin{align*}
                \rho(\Phi(x),\Phi(y)) & = \sup_{z \in X} |\Phi(x)(z) - \Phi(y)(z)| \\
                & = \sup_{z \in X}|d(x,z) - d(y,z)| \\
                & = d(x,y)
            \end{align*}
        \end{proof}
    \end{proof}
    \qed
\end{proof}

\begin{proposition}
    Let $Z$ be a metric space and $A \subseteq Z$. If $A$ is dense and every Cauchy sequence in $A$ converges in $Z$,
    then $Z$ is complete.
\end{proposition}

\section{Completion}

\begin{definition}[Completion]
    The \emph{completion} of a metric space $X$ consists of a complete metric space $Y$ and isometric embedding
    $i : X \rightarrow Y$ such that $Y = \overline{i(X)}$.
\end{definition}

\begin{theorem}
    The completion of a metric space is unique up to isometry.
\end{theorem}

\begin{proof}
    \pf
    \step{1}{\pflet{$i : X \rightarrow Y$ and $j : X \rightarrow Z$ be completions of $X$.}}
    \step{2}{Define $\phi : Y \rightarrow Z$ as follows. Given $y \in Y$, pick a sequence $(x_n)$ in $X$
    such that $i(x_n) \rightarrow y$ as $n \rightarrow \infty$. Then $\phi(y) = \lim_{n \rightarrow \infty} j(x_n)$}
    \step{3}{$\phi$ is an isometry.}
    \qed
\end{proof}

Existence follows from Theorem \ref{theorem:completion}. An alternative proof:

\begin{proof}
    \pf
    \step{1}{\pflet{$X$ be a metric space.}}
    \step{2}{\pflet{$Y$ be the set of all Cauchy sequences in $X$, quotiented by: $(x_n) = (y_n)$ iff $d(x_n,y_n) \rightarrow 0$
    as $n \rightarrow \infty$}}
    \step{3}{Define $D : Y^2 \rightarrow \RR$ by $D((x_n),(y_n)) = \lim_{n \rightarrow \infty} d(x_n,y_n)$}
    \step{4}{$D$ is a metric on $Y$ under which $Y$ is complete.}
    \step{5}{Define $h : X \rightarrow Y$ by $h(x) = (x, x, x, \ldots)$}
    \step{6}{$h$ is an embedding.}
    \step{7}{$h(X)$ is dense in $Y$.}
    \qed
\end{proof}

\section{Topolgically Complete Spaces}

\begin{definition}[Topologically Complete]
    A topological space $X$ is \emph{topologically complete} if and only if there exists a metric that induces the topology on $X$
    under which $X$ is complete.
\end{definition}

\begin{proposition}
    \label{proposition:closed_subspace_topolgically_complete}
    Every closed subspace of a topologically complete space is topologically complete.
\end{proposition}

\begin{proof}
    \pf\ Proposition \ref{proposition:closed_subspace_complete}. \qed
\end{proof}

\begin{proposition}[Choice]
    \label{proposition:product_topologically_complete}
    A product of topologically complete spaces is topologically complete.
\end{proposition}

\begin{proof}
    \pf
    \step{1}{\pflet{$\{X_\alpha\}_{\alpha \in J}$ be a sequence of topologically complete spaces.}}
    \step{2}{For each $\alpha \in C$, \pick\ a metric $d_\alpha$ that induces the topology on $X_n$
    under which $X_\alpha$ is complete.}
    \step{3}{For each $\alpha$, \pflet{$\overline{d_\alpha}$ be the standard bounded metric
    associated with $d_\alpha$.}}
    \step{4}{\pflet{$X = \prod_\alpha X_\alpha$}}
    \step{5}{Define $D : X^2 \rightarrow \RR$ by:
    \[ D(a,b) = \sup_\alpha \overline{d_\alpha}(a(\alpha), b(\alpha)) \]}
    \step{6}{$D$ is a metric on $X$.}
    \step{7}{$X$ is complete under $D$.}
    \begin{proof}
        \step{a}{\pflet{$(f_n)$ be a Cauchy sequence in $X$.}}
        \step{b}{For $\alpha \in J$, we have $(f_n(\alpha))$ is a Cauchy sequence is $X_\alpha$}
        \step{c}{For $\alpha \in J$, \pflet{$l(\alpha) = \lim_{n \rightarrow \infty} f_n(\alpha)$}}
        \step{d}{$f_n \rightarrow l$ as $n \rightarrow \infty$}
    \end{proof}
\end{proof}

\begin{proposition}
    \label{proposition:open_subspace_topologically_complete}
    An open subspace of a topologically complete space is topologically complete.
\end{proposition}

\begin{proof}
    \pf
    \step{1}{\pflet{$X$ be a topologically complete space.}}
    \step{2}{\pick\ a metric $d$ on $X$ that induces the topology on $X$
    under which $X$ is complete.}
    \step{3}{\pflet{$U$ be open in $X$.}}
    \step{3.5}{\assume{w.l.o.g. $U \neq X$}}
    \step{4}{Define $\phi : U \rightarrow \RR$ by $\phi(x) = 1 / d(x, X - U)$}
    \step{5}{Define $f : U \rightarrow X \times \RR$ by $f(x) = (x, \phi(x))$}
    \step{6}{$f$ is an embedding.}
    \begin{proof}
        \step{a}{$f$ is injective.}
        \step{b}{$f$ is continuous.}
        \step{c}{$f$ maps open sets to open sets in $f(U)$}
        \begin{proof}
            \pf\ For any $V \subseteq U$ open, we have $f(V) = (V \times \RR) \cap f(U)$ is open in $f(U)$.
        \end{proof}
    \end{proof}
    \step{7}{$f(U)$ is closed in $X \times \RR$}
    \begin{proposition}
        \pf\ Proposition \ref{proposition:graph_closed}.
    \end{proposition}
    \qedstep
    \begin{proof}
        \pf\ Proposition \ref{proposition:closed_subspace_topolgically_complete}.
    \end{proof}
    \qed
\end{proof}

\begin{proposition}
    Every $G_\delta$ subspace of a topologically complete space is topologically complete.
\end{proposition}

\begin{proof}
    \pf
    \step{1}{\pflet{$X$ be a topologically complete space.}}
    \step{2}{\pick\ a metric $d$ on $X$ that induces the topology on $X$
    under which $X$ is complete.}
    \step{3}{\pflet{$A$ be a $G_\delta$ set in $X$.}}
    \step{4}{\pick\ a sequence $(U_n)$ of open sets such that $A = \bigcap_n U_n$.}
    \step{5}{Define $f : A \rightarrow \prod_n U_n$ by $f(a) = (a, a, \ldots)$}
    \step{6}{$f$ is an embedding.}
    \step{7}{$f(A)$ is closed in $\prod_n U_n$}
    \begin{proof}
        \pf\ It is $\prod_n U_n \cap \Delta$.
    \end{proof}
    \step{8}{$\prod_n U_n$ is topologically complete.}
    \begin{proof}
        \step{a}{Each $U_n$ is topologically complete.}
        \begin{proof}
            \pf\ Proposition \ref{proposition:open_subspace_topologically_complete}.
        \end{proof}
        \qedstep
        \begin{proof}
            \pf\ Proposition \ref{proposition:product_topologically_complete}.
        \end{proof}
    \end{proof}
    \step{9}{$A$ is topologically complete.}
    \begin{proof}
        \pf\ Proposition \ref{proposition:closed_subspace_topologically_complete}.
    \end{proof}
\qed
\end{proof}

\begin{corollary}
    The space $\RR - \QQ$ of irrationals is topologically complete.
\end{corollary}

\section{Peano Spaces}

\begin{definition}[Peano Space]
    A \emph{Peano space} is a Hausdorff space that is the continuous image of $I$.
\end{definition}

\begin{theorem}
    $I^2$ is a Peano space.
\end{theorem}

\begin{proof}
    \pf
    \step{1}{Define the sequence of functions $f_n : I \rightarrow I^2$ as follows. $f_0$ is the path consisting of two line segments
    from $(0,0)$ to $(1/2,1/2)$ and from $(1/2,1/2)$ to $(1,0)$. $f_{n+1}$ is the path obtained from $f_n$ by replacing every triangular
    segment with four triangular segments (see p. 273 in Munkres).}
    \step{2}{\pflet{$d$ be the square metric on $\RR^2$}}
    \step{3}{\pflet{$\rho$ be the sup metric on $\mathcal{C}(I,I^2)$}}
    \step{4}{$(f_n)$ is a Cauchy sequence with respect to $\rho$}
    \begin{proof}
        \pf\ Since $\rho(f_n,f_{n+1}) \leq 1/2^n$
    \end{proof}
    \step{5}{\pflet{$f : I \rightarrow I^2$ be the limit of $(f_n)$}}
    \step{6}{$f$ is surjective.}
    \begin{proof}
        \step{a}{\pflet{$x \in I^2$}}
        \step{b}{For all $\epsilon > 0$, every $\epsilon$-neighbourhood of $x$ intersects $f(I)$}
        \begin{proof}
            \pf\ Since $f_n$ passes through every small square of side $1/2^n$.
        \end{proof}
        \step{c}{$x \in f(I)$}
        \begin{proof}
            \pf\ Since $f(I)$ is compact, hence closed.
        \end{proof}
    \end{proof}
    \qed
\end{proof}

\begin{corollary}
    For all $n \geq 1$, the space $I^n$ is a Peano space.
\end{corollary}

\begin{proposition}
    There exists a continuous surjective map $\RR \rightarrow \RR^2$.
\end{proposition}

\begin{proof}
    \pf\ Concatenate together a bunch of space-filling curves. \qed
\end{proof}

\begin{corollary}
    For $n \geq 1$, there exists a continuous surjective map $\RR \rightarrow \RR^n$.
\end{corollary}

\begin{proposition}
    Every Peano space is compact.
\end{proposition}

\begin{proof}
    \pf\ Theorem \ref{theorem:continuous_image_compact}. \qed
\end{proof}

\begin{proposition}
    Every Peano space is connected.
\end{proposition}

\begin{proof}
    \pf\ Theorem \ref{theorem:connected_continuous_image}. \qed
\end{proof}

\begin{proposition}
    Every Peano space is locally connected.
\end{proposition}

\begin{proof}
    \pf
    \step{1}{\pflet{$X$ be a Peano space.}}
    \step{2}{\pflet{$f : I \twoheadrightarrow X$ be continuous and surjective.}}
    \step{3}{\pflet{$y \in X$}}
    \step{4}{\pflet{$V$ be a neighbourhood of $y$}}
    \step{5}{\pick\ $x \in X$ such that $f(x) = y$}
    \step{6}{\pick\ a connected open neighbourhood $U$ of $x$ such that $U \subseteq \inv{f}(V)$}
    \step{7}{$f(U)$ is a connected open neighbourhood of $y$ that is included in $V$.}
    \qed
\end{proof}

\begin{proposition}
    Every Peano space is metrizable.
\end{proposition}

\begin{proof}
    \pf
    \step{1}{\pflet{$X$ be a Peano space.}}
    \step{2}{\pflet{$f : I \twoheadrightarrow X$ be continuous and surjective.}}
    \step{3}{$f$ is a perfect map.} % TODO Extract lemma
    \begin{proof}
        \step{a}{$f$ is a closed map.}
        \begin{proof}
            \pf\ Proposition \ref{proposition:closed_map_compact_Hausdorff}.
        \end{proof}
        \step{b}{For all $y \in X$ we have $\inv{f}(y)$ is compact.}
        \begin{proof}
            \pf\ It is a closed subspace of a compact space.
        \end{proof}
    \end{proof}
    \step{4}{$X$ is regular.}
    \begin{proof}
        \pf\ Proposition \ref{proposition:perfect_image_regular}.
    \end{proof}
    \step{5}{$X$ is second countable.}
    \begin{proof}
        \pf\ Proposition \ref{proposition:perfect_image_second_countable}.
    \end{proof}
    \step{6}{$X$ is metrizable.}
    \begin{proof}
        \pf\ Urysohn Metrization Theorem.
    \end{proof}
    \qed
\end{proof}

\begin{theorem}[Hahn-Mazurkiewicz Theorem]
    Every compact, connected, locally connected, metrizable space is a Peano space.
\end{theorem}

\begin{proof}
    \pf\ See J. G. Hocking and G. S. Young. \emph{Topology}. 1961. p. 129. \qed
\end{proof}

\begin{corollary}
    The space $I^\omega$ is a Peano space.
\end{corollary}

\section{Totally Bounded Metric Spaces}

\begin{definition}[Totally Bounded]
    A metric space $X$ is \emph{totally bounded} if and only if, for every $\epsilon > 0$, there exists a finite covering of $X$
    by $\epsilon$-balls.
\end{definition}

\begin{proposition}
    Every totally bounded metric space is bounded.
\end{proposition}

\begin{proof}
    \pf
    \step{1}{\pflet{$X$ be a totally bounded metric space.}}
    \step{2}{\pick\ $x_1$, \ldots, $x_n$ such that $B(x_1,1)$, \ldots, $B(x_n,1)$ cover $X$.}
    \step{3}{$\diam X \leq \max_{i,j} d(x_i,x_j) + 2$}
    \qed
\end{proof}

The following example shows that not every totally bounded space is bounded.

\begin{proposition}
    The real line $\RR$ is totally bounded under the metric $d(a,b) = \min(1,|a-b|)$.
\end{proposition}

\begin{theorem}[Choice]
    \label{theorem:compact_complete_totally_bounded}
    A metric space is compact if and only if it is complete and totally bounded.
\end{theorem}

\begin{proof}
    \pf
    \step{1}{Every compact metric space is complete.}
    \begin{proof}
        \pf\ Corollary \ref{corollary:compact_complete}.
    \end{proof}
    \step{2}{Every compact metric space is totally bounded.}
    \begin{proof}
        \pf\ Trivial.
    \end{proof}
    \step{3}{Every complete, totally bounded metric space is compact.}
    \begin{proof}
        \step{a}{\pflet{$X$ be a complete, totally bounded metric space.}}
        \step{b}{\pflet{$(x_n)$ be a sequence in $X$.}}
        \step{c}{\pick\ a sequence of infinite sets $\ZZ^+ \supseteq J_1 \supseteq J_2 \supseteq \cdots$
        such that, for each $k$, there exists an open ball $B$ of radius $1/k$ such that $\forall n \in J_k. x_n \in B$}
        \begin{proof}
            \step{i}{\assume{as induction hypothesis we have chosen $J_1$, \ldots, $J_k$}}
            \step{ii}{\pick\ an open ball $B$ of radius $1/(k+1)$ such that, for infinitely many $n \in J_k$, we have $x_n \in B$}
            \begin{proof}
                \pf\ Such a ball must exist since $X$ can be covered by finitely many open balls of radius $1 / (k+1)$.
            \end{proof}
            \step{iii}{\pflet{$J_{k+1} = \{ n \in J_k \mid x_n \in B \}$}}
        \end{proof}
        \step{d}{\pick\ a sequence of integers $n_1 < n_2 < \cdots$ such that $\forall r. n_r \in J_r$.}
        \begin{proof}
            \pf\ This is possible since each $J_r$ is infinite.
        \end{proof}
        \step{e}{$(x_{n_r})$ is a Cauchy sequence.}
        \begin{proof}
            \pf\ For $i,j \geq k$ we have $d(x_{n_i},x_{n_j}) \leq 2/k$.
        \end{proof}
    \end{proof}
    \qed
\end{proof}

\begin{proposition}[Choice]
    Let $(X_n)$ be a sequence of totally bounded metric spaces. Then $\prod_n X_n$ is totally bounded under the metric
    $D(x,y) = \sup_n \overline{d}(x_n,y_n)/n$.
\end{proposition}

\begin{proof}
    \pf
    \step{1}{\pflet{$\epsilon > 0$}}
    \step{2}{\pick\ $N$ such that $1/N \leq \epsilon$}
    \step{3}{\pick\ $K$ such that each of $X_1$, \ldots, $X_N$ can be covered by $K$ $\epsilon$-balls.}
    \step{3}{For $1 \leq i \leq N$, \pick\ a covering $B(x_{i1},i \epsilon)$, \ldots, $B(x_{iK},\epsilon)$ of $X_i$ by $i \epsilon$-balls.}
    \step{4}{\pick\ $x_i \in X_i$ for $i > N$}
    \step{4}{For $\alpha : \{ 1, \ldots, N \} \rightarrow \{1, \ldots, K \}$, \pflet{$x_\alpha$ be the point with $(x_\alpha)_i = x_{i\alpha(i)}$
    for $i \leq N$, and $(x_\alpha)_i = x_i$ for $i > N$} \prove{$\{B(x_\alpha,\epsilon)\}_\alpha$ covers $\prod_n X_n$}}
    \step{5}{\pflet{$y \in \prod_n X_n$}}
    \step{6}{\pick\ $\alpha$ such that for $1 \leq i \leq N$, $y_i \in B(x_{i\alpha_i}, i \epsilon)$}
    \step{7}{$y \in B(x_\alpha, \epsilon)$}
    \qed
\end{proof}

\section{Equicontinuity}

\begin{definition}[Equicontinuous]
    Let $X$ be a topological space. Let $Y$ be a metric space. Let $\mathcal{F}$ be a set of continuous functions from $X$ to $Y$.
    Let $x_0 \in X$. Then $\mathcal{F}$ is \emph{equicontinuous at $x_0$} if and only if, for all $\epsilon > 0$, there exists a
    neighbourhood $U$ of $x_0$ such that, for all $x \in U$ and $f \in \mathcal{F}$,
    \[ d(f(x),f(x_0)) < \epsilon \enspace . \]

    The set $\mathcal{F}$ is \emph{equicontinuous} if and only if it is equicontinuous at $x_0$ for all $x_0 \in X$.
\end{definition}

\begin{lemma}
    \label{lemma:totally_bounded_equicontinuous}
    Let $X$ be a topological space. Let $Y$ be a metric space. Let $\mathcal{F} \subseteq \mathcal{C}(X,Y)$. If $\mathcal{F}$
    is totally bounded under the uniform metric, then $\mathcal{F}$ is equicontinuous.
\end{lemma}

\begin{proof}
    \pf
    \step{1}{\pflet{$x_0 \in X$}}
    \step{2}{\pflet{$\epsilon > 0$}}
    \step{3}{\pick\ a finite set of $\epsilon/3$-balls $B(f_1,\epsilon/3)$, \ldots, $B(f_n,\epsilon/3)$ that cover $\mathcal{F}$.}
    \step{4}{\pick\ an open neighbourhood $U$ of $x_0$ such that, for all $x \in U$ and all $1 \leq i \leq n$, $d(f_i(x),f_i(x_0)) < \epsilon / 3$}
    \step{5}{\pflet{$x \in U$ and $f \in \mathcal{F}$}}
    \step{6}{\pick\ $i$ such that $f \in B(f_i,\epsilon)$}
    \step{7}{$d(f(x),f(x_0)) < \epsilon$}
    \begin{proof}
        \pf
        \begin{align*}
            d(f(x),f(x_0)) & \leq d(f(x),f_i(x)) + d(f_i(x),f_i(x_0)) + d(f_i(x_0),f(x_0)) \\
            & < \epsilon / 3 + \epsilon / 3 + \epsilon / 3 \\
            & = \epsilon
        \end{align*}
    \end{proof}
    \qed
\end{proof}

\begin{lemma}[Choice]
    \label{lemma:equicontinuous_totally_bounded}
    Let $X$ be a compact space. Let $Y$ be a compact metric space. Let $\mathcal{F} \subseteq \mathcal{C}(X,Y)$. If $\mathcal{F}$
    is equicontinuous then $\mathcal{F}$ is totally bounded under the sup metric.
\end{lemma}

\begin{proof}
    \pf
    \step{1}{\pflet{$\rho$ be the sup metric.}}
    \step{2}{\assume{$\mathcal{F}$ is equicontinuous.}}
    \step{3}{\pflet{$\epsilon > 0$}}
    \step{4}{\pflet{$\delta = \epsilon / 3$}}
    \step{5}{For $a \in X$, \pick\ an open neighbourhood $U_a$ of $a$ such that $\forall x \in U_a. \forall f \in \mathcal{F}. d(f(x),f(a)) < \delta$}
    \step{6}{\pick\ finitely many of these sets $U_{a_1}$, \ldots, $U_{a_k}$ that covers $X$.}
    \step{7}{\pick\ a finite open cover $V_1$, \ldots, $V_m$ of $Y$ by sets of diameter $< \delta$.}
    \step{8}{\pflet{$J$ be the set of all functions $\{1, \ldots, m \} \rightarrow \{1, \ldots, k\}$}}
    \step{9}{For all $\alpha \in J$ such that there exsits $f \in \mathcal{F}$ such that $\forall i. f(a_i) \in V_{\alpha(i)}$,
    \pick\ $f_\alpha \in \mathcal{F}$ such that $\forall i. f_\alpha(a_i) \in V_{\alpha(i)}$.
    \prove{The open balls $B(f_\alpha,\epsilon)$ cover $\mathcal{F}$.}}
    \step{10}{\pflet{$f \in \mathcal{F}$}}
    \step{11}{\pick\ $\alpha$ such that $\forall i. f(a_i) \in V_{\alpha(i)}$ \prove{$f \in B(f_\alpha, \epsilon)$}}
    \step{12}{\pflet{$x \in X$}}
    \step{13}{\pick\ $i$ such that $x \in U_i$}
    \step{14}{$d(f(x),f_\alpha(x)) < \epsilon$}
    \begin{proof}
        \pf
        \begin{align*}
            d(f(x),f_\alpha(x)) & \leq d(f(x),f(a_i)) + d(f(a_i),f_\alpha(a_i)) + d(f_\alpha(a_i),f_\alpha(x)) \\
            & < \delta + \delta + \delta \\
            & = \epsilon
        \end{align*}
    \end{proof}
    \qed
\end{proof}

\begin{corollary}
    \label{corollary:equicontinuous_totally_bounded}
    Let $X$ be a compact space. Let $Y$ be a compact metric space. Let $\mathcal{F} \subseteq \mathcal{C}(X,Y)$. If $\mathcal{F}$
    is equicontinuous then $\mathcal{F}$ is totally bounded under the uniform metric.
\end{corollary}

\begin{proposition}
    Every finite set of functions is equicontinuous.
\end{proposition}

\begin{proposition}
    If $(f_n)$ is a sequence of functions in $\mathcal{C}(X,Y)$ that converges uniformly then $\{ f_n \mid n \geq 1 \}$
    is equicontinuous.
\end{proposition}

\begin{proof}
    \pf
    \step{0}{\pflet{$f = \lim_{n \rightarrow \infty} f_n$}}
    \step{1}{\pflet{$x_0 \in X$}}
    \step{2}{\pflet{$\epsilon > 0$}}
    \step{3}{\pick\ $N$ such that $\forall n \geq N. \forall x \in X. d(f_n(x),f(x)) < \epsilon$}
    \step{4}{\pick\ an open neighbourhood $U$ of $x_0$ such that $\forall i \leq N. \forall x \in U. d(f_i(x),f(x)) < \epsilon$}
    \step{5}{$\forall n. \forall x \in U. d(f_n(x),f(x)) < \epsilon$}
    \qed
\end{proof}

\begin{proposition}
    Let $\mathcal{F}$ be a set of differentiable functions $f : \RR \rightarrow \RR$ such that, for all $x \in \RR$,
    there exists an open neighbourhood $U$ of $x$ such that $\{ f' \mid f \in \mathcal{F} \}$ is uniformly bounded on $U$.
    Then $\mathcal{F}$ is equicontinuous.
\end{proposition}

\begin{proof}
    \pf
    \step{1}{\pflet{$x_0 \in X$}}
    \step{2}{\pflet{$\epsilon > 0$}}
    \step{3}{\pick\ an open neighbourhood $U$ of $x_0$ such that $\{ f' \mid f \in \mathcal{f} \}$ is uniformly bounded on $U$.}
    \step{4}{\pick\ $M$ such that $\forall f \in \mathcal{F}. \forall x \in U. |f'(x)| \leq M$}
    \step{5}{\pick\ $\delta > 0$ such that $\delta < \epsilon / M$ and $(x_0 - \delta, x_0 + \delta) \subseteq U$}
    \step{6}{\pflet{$f \in \mathcal{F}$}}
    \step{7}{\pflet{$x \in (x_0 - \delta, x_0 + \delta)$} \prove{$|f(x_0) - f(x)| \leq \epsilon$}}
    \step{8}{\assume{w.l.o.g. $x \neq x_0$}}
    \step{9}{\pick\ $a$ between $x$ and $x_0$ such that $f'(a) = (f(x)-f(x_0))/(x - x_0)$}
    \step{10}{$|f'(a)| \leq M$}
    \step{11}{}
    \begin{proof}
        \pf
        \begin{align*}
            |f(x_0)-f(x)| & = |f'(a)||x-x0| \\
            & < M \delta \\
            & \leq \epsilon
        \end{align*}
    \end{proof}
    \qed
\end{proof}

\section{Pointwise Bounded Sets}

\begin{definition}[Pointwise Bounded]
    Let $X$ be a set. Let $Y$ be a metric space. Let $\mathcal{F} \subseteq Y^X$. Then $\mathcal{F}$ is \emph{pointwise bounded}
    if and only if, for all $x \in X$, the set $\{ f(x) \mid f \in \mathcal{F} \}$ is bounded.
\end{definition}

\begin{theorem}[Ascoli's Theorem, Classical Version]
    Let $X$ be a compact space. Let $Y$ be a metric space in which closed bounded subspaces are compact. 
    Give $\mathcal{C}(X, Y)$ the uniform topology.
    Let $\mathcal{F} \subseteq \mathcal{C}(X, Y)$. Then $\overline{\mathcal{F}}$ is compact if and only if
    $\mathcal{F}$ is equicontinuous and pointwise bounded.
\end{theorem}

\begin{proof}
    \pf
    \step{1}{\pflet{$\rho$ be the sup metric on $\mathcal{C}(X, Y)$}}
    \step{2}{\pflet{$\mathcal{G} = \overline{\mathcal{F}}$}}
    \step{3}{If $\mathcal{G}$ is compact then $\mathcal{G}$ is equicontinuous.}
    \begin{proof}
        \step{a}{\assume{$\mathcal{G}$ is compact.}}
        \step{b}{$\mathcal{G}$ is totally bounded under $\overline{\rho}$.}
        \begin{proof}
            \pf\ Theorem \ref{theorem:compact_complete_totally_bounded}.
        \end{proof}
        \step{c}{$\mathcal{G}$ is equicontinuous.}
        \begin{proof}
            \pf\ \ref{lemma:totally_bounded_equicontinuous}.
        \end{proof}
    \end{proof}
    \step{4}{If $\mathcal{G}$ is compact then $\mathcal{G}$ is pointwise bounded.}
    \begin{proof}
        \step{a}{\assume{$\mathcal{G}$ is compact.}}
        \step{b}{$\mathcal{G}$ is bounded under $\rho$.}
        \begin{proof}
            \pf\ \ref{proposition:compact_bounded}.
        \end{proof}
        \step{c}{$\mathcal{G}$ is pointwise bounded.}
    \end{proof}
    \step{5}{If $\mathcal{F}$ is equicontinuous and pointwise bounded then $\mathcal{G}$ is equicontinuous.}
    \begin{proof}
        \step{a}{\assume{$\mathcal{F}$ is equicontinuous and pointwise bounded.}}
        \step{b}{\pflet{$x_0 \in X$}}
        \step{c}{\pflet{$\epsilon > 0$}}
        \step{d}{\pick\ an open neighbourhood $U$ of $x_0$ such that $\forall x \in U. \forall f \in \mathcal{F}. d(f(x),f(x_0)) < \epsilon / 3$}
        \step{e}{\pflet{$x \in U$}}
        \step{f}{\pflet{$g \in \mathcal{G}$}}
        \step{g}{\pick\ $f \in \mathcal{F}$ such that $\rho(f,g) < \epsilon / 3$}
        \step{h}{$d(g(x),g(x_0)) < \epsilon$}
        \begin{proof}
            \pf
            \begin{align*}
                d(g(x),g(x_0)) & \leq d(g(x),f(x)) + d(f(x),f(x_0)) + d(f(x_0),g(x_0)) \\
                & < \epsilon /  3 + \epsilon / 3 + \epsilon / 3 \\
                & = \epsilon
            \end{align*}
        \end{proof}
    \end{proof}
    \step{6}{If $\mathcal{F}$ is equicontinuous and pointwise bounded then $\mathcal{G}$ is pointwise bounded.}
    \begin{proof}
        \step{a}{\assume{$\mathcal{F}$ is equicontinuous and pointwise bounded.}}
        \step{b}{\pflet{$a \in X$}}
        \step{c}{\pflet{$M = \diam \{ f(a) \mid f \in \mathcal{F} \}$}}
        \step{d}{\pflet{$g, g' \in \mathcal{G}$}}
        \step{e}{\pick\ $f, f' \in \mathcal{F}$ such that $\rho(f,g) < 1$ and $\rho(f',g') < 1$}
        \step{f}{$d(g(a),g'(a)) < M + 2$}
        \begin{proof}
            \pf
            \begin{align*}
                d(g(a),g'(a)) & \leq d(g(a),f(a)) + d(f(a),f'(a)) + d(f'(a).g'(a)) \\
                & < 1 + M + 1 \\
                & = M + 2
            \end{align*}
        \end{proof}
    \end{proof}
    \step{7}{If $\mathcal{G}$ is equicontinuous and pointwise bounded then there exists a compact $Y \subseteq \RR^n$
    such that $\forall g \in \mathcal{G}. g(X) \subseteq Y$}
    \begin{proof}
        \step{a}{\assume{$\mathcal{G}$ is equicontinuous and pointwise bounded.}}
        \step{b}{For $a \in X$, \pick\ an open neighbourhood $U_a$ of $a$ such that $\forall x \in U_a. \forall g \in \mathcal{G}, d(g(a),g(x)) < 1$}
        \step{c}{\pick\ finitely many of these sets $U_{a_1}$, \ldots, $U_{a_n}$ that cover $X$.}
        \step{d}{\pick\ $N$ such that $U_{a_1} \cup \cdots \cup U_{a_n} \subseteq B(0,N)$}
        \step{e}{$\forall g \in \mathcal{g}, g(X) \subseteq B(0,N+1)$}
        \step{f}{$\forall g \in \mathcal{g}, g(X) \subseteq \overline{B(0,N+1)}$}
    \end{proof}
    \step{8}{If $\mathcal{F}$ is equicontinuous and pointwise bounded then $\mathcal{G}$ is compact.}
    \begin{proof}
        \step{a}{\assume{$\mathcal{F}$ is equicontinuous and pointwise bounded.}}
        \step{b}{$\mathcal{G}$ is complete under $\rho$.}
        \begin{proof}
            \pf\ It is a closed subspace of the complete metric space $\mathcal{C}(X, Y)$ under $\rho$.
        \end{proof}
        \step{c}{$\mathcal{G}$ is totally bounded under $\rho$.}
        \begin{proof}
            \step{i}{$\mathcal{G}$ is equicontinuous.}
            \begin{proof}
                \pf\ \stepref{5}
            \end{proof}
            \step{ii}{$\mathcal{G}$ is pointwise bounded.}
            \begin{proof}
                \pf\ \stepref{6}
            \end{proof}
            \step{iii}{There exists a compact subspace $Y$ of $\RR^n$ such that $\mathcal{G} \subseteq \mathcal{C}(X,Y)$}
            \begin{proof}
                \pf\ \stepref{7}
            \end{proof}
            \step{iv}{$\mathcal{G}$ is totally bounded under $\rho$}
            \begin{proof}
                \pf\ Lemma \ref{lemma:equicontinuous_totally_bounded}.
            \end{proof}
        \end{proof}
        \step{d}{$\mathcal{G}$ is compact.}
        \begin{proof}
            \pf\ Theorem \ref{theorem:compact_complete_totally_bounded}.
        \end{proof}
    \end{proof}
    \qed
\end{proof}

\begin{corollary}
    Let $X$ be a compact space. Let $Y$ be a metric space in which closed bounded subspaces are compact.
    Give $\mathcal{C}(X, Y)$ the uniform topology.
    Let $\mathcal{F} \subseteq \mathcal{C}(X, Y)$. Then $\mathcal{F}$ is compact if and only if it is closed, bounded under the
    sup metric, and equicontinuous.
\end{corollary}

\begin{theorem}[Arzaela's Theorem]
    Let $X$ be a compact space. Let $(f_n)$ be a sequence of continuous functions $X \rightarrow \RR^n$. If $\{f_n \mid n \geq 1 \}$
    is pointwise bounded and equicontinuous, then $(f_n)$ has a uniformly convergent subsequence.
\end{theorem}

\begin{proof}
    \pf\ By Ascoli's Theorem, $\overline\{f_n \mid n \geq 1 \}$ under the uniform topology is compact hence sequentially compact. \qed
\end{proof}

\section{Vanishing at Infinity}

\begin{definition}[Vanish at Infinity]
    Let $X$ be a topological space. Let $f : X \rightarrow \RR$ be continuous. Then $f$ \emph{vanishes at infinity}
    if and only if, for every $\epsilon > 0$, there exists a compact $C \subseteq X$ such that $\forall x \in X - C. |f(x)| < \epsilon$.

    We write $\mathcal{C}_0(X, \RR)$ for the set of continuous functions that vanish at infinity.
\end{definition}

\begin{definition}[Vanish Uniformly at Infinity]
    Let $X$ be a topological space. Let $\mathcal{F} \subseteq \mathcal{C}(X,\RR)$. Then $\mathcal{F}$ \emph{vanishes uniformly at infinity}
    if and only if, for every $\epsilon > 0$, there exists a compact $C \subseteq X$ such that $\forall x \in X - C. \forall f \in \mathcal{F}.
    |f(x)| < \epsilon$.
\end{definition}

\begin{theorem}
    Let $X$ be a locally compact Hausdorff space. Give $\mathcal{C}_0(X, \RR)$ the uniform topology. Let $\mathcal{F} \subseteq
    \mathcal{C}_0(X, \RR)$. Then $\overline{\mathcal{F}}$ is compact if and only if $\mathcal{F}$ is pointwise bounded,
    equicontinuous, and vanishes uniformly at infinity.
\end{theorem}

\begin{proof}
    \pf
    \step{1}{\pflet{$Y$ be the one-point compactification of $X$.}}
    \step{2}{Give $\mathcal{C_0}(X,\RR)$ and $\mathcal{C}(Y,\RR)$ the sup metric.}
    \step{3}{Define $i : \mathcal{C_0}(X, \RR) \rightarrow \mathcal{C}(Y, \RR)$ by: $i(f)(x) = f(x)$ for $x \in X$,
    $i(f)(\infty) = 0$.}
    \begin{proof}
        \step{a}{\pflet{$f \in \mathcal{C_0}(X, \RR)$}}
        \step{b}{For all $x \in X$, $i(f)$ is continuous at $x$.}
        \begin{proof}
            \step{i}{\pflet{$x \in X$}}
            \step{ii}{\pflet{$\epsilon > 0$}}
            \step{iii}{\pick\ an open neighbourhood $U$ of $x$ in $X$ such that $f(U) \subseteq (f(x) - \epsilon, f(x) + \epsilon)$}
            \step{iv}{$U$ is an open neighbourhood of $x$ in $Y$ such that $i(f)(U) \subseteq (f(x) - \epsilon, f(x) + \epsilon)$}
        \end{proof}
        \step{c}{$i(f)$ is continuous at $\infty$.}
        \begin{proof}
            \step{i}{\pflet{$\epsilon > 0$}}
            \step{ii}{\pick\ a compact $C \subseteq X$ such that $\forall x \in X - C. |f(x)| < \epsilon$}
            \step{iii}{$Y - C$ is an open neighbourhood of $\infty$ such that $i(f)(Y - C) \subseteq (-\epsilon,\epsilon)$}
        \end{proof}
    \end{proof}
    \step{4}{$i$ is an isometric embedding.}
    \begin{proof}
        \step{a}{\pflet{$f, g \in \mathcal{C}_0(X, \RR)$}}
        \step{b}{$\rho(i(f),i(g)) = \rho(f,g)$}
        \begin{proof}
            \pf
            \begin{align*}
            \rho(i(f),i(g)) & = \sup_{y \in Y} |i(f)(y) - i(g)(y)| \\
            & = \max(\sup_{x \in X} |f(x) - g(x)|, |i(f)(\infty) - i(g)(\infty)|) \\
            & = \max(\rho(f,g),0) \\
            & = \rho(f,g)
            \end{align*}
        \end{proof}
    \end{proof}
    \step{5}{$\ran i$ is closed.}
    \begin{proof}
        \step{a}{\pflet{$(f_n)$ be a sequence of functions in $\mathcal{C}_0(X,\RR)$ and $i(f_n) \rightarrow g$
        in $\mathcal{C}(Y, \RR)$} \prove{$g \in i(\mathcal{C}_0(X, \RR))$}}
        \step{b}{\pflet{$f = g \restriction X$}}
        \step{c}{$g(\infty) = 0$}
        \begin{proof}
            \pf $g(\infty) = \lim_{n \rightarrow \infty} f_n(\infty) = 0$ by
            Propostion \ref{proposition:evaluation_continuous}.
        \end{proof}
        \step{d}{$f \in \mathcal{C}_0(X, \RR)$}
        \begin{proof}
            \step{i}{\pflet{$\epsilon > 0$}}
            \step{ii}{\pick\ $C \subseteq X$ compact such that
            $Y - C \subseteq \inv{g}((-\epsilon, \epsilon))$}
            \step{iii}{$\forall x \in X - C. |f(x)| < \epsilon$}
        \end{proof}
        \step{e}{$i(f) = g$}
    \end{proof}
    \step{6}{$i(\mathcal{F})$ is pointwise bounded if and only if $\mathcal{F}$ is pointwise bounded.}
    \step{7}{For $x \in X$, we have $i(\mathcal{F})$ is equicontinuous at $x$ if and only if $\mathcal{F}$ is equicontinuous at $x$.}
    \step{8}{$i(\mathcal{F})$ is equicontinuous at $\infty$ if and only if $\mathcal{F}$ vanishes uniformly at infinity.}
    \step{9}{$\overline{\mathcal{F}}$ is compact if and only if $\mathcal{F}$ is pointwise bounded,
    equicontinuous, and vanishes uniformly at infinity.}
    \begin{proof}
        \pf\ By Ascoli's Theorem.
    \end{proof}
    \qed
\end{proof}

\section{Hausdorff Metric}

\begin{definition}
    Let $X$ be a metric space. Let $\mathcal{H}$ be the set of all nonempty closed bounded subsets of $X$. The \emph{Hausdorff metric}
    $D : \mathcal{H}^2 \rightarrow \RR$ is defined by: $D(A,B) = \inf \{ \epsilon \mid A \subseteq U(B, \epsilon) \text{ and } B \subseteq U(A, \epsilon) \}$.

    We prove this is a metric.
\end{definition}

\begin{proof}
    \pf
    \step{0}{For all $A, B \in \mathcal{H}$, there exists $\epsilon$ such that $A \in U(B, \epsilon)$.}
    \begin{proof}
        \step{a}{\pick\ $a \in A$ and $b \in B$}
        \step{b}{\pflet{$\epsilon = \diam A + d(a,b)$}}
        \step{c}{\pflet{$x \in A$}}
        \step{d}{$d(x,b) \leq \epsilon$}
        \step{e}{$d(x,B) \leq \epsilon$}
        \step{f}{$x \in U(B, \epsilon + 1)$}
    \end{proof}
    \step{1}{$\forall A,B \in \mathcal{H}. D(A,B) \geq 0$}
    \begin{proof}
        \pf\ Trivial.
    \end{proof}
    \step{2}{$\forall A,B \in \mathcal{H}. D(A,B) = 0 \Rightarrow A = B$}
    \begin{proof}
        \step{a}{\pflet{$A, B \in \mathcal{H}$}}
        \step{b}{\assume{$D(A,B) = 0$}}
        \step{c}{$A \subseteq B$}
        \begin{proof}
            \step{i}{\pflet{$x \in A$}}
            \step{ii}{\pflet{$\epsilon > 0$} \prove{$B(x,\epsilon)$ intersects $B$}}
            \step{iii}{\pick\ $\delta < \epsilon$ such that $A \subseteq U(B, \delta)$}
            \step{iv}{$x \in U(B, \delta)$}
            \step{v}{$d(x,B) < \delta$}
            \step{vi}{\pick\ $b \in B$ such that $d(x,b) < \delta$}
            \step{vii}{$b \in B(x, \epsilon) \cap B$}
        \end{proof}
        \step{d}{$B \subseteq A$}
        \begin{proof}
            \pf\ Similar.
        \end{proof}
    \end{proof}
    \step{3}{$\forall A, B \in \mathcal{H}. D(A,B) = D(A,B)$}
    \begin{proof}
        \pf\ Trivial.
    \end{proof}
    \step{4}{$\forall A, B, C \in \mathcal{H}. D(A,C) \leq D(A,B) + D(B,C)$}
    \begin{proof}
        \step{a}{\pflet{$A, B, C \in \mathcal{H}$}}
        \step{b}{For all $\delta > D(A,B)$ and $\epsilon > D(B,C)$, we have $A \subseteq U(C, \delta + \epsilon)$ and
        $C \subseteq U(A, \delta + \epsilon)$.}
        \begin{proof}
            \step{i}{\pflet{$\delta > D(A,B)$}}
            \step{ii}{\pflet{$\epsilon > D(B,C)$}}
            \step{iii}{\pflet{$a \in A$}}
            \step{iv}{$a \in U(B, \delta)$}
            \step{v}{\pick\ $b \in B$ such that $d(a,b) < \delta$}
            \step{vi}{$b \in U(C, \epsilon)$}
            \step{vii}{\pick\ $c \in C$ such that $d(b,c) < \epsilon$}
            \step{viii}{$d(a,c) < \delta + \epsilon$}
        \end{proof}
    \end{proof}
    \qed
\end{proof}

\begin{proposition}[Choice]
    \label{proposition:Hausdorff_metric_complete}
    Let $X$ be a complete metric space. Then the set $\mathcal{H}$ of all nonempty closed bounded subsets of $X$ under the Hausdorff metric
    is complete.
\end{proposition}

\begin{proof}
    \pf
    \step{1}{\pflet{$(A_n)$ be a Cauchy sequence in $\mathcal{H}$}}
    \step{2}{\assume{w.l.o.g. $D(A_n,A_{n+1}) < 1/2^n$ for all $n$.}}
    \step{3}{\pflet{$A$ be the set of all points $l$ that are the limit of some sequence $(x_n)$ in $X$
    with $x_n \in A_n$ for all $n$.}}
    \step{4}{$\overline{A} \in \mathcal{H}$}
    \begin{proof}
        \step{a}{$A$ is nonempty.}
        \begin{proof}
            \step{i}{\pick\ a sequence $(a_n)$ such that $\forall n. a_n \in A_n$}
            \step{ii}{$(a_n)$ is Cauchy.}
            \begin{proof}
                \pf\ For all $n$ we have $d(a_n,a_{n+1}) < 1/2^n$.
            \end{proof}
            \step{iii}{$\lim_{n \rightarrow \infty} a_n \in A$}
        \end{proof}
        \step{bb}{$A$ is bounded.}
        \begin{proof}
            \step{i}{\pflet{$x, y \in A$}}
            \step{ii}{\pick\ sequences $(x_n)$, $(y_n)$ with $x_n, y_n \in A_n$
            such that $x_n \rightarrow x$ and $y_n \rightarrow y$}
            \step{iii}{$d(x,y) < \diam A_1 + 2$}
            \begin{proof}
                \pf
                \begin{align*}
                    d(x,y) & \leq d(x,x_1) + d(x_1, y_1) + d(y_1,y) \\
                    & < 1 + \diam A_1 + 1 \\
                    & = \diam A_1 + 2
                \end{align*}
            \end{proof}
        \end{proof}
        \step{b}{$\overline{A}$ is bounded.} % TODO Extract lemma
        \begin{proof}
            \step{i}{\pflet{$x,y \in \overline{A}$}}
            \step{ii}{\pick\ $a \in A \cap B(x,1)$}
            \step{iii}{\pick\ $b \in A \cap B(y,1)$}
            \step{iv}{$d(a,b) \leq \diam A$}
            \step{v}{$d(x,y) \leq \diam A + 2$}
        \end{proof}
    \end{proof}
    \step{5}{$A_n \rightarrow \overline{A}$ as $n \rightarrow \infty$}
    \begin{proof}
        \step{a}{\pflet{$\epsilon > 0$}}
        \step{b}{\pick\ $N \geq 3$ such that $1/2^{N-3} < \epsilon$}
        \step{c}{\pflet{$n \geq N$} \prove{$D(A_n, \overline{A}) \leq \epsilon / 2$}}
        \step{d}{$A_n \subseteq U(\overline{A}, \epsilon / 2)$}
        \begin{proof}
            \step{i}{\pflet{$x_n \in A_n$} \prove{$d(x, \overline{A}) < \epsilon / 2$}}
            \step{ii}{Extend $x_n$ to a sequence $(x_m)_m$ where $x_m \in X_m$ for all $m$}
            \step{iii}{$(x_m)$ is Cauchy}
            \step{iv}{\pflet{$l = \lim_{m \rightarrow \infty} x_m$}}
            \step{v}{$\forall m \geq n. d(x_n,x_m) < 1/2^{n-1}$}
            \step{vi}{$d(x_n,l) < 1/2^{n-1}$}
            \step{vii}{$d(x_n,l) < \epsilon / 2$}
        \end{proof}
        \step{e}{$\overline{A} \subseteq U(A_n, \epsilon / 2)$}
        \begin{proof}
            \step{i}{\pflet{$x \in \overline{A}$}}
            \step{ii}{\pick{$y \in A$ such that $d(x,y) < \epsilon / 4$}}
            \step{iii}{\pick\ a sequence $(x_m)$ with limit $y$ such that $\forall m. x_m \in A_m$}
            \step{iv}{$d(x,x_n) < \epsilon / 2$}
        \end{proof}
    \end{proof}
    \qed
\end{proof}

\begin{proposition}[Choice]
    \label{proposition:Hausdorff_metric_totally_bounded}
    Let $X$ be a totally bounded metric space. Then the set $\mathcal{H}$ of all nonempty closed bounded subsets of $X$ under the Hausdorff metric
    is totally bounded.
\end{proposition}

\begin{proof}
    \pf
    \step{1}{\pflet{$\epsilon > 0$}}
    \step{2}{\pflet{$\delta = \epsilon / 2$}}
    \step{3}{\pick\ $S$ a finite subset of $X$ such that $\{ B(s,\delta) \mid s \in S \}$ covers $X$.}
    \step{4}{\pflet{$\mathcal{A} = \pow S - \{ \emptyset \}$} \prove{$\{ B(A,\epsilon) \mid A \in \mathcal{A} \}$ covers $\mathcal{H}$}}
    \step{5}{$\mathcal{A} \subseteq \mathcal{H}$}
    \step{6}{\pflet{$B \in \mathcal{H}$}}
    \step{7}{\pflet{$A = U(B,\delta) \cap S$}}
    \step{8}{$A \in \mathcal{A}$}
    \begin{proof}
        \step{i}{\pick\ $b \in B$}
        \step{ii}{\pick\ $s \in S$ such thath $b \in B(s, \delta)$}
        \step{iii}{$s \in A$}
        \step{iv}{$A \neq \emptyset$}
    \end{proof}
    \step{9}{$A \subseteq U(B, \delta)$}
    \step{10}{$B \subseteq U(A, \delta)$}
    \begin{proof}
        \step{a}{\pflet{$x \in B$}}
        \step{b}{\pick\ $s \in S$ such that $x \in B(s, \delta)$}
        \step{c}{$s \in A$}
        \step{d}{$d(x,s) < \delta$}
    \end{proof}
    \step{11}{$B \in B(A,\epsilon)$}
    \qed
\end{proof}

\begin{theorem}
    Let $X$ be a compact metric space. Then the set $\mathcal{H}$ of all nonempty closed bounded subsets of $X$ under the Hausdorff metric
    is compact.
\end{theorem}

\begin{proof}
    \pf\ Theorem \ref{theorem:compact_complete_totally_bounded},
    Proposition \ref{proposition:Hausdorff_metric_complete},
    Proposition \ref{proposition:Hausdorff_metric_totally_bounded}.
    \qed
\end{proof}

\begin{proposition}
    Let $X$ and $Y$ be metric spaces. Give $X \times Y$ the square metric. Let $\mathcal{H}$ be the set of all nonempty closed bounded
    subsets of $X \times Y$ under the Hausdorff metric. Give $\mathcal{C}(X,Y)$ the uniform metric. Let $gr : \mathcal{C}(X,Y) \rightarrow
    \mathcal{H}$ be the function that maps any continuous function $f : X \rightarrow Y$ to its graph $\{ (x, f(x)) \mid x \in X \}$.
    Then $gr$ is uniformly continuous.
\end{proposition}

\begin{proof}
    \pf
    \step{1}{\pflet{$\epsilon > 0$}}
    \step{2}{\pflet{$\delta = \min(\epsilon / 3, 1)$}}
    \step{3}{\pflet{$f,g \in \mathcal{C}(X,Y)$}}
    \step{4}{\assume{$\overline{\rho}(f,g) < \delta$}}
    \step{5}{$D(gf(f),gf(g)) \leq \epsilon / 2$}
    \begin{proof}
        \step{a}{$gf(f) \subseteq U(gf(g),\epsilon / 2)$}
        \begin{proof}
            \step{i}{\pflet{$(x,f(x)) \in gf(f)$}}
            \step{ii}{$\sigma((x,f(x)),(x,g(x))) = d(f(x),g(x)) < \delta$}
            \step{iii}{$d((x,f(x)),gr(g)) \leq \delta < \epsilon / 2$}
        \end{proof}
        \step{b}{$gf(g) \subseteq U(gf(f),\epsilon / 2)$}
        \begin{proof}
            \pf\ Similar.
        \end{proof}
    \end{proof}
    \qed
\end{proof}

\begin{proposition}
    Let $X$ and $Y$ be metric spaces. Give $X \times Y$ the square metric. Let $\mathcal{H}$ be the set of all nonempty closed bounded
    subsets of $X \times Y$ under the Hausdorff metric. Give $\mathcal{C}(X,Y)$ the uniform metric. Let $gr : \mathcal{C}(X,Y) \rightarrow
    \mathcal{H}$ be the function that maps any continuous function $f : X \rightarrow Y$ to its graph $\{ (x, f(x)) \mid x \in X \}$.
    Let $f : X \rightarrow Y$ be uniformly continuous. Then $\inv{gr} : gr(\mathcal{C}(X,Y)) \rightarrow \mathcal{C}(X,Y)$
    is continuous at $gr(f)$.
\end{proposition}

\begin{proof}
    \pf
    \step{1}{\pflet{$\epsilon > 0$}}
    \step{2}{Pick $\delta$ such that $\delta < \epsilon / 4$ and $\forall x,y \in X. d(x,y) < \delta \Rightarrow d(f(x),g(y)) < \epsilon / 4$}
    \step{3}{\pflet{$g \in \mathcal{C}(X,Y)$ with $D(gr(f),gr(g)) < \delta$}}
    \step{4}{$\overline{\rho}(f,g) < \epsilon$}
    \begin{proof}
        \step{i}{\pflet{$x \in X$}}
        \step{ii}{$d((x,g(x)),gr(f)) < \delta$}
        \step{iii}{\pick\ $y \in X$ such that $\sigma((y,f(y)),(x,g(x))) < \delta$}
        \step{iv}{$d(x,y) < \delta$}
        \step{v}{$d(f(y),g(x)) < \delta$}
        \step{vi}{$d(f(x),g(x)) < \epsilon / 2$}
        \begin{proof}
            \pf
            \begin{align*}
                d(f(x),g(x)) & \leq d(f(x),f(y)) + d(f(y),g(x)) \\
                & < \epsilon / 4 + \epsilon / 4 \\
                & = \epsilon / 2
            \end{align*}
        \end{proof}
    \end{proof}
    \qed
\end{proof}

\begin{corollary}
    Let $X$ be a compact metric space and $Y$ a metric space. Give $X \times Y$ the square metric. Let $\mathcal{H}$ be the set of all nonempty closed bounded
    subsets of $X \times Y$ under the Hausdorff metric. Give $\mathcal{C}(X,Y)$ the uniform metric. Let $gr : \mathcal{C}(X,Y) \rightarrow
    \mathcal{H}$ be the function that maps any continuous function $f : X \rightarrow Y$ to its graph $\{ (x, f(x)) \mid x \in X \}$.
    Then $gr$ is an embedding.    
\end{corollary}

\begin{proof}
    \pf\ By the Uniform Continuity Theorem. \qed
\end{proof}

\begin{proposition}
    Let $X = (0,+\infty)$ and $Y = (0,+\infty)$. Let $f : X \rightarrow Y$ be the function $f(x) = 1 / x$.
    Then Then $\inv{gr} : gr(\mathcal{C}(X,Y)) \rightarrow \mathcal{C}(X,Y)$
    is not continuous at $gr(f)$.
\end{proposition}

\begin{proof}
    \pf
    \step{1}{\pflet{$\epsilon = 1$}}
    \step{2}{\pflet{$\delta > 0$}}
    \step{3}{Define $g : X \rightarrow Y$ by $g(x) = (x+\delta/3)^2$}
    \step{4}{$D(gr(f),gr(g)) < \delta$}
    \begin{proof}
        \step{a}{$gr(f) \subseteq U(gr(g),\delta/2)$}
        \begin{proof}
            \pf\ Since $d((x,x^2),(x+\delta/3,x^2)) = \delta/3$
        \end{proof}
        \step{b}{$gf(g) \subseteq U(gr(f),\delta/2)$}
        \begin{proof}
            \pf\ Similar.
        \end{proof}
    \end{proof}
    \step{5}{$\overline{\rho}(f,g) \geq \epsilon$}
    \begin{proof}
        \step{a}{\pflet{$x = 3\delta/2$}}
        \begin{align*}
            |f(x)-g(x)| & = 2x\delta/3+\delta^2/9 \\
            & = 1 + \delta^2/9 \\
            & > \epsilon
        \end{align*}
    \end{proof}
    \qed
\end{proof}

\section{Topology of Compact Convergence}

\begin{definition}[Topology of Compact Convergence]
    Let $X$ be a topological space and $Y$ a metric space. The \emph{topology of compact convergence} on $Y^X$ is the topology
    generated by the basis $\BB = \{ B_C(f, \epsilon) \mid C \subseteq X \text{ nonempty and compact}, f \in Y^X, \epsilon > 0 \}$
    where
    \[ B_C(f, \epsilon) = \{ g \in Y^X \mid \sup_{x \in C} d(f(x), g(x)) < \epsilon \} \enspace . \]

    We prove that this is a basis for a topology on $Y^X$.
\end{definition}

\begin{proof}
    \pf
    \step{0}{\pick\ $x_0 \in X$}
    \step{1}{For all $f \in Y^X$, we have $f \in B_{\{x_0\}}(f,1)$}
    \step{2}{For all $B_1, B_2 \in \BB$ and $f \in B_1 \cap B_2$, there exists $B_3 \in \BB$ such that $f \in B_3 \subseteq B_1 \cap B_2$}
    \begin{proof}
        \step{a}{\pflet{$f \in B_C(g,\epsilon_1) \cap B_D(h,\epsilon_2)$}}
        \step{b}{\pflet{$\delta = \min(\epsilon_1/2 - \sup_{x \in C} d(f(x),g(x)),\epsilon_2/2 - \sup_{x \in D} d(f(x),h(x)))$}}
        \step{c}{\pflet{$B_3 = B_{C \cup D}(f,\delta)$}}
        \step{d}{$B_3 \subseteq B_C(g,\epsilon_1)$}
        \begin{proof}
            \step{i}{\pflet{$k \in B_3$}}
            \step{ii}{\pflet{$x \in C$}}
            \step{iii}{$d(k(x),g(x)) < \epsilon_1$}
            \begin{proof}
                \pf
                \begin{align*}
                    d(k(x),g(x)) & \leq d(k(x),f(x)) + d(f(x),g(x)) \\
                    & < \delta + d(f(x),g(x)) \\
                    & \leq \epsilon_1/2
                \end{align*}
            \end{proof}
        \end{proof}
        \step{e}{$B_3 \subseteq B_D(h,\epsilon_2)$}
        \begin{proof}
            \pf\ Similar
        \end{proof}
    \end{proof}
    \qed
\end{proof}

\begin{theorem}
    \label{theorem:converge_topology_compact_convergence}
    Let $X$ be a topological space. Let $Y$ be a metric space. Let $(f_n)$ be a sequence of functions $X \rightarrow Y$
    and $f : X \rightarrow Y$. Then $f_n \rightarrow f$ in the topology of compact convergence if and only if, for every
    compact subspace $C$ of $X$, we have $f_n \restriction C$ converges uniformly to $f \restriction C$.
\end{theorem}

\begin{proof}
    \pf
    \step{1}{If $f_n \rightarrow f$ in the topology of compact convergence then, for every
    compact subspace $C$ of $X$, we have $f_n \restriction C$ converges uniformly to $f \restriction C$.}
    \begin{proof}
        \step{a}{\assume{$f_n \rightarrow f$ in the topology of compact convergence.}}
        \step{b}{\pflet{$C \subseteq X$ be compact.}}
        \step{c}{\pflet{$\epsilon > 0$}}
        \step{d}{\pick\ $N$ such that $\forall n \geq N. f_n \in B_C(f,\epsilon)$.}
        \step{e}{$\forall n \geq N. \overline{\rho}(f_n,f) < \epsilon$}
    \end{proof}
    \step{2}{If, for every
    compact subspace $C$ of $X$, we have $f_n \restriction C$ converges uniformly to $f \restriction C$,
    then $f_n \rightarrow f$ in the topology of compact convergence.}
    \begin{proof}
        \step{a}{\assume{for every
        compact subspace $C$ of $X$, we have $f_n \restriction C$ converges uniformly to $f \restriction C$}}
        \step{b}{\pflet{$f \in B_C(g,\epsilon)$}}
        \step{bb}{\assume{w.l.o.g. $\epsilon < 1$}}
        \step{c}{\pick\ $N$ such that $\forall n \geq N. \overline{\rho}(f_n,f) < \epsilon/2 - \sup_{x \in C} d(f(x),g(x))$}
        \step{d}{\pflet{$n \geq N$} \prove{$f_n \in B_C(g,\epsilon)$}}
        \step{e}{\pflet{$x \in C$}}
        \step{f}{$d(f_n(x),g(x)) < \epsilon / 2$}
        \begin{proof}
            \pf
            \begin{align*}
                d(f_n(x),g(x)) & \leq d(f_n(x),f(x)) + d(f(x),g(x)) \\
                & \leq \overline{\rho}(f_n,f) + \sup_{x \in C} d(f(x),g(x)) & (\overline{\rho}(f_n,f) < 1) \\
                & < \epsilon / 2
            \end{align*}
        \end{proof}
    \end{proof}
    \qed
\end{proof}

\begin{theorem}[Choice]
    Let $X$ be a compactly generated space. Let $Y$ be a metric space. Then $\mathcal{C}(X,Y)$ is closed in $Y^X$
    under the topology of compact convergence.
\end{theorem}

\begin{proof}
    \pf
    \step{1}{\pflet{$f$ be a limit point of $\mathcal{C}(X,Y)$.} \prove{$f$ is continuous.}}
    \step{2}{\pflet{$C \subseteq X$ be compact.} \prove{$f \restriction C : C \rightarrow Y$ is continuous.}}
    \step{3}{For all $n \in \ZZ^+$, \pick\ $f_n \in B_C(f,1/n) \cap \mathcal{C}(X,Y)$}
    \step{4}{$f_n \restriction C$ converges uniformly to $f \restriction C$}
    \begin{proof}
        \pf\ Theorem \ref{theorem:converge_topology_compact_convergence}.
    \end{proof}
    \step{5}{$f \restriction C$ is continuous.}
    \begin{proof}
        \pf\ By the Uniform Limit Theorem.
    \end{proof}
    \qedstep
    \begin{proof}
        \pf\ Lemma \ref{lemma:compactly_generated_continuous}.
    \end{proof}
    \qed
\end{proof}

\begin{corollary}
    Let $X$ be a compactly generated space. Let $Y$ be a metric space. Let $f_n : X \rightarrow Y$ be a sequence
    of continuous functions. Let $f : X \rightarrow Y$. If $f_n \rightarrow f$ in the topology of compact convergence,
    then $f$ is continuous.
\end{corollary}

\begin{proposition}
    The topology of compact convergence is finer than the product topology.
\end{proposition}

\begin{proof}
    \pf
    \step{1}{\pflet{$X$ be a topological space and $Y$ a metric space.}}
    \step{2}{\pflet{$x \in X$ and $V \subseteq Y$ be open.} \prove{$\inv{\pi_x}(V)$ is open in the topology of compact convergence.}}
    \step{3}{\pflet{$f \in \inv_{\pi_x}(V)$}}
    \step{4}{\pick\ $\epsilon > 0$ such that $B(f(x),\epsilon) \subseteq V$}
    \step{5}{$f \in B_{\{x\}}(f,\epsilon) \subseteq \inv{\pi_x}(V)$}
    \qed
\end{proof}

\begin{proposition}
    Let $X$ be a discrete topological space and $Y$ a metric space. Then the topology of compact convergence
    is the same as the product topology on $Y^X$.
\end{proposition}

\begin{proof}
    \pf
    \step{1}{\pflet{$C \subseteq X$ be compact, $f : X \rightarrow Y$ and $\epsilon > 0$}
    \prove{$B_C(f,\epsilon)$ is open in the product topology.}}
    \step{2}{\pflet{$C = \{ x_1, \ldots, x_n \}$}}
    \begin{proof}
        \pf\ In a discrete space, every compact subspace is finite.
    \end{proof}
    \step{2}{\pflet{$g \in B_C(f, \epsilon)$}}
    \step{3}{\pick\ $\delta < \epsilon - d(f(x_i),g(x_i))$ for all $i$.} 
    \step{4}{$\bigcap_{i=1}^n \inv{\pi_{x_i}}((g(x_i)-\delta,g(x_i)+\delta))
    \subseteq B_C(f, \epsilon)$}
    \qed
\end{proof}

\begin{proposition}
    The topology of compact convergence is coarser than the uniform topology.
\end{proposition}

\begin{proof}
    \pf
    \step{1}{\pflet{$X$ be a topological space and $Y$ a metric space.}}
    \step{2}{\pflet{$C \subseteq X$ be compact, $f : X \rightarrow Y$, $\epsilon > 0$} \prove{$B_C(f,\epsilon)$ is open in the
    uniform topology.}}
    \step{3}{\pflet{$g \in B_C(f,\epsilon)$}}
    \step{4}{\pflet{$\delta = \epsilon - \sup_{x \in C} d(f(x),g(x))$}}
    \step{5}{$B(g,\delta) \subseteq B_C(f,\epsilon)$}
    \qed
\end{proof}

\begin{proposition}
    Let $X$ be a compact space and $Y$ a metric space. Then the topology of compact convergence and the uniform topology 
\end{proposition}
\chapter{Manifolds}

\section{Manifolds}

\begin{definition}[Manifold]
    Let $m \geq 1$. An \emph{$m$-manifold} is a second countable Hausdorff space such that every point has a neighbourhood that is
    homeomorphic to an open subset of $\RR^m$.

    A \emph{curve} is a 1-manifold.

    A \emph{surface} is a 2-manifold.
\end{definition}

\begin{theorem}
    Every compact manifold is imbeddable in $\RR^N$ for some $N$.
\end{theorem}

\begin{proof}
    \pf\ From Theorem \ref{theorem:imbed_in_RN}. \qed
\end{proof}

\begin{proposition}
    Every manifold is regular.
\end{proposition}

\begin{proof}
    \pf
    \step{1}{\pflet{$X$ be an $m$-manifold.}}
    \step{2}{\pflet{$x \in X$}}
    \step{3}{\pflet{$U$ be an open neighbourhood of $x$.}}
    \step{4}{\pick\ an open neighbourhood $V$ of $x$ that is homeomorphic to an open subspace of $\RR^m$.}
    \step{5}{$V$ is regular.}
    \step{6}{There exists an open neighbourhood $W$ of $x$ such that $\overline{W} \subseteq U \cap V$.}
    \qed
\end{proof}

\begin{corollary}
    Every manifold is metrizable.
\end{corollary}

\begin{proposition}
    Let $X$ be a compact Hausdorff space such that every point has an open neighbourhood homeomorphic with an open subspace of $\RR^m$.
    Then $X$ is an $m$-manifold.
\end{proposition}

\begin{proof}
    \pf\ By Theorem \ref{theorem:imbed_in_RN}, $X$ can be embedded in $\RR^N$ for some $N$, hence is second countable. \qed
\end{proof}