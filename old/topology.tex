\newcommand{\support}{\ensuremath{\operatorname{support}}}

\chapter{Topological Spaces}

\section{Topologies}
\begin{definition}[Topology]
    A \emph{topology} on a set $X$ is a set $\TT \subseteq \pow X$ such that:
    \begin{itemize}
        \item $X \in \TT$.
        \item For all $\UU \subseteq \TT$ we have $\bigcup \UU \in \TT$.
        \item For all $U, V \in \TT$ we have $U \cap V \in \TT$.
    \end{itemize}
    We call the elements of $X$ \emph{points} and the elements of $\TT$ \emph{open sets}.
\end{definition}

\begin{definition}[Topological Space]
    A \emph{topological space} $X$ consists of a set $X$ and a topology on $X$.
\end{definition}

\begin{definition}[Discrete Space]
    For any set $X$, the \emph{discrete} topology on $X$ is $\pow X$.
\end{definition}

\begin{definition}[Indiscrete Space]
    For any set $X$, the \emph{indiscrete} or \emph{trivial} topology on $X$ is $\{ \emptyset, X \}$.
\end{definition}

\begin{definition}[Finite Complement Topology]
    For any set $X$, the \emph{finite complement topology} on $X$ is $\{ U \in \pow X \mid X - U \text{ is finite} \} 
    \cup \{ \emptyset \}$.
\end{definition}

\begin{definition}[Countable Complement Topology]
    For any set $X$, the \emph{countable complement topology} on $X$ is $\{ U \in \pow X \mid X - U \text{ is countable} \} 
    \cup \{ \emptyset \}$.
\end{definition}

\begin{definition}[Finer, Coarser]
    Suppose that $\TT$ and $\TT'$ are two topologies on a given set $X$. If $\TT' \supseteq \TT$, we say that $\TT'$ is \emph{finer} than
    $\TT$; if $\TT'$ \emph{properly} contains $\TT$, we say that $\TT'$ is \emph{strictly} finer than $\TT$. We also say that $\TT$ is
    \emph{coarser} than $\TT'$, or \emph{stricly} coarser, in these two respective situations. We say $\TT$ is \emph{comparable} with
    $\TT'$ if either $\TT' \supseteq \TT$ or $\TT \supseteq \TT'$.
\end{definition}

\begin{proposition}
    On any set $X$, the discrete topology is the finest topology.
\end{proposition}

\begin{proposition}
    On any set $X$, the indiscrete topology is the coarsest topology.
\end{proposition}

\begin{proposition}
    On any set $X$, the countable complement topology is finer than the finite complement topology.
\end{proposition}

\begin{definition}[Open Neighbourhood]
    Let $X$ be a topological space and $x \in X$. An \emph{open neighbourhood} of $x$ is an open set $U$ such that $x \in U$.
\end{definition}

\begin{proposition}
    \label{proposition:open}
    Let $X$ be a topological space and $U \subseteq X$. Then $U$ is open if and only if, for all $x \in U$,
    there exists an open neighbourhood $V$ of $x$ such that $V \subseteq U$.
\end{proposition}

\begin{proof}
    \pf
    \step{1}{$\Rightarrow$}
    \begin{proof}
        \pf\ Take $V = U$
    \end{proof}
    \step{2}{$\Leftarrow$}
    \begin{proof}
        \pf\ We have $U = \bigcup \{ V \text{ open in } X \mid V \subseteq U \}$.
    \end{proof}
    \qed
\end{proof}

\begin{proposition}
    Let $X$ be a set and $\TT$ a nonempty set of topologies on $X$. Then $\bigcap \TT$ is a topology on $X$.
\end{proposition}

\begin{proof}
    \pf
    \step{1}{$X \in \bigcap \TT$}
    \begin{proof}
        \pf\ Since $X$ is in every member of $\TT$.
    \end{proof}
    \step{2}{$\bigcap \TT$ is closed under union.}
    \begin{proof}
        \step{a}{\pflet{$\UU \subseteq \bigcap \TT$}}
        \step{b}{For all $T \in \TT$ we have $\UU \subseteq T$}
        \step{c}{For all $T \in \TT$ we have $\bigcup \UU \in T$}
        \step{d}{$\bigcup \UU \in \bigcap \TT$}
    \end{proof}
    \step{3}{$\bigcap \TT$ is closed under binary intersection.}
    \begin{proof}
        \step{a}{\pflet{$U,V \in \bigcap \TT$}}
        \step{b}{For all $T \in \TT$ we have $U,V \in T$}
        \step{c}{For all $T \in \TT$ we have $U \cap V \in T$}
        \step{d}{$U \cap V \in \bigcap \TT$}
    \end{proof}
    \qed
\end{proof}

\begin{corollary}
    For any set $X$, the set of topologies on $X$ is a complete lattice under $\subseteq$.
\end{corollary}

\section{Closed Sets}

\begin{definition}[Closed Set]
    Let $X$ be a topological space and $A \subseteq X$. Then $A$ is \emph{closed} if and only if $X -
    A$ is open.
\end{definition}

\begin{proposition}
    In a discrete space, every set is closed.
\end{proposition}

\begin{proposition}
    In an indiscrete space $X$, the only closed sets are $\emptyset$ and $X$.
\end{proposition}

\begin{proposition}
    In the finite complement topology on $X$, the closed sets are the finite sets and $X$.
\end{proposition}

\begin{proposition}
    In the countable complement topology on $X$, the closed sets are the countable sets and $X$.
\end{proposition}

\begin{proposition}
    Let $\TT$ and $\TT'$ be two topologies on a set $X$. Then $\TT \subseteq \TT'$ if and only if every closed set in $\TT$
    is closed in $\TT'$.
\end{proposition}

\begin{lemma}
    The empty set is closed.
\end{lemma}

\begin{proof}
    \pf\ Since the whole space $X$ is always open. \qed
\end{proof}

\begin{lemma}
    \label{lemma:closed_whole_set}
    In any topological space $X$, the set $X$ is closed.
\end{lemma}

\begin{proof}
    \pf\ Since $\emptyset$ is open. \qed
\end{proof}

\begin{lemma}
    \label{lemma:closed_intersection}
    The intersection of a nonempty set of closed sets is closed.
\end{lemma}

\begin{proof}
    \pf\ Let $\mathcal{C}$ be a nonempty set of closed sets. Then $X - \bigcap \mathcal{C}
    = \bigcup \{ X - C \mid C \in \mathcal{C} \}$ is open. \qed
\end{proof}

\begin{lemma}
    \label{lemma:closed_union}
    The union of two closed sets is closed.
\end{lemma}

\begin{proof}
    \pf\ Let $C$ and $D$ be closed. Then $X - (C \cup D) = (X - C) \cap (X - D)$
    is open. \qed
\end{proof}

\begin{proposition}
    Let $X$ be a set and $\mathcal{C} \subseteq \pow X$ a set such that:
    \begin{enumerate}
        \item $\emptyset \in \mathcal{C}$
        \item $X \in \mathcal{C}$
        \item For all $\AA \subseteq \mathcal{C}$ nonempty we have $\bigcap \AA \in \mathcal{C}$
        \item For all $C, D \in \mathcal{C}$ we have $C \cup D \in \mathcal{C}$.
    \end{enumerate}
    Then there exists a unique topology $\TT$ such that $\mathcal{C}$ is the set of closed sets, namely
    \[ \TT = \{ X - C \mid C \in \mathcal{C} \} \]
\end{proposition}

\begin{proof}
    \pf
    \step{1}{\pflet{$\TT = \{ X - C \mid C \in \mathcal{C} \}$}}
    \step{2}{$\TT$ is a topology}
    \begin{proof}
        \step{a}{$X \in \TT$}
        \begin{proof}
            \pf\ Since $\emptyset \in \mathcal{C}$
        \end{proof}
        \step{b}{For all $\UU \subseteq \TT$ we have $\bigcup \UU \in \TT$}
        \begin{proof}
            \step{i}{\pflet{$\UU \subseteq \TT$}}
            \step{ii}{\case{$\UU = \emptyset$}}
            \begin{proof}
                \pf\ In this case $\bigcup \UU = \emptyset \in \TT$ since $X \in \mathcal{C}$
            \end{proof}
            \step{iii}{\case{$\UU \neq \emptyset$}}
            \begin{proof}
                \pf\ In this case $X - \bigcup \UU = \bigcap \{ X - U \mid U \in \UU \} \in
                \mathcal{C}$.
            \end{proof}
        \end{proof}
        \step{c}{For all $U, V \in \TT$ we have $U \cap V \in \TT$}
        \begin{proof}
            \pf\ Since $X - (U \cap V) = (X - U) \cup (X - V) \in \mathcal{C}$.
        \end{proof}
    \end{proof}
    \step{3}{$\mathcal{C}$ is the set of all closed sets in $\TT$}
    \begin{proof}
        \pf
        \begin{align*}
            & C \text{ is closed in } \TT \\
            \Leftrightarrow & X - C \in \TT \\
            \Leftrightarrow & C \in \mathcal{C}
        \end{align*}
    \end{proof}
    \step{4}{If $\TT'$ is a topology and $\mathcal{C}$ is the set of closed sets in $\TT'$ then
    $\TT' = \TT$}
    \begin{proof}
        \pf\ 
        We have
        \begin{align*}
            & U \in \TT \\
            \Leftrightarrow & X - U \in \mathcal{C} \\
            \Leftrightarrow & X - U \text{ is closed in } \TT' \\
            \Leftrightarrow & U \in \TT'
        \end{align*}
    \end{proof}
    \qed
\end{proof}

\begin{proposition}
    If $U$ is open and $A$ is closed then $U - A$ is open.
\end{proposition}

\begin{proof}
    \pf\ $U - A = U \cap (X - A)$ is the intersection of two open sets. \qed
\end{proof}

\begin{proposition}
    If $U$ is open and $A$ is closed then $A - U$ is closed.
\end{proposition}

\begin{proof}
    \pf\ $A - U = A \cap (X - U)$ is the intersection of two closed sets. \qed
\end{proof}

\section{Interior}

\begin{definition}[Interior]
    Let $X$ be a topological space and $A \subseteq X$. The \emph{interior} of $A$, $\Int A$, is the union
    of all the open subsets of $A$.
\end{definition}

\begin{proposition}
    In a discrete space, we have $\Int A = A$.
\end{proposition}

\begin{proposition}
    In an indiscrete space $X$, we have $\Int X = X$, and $\Int A = \emptyset$ for $A \subset X$.
\end{proposition}

\begin{proposition}
    In a finite complement or countable complement topology, we have $\Int A = A$ if $A$ is open, and $\Int A = \emptyset$ otherwise.
\end{proposition}

\begin{proposition}
    \label{proposition:interior_open}
    The interior of a set $A$ is the largest open set $U$ such that $U \subseteq A$.
\end{proposition}

\begin{proof}
    \pf\ Immediate from definition. \qed
\end{proof}

\begin{corollary}
    \label{corollary:open_interior}
    A set $A$ is open if and only if $A = \Int A$.
\end{corollary}

\section{Closure}

\begin{definition}[Closure]
    Let $X$ be a topological space and $A \subseteq X$. The \emph{closure} of $A$, $\overline{A}$, is the
    intersection of all the closed sets that include $A$.
\end{definition}

This intersection exists since $X$ is a closed set that includes A (Lemma \ref{lemma:closed_whole_set}).

\begin{proposition}
    In a discrete space, we have $\overline{A} = A$.
\end{proposition}

\begin{proposition}
    In an indiscrete space, we have $\overline{\emptyset} = \emptyset$, and $\overline{A} = X$ for any nonempty $A$.
\end{proposition}

\begin{proposition}
    In a finite complement or countable complement topology on a set $X$, we have $\overline{A} = A$ if $A$ is closed,
    and $\overline{A} = X$ otherwise.
\end{proposition}

\begin{proposition}
    \label{proposition:closure_smallest}
    The closure of a set $A$ is the smallest closed set $C$ such that $A \subseteq C$.
\end{proposition}

\begin{proof}
    \pf\ Immediate from definition. \qed
\end{proof}

\begin{corollary}
    A set $A$ is closed if and only if $A = \overline{A}$.
\end{corollary}

\begin{proposition}
    \label{proposition:closure_neighbourhood}
    Let $X$ be a topological space, $A \subseteq X$ and $x \in X$. Then $x \in \overline{A}$ if and only if
    every ooen neighbourhood of $x$ intersects $A$.
\end{proposition}

\begin{proof}
    \pf\ We have
    \begin{align*}
        & x \in \overline{A} \\
        \Leftrightarrow & \forall C. C \text{ closed } \wedge A \subseteq C \Rightarrow x \in C \\
        \Leftrightarrow & \forall U. U \text{ open } \wedge A \cap U = \emptyset \Rightarrow x \notin U \\
        \Leftrightarrow & \forall U. U \text{ open } \wedge x \in U \Rightarrow U \text{ intersects } A & \qed
    \end{align*}
\end{proof}

\begin{proposition}
    \label{proposition:closure_monotone}
    If $A \subseteq B$ then $\overline{A} \subseteq \overline{B}$.
\end{proposition}

\begin{proof}
    \pf\ This holds because $\overline{B}$ is a closed set that includes $A$. \qed
\end{proof}

\begin{proposition}
    \label{proposition:closure_int_complement}
    \[ \overline{X - A} = X - \Int A \]
\end{proposition}

\begin{proof}
    \pf
    \step{1}{$\overline{X - A} \subseteq X - \Int A$}
    \begin{proof}
        \step{a}{$\Int A \subseteq A$}
        \step{b}{$X - A \subseteq X - \Int A$}
        \step{c}{$\overline{X - A} \subseteq X - \Int A$}
    \end{proof}
    \step{2}{$X - \Int A \subseteq \overline{X - A}$}
    \begin{proof}
        \step{a}{$X - A \subseteq \overline{X - A}$}
        \step{b}{$X - \overline{X - A} \subseteq A$}
        \step{c}{$X - \overline{X - A} \subseteq \Int A$}
    \end{proof}
    \qed
\end{proof}

\begin{proposition}
    \[ X - \overline{A} = \Int (X - A) \]
\end{proposition}

\begin{proof}
    \pf\ Dual. \qed
\end{proof}

\begin{proposition}
    \label{proposition:closure_union}
    \[ \overline{A \cup B} = \overline{A} \cup \overline{B} \]
\end{proposition}

\begin{proof}
    \pf
    \step{1}{$\overline{A} \subseteq \overline{A \cup B}$}
    \begin{proof}
        \pf\ By Proposition \ref{proposition:closure_monotone}.
    \end{proof}
    \step{2}{$\overline{B} \subseteq \overline{A \cup B}$}
    \begin{proof}
        \pf\ By Proposition \ref{proposition:closure_monotone}.
    \end{proof}
    \step{3}{$\overline{A \cup B} \subseteq \overline{A} \cup \overline{B}$}
    \begin{proof}
        \pf\ Since $\overline{A} \cup \overline{B}$ is a closed set that includes $A \cup B$.
    \end{proof}
    \qed
\end{proof}

\begin{proposition}
    \label{proposition:every_closure_maximal_fip}
    Let $X$ be a topological space. Let $\DD$ be a set of subsets of $X$ that is maximal with respect to the finite intersection property. Let $x \in X$.
    Then the following are equivalent:
    \begin{enumerate}
        \item For all $D \in \DD$ we have $x \in \overline{D}$
        \item Every open neighbourhood of $x$ is in $\DD$.
    \end{enumerate}
\end{proposition}

\begin{proof}
    \pf
    \step{1}{$1 \Rightarrow 2$}
    \begin{proof}
        \step{a}{\assume{For all $D \in \DD$ we have $x \in \overline{D}$}}
        \step{b}{\pflet{$U$ be an open neighbourhood of $x$}}
        \step{c}{$\DD \cup \{U\}$ satisfies the finite intersection property.}
        \begin{proof}
            \step{i}{\pflet{$D_1, \ldots, D_n \in \DD$}}
            \step{ii}{$D_1 \cap \cdots \cap D_n \in \DD$}
            \begin{proof}
                \pf\ Lemma \ref{lemma:finite_intersection_maximal}.
            \end{proof}
            \step{iii}{$x \in \overline{D_1 \cap \cdots \cap D_n}$}
            \begin{proof}
                \pf\ \stepref{a}, \stepref{ii}
            \end{proof}
            \step{iv}{$D_1 \cap \cdots \cap D_n \cap U \neq \emptyset$}
            \begin{proof}
                \pf\ Proposition \ref{proposition:closure_neighbourhood}, \stepref{b}, \stepref{iii}.
            \end{proof}
        \end{proof}
        \step{d}{$\DD = \DD \cup \{U\}$}
        \begin{proof}
            \pf\ By the maximality of $\DD$.
        \end{proof}
        \step{e}{$U \in \DD$}
    \end{proof}
    \step{2}{$2 \Rightarrow 1$}
    \begin{proof}
        \step{a}{\assume{Every neighbourhood of $x$ is in $\DD$.}}
        \step{b}{\pflet{$D \in \DD$}}
        \step{c}{Every neighbourhood of $x$ intersects $D$.}
        \begin{proof}
            \pf\ From \stepref{a}, \stepref{b} and the fact that $\DD$ satisfies the finite intersection property.
        \end{proof}
        \step{d}{$x \in \overline{D}$}
        \begin{proof}
            \pf\ Proposition \ref{proposition:closure_neighbourhood}, \stepref{c}.
        \end{proof}
    \end{proof}
    \qed
\end{proof}

\section{Boundary}

\begin{definition}[Boundary]
    The \emph{boundary} of a set $A$ is the set $\partial A = \overline{A} \cap \overline{X - A}$.
\end{definition}

\begin{proposition}
    In a discrete space we have $\partial A = \emptyset$.
\end{proposition}

\begin{proposition}
    In an indiscrete space $X$, we have $\partial A = \emptyset$ if $A = \emptyset$ or $A = X$,
    and $\partial A = X$ for all other $X$.
\end{proposition}

\begin{proposition}
    \label{proposition:int_boundary_disjoint}
    \[ \Int A \cap \partial A = \emptyset \]
\end{proposition}

\begin{proof}
    \pf\ Since $\overline{X - A} = X - \Int A$ (Proposition \ref{proposition:closure_int_complement}). \qed
\end{proof}

\begin{proposition}
    \label{proposition:closure_int_boundary}
    \[ \overline{A} = \Int A \cup \partial A \]
\end{proposition}

\begin{proof}
    \pf
    \begin{align*}
        \Int A \cup \partial A & = \Int A \cup (\overline{A} \cap \overline{X - A}) \\
        & = (\Int A \cup \overline{A}) \cap (\Int A \cup \overline{X - A}) \\
        & = \overline{A} \cap X \\
        & = \overline{A}
    \end{align*}
\end{proof}

\begin{proposition}
    $\partial A = \emptyset$ if and only if $A$ is open and closed.
\end{proposition}

\begin{proof}
    \pf\ If $\partial A = \emptyset$ then $\overline{A} = \Int A$ by Proposition \ref{proposition:closure_int_boundary}.
\end{proof}

\begin{proposition}
    \[ \partial A = \overline{A} - \Int A \]
\end{proposition}

\begin{proof}
    \pf\ Propositions \ref{proposition:int_boundary_disjoint}, \ref{proposition:closure_int_boundary}. \qed
\end{proof}

\begin{corollary}
    A set $U$ is open if and only if $\partial U = \overline{U} - U$.
\end{corollary}

\begin{proof}
    \pf
    \begin{align*}
        & \partial U = \overline{U} - U \\
        \Leftrightarrow & \overline{U} - \Int U = \overline{U} - U \\
        \Leftrightarrow & \Int U = U & \qed
    \end{align*}
\end{proof}

\section{Limit Points}

\begin{definition}[Limit Point]
    Let $X$ be a topological space, $a \in X$ and $A \subseteq X$. Then $a$ is a \emph{limit point},
    \emph{cluster point} or \emph{point of accumulation} for $A$ if and only if every open neighbourhood of $a$
    intersects $A$ at a point other than $a$.
\end{definition}

\begin{proposition}
    In a discrete space, no set has a limit point.
\end{proposition}

\begin{proposition}
    \label{proposition:indiscrete_limit_point}
    In an indiscrete topology, every point is a limit point of any set that has more than one point.
\end{proposition}

\begin{proof}
    \pf\ Let $X$ be an indiscrete space. Let $A$ be a set with more than one point and $x$ be a point.
    The only neighbourhood of $x$ is $X$, which must intersect $A$ at a point other than $x$. \qed
\end{proof}

\begin{lemma}
    The point $a$ is an accumulation point for $A$ if and only if $a \in \overline{A - \{a\}}$.
\end{lemma}

\begin{proof}
    \pf\ From Proposition \ref{proposition:closure_neighbourhood}. \qed
\end{proof}

\begin{theorem}
    \label{theorem:limit_point_closure}
    Let $X$ be a topological space and $A \subseteq X$. Let $A'$ be the set of all limit points of $A$.
    Then $\overline{A} = A \cup A'$.
\end{theorem}

\begin{proof}
    \pf
    \step{1}{For all $x \in \overline{A}$, if $x \notin A$ then $x \in A'$}
    \begin{proof}
        \pf\ From Proposition \ref{proposition:closure_neighbourhood}.
    \end{proof}
    \step{2}{$A \subseteq \overline{A}$}
    \begin{proof}
        \pf\ Proposition \ref{proposition:closure_smallest}.
    \end{proof}
    \step{3}{$A' \subseteq \overline{A}$}
    \begin{proof}
        \pf\ From Proposition \ref{proposition:closure_neighbourhood}.
    \end{proof}
    \qed
\end{proof}

\begin{corollary}
    \label{corollary:closed_limit_point}
    A set is closed if and only if it contains all its limit points.
\end{corollary}

\begin{lemma}
    \label{lemma:limit_point_subset}
    Let $X$ be a topological space and $A \subseteq B \subseteq X$. Then every limit
    point of $A$ is a limit point of $B$.
\end{lemma}

\begin{proof}
    \pf\ Immediate from definitions. \qed
\end{proof}

\section{Basis for a Topology}

\begin{definition}[Basis]
    If $X$ is a set, a \emph{basis} for a topology on $X$ is a set $\BB \subseteq \pow X$ called \emph{basis elements} such that
    \begin{enumerate}
        \item For all $x \in X$, there exists $B \in \BB$ such that $x \in B$.
        \item For all $B_1, B_2 \in \BB$ and $x \in B_1 \cap B_2$, there exists $B_3 \in \BB$ such that
        $x \in B_3 \subseteq B_1 \cap B_2$.
    \end{enumerate}

    If $\BB$ satisfies these two conditions, then we define the topology \emph{generated} by $\BB$ to be
    $\TT = \{ U \in \pow X \mid \forall x \in U. \exists B \in \BB. x \in B \subseteq U \}$.
\end{definition}

We prove this is a topology.

\begin{proof}
    \pf
    \step{1}{$X \in \TT$}
    \begin{proof}
        \pf\ For all $x \in X$ there exists $B \in \BB$ such that $x \in B \subseteq X$ by condition 1.
    \end{proof}
    \step{2}{For all $\UU \subseteq \TT$ we have $\bigcup \UU \in \TT$}
    \begin{proof}
        \step{a}{\pflet{$\UU \subseteq \TT$}}
        \step{b}{\pflet{$x \in \bigcup \UU$}}
        \step{c}{\pick\ $U \in \UU$ such that $x \in U$}
        \step{d}{\pick\ $B \in \BB$ such that $x \in B \subseteq U$}
        \begin{proof}
            \pf\ Since $U \in \TT$ by \stepref{a} and \stepref{c}.
        \end{proof}
        \step{e}{$x \in B \subseteq \bigcup \UU$}
    \end{proof}
    \step{3}{For all $U, V \in \TT$ we have $U \cap V \in \TT$}
    \begin{proof}
        \step{a}{\pflet{$U, V \in \TT$}}
        \step{b}{\pflet{$x \in U \cap V$}}
        \step{c}{\pick\ $B_1 \in \BB$ such that $x \in B_1 \subseteq U$}
        \step{d}{\pick\ $B_2 \in \BB$ such that $x \in B_2 \subseteq V$}
        \step{e}{\pick\ $B_3 \in \BB$ such that $x \in B_3 \subseteq B_1 \cap B_2$}
        \begin{proof}
            \pf\ By condition 2.
        \end{proof}
        \step{f}{$x \in B_3 \subseteq U \cap V$}
    \end{proof}
    \qed
\end{proof}

\begin{lemma}
    \label{lemma:basis_unions}
    Let $X$ be a set. Let $\BB$ be a basis for a topology $\TT$ on $X$. Then $\TT$ is the set of all unions of 
    subsets of $\BB$.
\end{lemma}

\begin{proof}
    \pf
    \step{1}{For all $U \in \TT$, there exists $\AA \subseteq \BB$ such that $U = \bigcup \AA$}
    \begin{proof}
        \step{a}{\pflet{$U \in \TT$}}
        \step{b}{\pflet{$\AA = \{ B \in \BB \mid B \subseteq U \}$}}
        \step{c}{$U \subseteq \bigcup \AA$}
        \begin{proof}
            \step{i}{\pflet{$x \in U$}}
            \step{ii}{\pick\ $B \in \BB$ such that $x \in B \subseteq U$}
            \begin{proof}
                \pf\ Since $\BB$ is a basis for $\TT$.
            \end{proof}
            \step{iii}{$x \in B \in \AA$}
        \end{proof}
        \step{d}{$\bigcup \AA \subseteq U$}
        \begin{proof}
            \pf\ From the definition of $\AA$ (\stepref{b}).
        \end{proof}
    \end{proof}
    \step{2}{For all $\AA \subseteq \BB$ we have $\bigcup \AA \in \TT$}
    \begin{proof}
        \step{a}{$\BB \subseteq \TT$}
        \begin{proof}
            \pf\ If $B \in \BB$ and $x \in B$, then there exists $B' \in \BB$ such that $x \in B' \subseteq B$, namely $B' = B$.
        \end{proof}
        \qedstep
        \begin{proof}
            \pf\ Since $\TT$ is closed under union.
        \end{proof}
    \end{proof}
    \qed
\end{proof}

\begin{corollary}
    \label{cor:basis_open}
    Let $X$ be a set. Let $\BB$ be a basis for a topology $\TT$ on $X$. Then $\TT$ is the coarsest topology
    that includes $\BB$.
\end{corollary}

\begin{proof}
    \pf\ Since every topology that includes $\BB$ includes all unions of subsets of $\BB$. \qed
\end{proof}

\begin{lemma}
    \label{lemma:basis}
    Let $X$ be a topological space. Suppose that $\CC$ is a set of open sets such that, for every open set $U$ and every point $x \in U$,
    there exists $C \in \CC$ such that $x \in C \subseteq U$. Then $\CC$ is a basis for the topology on $X$.
\end{lemma}

\begin{proof}
    \pf
    \step{1}{For all $x \in X$, there exists $C \in \CC$ such that $x \in C$}
    \begin{proof}
        \pf\ Immediate from hypothesis.
    \end{proof}
    \step{2}{For all $C_1, C_2 \in \CC$ and $x \in C_1 \cap C_2$, there exists $C_3 \in \CC$ such that $x \in C_3 \subseteq C_1 \cap C_2$}
    \begin{proof}
        \pf\ Since $C_1 \cap C_2$ is open.
    \end{proof}
    \step{3}{Every open set is open in the topology generated by $\CC$}
    \begin{proof}
        \pf\ Immediate from hypothesis.
    \end{proof}
    \step{4}{Every union of a subset of $\CC$ is open.}
    \begin{proof}
        \pf\ Since every member of $\CC$ is open.
    \end{proof}
    \qed
\end{proof}

\begin{lemma}
    Let $\BB$ and $\BB'$ be bases for the topologies $\TT$ and $\TT'$ respectively on the set $X$. Then the following are equivalent.
    \begin{enumerate}
        \item $\TT \subseteq \TT'$
        \item For all $B \in \BB$ and $x \in B$, there exists $B' \in \BB'$ such that $x \in B' \subseteq B$.
    \end{enumerate}
\end{lemma}

\begin{proof}
    \pf
    \step{1}{$1 \Rightarrow 2$}
    \begin{proof}
        \step{a}{\assume{$\TT \subseteq \TT'$}}
        \step{b}{\pflet{$B \in \BB$ and $x \in B$}}
        \step{c}{$B \in \TT$}
        \begin{proof}
            \pf\ Corollary \ref{cor:basis_open}.
        \end{proof}
        \step{d}{$B \in \TT'$}
        \begin{proof}
            \pf\ By \stepref{a}
        \end{proof}
        \step{e}{There exists $B' \in \BB'$ such that $x \in B' \subseteq B$}
        \begin{proof}
            \pf\ Since $\BB'$ is a basis for $\TT'$.
        \end{proof}
    \end{proof}
    \step{2}{$2 \Rightarrow 1$}
    \begin{proof}
        \step{a}{\assume{2}}
        \step{b}{\pflet{$U \in \TT$} \prove{$U \in \TT'$}}
        \step{c}{\pflet{$x \in U$} \prove{There exists $B' \in \BB'$ such that $x \in B' \subseteq U$}}
        \step{d}{\pick\ $B \in \BB$ such that $x \in B \subseteq U$}
        \begin{proof}
            \pf\ Since $\BB$ is a basis for $\TT$.
        \end{proof}
        \step{e}{\pick\ $B' \in \BB'$ such that $x \in B' \subseteq B$}
        \begin{proof}
            \pf\ By \stepref{a}.
        \end{proof}
        \step{f}{$x \in B' \subseteq U$}
    \end{proof}
    \qed
\end{proof}

\begin{theorem}
    Let $X$ be a topological space, $A \subseteq X$ and $x \in X$. Let $\BB$ be a basis for $X$. Then $x \in 
    \overline{A}$ if and only if, for all $B \in \BB$, if $x \in B$ then $B$ intersects $A$.
\end{theorem}

\begin{proof}
    \pf
    \step{1}{If $x \in \overline{A}$ then, for all $B \in \BB$, if $x \in B$ then $B$ intersects $A$.}
    \begin{proof}
        \pf\ This follows from Theorem \ref{theorem:closure_neighbourhood} since every element of $\BB$ is open
        (Corollary \ref{cor:basis_open}).
    \end{proof}
    \step{2}{Suppose that, for all $B \in \BB$, if $x \in B$ then $B$ intersects $A$. Then $x \in \overline{A}$.}
    \begin{proof}
        \step{a}{\assume{For all $B \in \BB$, if $x \in B$ then $B$ intersects $A$.}}
        \step{b}{\pflet{$U$ be an open set that contains $x$} \prove{$U$ intersects $A$.}}
        \step{c}{\pick\ $B \in \BB$ such that $x \in B \subseteq U$.}
        \step{d}{$B$ intersects $A$.}
        \begin{proof}
            \pf\ From \stepref{a}.
        \end{proof}
        \step{e}{$U$ intersects $A$.}
        \qedstep
        \begin{proof}
            \pf\ By Theorem \ref{theorem:closure_neighbourhood}.
        \end{proof}
    \end{proof}
    \qed
\end{proof}

\begin{definition}[Lower Limit Topology on the Real Line]
    The \emph{lower limit topology on the real line} is the topology on $\RR$ generated by the basis consisting of all half-open intervals
    of the form $[a,b)$.

    We write $\RR_l$ for the topological space $\RR$ under the lower limit topology.
\end{definition}

We prove this is a basis for a topology.

\begin{proof}
    \pf
    \step{1}{For all $x \in \RR$ there exists an interval $[a,b)$ such that $x \in [a,b)$.}
    \begin{proof}
        \pf\ Take $[a,b) = [x,x+1)$.
    \end{proof}
    \step{2}{For any open intervals $[a,b)$, $[c,d)$ if $x \in [a,b) \cap [c,d)$, then there exists an interval $[e,f)$ such that
    $x \in [e,f) \subseteq [a,b) \cap [c,d)$}
    \begin{proof}
        \pf\ Take $[e,f) = [\max (a,c),\min (b,d))$.
    \end{proof}
    \qed
\end{proof}

\begin{definition}[$K$-topology on the Real Line]
    Let $K = \{ 1/n \mid n \in \ZZ^+ \}$.

    The \emph{$K$-topology on the real line} is the topology on $\RR$ generated by the basis consisting of all open intervals
    $(a,b)$ and all sets of the form $(a,b) - K$.

    We write $\RR_K$ for the topological space $\RR$ under the $K$-topology.
\end{definition}

We prove this is a basis for a topology.

\begin{proof}
    \pf
    \step{1}{For all $x \in \RR$ there exists an open interval $(a,b)$ such that $x \in (a,b)$.}
    \begin{proof}
        \pf\ Take $(a,b) = (x-1,x+1)$.
    \end{proof}
    \step{2}{For any basic open sets $B_1$, $B_2$ if $x \in B_1 \cap B_2$, then there exists a basic open set $B_3$ such that
    $x \in B_3 \subseteq B_1 \cap B_2$}
    \begin{proof}
        \step{a}{\case{$B_1 = (a,b)$, $B_2 = (c,d)$}}
        \begin{proof}
            \pf\ Take $B_3 = (\max (a,c), \min (b,d))$.
        \end{proof}
        \step{b}{\case{$B_1 = (a,b)$ or $(a,b) - K$, $B_2 = (c,d)$ or $(c,d) - K$, and they are
        not both open intervals.}}
        \begin{proof}
            \pf\ Take $B_3 = (\max (a,c), \min (b,d)) - K$.
        \end{proof}
    \end{proof}
    \qed
\end{proof}

\begin{lemma}
    The lower limit topology and the $K$-topology are incomparable.
\end{lemma}

\begin{proof}
    \pf
    \step{1}{The interval $[10,11)$ is not open in the $K$-topology.}
    \begin{proof}
        \pf\ There is no open interval $(a,b)$ such that $10 \in (a,b) \subseteq [10,11)$ or
        $10 \in (a,b) - K \subseteq [10,11)$.
    \end{proof}
    \step{2}{The set $(-1,1) - K$ is not open in the lower limit topology.}
    \begin{proof}
        \pf\ There is no half-open interval $[a,b)$ such that $0 \in [a,b) \subseteq (-1,1) - K$, since
        there must be a positive integer $n$ with $1/n \in [a,b)$.
    \end{proof}
    \qed
\end{proof}

\section{Subbases}

\begin{definition}[Subbasis]
    A \emph{subbasis} $\mathcal{S}$ for a topology on $X$ is a set $\mathcal{S} \subseteq \pow X$ such that
    $\bigcup \mathcal{S} = X$.

    The topology \emph{generated} by the subbasis $\mathcal{S}$ is the set of all unions of finite
    intersections of elements of $\mathcal{S}$.
\end{definition}

We prove this is a topology.

\begin{proof}
    \pf
    \step{1}{The set $\mathcal{B}$ of all finite intersections of elements of $\mathcal{S}$ forms a basis for a topology
    on $X$.}
    \begin{proof}
        \step{a}{$\bigcup \mathcal{B} = X$}
        \begin{proof}
            \pf\ Since $\mathcal{S} \subseteq \BB$.
        \end{proof}
        \step{b}{$\BB$ is closed under binary intersection.}
        \begin{proof}
            \pf\ By definition.
        \end{proof}
    \end{proof}
    \qedstep
    \begin{proof}
        \pf\ By Lemma \ref{lemma:basis_unions}.
    \end{proof}
    \qed
\end{proof}

We have simultaneously proved:

\begin{proposition}
    \label{proposition:subbasis_basis}
    Let $\SS$ be a subbasis for the topology on $X$.
    Then the set of all finite intersections of elements of $\SS$
    is a basis for the topology on $X$.
\end{proposition}

\begin{proposition}
    \label{proposition:subbasis_coarsest}
    Let $X$ be a set. Let $\mathcal{S}$ be a subbasis for a topology $\TT$ on $X$. Then $\TT$ is the coarsest topology
    that includes $\mathcal{S}$.
\end{proposition}

\begin{proof}
    \pf\ Since every topology that includes $\mathcal{S}$ includes every union of finite intersections of
    elements of $\mathcal{S}$. \qed
\end{proof}

\section{Local Basis at a Point}

\begin{definition}[Local Basis]
    Let $X$ be a topological space and $a \in X$. A \emph{(local) basis at $a$} is a set $\BB$ of open neighbourhoods of $a$ such that every neighbourhood of $a$ includes some
    member of $\BB$.
\end{definition}

\begin{lemma}
    \label{lemma:countable_local_basis}
    If there exists a countable local basis at a point $a$, then there exists a countable local basis $\{ B_n \mid n \geq 1 \}$ such that $B_1 \supseteq B_2 \supseteq \cdots$. 
\end{lemma}

\begin{proof}
    \pf\ Given a countable local basis $\{ C_n \mid n \geq 1 \}$, take $B_n = C_1 \cap \cdots \cap C_n$. \qed
\end{proof}

\section{Nets and Convergence}

\begin{definition}[Net]
    Let $X$ be a topological space. A \emph{net} in $X$ consists of a directed poset $J$
    and a family $(x_\alpha)_{\alpha \in J}$ of points of $X$ indexed by $J$.
\end{definition}

\begin{definition}[Convergence]
    Let $X$ be a topological space. Let $(x_\alpha)_{\alpha \in J}$ be a net in $X$
    and $l \in X$. Then $(x_\alpha)$ \emph{converges} to the \emph{limit} $l$
    iff, for every open neighbourhood $U$ of $l$, there exists $\alpha \in J$ such that,
    for all $\beta \geq \alpha$, we have $x_\beta \in U$.
\end{definition}

\begin{lemma}[Choice]
    \label{lemma:closure_net}
    Let $X$ be a topological space. Let $A \subseteq X$ and $l \in X$. 
    Then $l \in \overline{A}$ if and only if there exists a net of points in $A$
    that converges to $l$.
\end{lemma}

\begin{proof}
    \pf
    \step{1}{If $l \in \overline{A}$ then there exists a net of points in $A$
    that converges to $l$.}
    \begin{proof}
        \step{a}{\assume{$l \in \overline{A}$}}
        \step{b}{\pflet{$J$ be the set of open neighbourhoods of $l$ under $\supseteq$}}
        \step{c}{For $U \in J$, \pick\ $a_U \in U \cap A$ \prove{$a_U \rightarrow l$
        as $U \rightarrow \infty$}}
        \begin{proof}
            \pf\ Such an $a_U$ exists by Proposition \ref{proposition:closure_neighbourhood}.
        \end{proof}
        \step{d}{\pflet{$U$ be an open neighbourhood of $l$.}}
        \step{e}{For any $V \subseteq U$ we have $a_V \in U$.}
    \end{proof}
    \step{2}{If there exists a net of points in $A$ that coverges to $l$,
    then $l \in \overline{A}$.}
    \begin{proof}
        \step{1}{\pflet{$(a_\alpha)_{\alpha \in J}$ be a sequence of points in 
        $A$ that converges to $l$.}}
        \step{2}{\pflet{$U$ be an open neighbourhood of $l$.}}
        \step{3}{\pick\ $\alpha \in J$ such that,
        for all $\beta \geq \alpha$, we have $a_\beta \in U$.}
        \step{4}{$a_\alpha \in U \cap A$}
        \qedstep
        \begin{proof}
            \pf\ Proposition \ref{proposition:closure_neighbourhood}.
        \end{proof}
    \end{proof}
    \qed
\end{proof}

\begin{proposition}
    \label{proposition:convergence_basis}
    Let $X$ be a topological space. Let $\BB$ be a basis for the topology on $X$.
    Let $(a_\alpha)$ be a net in $X$ and $l \in X$. Then $a_\alpha \rightarrow l$ if and only if, for every $B \in \BB$ with $l \in B$,
    there exists $\alpha$ such that, for all $\beta \geq \alpha$, we have $a_\beta \in B$.
\end{proposition}

\begin{proof}
    \pf
    \step{1}{If $a_\alpha \rightarrow l$ then, for every $B \in \BB$ with $l \in B$,
    there exists $\alpha$ such that, for all $\beta \geq \alpha$, we have $a_\beta \in B$.}
    \begin{proof}
        \pf\ Since every element of $\BB$ is open (Corollary \ref{cor:basis_open}).
    \end{proof}
    \step{2}{If, for every $B \in \BB$ with $l \in B$,
    there exists $\alpha$ such that, for all $\beta \geq \alpha$, we have $a_\beta \in B$,
    then $a_\alpha \rightarrow l$ as $n \rightarrow \infty$.}
    \begin{proof}
        \step{a}{\assume{for every $B \in \BB$ with $l \in B$, there exists $\alpha$ such that, for all $\beta \geq \alpha$,
        we have $a_\beta \in B$}}
        \step{b}{\pflet{$U$ be a neighbourhood of $l$.}}
        \step{c}{\pick\ $B \in \BB$ such that $l \in B \subseteq U$}
        \step{d}{\pick\ $\alpha$ such that, for all $\beta \geq \alpha$, we have $a_\beta \in B$}
        \begin{proof}
            \pf\ From \stepref{a}.
        \end{proof}
        \step{e}{For all $\beta \geq \alpha$ we have $a_\beta \in U$}
    \end{proof}
    \qed
\end{proof}

\begin{lemma}
    \label{lemma:converge_constant}
    If a net $(a_\alpha)$ is constant with $a_\alpha = l$ for all $\alpha$, then $a_\alpha \rightarrow l$.
\end{lemma}

\begin{proof}
    \pf\ Immediate from definitions. \qed
\end{proof}

\section{Locally Finite Sets}

\begin{definition}[Locally Finite]
    Let $X$ be a topological space and $\{ A_\alpha \}$ a family of subsets of $X$. Then $\AA$ is \emph{locally finite} if and only if every point in $X$ has a neighbourhood that
    intersects $A_\alpha$ for only finitely many $\alpha$.
\end{definition}

\begin{proposition}
    A $T_1$ space with a locally finite basis is discrete.
\end{proposition}

\begin{proof}
    \pf
    \step{1}{\pflet{$X$ be a $T_1$ space with a locally finite basis.}}
    \step{2}{\pick\ a locally finite basis $\BB$.}
    \step{3}{\pflet{$x \in X$}}
    \step{4}{\pick\ an open neighbourhood $U$ of $x$ that intersects only finitely many elements of $\BB$, say $B_1$, \ldots, $B_n$.}
    \step{5}{$U \cap B_1 \cap \cdots \cap B_n = \{x\}$}
    \begin{proof}
        \step{i}{\pflet{$y \in U \cap B_1 \cap \cdots \cap B_n$}}
        \step{ii}{\assume{for a contradiction $x \neq y$}}
        \step{iii}{\pick\ $B \in \BB$ such that $y \in B \subseteq X - \{x\}$}
        \step{iv}{\pick\ $i$ such that $B = B_i$}
        \begin{proof}
            \pf\ Since $U$ intersects $B$ in $y$.
        \end{proof}
        \qedstep
        \begin{proof}
            \pf\ This contradicts the fact that $x \in B_i$.
        \end{proof}
    \end{proof}
    \step{6}{$\{x\}$ is open.}
    \qed
\end{proof}

\begin{lemma}
    Let $X$ be a topological space. Let $\{A_\alpha\}_{\alpha \in J}$ be a locally finite family of subsets of $X$. 
    Then every subfamily of $\{A_\alpha\}_{\alpha \in J}$ is locally finite.
\end{lemma}

\begin{proof}
    \pf
    \step{2}{\pflet{$\{ A_\alpha \}_{\alpha \in K}$ be a subfamily.}}
    \step{3}{\pflet{$x \in X$}}
    \step{4}{\pick\ an open neighbourhood $U$ of $x$ that intersects $A_\alpha$ for only finitely many $\alpha \in J$.}
    \step{5}{$U$ intersects $A_\alpha$ for only finitely many $\alpha \in K$.}
    \qed
\end{proof}

\begin{lemma}
    \label{lemma:closures_locally_finite}
    Let $X$ be a topological space. Let $\{A_\alpha\}_{\alpha \in J}$ be a locally finite family of subsets of $X$. 
    Then $\{ \overline{A_\alpha} \}_{\alpha \in J}$ is locally finite.
\end{lemma}

\begin{proof}
    \pf
    \step{1}{\pflet{$x \in X$}}
    \step{2}{\pick\ a neighbourhood $U$ of $x$ that intersects $A_\alpha$ for only finitely many $\alpha \in J$.}
    \step{3}{For all $\alpha \in J$, if $U$ intersects $\overline{A_\alpha}$ then $U$ intersects $A_\alpha$.}
    \step{4}{$U$ intersects $\overline{A_\alpha}$ for only finitely many $\alpha \in J$.}
    \qed
\end{proof}

\begin{lemma}
    Let $X$ be a topological space. Let $\{A_\alpha\}_{\alpha \in J}$ be a locally finite family of subsets of $X$. 
    Then $\overline{\bigcup_{\alpha \in J} A_\alpha} = \bigcup_{\alpha \in J} \overline{A_\alpha}$.
\end{lemma}

\begin{proof}
    \pf
    \step{1}{$\overline{\bigcup_{\alpha \in J} A_\alpha} \subseteq \bigcup_{\alpha \in J} \overline{A_\alpha}$}
    \begin{proof}
        \step{a}{\pflet{$x \in \overline{\bigcup_{\alpha \in J} A_\alpha}$}}
        \step{b}{\pick\ a neighbourhood $U$ of $x$ that intersects $A_\alpha$ for only finitely
        $\alpha \in J$, say $\alpha_1$, \ldots, $\alpha_n$}
        \step{c}{\assume{for a contradiction $x$ is not in any of $\overline{A_{\alpha_1}}$, \ldots, $\overline{A_{\alpha_n}}$}}
        \step{d}{$U - \overline{A_{\alpha_1}} - \cdots - \overline{A_{\alpha_n}}$ is a neighbourhood of $x$ that does not intersect 
        $\bigcup_{\alpha \in J} A_\alpha$}
        \qedstep
        \begin{proof}
            \pf\ This contradicts \stepref{a}.
        \end{proof}
    \end{proof}
    \step{2}{$\bigcup_{\alpha \in J} \overline{A_\alpha} \subseteq \overline{\bigcup_{\alpha \in J} A_\alpha}$}
    \begin{proof}
        \pf\ Proposition \ref{proposition:closure_monotone}.
    \end{proof}
    \qed
\end{proof}

\begin{corollary}
    \label{corollary:union_locally_finite_closed}
    The union of a locally finite set of closed sets is closed.
\end{corollary}

\section{Open Maps}

\begin{definition}[Open Map]
    Let $X$ and $Y$ be topological spaces and $f : X \rightarrow Y$. Then $f$ is an \emph{open map} if and
    only if, for every open set $U$ in $X$, the set $f(U)$ is open in $Y$.
\end{definition}

\begin{lemma}
    \label{lemma:open_map_basis}
    Let $X$ and $Y$ be topological spaces and $f : X \rightarrow Y$. Let $\BB$ be a basis for the topology on $X$.
    If $f(B)$ is open in $Y$ for all $B \in \BB$, then $f$ is an open map.
\end{lemma}

\begin{proof}
    \pf\ From Lemma \ref{lemma:basis_unions}. \qed
\end{proof}

\begin{proposition}
    \label{proposition:open_map_basis}
    Let $X$ and $Y$ be topological spaces. Let $\BB$ be a basis for the topology on $X$.
    Let $f : X \rightarrow Y$. Suppose that, for all $B \in \BB$, we have $f(B)$
    is open tn $Y$. Then $f$ is an open map.
\end{proposition}

\begin{proof}
    \pf\ For any $\AA \subseteq \BB$, we have $f(\bigcup \AA) = \bigcup_{B \in \BB} f(B)$
    is open in $Y$. The result follows from Lemma \ref{lemma:basis_unions}. \qed
\end{proof}

\section{Continuous Functions}

\begin{definition}[Continuous]
    Let $X$ and $Y$ be topological spaces. A function $f : X \rightarrow Y$ is \emph{continuous} if and only
    if, for every open set $V$ in $Y$, the set $\inv{f}(V)$ is open in $X$.
\end{definition}

\begin{proposition}
    \label{proposition:continuous_basis}
    Let $X$ and $Y$ be topological spaces and $f : X \rightarrow Y$. Let $\BB$ be a basis for $Y$. Then $f$
    is continuous if and only if, for all $B \in \BB$, we have $\inv{f}(B)$ is open in $X$.
\end{proposition}

\begin{proof}
    \pf
    \step{1}{If $f$ is continuous then, for all $B \in \BB$, we have $\inv{f}(B)$ is open in $X$.}
    \begin{proof}
        \pf\ Since every element of $B$ is open (Lemma \ref{lemma:basis_unions}).
    \end{proof}
    \step{2}{Suppose that, for all $B \in \BB$, we have $\inv{f}(B)$ is open in $X$. Then $f$ is continuous.}
    \begin{proof}
        \step{a}{\assume{For all $B \in \BB$, we have $\inv{f}(B)$ is open in $X$.}}
        \step{b}{\pflet{$V$ be open in $Y$.}}
        \step{c}{\pick\ $\AA \subseteq \BB$ such that $V = \bigcup \AA$}
        \begin{proof}
            \pf\ By Lemma \ref{lemma:basis_unions}.
        \end{proof}
        \step{d}{$\inv{f}(V)$ is open in $X$.}
        \begin{proof}
            \pf
            \begin{align*}
                \inv{f}(V) & = \inv{f} \left( \bigcup \AA \right) \\
                & = \bigcup_{B \in \AA} \inv{f}(B)
            \end{align*}
        \end{proof}
    \end{proof}
    \qed
\end{proof}

\begin{proposition}
    \label{proposition:continuous_subbasis}
    Let $X$ and $Y$ be topological spaces and $f : X \rightarrow Y$. Let $\SS$ be a subbasis for $Y$. Then $f$
    is continuous if and only if, for all $S \in \SS$, we have $\inv{f}(S)$ is open in $X$.
\end{proposition}

\begin{proof}
    \pf
    \step{1}{If $f$ is continuous then, for all $S \in \SS$, we have $\inv{f}(S)$ is open in $X$.}
    \begin{proof}
        \pf\ Since every element of $S$ is open.
    \end{proof}
    \step{2}{Suppose that, for all $S \in \SS$, we have $\inv{f}(S)$ is open in $X$. Then $f$ is continuous.}
    \begin{proof}
        \step{a}{\assume{For all $S \in \SS$, we have $\inv{f}(S)$ is open in $X$.}}
        \step{b}{\pflet{$S_1, \ldots, S_n \in \SS$}}
        \step{c}{$\inv{f}(S_1 \cap \cdots \cap S_n)$ is open in $A$}
        \begin{proof}
            \pf\ Since $\inv{f}(S_1 \cap \cdots \cap S_n) = \inv{f}(S_1) \cap \cdots \cap \inv{f}(S_n)$.
        \end{proof}
        \qedstep
        \begin{proof}
            \pf\ By Propositions \ref{proposition:continuous_basis} and \ref{proposition:subbasis_basis}.
        \end{proof}
    \end{proof}
    \qed
\end{proof}

\begin{proposition}
    Let $X$ and $Y$ be topological spaces and $f : X \rightarrow Y$. Let $\SS$ be a basis for $Y$. Then $f$
    is continuous if and only if, for all $V \in \SS$, we have $\inv{f}(V)$ is open in $X$.
\end{proposition}

\begin{proof}
    \pf
    \step{1}{If $f$ is continuous then, for all $V \in \SS$, we have $\inv{f}(V)$ is open in $X$.}
    \begin{proof}
        \pf\ Since every element of $\SS$ is open.
    \end{proof}
    \step{2}{Suppose that, for all $V \in \SS$, we have $\inv{f}(V)$ is open in $X$. Then $f$ is continuous.}
    \begin{proof}
        \step{a}{\assume{For all $V \in \SS$, we have $\inv{f}(V)$ is open in $X$.}}
        \step{b}{For every set $B$ that is the finite intersection of elemets of $\SS$, we have
        $\inv{f}(B)$ is open in $X$.}
        \begin{proof}
            \pf\ Because $\inv{f}(V_1 \cap \cdots \cap V_n) = \inv{f}(V_1) \cap \cdots \cap \inv{f}(V_n)$.
        \end{proof}
        \qedstep
        \begin{proof}
            \pf\ From Propositions \ref{proposition:subbasis_basis} and \ref{proposition:continuous_basis}.
        \end{proof}
    \end{proof}
    \qed
\end{proof}

\begin{definition}[Continuous at a Point]
    Let $X$ and $Y$ be topological spaces. Let $f : X \rightarrow Y$ and $x \in X$. Then $f$ is
    \emph{continuous at $x$} if and only if, for every neighbourhood $V$ of $f(x)$, there exists a
    neighbourhood $U$ of $x$ such that $f(U) \subseteq V$.
\end{definition}

\begin{theorem}
    \label{theorem:continuous}
    Let $X$ and $Y$ be topological spaces and $f : X \rightarrow Y$. Then the following are equivalent:
    \begin{enumerate}
        \item $f$ is continuous.
        \item For all $A \subseteq X$, we have $f(\overline{A}) \subseteq \overline{f(A)}$
        \item For all $B \subseteq Y$ closed, we have $\inv{f}(B)$ is closed in $X$.
        \item $f$ is continuous at every point of $X$.
    \end{enumerate}
\end{theorem}

\begin{proof}
    \pf
    \step{1}{$1 \Rightarrow 2$}
    \begin{proof}
        \step{a}{\assume{$f$ is continuous.}}
        \step{b}{\pflet{$A \subseteq X$}}
        \step{c}{\pflet{$x \in \overline{A}$} \prove{$f(x) \in \overline{f(A)}$}}
        \step{d}{\pflet{$V$ be a neighbourhood of $f(x)$}}
        \step{e}{$\inv{f}(V)$ is a neighbourhood of $x$}
        \step{f}{\pick\ $y \in A \cap \inv{f}(V)$}
        \begin{proof}
            \pf\ By Theorem \ref{theorem:closure_neighbourhood}.
        \end{proof}
        \step{g}{$f(y) \in V \cap f(A)$}
        \qedstep
        \begin{proof}
            \pf\ By Theorem \ref{theorem:closure_neighbourhood}.
        \end{proof}
    \end{proof}
    \step{2}{$2 \Rightarrow 3$}
    \begin{proof}
        \step{a}{\assume{2}}
        \step{b}{\pflet{$B$ be closed in $Y$}}
        \step{c}{\pflet{$x \in \overline{\inv{f}(B)}$ \prove{$x \in \inv{f}(B)$}}}
        \step{d}{$f(x) \in B$}
        \begin{proof}
            \pf
            \begin{align*}
                f(x) & \in f(\overline{\inv{f}(B)}) \\
                & \subseteq \overline{f(\inv{f}(B))} & (\text{\stepref{a}})\\
                & \subseteq \overline{B} & (Proposition \ref{proposition:closure_monotone}) \\
                & = B
            \end{align*}
        \end{proof}
    \end{proof}
    \step{3}{$3 \Rightarrow 1$}
    \begin{proof}
        \step{a}{\assume{3}}
        \step{b}{\pflet{$V$ be open in $Y$}}
        \step{c}{$Y - V$ is closed in $Y$}
        \step{d}{$\inv{f}(Y - V)$ is closed in $X$}
        \step{e}{$X - \inv{f}(V)$ is closed in $X$}
        \step{f}{$\inv{f}(V)$ is open in $X$}
    \end{proof}
    \step{4}{$1 \Rightarrow 4$}
    \begin{proof}
        \pf\ For any neighbourhood $V$ of $f(x)$, the set $U = \inv{f}(V)$ is a neighbourhood of $x$ such that
        $f(U) \subseteq V$.
    \end{proof}
    \step{5}{$4 \Rightarrow 1$}
    \begin{proof}
        \step{a}{\assume{4}}
        \step{b}{\pflet{$V$ be open in $Y$}}
        \step{c}{\pflet{$x \in \inv{f}(V)$}}
        \step{d}{$V$ is a neighbourhood of $f(x)$}
        \step{e}{\pick\ a neighbourhood $U$ of $x$ such that $f(U) \subseteq V$}
        \step{f}{$x \in U \subseteq \inv{f}(V)$}
        \qedstep
        \begin{proof}
            \pf\ By Lemma \ref{lemma:open}.
        \end{proof}
    \end{proof}
    \qed
\end{proof}

\begin{theorem}
    A constant function is continuous.
\end{theorem}

\begin{proof}
    \pf\ Let $X$ and $Y$ be topological spaces. Let $b \in Y$, and let $f : X \rightarrow Y$ be the constant
    function with value $b$. For any open $V \subseteq Y$, the set $\inv{f}(V)$ is either $X$ (if $b \in V$)
    or $\emptyset$ (if $b \notin V$). \qed
\end{proof}

\begin{theorem}
    \label{theorem:continuous_composite}
    The composite of two continuous functions is continuous.
\end{theorem}

\begin{proof}
    \pf\ Let $f : X \rightarrow Y$ and $g : Y \rightarrow Z$ be continuous. For any $V$ open in $Z$,
    we have $\inv{(g \circ f)}(V) = \inv{f}(\inv{g}(V))$ is open in $X$. \qed
\end{proof}

\begin{theorem}
    \label{theorem:continuous_restriction}
    Let $f : X \rightarrow Y$ be a continuous function and $A$ be a subspace of $X$. Then the restriction
    $f \restriction A : A \rightarrow Y$ is continuous.
\end{theorem}

\begin{proof}
    \pf\ Let $V$ be open in $Y$. Then $\inv{(f \restriction A)}(V) = \inv{f}(V) \cap A$ is open in $A$. \qed
\end{proof}

\begin{theorem}
    Let $f : X \rightarrow Y$ be continuous. Let $Z$ be a subspace of $Y$ such that $f(X) \subseteq Z$. Then
    the corestriction $f : X \rightarrow Z$ is continuous.
\end{theorem}

\begin{proof}
    \pf
    \step{1}{\pflet{$V$ be open in $Z$.}}
    \step{2}{\pick\ $U$ open in $Y$ such that $V = U \cap Z$.}
    \step{3}{$\inv{f}(V) = \inv{f}(U)$}
    \step{4}{$\inv{f}(V)$ is open in $X$.}
    \qed
\end{proof}

\begin{theorem}
    Let $f : X \rightarrow Y$ be continuous. Let $Z$ be a space such that $Y$ is a subspace of $Z$.
    Then the expansion $f : X \rightarrow Z$ is continuous.
\end{theorem}

\begin{proof}
    \pf\ Let $V$ be open in $Z$. Then $\inv{f}(V) = \inv{f}(V \cap Y)$ is open in $X$. \qed
\end{proof}

\begin{theorem}
    Let $X$ and $Y$ be topological spaces. Let $f : X \rightarrow Y$. Suppose $\UU$ is a set of open sets
    in $X$ such that $X = \bigcup \UU$ and, for all $U \in \UU$, we have $f \restriction U : U \rightarrow Y$
    is continuous. Then $f$ is continuous.
\end{theorem}

\begin{proof}
    \pf
    \step{1}{\pflet{$V$ be open in $Y$}}
    \step{2}{$\inv{f}(V) = \bigcup_{U \in \UU} \inv{(f \restriction U)}(V)$}
    \step{3}{For all $U \in \UU$, we have $\inv{(f \restriction U)}(V)$ is open in $U$.}
    \step{4}{For all $U \in \UU$, we have $\inv{(f \restriction U)}(V)$ is open in $X$.}
    \begin{proof}
        \pf\ Lemma \ref{lemma:subspace_open}.
    \end{proof}
    \qed
\end{proof}

\begin{proposition}
    Let $X$ and $X'$ be the same set $X$ under two topologies $\TT$ and $\TT'$. Let $i : X \rightarrow X'$
    be the identity function. Then $i$ is continuous if and only if $\TT' \subseteq \TT$.
\end{proposition}

\begin{proof}
    \pf\ Immediate from definitions. \qed
\end{proof}

\begin{proposition}
    Let $f : \RR \rightarrow \RR$ and $a \in \RR$. Then $f$ is continuous on the right at $a$ if and only if
    $f$ is continuous at $a$ as a function $\RR_l \rightarrow \RR$.
\end{proposition}

\begin{proof}
    \pf
    \step{1}{If $f$ is continuous on the right at $a$ then $f$ is continuous at $a$ as a function $\RR_l
    \rightarrow \RR$.}
    \begin{proof}
        \step{a}{\assume{$f$ is continuous on the right at $a$.}}
        \step{b}{\pflet{$V$ be a neighbourhood of $f(a)$}}
        \step{c}{\pick\ $b$, $c$ such that $f(a) \in (b,c) \subseteq V$.}
        \step{d}{\pflet{$\epsilon = \min(c-f(a),f(a)-b)$}}
        \step{e}{\pick\ $\delta > 0$ such that, for all $x$, if $a < x < a + \delta$ then
        $|f(x)-f(a)|< \epsilon$}
        \step{f}{\pflet{$U = [a,a+\delta)$}}
        \step{g}{$f(U) \subseteq V$}
    \end{proof}
    \step{2}{If $f$ is continuous at $a$ as a function $\RR_l \rightarrow \RR$ then $f$ is continuous on
    the right at $a$.}
    \begin{proof}
        \step{a}{\assume{$f$ is continuous at $a$ as a function $\RR_l \rightarrow \RR$}}
        \step{b}{\pflet{$\epsilon > 0$}}
        \step{c}{\pick\ a neighbourhood $U$ of $a$ such that $f(U) \subseteq (f(a)-\epsilon, f(a)+\epsilon)$}
        \step{d}{\pick\ $b$, $c$ such that $a \in [b,c) \subset U$}
        \step{e}{\pflet{$\delta = c - a$}}
        \step{f}{For all $x$, if $a < x < a + \delta$ then $|f(x)-f(a)| < \epsilon$}
    \end{proof}
    \qed
\end{proof}

\begin{lemma}
    \label{lemma:continuous_open_subspace}
    Let $f : X \rightarrow Y$. Let $Z$ be an open subspace of $X$ and $a \in Z$. If $f \restriction Z$ is continuous at $a$ then $f$ is continuous at $a$.
\end{lemma}

\begin{proof}
    \pf
    \step{1}{\pflet{$V$ be a neighbourhood of $f(x)$}}
    \step{2}{\pick\ a neighbourhood $W$ of $x$ in $Z$ such that $f(W) \subseteq V$}
    \step{3}{$W$ is a neighbourhood of $x$ in $X$ such that $f(W) \subseteq V$}
    \begin{proof}
        \pf\ Lemma \ref{lemma:subspace_open}.
    \end{proof}
    \qed
\end{proof}

\begin{proposition}
    \label{proposition:converge_continuous}
    Let $X$ and $Y$ be topological spaces and $f : X \rightarrow Y$.
    Then $f$ is continuous if and only if, for any net
    $(a_\alpha)_{\alpha \in J}$ in $X$ and $l \in X$,
    if $a_\alpha \rightarrow l$ as $\alpha \rightarrow \infty$ in $X$ then
    $f(a_\alpha) \rightarrow f(l)$ as $\alpha \rightarrow \infty$.
\end{proposition}

\begin{proof}
    \pf
    \step{1}{If $f$ is continuous then, for every net $(a_\alpha)_{\alpha \in J}$
    in $X$ and $l \in X$, if $a_\alpha \rightarrow l$ as $\alpha \rightarrow \infty$
    then $f(a_\alpha) \rightarrow f(l)$ as $\alpha \rightarrow \infty$}
    \begin{proof}
        \step{a}{\assume{$f$ is continuous.}}
        \step{b}{\pflet{$(a_\alpha)_{\alpha \in J}$ be a net in $X$}}
        \step{c}{\pflet{$l \in X$}}
        \step{d}{\assume{$a_\alpha \rightarrow l$ as $\alpha \rightarrow \infty$}}
        \step{1}{\pflet{$V$ be a neighbourhood of $f(l)$}}
        \step{2}{\pick\ a neighbourhood $U$ of $l$ such that $f(U) \subseteq V$}
        \step{3}{\pick\ $\alpha \in J$ such that, for all $\beta \geq \alpha$, we have $a_\beta \in U$}
        \step{4}{For all $\beta \geq \alpha$ we have $f(a_\beta) \in V$}    
    \end{proof}
    \step{2}{If, for every net $(a_\alpha)_{\alpha \in J}$
    in $X$ and $l \in X$, if $a_\alpha \rightarrow l$ as $\alpha \rightarrow \infty$
    then $f(a_\alpha) \rightarrow f(l)$ as $\alpha \rightarrow \infty$,
    then $f$ is continuous.}
    \begin{proof}
        \step{a}{\assume{for every net $(a_\alpha)_{\alpha \in J}$
        in $X$ and $l \in X$, if $a_\alpha \rightarrow l$ as $\alpha \rightarrow \infty$
        then $f(a_\alpha) \rightarrow f(l)$ as $\alpha \rightarrow \infty$}
        \prove{For every $A \subseteq X$, we have $f(\overline{A}) \subseteq
        \overline{f(A)}$}}
        \step{b}{\pflet{$A \subseteq X$}}
        \step{c}{\pflet{$x \in \overline{A}$} \prove{$f(x) \in \overline{f(A)}$}}
        \step{d}{\pick\ a net $(a_\alpha)_{\alpha \in J}$ of points in $A$
        that converges to $x$}
        \begin{proof}
            \pf\ Lemma \ref{lemma:closure_net}.
        \end{proof}
        \step{e}{$(f(a_\alpha))_{\alpha \in J}$ is a net of points in $f(A)$
        that converges to $f(x)$}
        \begin{proof}
            \pf\ From \stepref{a}.
        \end{proof}
        \step{f}{$f(x) \in \overline{f(A)}$}
        \begin{proof}
            \pf\ Lemma \ref{lemma:closure_net}.
        \end{proof}
        \qedstep
        \begin{proof}
            \pf\ Theorem \ref{theorem:continuous}.
        \end{proof}
    \end{proof}
    \qed
\end{proof}

\section{Closed Maps}

\begin{definition}[Closed Map]
    Let $X$ and $Y$ be topological spaces and $f : X \rightarrow Y$.
    Then $f$ is a \emph{closed map} if and only if, for every closed set
    $A$ in $X$, we have $f(A)$ is closed in $Y$.
\end{definition}

\begin{lemma}
    \label{lemma:closed_map_neighbourhoods}
    Let $X$ and $Y$ be topological spaces. Let $f : X \rightarrow Y$
    be a closed map. Let $B \subseteq Y$. Let $U$ be an open
    neighbourhood of $\inv{p}(B)$. Then there exists an open neighbourhood
    $V$ of $B$ such that $\inv{p}(V) \subseteq U$.
\end{lemma}

\begin{proof}
    \pf
    \step{1}{\pflet{$V = Y - p(X - U)$}}
    \step{2}{$V$ is open}
    \begin{proof}
        \pf\ Since $p(X - U)$ is closed.
    \end{proof}
    \step{3}{$B \subseteq V$}
    \begin{proof}
        \pf\ If $p(x) = b \in B$ then $x \in U$.
    \end{proof}
    \step{4}{$\inv{p}(V) \subseteq U$}
    \begin{proof}
        \pf\ If $p(x) \in V$ then $x \in U$.
    \end{proof}
    \qed
\end{proof}

\section{Homeomorphisms}

\begin{definition}[Homeomorphism]
    Let $X$ and $Y$ be topological spaces. A \emph{Homeomorphism} $f$ between $X$ and $Y$, $f : X \cong Y$,
    is a bijection $f : X \rightarrow Y$ such that both $f$ and $\inv{f}$ are continuous.
\end{definition}

\begin{lemma}
    \label{lemma:homeomorphism}
    Let $X$ and $Y$ be topological spaces and $f : X \rightarrow Y$ a bijection. Then the following are
    equivalent:
    \begin{enumerate}
        \item $f$ is a homeomorphism.
        \item $f$ is continuous and an open map.
        \item $f$ is continuous and a closed map.
        \item For any $U \subseteq X$, we have $U$ is open if and only if $f(U)$ is open.
    \end{enumerate}
\end{lemma}

\begin{proof}
    \pf\ Immediate from definitions. \qed
\end{proof}

\begin{proposition}
    Let $X$ and $X'$ be the same set $X$ under two topologies $\TT$ and $\TT'$. Let $i : X \rightarrow X'$
    be the identity function. Then $i$ is a homeomorphism if and only if $\TT = \TT'$.
\end{proposition}

\begin{proof}
    \pf\ Immediate from definitions. \qed
\end{proof}

\begin{definition}[Topological Property]
    Let $P$ be a property of topological spaces. Then $P$ is a \emph{topological} property if and only if,
    for any spaces $X$ and $Y$, if $P$ holds of $X$ and $X \cong Y$ then $P$ holds of $Y$.
\end{definition}

\begin{definition}[Topological Imbedding]
    Let $X$ and $Y$ be topological spaces and $f : X \rightarrow Y$. Then $f$ is a \emph{topological
    imbedding} if and only if the corestriction $f : X \rightarrow f(X)$ is a homeomorphism.
\end{definition}

\begin{proposition}
    \label{proposition:imbedding_product}
    Let $X$ and $Y$ be topological spaces and $a \in X$. The function $i : Y \rightarrow X \times Y$ that maps $y$ to $(a,y)$ is an imbedding.
\end{proposition}

\begin{proof}
    \pf
    \step{1}{$i$ is injective}
    \step{2}{$i$ is continuous.}
    \begin{proof}
        \pf\ For $U$ open in $X$ and $V$ open in $Y$, we have $\inv{i}(U \times V)$ is $V$ if $a \in U$, and $\emptyset$ if $a \notin U$.
    \end{proof}
    \step{3}{$i : Y \rightarrow i(Y)$ is an open map.}
    \begin{proof}
        \pf\ For $V$ open in $Y$ we have $i(V) = (X \times V) \cap i(Y)$.
    \end{proof}
    \qed
\end{proof}

\section{The Order Topology}

\begin{definition}[Order Topology]
    Let $X$ be a linearly ordered set with at least two points. The \emph{order topology} on $X$ is the topology
    generated by the basis $\mathcal{B}$ consisting of:
    \begin{itemize}
        \item all open intervals $(a,b)$;
        \item all intervals of the form $[\bot,b)$ where $\bot$ is least in $X$;
        \item all intervals of the form $(a,\top]$ where $\top$ is greatest in $X$.
    \end{itemize}
\end{definition}

We prove this is a basis for a topology.

\begin{proof}
    \pf
    \step{1}{For all $x \in X$ there exists $B \in \mathcal{B}$ such that $x \in B$.}
    \begin{proof}
        \step{a}{\pflet{$x \in X$}}
        \step{b}{\case{$x$ is greatest in $X$.}}
        \begin{proof}
            \step{i}{\pick\ $y \in X$ with $y \neq x$}
            \step{ii}{$x \in (y,x] \in \BB$}
        \end{proof}
        \step{c}{\case{$x$ is least in $X$.}}
        \begin{proof}
            \step{i}{\pick\ $y \in X$ with $y \neq x$}
            \step{ii}{$x \in [x,y) \in \BB$}
        \end{proof}
        \step{d}{\case{$x$ is neither greatest nor least in $X$.}}
        \begin{proof}
            \step{i}{\pick\ $a, b \in X$ with $a < x$ and $x < b$}
            \step{ii}{$x \in (a,b) \in \BB$}
        \end{proof}
    \end{proof}
    \step{2}{For all $B_1, B_2 \in \mathcal{B}$ and $x \in B_1 \cap B_2$, there exists $B_3 \in \BB$
    such that $x \in B_3 \subseteq B_1 \cap B_2$}
    \begin{proof}
        \step{i}{\pflet{$B_1, B_2 \in \BB$ and $x \in B_1 \cap B_2$}}
        \step{ii}{\case{$B_1 = (a,b)$, $B_2 = (c,d)$}}
        \begin{proof}
            \pf\ Take $B_3 = (\max (a,c), \min (b,d))$.
        \end{proof}
        \step{iii}{\case{$B_1 = (a,b)$, $B_2 = [\bot,d)$}}
        \begin{proof}
            \pf\ Take $B_3 = (a, \min (b,d))$.
        \end{proof}
        \step{iv}{\case{$B_1 = (a,b)$, $B_2 = (c,\top]$}}
        \begin{proof}
            \pf\ Take $B_3 = (\max (a,c), b)$.
        \end{proof}
        \step{v}{\case{$B_1 = [\bot,b)$, $B_2 = [\bot,d)$}}
        \begin{proof}
            \pf\ Take $B_3 = [\bot, \min (b,d))$.
        \end{proof}
        \step{vi}{\case{$B_1 = [\bot, b)$, $B_2 = (c, \top]$}}
        \begin{proof}
            \pf\ Take $B_3 = (c, b)$.
        \end{proof}
    \end{proof}
    \qed
\end{proof}

\begin{lemma}
    \label{lemma:open_rays_subbasis}
    Let $X$ be a linearly ordered set. Then the open rays form a subbasis for the order topology on $X$.
\end{lemma}

\begin{proof}
    \pf
    \step{1}{Every open ray is open.}
    \begin{proof}
        \step{a}{For all $a \in X$, the ray $(- \infty, a)$ is open.}
        \begin{proof}
            \step{i}{\pflet{$x \in (- \infty, a)$}}
            \step{ii}{\case{$x$ is least in $X$}}
            \begin{proof}
                \pf\ $x in [x, a) = (- \infty, a)$.
            \end{proof}
            \step{iii}{\case{$x$ is not least in $X$}}
            \begin{proof}
                \step{one}{\pick\ $y < x$}
                \step{two}{$x \in (y, a) \subseteq (- \infty, a)$}
            \end{proof}
        \end{proof}
        \step{b}{For all $a \in X$, the ray $(a, + \infty)$ is open.}
        \begin{proof}
            \pf\ Similar.
        \end{proof}
    \end{proof}
    \step{2}{Every basic open set is a finite intersection of open rays.}
    \begin{proof}
        \pf\ We have $(a,b) = (a,+\infty) \cap (-\infty, b)$, $[\bot,b) = (- \infty, b)$ and
        $(a, \top] = (a, + \infty)$.
    \end{proof}
    \qed
\end{proof}

\begin{definition}[Standard Topology on the Real Line]
    The \emph{standard topology on the real line} is the order topology on $\RR$ generated by the standard
    order.
\end{definition}

\begin{lemma}
    The lower limit topology is strictly finer than the standard topology on $\RR$.
\end{lemma}

\begin{proof}
    \pf
    \step{1}{Every open interval is open in the lower limit topology.}
    \begin{proof}
        \pf\ If $x \in (a,b)$ then $x \in [x,b) \subseteq (a,b)$.
    \end{proof}
    \step{2}{The half-open interval $[0,1)$ is not open in the standard topology.}
    \begin{proof}
        \pf\ There is no open interval $(a,b)$ such that $0 \in (a,b) \subseteq [0,1)$.
    \end{proof}
    \qed
\end{proof}

\begin{lemma}
    The $K$-topology is strictly finer than the standard topology on $\RR$.
\end{lemma}

\begin{proof}
    \pf
    \step{1}{Every open interval is open in the $K$-topology.}
    \begin{proof}
        \pf\ Corollary \ref{cor:basis_open}.
    \end{proof}
    \step{2}{The set $(-1,1) - K$ is not open in the standard topology.}
    \begin{proof}
        \pf\ There is no open interval $(a,b)$ such that $0 \in (a,b) \subseteq (-1,1) - K$, since
        there must be a positive integer $n$ with $1/n \in (a,b)$.
    \end{proof}
    \qed
\end{proof}

\begin{lemma}
    \label{lemma:order_topology_closed}
    Let $X$ be a topological space.
    Let $Y$ be a linearly ordered set in the order topology. Let $f, g : X \rightarrow Y$ be continuous. Then $C = \{ x \in X \mid f(x) \leq g(x) \}$ is closed.
\end{lemma}

\begin{proof}
    \pf
    \step{1}{\pflet{$x \in X - C$}}
    \step{2}{$f(x) > g(x)$ \prove{There exists a neighbourhood $U$ of $x$ such that $U \subseteq X - C$}}
    \step{3}{\case{There exists $y$ such that $g(x) < y < f(x)$}}
    \begin{proof}
        \pf\ Take $U = \inv{g}((-\infty, y)) \cup \inv{f}(y,+\infty)$.
    \end{proof}
    \step{4}{\case{There is no $y$ such that $g(x) < y < f(x)$}}
    \begin{proof}
        \pf\ Take $U = \inv{g}((-\infty, f(x))) \cup \inv{f}(g(x),+\infty)$.
    \end{proof}
    \qed
\end{proof}

\begin{proposition}
    Let $X$ be a topological space.
    Let $Y$ be a linearly ordered set in the order topology. Let $f, g : X \rightarrow Y$ be continuous. Define $h : X \rightarrow Y$ by $h(x) = \min(f(x),g(x))$.
    Then $h$ is continuous.
\end{proposition}

\begin{proof}
    \pf\ By the Pasting Lemma applied to $\{ x \in X \mid f(x) \leq g(x) \}$ and $\{ x \in X \mid g(x) \leq f(x) \}$, which are closed by Lemma \ref{lemma:order_topology_closed}.
\end{proof}

\begin{proposition}
    \label{proposition:homeomorphism_order_topology}
    Let $X$ and $Y$ be linearly ordered sets in the order topology. Let $f : X \rightarrow Y$ be strictly monotone and surjective. Then $f$ is a homeomorphism.
\end{proposition}

\begin{proof}
    \pf
    \step{1}{$f$ is bijective.}
    \begin{proof}
        \pf\ Proposition \ref{proposition:iso_strictly_monotone_sujective}.
    \end{proof}
    \step{2}{$f$ is continuous.}
    \begin{proof}
        \step{a}{For all $y \in Y$ we have $\inv{f}((y, +\infty))$ is open.}
        \begin{proof}
            \step{i}{\pflet{$y \in Y$}}
            \step{ii}{\pick{$x \in X$ such that $f(x) = y$}}
            \begin{proof}
                \pf\ Since $f$ is surjective.
            \end{proof}
            \step{iii}{$\inv{f}((y, +\infty)) = (x, +\infty)$}
            \begin{proof}
                \pf\ By strict monotoncity.
            \end{proof}
        \end{proof}
        \step{b}{For all $y \in Y$ we have $\inv{f}((-\infty, y))$ is open.}
        \begin{proof}
            \pf\ Similar.
        \end{proof}
    \end{proof}
    \step{3}{$\inv{f}$ is continuous.}
    \begin{proof}
        \step{a}{For all $x \in X$ we have $f((x, +\infty))$ is open.}
        \begin{proof}
            \pf\ $f((x, +\infty)) = (f(x), +\infty)$.
        \end{proof}
        \step{b}{For all $x \in X$ we have $f((-\infty, x))$ is open.}
        \begin{proof}
            \pf\ $f((-\infty, x)) = (-\infty, f(x))$.
        \end{proof}
    \end{proof}
    \qed
\end{proof}

\begin{proposition}
    Every continuous function $S_\Omega \rightarrow \RR$
    is eventually constant.
\end{proposition}

\begin{proof}
    \pf
    \step{1}{\pflet{$f : S_\Omega \rightarrow \RR$ be continuous.}}
    \step{2}{For every $\epsilon > 0$, there exists $\alpha \in S_\Omega$
    such that, for all $\beta > \alpha$, we have $|f(\beta) - f(\alpha)| < \epsilon$}
    \begin{proof}
        \step{a}{\pflet{$\epsilon > 0$}}
        \step{b}{\assume{for a contradiction that, for all $\alpha \in S_\Omega$,
        there exists $\beta > \alpha$ such that $|f(\beta) - f(\alpha)| \geq \epsilon$}}
        \step{c}{Define the sequence $(\alpha_n)$ in $S_\Omega$ by:
        $\alpha_0 = 0$; $\alpha_{n+1}$ is the least element $> \alpha_n$
        such that $|f(\alpha_{n+1}) - f(\alpha_n)| \geq \epsilon$}
        \step{d}{\pflet{$\beta = \sup_n \alpha_n$}}
        \step{e}{$f(\alpha_n) \rightarrow f(\beta)$ as $n \rightarrow \infty$}
        \step{f}{There exists $N$ such that, for all $n \geq N$, we have $|f(\alpha_n) - f(\beta)| < \epsilon / 2$}
        \step{g}{$|f(\alpha_N) - f(\alpha_{N+1})| < \epsilon$}
        \qedstep
        \qed
    \end{proof}
    \step{3}{For $n \in \ZZ^+$, \pflet{$\alpha_n$ be the least element in $S_\Omega$
    such that, for all $\beta > \alpha_n$, we have $|f(\beta) - f(\alpha_n)| < 1/n$}}
    \step{4}{\pflet{$\beta = \sup_n \alpha_n$}}
    \step{5}{For all $\gamma > \beta$ we have $f(\gamma) = f(\beta)$}
    \begin{proof}
        \pf\ Since $|f(\gamma) - f(\beta)| < 1/n$ for all $n \in \ZZ^+$.
    \end{proof}
    \qed
\end{proof}

\begin{theorem}
    Let $X$ be a linearly ordered set. Let $(s_n)$ be a monotone increasing sequence in $X$ with a supremum $s$.
    Then $s_n \rightarrow s$ as $n \rightarrow \infty$.
\end{theorem}

\begin{proof}
    \pf
    \step{1}{\assume{$s$ is not least in $X$.}}
    \begin{proof}
        \pf\ Otherwise $(s_n)$ is the constant sequence $s$ and the result follows from Lemma \ref{lemma:converge_constant}.
    \end{proof}
    \step{2}{\pflet{$U$ be a neighbourhood of $s$.}}
    \step{3}{\pick{$a < s$ such that $(a,s] \subseteq U$}}
    \step{4}{\pick\ $N$ such that $a < a_N$.}
    \step{5}{For all $n \geq N$ we have $a_n \in (a,s]$}
    \step{6}{For all $n \geq N$ we have $a_n \in U$.}
    \qed
\end{proof}

\begin{theorem}
    If $\sum_{i=0}^\infty a_i = s$ and $\sum_{i=0}^\infty b_i = t$ then $\sum_{i=0}^\infty (ca_i + b_i) = cs+t$.
\end{theorem}

\begin{proof}
    \pf $\sum_{i=0}^N (ca_i + b_i) = c \sum_{i=0}^N a_i + \sum_{i=0}^N b_i \rightarrow cs+t$ as $n \rightarrow \infty$. \qed
\end{proof}

\begin{theorem}[Comparison Test]
    If $|a_i| \leq b_i$ for all $i$ and $\sum_{i=0}^\infty b_i$ converges then $\sum_{i=0}^\infty a_i$ converges.
\end{theorem}

\begin{proof}
    \pf
    \step{1}{$\sum_{i=0}^\infty |a_i|$ converges}
    \begin{proof}
        \pf\ The partial sums $\sum_{i=0}^N |a_i|$ form an increasing sequence bounded above by $\sum_{i=0}^\infty b_i$.
    \end{proof}
    \step{2}{\pflet{$c_i = |a_i| + a_i$ for all $i$}}
    \step{3}{$\sum_{i=0}^\infty c_i$ converges}
    \begin{proof}
        \pf\ Each $c_i$ is either $2|a_i|$ or 0. So the partial sums $\sum_{i=0}^N c_i$ form an increasing sequence bounded above by $2 \sum_{i=0}^\infty b_i$.
    \end{proof}
    \qedstep
    \begin{proof}
        \pf\ Since $a_i = c_i - |a_i|$.
    \end{proof}
    \qed
\end{proof}

\begin{corollary}
    If $\sum_{i=0}^\infty |a_i|$ converges then $\sum_{i=0}^\infty a_i$ converges.
\end{corollary}

\begin{theorem}[Weierstrass $M$-test]
    Let $X$ be a set and $(f_n : X \rightarrow \RR)$ be a sequence of functions. Let
    \[ s_n(x) = \sum_{i=0}^n f_i(x) \]
    for all $n$, $x$. Suppose $|f_i(x)| \leq M_i$ for all $i \geq 0$ and $x \in X$.
    If the series $\sum_{i=0}^\infty M_i$ converges, then the sequence $(s_n)$ converges uniformly to
    \[ s(x) = \sum_{i=0}^\infty f_i(x) \enspace . \]
\end{theorem}

\begin{proof}
    \pf
    \step{1}{\pflet{$r_n = \sum_{i=n+1}^\infty M_i$ for all $n$}}
    \step{2}{Given $0 \leq n < k$, we have $|s_k(x) - s_n(x)| \leq r_n$}
    \begin{proof}
        \pf
        \begin{align*}
            |s_k(x) - s_n(x)| & = |\sum_{i=n+1}^k f_i(x)| \\
            & \leq \sum_{i=n+1}^k |f_i(x)| \\
            & \leq \sum_{i=n+1}^k M_i \\
            & \leq r_n
        \end{align*}
    \end{proof}
    \step{3}{Given $n \geq 0$ we have $|s(x) - s_n(x)| \leq r_n$}
    \begin{proof}
        \pf\ By taking the limit $k \rightarrow \infty$ in \stepref{2}.
    \end{proof}
    \qedstep
    \begin{proof}
        \pf\ Since $r_n \rightarrow 0$ as $n \rightarrow \infty$.
    \end{proof}
    \qed
\end{proof}

\section{The nth Root Function}

\begin{proposition}
    For all $n \geq 1$, the function $f : \RR_{\geq 0} \rightarrow \RR_{\geq 0}$ defined by $f(x) = x^n$ is a homemorphism.
\end{proposition}

\begin{proof}
    \pf
    \step{1}{$f$ is strictly monotone.}
    \begin{proof}
        \step{a}{\pflet{$x,y \in \RR$ with $0 \leq x < y$}}
        \step{b}{$x^n < y^n$}
        \begin{proof}
            \begin{align*}
                y^n - x^n & = (y - x)(y^{n-1} + y^{n-2}x + y^{n-3}x^2 + \cdots + x^{n-1}) \\
                & > 0
            \end{align*}
        \end{proof}
    \end{proof}
    \step{2}{$f$ is surjective.}
    \begin{proof}
        \step{a}{\pflet{$y \in \RR_{\geq 0}$}}
        \step{b}{\pick\ $x \in \RR$ such that $y \leq x^n$}
        \begin{proof}
            \pf\ If $y \leq 1$ take $x = 1$, otherwise take $x = y$.
        \end{proof}
        \step{c}{There exists $x' \in [0,x]$ such that $(x')^n = y$}
        \begin{proof}
            \pf\ By the Intermediate Value Theorem.
        \end{proof}
    \end{proof}
    \qedstep
    \begin{proof}
        \pf\ Proposition \ref{proposition:homeomorphism_order_topology}.
    \end{proof}
    \qed
\end{proof}

\begin{definition}
    For $n \geq 1$, the \emph{$n$th root function} is the function $\RR_{\geq 0} \rightarrow \RR_{\geq 0}$ that is the inverse of $\lambda x.x^n$.
\end{definition}

\section{The Product Topology}

\begin{definition}[Product Topology]
    Let $\{ A_i \}_{i \in I}$ be a family of topological spaces. The \emph{product topology} on $\prod_{i \in I} A_i$
    is the topology generated by the subbasis consisting of the sets of the form $\inv{\pi_i}(U)$ where $i \in I$
    and $U$ is open in $A_i$.

    In the case all the $A_i$ are equal, the product topology on $A^I$ is also called the \emph{topology of pointwise convergence}
    or the \emph{point-open topology}.
\end{definition}

\begin{proposition}
    \label{proposition:product_basis}
    The product topology on $\prod_{i \in I} A_i$ is generated by the basis consisting of all sets of the form $\prod_{i \in I} U_i$
    where $\{ U_i \}_{i \in I}$ is a family such that each $U_i$ is an open set in $A_i$ and $U_i = A_i$ for all but finitely many $i$.
\end{proposition}

\begin{proof}
    \pf\ From Proposition \ref{proposition:subbasis_basis}. \qed
\end{proof}

\begin{proposition}
    \label{proposition:closed_product}
    If $A_i$ is closed in $X_i$ for all $i \in I$ then $\prod_{i \in I} A_i$ is closed in $\prod_{i \in I} X_i$.
\end{proposition}

\begin{proof}
    \pf
    \[ (\prod_{i \in I} X_i) - (\prod_{i \in I} A_i) = \bigcup_{j \in I} \left( \prod_{i \in I} X_i - \inv{\pi_j}(A_j) \right) \qed \]
\end{proof}

\begin{proposition}
    \label{proposition:open_map_projections}
    Let $\{ A_i \}_{i \in I}$ be a family of topological spaces. Then the projections $\pi_i : \prod_{i \in I} A_i \rightarrow A_i$ are open maps.
\end{proposition}

\begin{proof}
    \pf\ From Lemma \ref{lemma:open_map_basis}. \qed
\end{proof}

\begin{example}
    The projections are not always closed maps. For example, $\pi_1 : \RR^2 \rightarrow \RR$ maps the closed set $\{ (x,1/x) \mid x > 0 \}$ to $(0, + \infty)$.
\end{example}

\begin{proposition}
    Let $\{ X_i \}_{i \in I}$ be a family of sets. For $i \in I$, let $\TT_i$ and $\UU_i$ be topologies
    on $X_i$. Let $\mathcal{P}$ be the product topology on $\prod_{i \in I} X_i$ generated by
    the topologies $\TT_i$, and $\mathcal{Q}$ the product topology on the same set generated by the topologies
    $\UU_i$. Then $\mathcal{P} \subseteq \mathcal{Q}$ if and only if $\TT_i \subseteq \UU_i$ for all $i$.
\end{proposition}

\begin{proof}
    \pf
    \step{2}{If $\TT_i \subseteq \UU_i$ for all $i$ then $\mathcal{P} \subseteq \mathcal{Q}$}
    \begin{proof}
        \pf\ By Corollary \ref{cor:basis_open}.
    \end{proof}
    \step{3}{If $\mathcal{P} \subseteq \mathcal{Q}$ then $\TT_i \subseteq \UU_i$ for all $i$}
    \begin{proof}
        \step{a}{\assume{$\mathcal{P} \subseteq \mathcal{Q}$}}
        \step{b}{\pflet{$i \in I$}}
        \step{c}{\pflet{$U \in \TT_i$}}
        \step{d}{\pflet{$U_i = U$ and $U_j = X_j$ for $j \neq i$}}
        \step{e}{$\prod_{i \in I} U_i \in \mathcal{P}$}
        \step{f}{$\prod_{i \in I} U_i \in \mathcal{Q}$}
        \step{g}{$U \in \UU_i$}
        \begin{proof}
            \pf\ From Proposition \ref{proposition:open_map_projections}.
        \end{proof}
    \end{proof}
    \qed
\end{proof}

\begin{proposition}[Choice]
    \label{proposition:closure_in_product_topology_without_choice}
    Let $\{ X_i \}_{i \in I}$ be a family of topological spaces. Let $A_i \subseteq X_i$ for all $i \in I$.
    Then
    \[ \overline{\prod_{i \in I} A_i} \subseteq \prod_{i \in I} \overline{A_i} \enspace . \]
\end{proposition}

\begin{proof}
    \step{a}{For all $i \in I$ we have $A_i \subseteq \overline{A_i}$}
    \begin{proof}
        \pf\ Lemma \ref{lemma:closure_subset}.
    \end{proof}
    \step{c}{$\prod_{i \in I} A_i \subseteq \prod_{i \in I} \overline{A_i}$}
    \qedstep
    \begin{proof}
        \pf\ Since $\prod_{i \in I} A_i$ is closed by Proposition \ref{proposition:closed_product}.
    \end{proof}
\end{proof}

\begin{proposition}[Choice]
    \label{proposition:closure_in_product_topology}
    Let $\{ X_i \}_{i \in I}$ be a family of topological spaces. Let $A_i \subseteq X_i$ for all $i \in I$.
    Then
    \[ \prod_{i \in I} \overline{A_i} = \overline{\prod_{i \in I} A_i} \enspace . \]
\end{proposition}

\begin{proof}
    \pf
    \step{1}{$\overline{\prod_{i \in I} A_i} \subseteq \prod_{i \in I} \overline{A_i}$}
    \begin{proof}
        \pf\ Proposition \ref{proposition:closure_in_product_topology_without_choice}.
    \end{proof}
    \step{2}{$\prod_{i \in I} \overline{A_i} \subseteq \overline{\prod_{i \in I} A_i}$}
    \begin{proof}
        \step{a}{\pflet{$x \in \prod_{i \in I} \overline{A_i}$}}
        \step{b}{\pflet{$U$ be a neighbourhood of $x$}}
        \step{c}{\pick\ $V_i$ open in $X_i$ such that $x \in \prod_{i \in I} V_i \subseteq U$ with $V_i = X_i$ except for $i = i_1, \ldots, i_n$}
        \step{d}{For $i \in I$, pick $a_i \in V_i \cap A_i$}
        \begin{proof}
            \pf\ By Theorem \ref{theorem:closure_neighbourhood} and \stepref{a} using the Axiom of Choice.
        \end{proof}
        \step{f}{$U$ intersects $\prod_{i \in I} A_i$}
        \qedstep
        \begin{proof}
            \pf\ $a \in U \cap \prod_{i \in I} A_i$
        \end{proof}
    \end{proof}
    \qed
\end{proof}

\begin{example}
    The closure of $\RR^\infty$ in $\RR^\omega$ is $\RR^\omega$
\end{example}

\begin{proof}
    \pf
    \step{1}{\pflet{$a \in \RR^\omega$}}
    \step{2}{\pflet{$U$ be any neighbourhoods of $a$.}}
    \step{3}{\pick\ $U_n$ open in $\RR$ for all $n$ such that $a \in \prod_{n \geq 0} U_n \subseteq U$ and $U_n = \RR$ for all $n$ except $n_1$, \ldots, $n_k$}
    \step{4}{\pflet{$b_n = a_n$ for $n = n_1, \ldots, n_k$ and $b_n = 0$ for all other $n$}}
    \step{5}{$b \in \RR^\infty \cap U$}
    \qedstep
    \begin{proof}
        \pf\ From Theorem \ref{theorem:closure_neighbourhood}.
    \end{proof}
    \qed
\end{proof}

\begin{proposition}
    \label{proposition:converge_product}
    Let $\{ X_i \}_{i \in I}$ be a family of topological spaces. Let $(a_\alpha)_{\alpha \in J}$ be a net in $\prod_{i \in I} X_i$ and $l \in \prod_{i \in I} X_i$.
    Then $a_\alpha \rightarrow l$ as $\alpha \rightarrow \infty$ if and only if, for all $i \in I$, we have $\pi_i(a_\alpha) \rightarrow \pi_i(l)$ as $\alpha \rightarrow \infty$.
\end{proposition}

\begin{proof}
    \pf
    \step{1}{If $a_\alpha \rightarrow l$ as $\alpha \rightarrow \infty$ then, for all $i \in I$, we have $\pi_i(a_\alpha) \rightarrow \pi_i(l)$ as $n \rightarrow \infty$}
    \begin{proof}
        \pf\ Proposition \ref{proposition:converge_continuous}. %TODO
    \end{proof}
    \step{2}{If, for all $i \in I$, we have $\pi_i(a_\alpha) \rightarrow \pi_i(l)$ as $\alpha \rightarrow \infty$, then $a_\alpha \rightarrow l$ as $\alpha \rightarrow \infty$}
    \begin{proof}
        \step{a}{\assume{For all $i \in I$, we have $\pi_i(a_\alpha) \rightarrow \pi_i(l)$ as $\alpha \rightarrow \infty$}}
        \step{b}{\pflet{$V$ be a neighbourhood of $l$}}
        \step{c}{\pick\ open sets $U_i$ in $X_i$ such that $l \in \prod_{i \in I} U_i \subseteq V$ and $U_i = X_i$ for all $i$ except $i = i_1, \ldots, i_k$}
        \step{d}{For $j = 1, \ldots, k$, \pick\ $\alpha_j$ such that, for all $\beta \geq \alpha_j$, we have $\pi_{i_j}(a_\beta) \in U_{i_j}$}
        \step{e}{\pick\ $\alpha \in J$ such that $\alpha_1, \ldots, \alpha_k \leq \alpha$}
        \step{f}{For all $\beta \geq \alpha$ we have $a_\beta \in V$}
    \end{proof}
    \qed
\end{proof}

\begin{corollary}
    Let $X$ be a set and $Y$ a topological space. Let $(f_n)$ be a sequence in $Y^X$ and $f \in Y^X$.
    Then $f_n \rightarrow f$ in the product topology if and only if $f_n$ converges to $f$ pointwise,
    i.e. $\forall x \in X. f_n(x) \rightarrow f(x)$.
\end{corollary}

\begin{corollary}
    There exist topological spaces $X$, $Y$ such that $\mathcal{C}(X,Y)$ is not closed in $Y^X$.
\end{corollary}

\begin{proof}
    \pf\ Take $X = I$ and $Y = \RR$. Define $f_n : X \rightarrow Y$ by $f_n(x) = x^n$, and $f : X \rightarrow Y$ by
    $f(x) = 0$ for $x < 1$ and $f(1) = 1$. Then $f_n \rightarrow f$ in $Y^X$, but $f_n \in \mathcal{C}(X,Y)$ for all $n$
    and $f \notin \mathcal{C}(X,Y)$.
\end{proof}

\begin{theorem}
    \label{theorem:product_continuous}
        Let $A$ be a topological space and $\{ X_i \}_{i \in I}$ be a family of topological spaces. Let $f : A \rightarrow \prod_{i \in I} X_i$ be a function.
        If $\pi_i \circ f$ is continuous for all $i \in I$ then $f$ is continuous.
\end{theorem}

\begin{proof}
    \pf
    \step{1}{\pflet{$i \in I$ and $U$ be open in $X_i$}}
    \step{2}{$\inv{f}(\inv{\pi_i}(U))$ is open in $A$}
    \qedstep
    \begin{proof}
        \pf\ Proposition \ref{proposition:continuous_subbasis}.
    \end{proof}
    \qed
\end{proof}

\begin{proposition}
    Let $f : A \rightarrow B$ and $g : C \rightarrow D$ be continuous. Define $f \times g : A \times C \rightarrow B \times D$ by
    \[ (f \times g)(a,c) = (f(a), g(c)) \enspace . \]
    Then $f \times g$ is continuous.
\end{proposition}

\begin{proof}
    \pf\ $\pi_1 \circ (f \times g) = f \circ \pi_1$ and $\pi_2 \circ (f \times g) = g \circ \pi_2$ are continuous
    by Theorem \ref{theorem:continuous_composite}. The result follows by Theorem \ref{theorem:product_continuous}.
\end{proof}

\begin{proposition}
    Let $X$ be a set and $Y$ a topological space. The product topology on $Y^X$ is the topology generated by the subbasis
    \[ \{ S(x,U) \mid x \in X, U \text{ open in } Y \} \]
    where
    \[ S(x,U) = \{ f : X \rightarrow Y \mid f(x) \in U \} \enspace . \]
\end{proposition}

\subsection{Continuous in Each Variable Separately}

\begin{definition}[Continuous in Each Variable Separately]
    Let $F : X \times Y \rightarrow Z$. Then $F$ is \emph{continuous in each
    variable separately} if and only if:
    \begin{itemize}
        \item for every $a \in X$ the function $\lambda y \in Y. F(a,y)$ is continuous;
        \item for every $b \in Y$ the function $\lambda x \in X. F(x,b)$ is continuous.
    \end{itemize}
\end{definition}

\begin{proposition}
    Let $F : X \times Y \rightarrow Z$. If $F$ is continuous then $F$ is continuous in each
    variable separately.
\end{proposition}

\begin{proof}
    \pf\ For $a \in X$, the function $\lambda y \in Y. F(a,y)$ is $F \circ i$ where $i : Y \rightarrow X \times Y$ maps $y$ to $(a,y)$.
    We have $i$ is continuous by Proposition \ref{proposition:imbedding_product}, hence $F \circ i$ is continuous by Theorem \ref{theorem:continuous_composite}.

    Similarly for $\lambda x \in X. F(x,b)$ for $b \in Y$. \qed
\end{proof}

\begin{example}
    Define $F : \RR \times \RR \rightarrow \RR$ by
    \[ F(x,y) = \begin{cases}
        xy / (x^2 + y^2) & \text{if } (x,y) \neq (0,0) \\
        0 & \text{if } (x,y) = (0,0)
    \end{cases} \]
    Then $F$ is continuous in each variable separately but not continuous.
\end{example}

\begin{proposition}
    \label{proposition:open_map_product}
    Let $f : A \rightarrow C$ and $g : B \rightarrow D$ be open maps. Then $f \times g : A \times B \rightarrow C \times D$ is an open map.
\end{proposition}

\begin{proof}
    \pf\ Given $U$ open in $A$ and $V$ open in $B$. Then $(f \times g)(U \times V) = f(U) \times g(V)$ is open in $C \times D$.
    The result follows from Proposition \ref{proposition:open_map_basis}. \qed
\end{proof}

\begin{definition}[Sorgenfrey Plane]
    The \emph{Sorgenfrey plane} is $\RR_l^2$.    
\end{definition}

\section{The Subspace Topology}

\begin{definition}[Subspace Topology]
    Let $X$ be a topological space and $Y \subseteq X$. The \emph{subspace topology} on $Y$ is
    $\TT = \{ U \cap Y \mid U \text{ is open in } X \}$.
\end{definition}

We prove this is a topology.

\begin{proof}
    \pf
    \step{1}{$Y \in \TT$}
    \begin{proof}
        \pf\ Since $Y = X \cap Y$
    \end{proof}
    \step{2}{For all $\UU \subseteq \TT$, we have $\bigcup \UU \in \TT$}
    \begin{proof}
        \step{a}{\pflet{$\UU \subseteq \TT$}}
        \step{b}{\pflet{$\VV = \{ V \text{ open in } X \mid V \cap Y \in \UU \}$}}
        \step{c}{$\bigcup \UU = \left( \bigcup \VV  \right) \cap Y$}
    \end{proof}
    \step{3}{For all $U, V \in \TT$, we have $U \cap V \in \TT$}
    \begin{proof}
        \step{a}{\pflet{$U, V \in \TT$}}
        \step{b}{\pick\ $U'$, $V'$ open in $X$ such that $U = U' \cap Y$ and $V = V' \cap Y$}
        \step{c}{$(U \cap V) = (U' \cap V') \cap Y$}
    \end{proof}
    \qed
\end{proof}

\begin{theorem}
    \label{theorem:closed}
    Let $X$ be a topological space and $Y$ a subspace of $X$. Let $A \subseteq Y$. Then $A$ is closed in $Y$
    if and only if there exists a closed set $C$ in $X$ such that $A = C \cap Y$.
\end{theorem}

\begin{proof}
    \pf\ 
    We have
    \begin{align*}
        & A \text{ is closed in } Y \\
        \Leftrightarrow & Y - A \text{ is open in } Y \\
        \Leftrightarrow & \exists U \text{ open in } X. Y - A = Y \cap U \\
        \Leftrightarrow & \exists C \text{ closed in } X. Y - A = Y \cap (X - U) \\
        \Leftrightarrow & \exists C \text{ closed in } X. A = Y \cap U & \qed
    \end{align*}
\end{proof}

\begin{theorem}
    Let $Y$ be a subspace of $X$. Let $A \subseteq Y$. Let $\overline{A}$ be the closure of $A$ in $X$.
    Then the closure of $A$ in $Y$ is $\overline{A} \cap Y$.
\end{theorem}

\begin{proof}
    \pf\ The closure of $A$ in $Y$ is
    \begin{align*}
        & \bigcap \{ C \text{ closed in } Y \mid A \subseteq C \} \\
        = & \bigcap \{ D \cap Y \mid D \text{ closed in } X, A \subseteq D \cap Y \} 
            & (\text{Theorem \ref{theorem:closed}})\\
        = & \bigcap \{ D \mid D \text{ closed in } X, A \subseteq D \} \cap Y \\
        = & \overline{A} \cap Y & \qed
    \end{align*}
\end{proof}

\begin{lemma}
    \label{lemma:subspace_basis}
    Let $X$ be a topological space and $Y \subseteq X$. Let $\BB$ be a basis for the topology on $X$.
    Then $\BB' = \{ B \cap Y \mid B \in \BB \}$ is a basis for the subspace topology on $Y$.
\end{lemma}

\begin{proof}
    \pf
    \step{1}{Every element in $\BB'$ is open in $Y$}
    \step{2}{For every open set $U$ in $Y$ and point $y \in U$, there exists $B' \in \BB'$ such that
    $y \in B' \subseteq U$}
    \begin{proof}
        \step{a}{\pflet{$U$ be open in $Y$ and $y \in U$}}
        \step{b}{\pick\ $V$ open in $X$ such that $U = V \cap Y$}
        \step{c}{\pick\ $B \in \BB$ such that $y \in B \subseteq V$}
        \step{d}{\pflet{$B' = B \cap Y$}}
        \step{e}{$B' \in \BB'$}
        \step{f}{$y \in B' \subseteq U$}
    \end{proof}
    \qedstep
    \begin{proof}
        \pf\ By Lemma \ref{lemma:basis}.
    \end{proof}
    \qed
\end{proof}

\begin{lemma}
    \label{lemma:subspace_subbasis}
    Let $X$ be a topological space and $Y \subseteq X$. Let $\SS$ be a basis for the topology on $X$.
    Then $\SS' = \{ S \cap Y \mid S \in \SS \}$ is a subbasis for the subspace topology on $Y$.
\end{lemma}

\begin{proof}
    \pf\ The set $\{ B \cap Y \mid B \text{ is a finite intersection of elements of } \SS \}$ is a basis
    for the subspace topology by Lemma \ref{lemma:subspace_basis}, and this is the set of all finite
    intersections of elements of $\SS'$. \qed
\end{proof}

\begin{lemma}
    \label{lemma:subspace_open}
    Let $Y$ be a subspace of $X$. If $U$ is open in $Y$ and $Y$ is open in $X$ then $U$ is open in $X$.
\end{lemma}

\begin{proof}
    \pf
    \step{1}{\pick\ $V$ open in $X$ such that $U = V \cap Y$}
    \step{2}{$U$ is open in $X$}
    \begin{proof}
        \pf\ Since it is the intersection of two open sets $V$ and $Y$.
    \end{proof}
    \qed
\end{proof}

\begin{theorem}
    \label{theorem:closed_subspace}
    Let $Y$ be a subspace of $X$ and $A \subseteq Y$. If $A$ is closed in $Y$ and $Y$ is closed in $X$
    then $A$ is closed in $X$.
\end{theorem}

\begin{proof}
    \pf\ Pick a closed set $C$ in $X$ such that $A = C \cap Y$ (Theorem \ref{theorem:closed}). Then
    $A$ is the intersection of two sets closed in $X$, hence $A$ is closed in $X$ (Lemma \ref{lemma:closed_intersection}). \qed
\end{proof}

\begin{theorem}[Pasting Lemma]
    Let $X$ and $Y$ be topological spaces and $f : X \rightarrow Y$. Let $\{ A_\alpha \}$ be a locally finite family of closed subsets of $X$ that cover $X$. Suppose $f \restriction A_\alpha$
    is continuous for all $\alpha$. Then $f$ is continuous.
\end{theorem}

\begin{proof}
    \pf
    \step{0}{Let $X$ and $Y$ be topological spaces and $f : X \rightarrow Y$. Let $A$ and $B$ be closed subsets of $X$ such that $X = A \cup B$. 
    Suppose $f \restriction A$ and $f \restriction B$ are continuous. Then $f$ is continuous.}
    \begin{proof}
        \step{1}{\pflet{$C \subseteq Y$ be closed.}}
        \step{2}{$\inv{h}(C) = \inv{f}(C) \cup \inv{g}(C)$}
        \step{3}{$\inv{f}(C)$ and $\inv{g}(C)$ are closed in $X$.}
        \begin{proof}
            \pf\ Theorems \ref{theorem:continuous} and \ref{theorem:closed_subspace}.
        \end{proof}
        \step{4}{$\inv{h}(C)$ is closed in $X$.}
        \begin{proof}
            \pf\ Lemma \ref{lemma:closed_union}.
        \end{proof}
        \qedstep
        \begin{proof}
            \pf\ Theorem \ref{theorem:continuous}.
        \end{proof}    
    \end{proof}
    \step{1}{ Let $X$ and $Y$ be topological spaces and $f : X \rightarrow Y$. Let $\{ A_\alpha \}$ be a finite family of closed subsets of $X$ that cover $X$. Suppose $f \restriction A_\alpha$
    is continuous for all $\alpha$. Then $f$ is continuous.}
    \begin{proof}
        \pf\ From \stepref{0} by induction.
    \end{proof}
    \step{2}{ Let $X$ and $Y$ be topological spaces and $f : X \rightarrow Y$. Let $\{ A_\alpha \}$ be a locally finite family of closed subsets of $X$ that cover $X$. Suppose $f \restriction A_\alpha$
    is continuous for all $\alpha$. Then $f$ is continuous.}
    \begin{proof}
        \step{1a}{\pflet{$x \in X$} \prove{$f$ is continuous at $x$}}
        \step{2}{\pick\ a neighbourhood $U$ of $x$ that intersects $A_\alpha$ for only finitely many $\alpha$.}
        \step{3}{$f \restriction U$ is continuous}
        \begin{proof}
            \pf\ By \stepref{1}.
        \end{proof}
        \qedstep
        \begin{proof}
            \pf\ Lemma \ref{lemma:continuous_open_subspace}.
        \end{proof}
    \end{proof}
    \qed
\end{proof}

The following example shows that we cannot remove the assumption of local finiteness.

\begin{example}
    Define $f : [-1,1] \rightarrow \RR$ by: $f(x) = 1$ if $x < -1$, $f(x) = 0$ if $x > 1$. Let $C_n = [-1,-1/n]$ for $n \geq 1$, and $D = [0,1]$. Then
    $[-1,1] = \bigcup_{n=1}^\infty C_n \cup D$ and $f$ is continuous on each $C_n$ and each $D$, but $f$ is not continuous on $[-1,1]$.    
\end{example}

\begin{theorem}
    Let $\{ X_i \}_{i \in I}$ be a family of topological spaces. Let $A_i$ be a subspace of $X_i$ for all $i \in I$.
    Then the product topology on $\prod_{i \in I} A_i$ is the same as the topology it inherits as a subspace of
    $\prod_{i \in I} X_i$.
\end{theorem}

\begin{proof}
    \pf\ The product topology is generated by the subbasis
    \begin{align*}
        & \{ \inv{\pi_i}(U) \mid i \in I, U \text{ open in } A_i \} \\
        = & \{ \inv{\pi_i}(V) \cap A_i \mid i \in I, V \text{ open in } X_i \} \\
        = & \{ \inv{\pi_i}(V) \mid i \in I, V \text{ open in } X_i \} \cap \prod_{i \in I} A_i
    \end{align*}
    and this is a subbasis for the subspace topology by Lemma \ref{lemma:subspace_subbasis}. \qed
\end{proof}

\begin{theorem}
    Let $X$ be an ordered set in the order topology. Let $Y \subseteq X$ be an interval. Then the order topology
    on $Y$ is the same as the subspace topology on $Y$.
\end{theorem}

\begin{proof}
    \pf
    \step{1}{The order topology is finer than the subspace topology.}
    \begin{proof}
        \step{a}{For every open ray $R$ in $X$, the set $R \cap Y$ is open in the order topology.}
        \begin{proof}
            \step{i}{For all $a \in X$, we have $(-\infty, a) \cap Y$ is open in the order topology.}
            \begin{proof}
                \step{one}{\case{For all $y \in Y$ we have $y < a$}}
                \begin{proof}
                    \pf\ In this case $(-\infty, a) \cap Y = Y$.
                \end{proof}
                \step{two}{\case{For all $y \in Y$ we have $a < y$}}
                \begin{proof}
                    \pf\ In this case $(-\infty, a) \cap Y = \emptyset$.
                \end{proof}
                \step{three}{\case{There exists $y \in Y$ such that $y \leq a$ and $y \in Y$ such that
                $a \leq y$}}
                \begin{proof}
                    \step{aa}{$a \in Y$}
                    \begin{proof}
                        \pf\ Because $Y$ is an interval.
                    \end{proof}
                    \step{bb}{$(-\infty, a) \cap Y = \{ y \in Y \mid y < a \}$}
                \end{proof}
            \end{proof}
            \step{ii}{For all $a \in X$, we have $(a, +\infty) \cap Y$ is open in the order topology.}
            \begin{proof}
                \pf\ Similar.
            \end{proof}
        \end{proof}
        \qedstep
        \begin{proof}
            \pf\ By Lemmas \ref{lemma:open_rays_subbasis} and \ref{lemma:subspace_subbasis} and Proposition
            \ref{proposition:subbasis_coarsest}.
        \end{proof}
    \end{proof}
    \step{2}{The subspace topology is finer than the order topology.}
    \begin{proof}
        \step{a}{Every open ray in $Y$ is open in the subspace topology.}
        \begin{proof}
            \pf\ For any $a \in Y$ we have $(-\infty, a)_Y = (-\infty, a)_X \cap Y$
            and $(a, +\infty)_Y = (a,+\infty)_X \cap Y$.
        \end{proof}
        \qedstep
        \begin{proof}
            \pf\ By Lemma \ref{lemma:open_rays_subbasis} and  Proposition
            \ref{proposition:subbasis_coarsest}
        \end{proof}
    \end{proof}
    \qed
\end{proof}

This example shows that we cannot remove the hypothesis that $Y$ is an interval:

\begin{example}
    The order topology on $I_o^2$ is different from the subspace topology as a subspace of $\RR^2$ under
    the dictionary order topology.
\end{example}

\begin{proof}
    \pf\ The set $\{ 1/2 \} \times (1/2, 1)$ is open in the subspace topology but not in the order topology.
    \qed
\end{proof}

\begin{proposition}
    \label{proposition:subspace_transitive}
    Let $X$ be a topological space, $Y$ a subspace of $X$, and $Z$ a subspace of $Y$. Then the subspace
    topology on $Z$ inherited from $X$ is the same as the subspace topology on $Z$ inherited from $Y$.
\end{proposition}

\begin{proof}
    \pf\ The subspace topology inherited from $Y$ is
    \begin{align*}
        & \{ V \cap Z \mid V \text{ open in } Y \} \\
        = & \{ U \cap Y \cap Z \mid U \text{ open in } X \} \\
        = & \{ U \cap Z \mid U \text{ open in } X \}
    \end{align*}
    which is the subspace topology inherited from $X$. \qed
\end{proof}

\begin{definition}[Unit Circle]
    The \emph{unit circle} $S^1$ is $\{ (x,y) \in \RR^2 \mid x^2 + y^2 = 1 \}$ as a subspace of $\RR^2$.
\end{definition}

\begin{definition}[Unit 2-sphere]
    The \emph{unit 2-sphere} is $S^2 = \{ (x,y,z) \mid x^2 + y^2 + z^2 \leq 1 \}$ as a subspace of $\RR^3$.
\end{definition}

\begin{proposition}
    Let $f : X \rightarrow Y$ be an open map and $A \subseteq X$ be open. Then the restriction $f \restriction A : A \rightarrow f(A)$ is an open map.
\end{proposition}

\begin{proof}
    \pf
    \step{1}{\pflet{$U$ be open in $A$}}
    \step{2}{$U$ is open in $X$}
    \begin{proof}
        \pf\ Lemma \ref{lemma:subspace_open}.
    \end{proof}
    \step{3}{$f(U)$ is open in $Y$}
    \step{4}{$f(U)$ is open in $f(A)$}
    \begin{proof}
        \pf\ Since $f(U) = f(U) \cap f(A)$.
    \end{proof}
    \qed
\end{proof}

\begin{example}
    This example shows that we cannot remove the hypothesis that $A$ is open.

    Let $A = \{ (x,y) \in \RR^2 \mid (x > 0 \text{ and } y = 1/x) \text{ or } x=y=0 \}$.
    Then $\pi_1 : \RR^2 \rightarrow \RR$ is an open map, but $\pi_1 \restriction A : A \rightarrow [0, + \infty)$
    is not, because it maps the set $\{ 0,0 \}$ which is open in $A$ to $\{0\}$ which is not open in $[0,+\infty)$.
\end{example}

\begin{proposition}
    \label{proposition:limit_point_subspace}
    Let $Y$ be a subspace of $X$. Let $A \subseteq Y$ and $l \in Y$. Then $l$ is a limit point
    of $A$ in $Y$ if and only if $l$ is a limit point of $A$ in $X$.
\end{proposition}

\begin{proof}
    \pf\ Both are equivalent to the condition that any neighbourhood of $l$ in $X$ intersects $A$ in a point other than $l$. \qed
\end{proof}

\begin{theorem}
    If $A$ is a subspace of $X$ then the inclusion $j : A \rightarrow X$ is continuous.
\end{theorem}

\begin{proof}
    \pf\ For any $V$ open in $X$, we have $\inv{j}(V) = V \cap A$ is open in $A$. \qed
\end{proof}

\section{The Box Topology}

\begin{definition}[Box Topology]
    Let $\{ A_i \}_{i \in I}$ be a family of topological spaces. The \emph{box topology} on $\prod_{i \in I} A_i$ is the topology generated by the set of all sets
    of the form $\prod_{i \in I} U_i$ where $\{ U_i \}_{i \in I}$ is a family such that each $U_i$ is open in $A_i$.
\end{definition}

This is a basis since it covers $\prod_{i \in I} A_i$ and is closed under intersection.

\begin{proposition}
    The box topology is finer than the product topology.
\end{proposition}

\begin{proof}
    \pf\ From Proposition \ref{proposition:product_basis}. \qed
\end{proof}

\begin{corollary}
    \label{corollary:closed_box}
        If $A_i$ is closed in $X_i$ for all $i \in I$ then $\prod_{i \in I} A_i$ is closed in $\prod_{i \in I} X_i$ under the box topology.    
\end{corollary}

\begin{proof}
    \pf\ From Proposition \ref{proposition:closed_product}.
\end{proof}

\begin{proposition}[AC]
    Let $\{ A_i \}_{i \in I}$ be a family of topological spaces. For $i \in I$, let $\BB_i$ be a basis for the topology on $A_i$. Then $\BB = \{ \prod_{i \in I} B_i \mid
    \forall i \in I. B_i \in \BB_i \}$ is a basis for the box topology on $\prod_{i \in I} A_i$.
\end{proposition}

\begin{proof}
    \pf
    \step{1}{Every set of the form $\prod_{i \in I} B_i$ is open.}
    \step{2}{For every point $a \in \prod_{i \in I} A_i$ and every open set $U$ with $a \in U$, there exists $B \in \BB$ such that $a \in B \subseteq U$.}
    \begin{proof}
        \step{a}{\pflet{$U$ be open and $a \in U$}}
        \step{b}{\pick\ a family $\{ U_i \}_{i \in I}$ such that each $U_i$ is open in $A_i$ and $a \in \prod_{i \in I} U_i \subseteq U$.}
        \step{c}{For $i \in I$, \pick\ $B_i \in \BB_i$ such that $a_i \in B_i \subseteq U_i$}
        \begin{proof}
            \pf\ Using the Axiom of Choice.
        \end{proof}
        \step{d}{$a \in \prod_{i \in I} B_i \subseteq U$}
    \end{proof}
    \qedstep
    \begin{proof}
        \pf\ Lemma \ref{lemma:basis}.
    \end{proof}
    \qed
\end{proof}

\begin{theorem}
    Let $\{ X_i \}_{i \in I}$ be a family of topological spaces. Let $A_i$ be a subspace of $X_i$ for all $i \in I$.
    Give $\prod_{i \in I} X_i$ the box topology.
    Then the box topology on $\prod_{i \in I} A_i$ is the same as the topology it inherits as a subspace of
    $\prod_{i \in I} X_i$.
\end{theorem}

\begin{proof}
    \pf\ The box topology is generated by the basis
    \begin{align*}
        & \{ \prod_{i \in I} U_i \mid \forall i \in I, U_i \text{ open in } A_i \} \\
        = & \{ \prod_{i \in I} (V_i \cap A_i) \mid \forall i \in I, V_i \text{ open in } X_i \} \\
        = & \{ \prod_{i \in I} V_i \mid \forall i \in I, V_i \text{ open in } X_i \} \cap \prod_{i \in I} A_i
    \end{align*}
    and this is a basis for the subspace topology by Lemma \ref{lemma:subspace_basis}. \qed
\end{proof}

\begin{proposition}[AC]
    Let $\{ X_i \}_{i \in I}$ be a family of topological spaces. Give $\prod_{i \in I} X_i$ the box topology.
    Let $A_i \subseteq X_i$ for all $i \in I$.
    Then
    \[ \prod_{i \in I} \overline{A_i} = \overline{\prod_{i \in I} A_i} \enspace . \]
\end{proposition}

\begin{proof}
    \pf
    \step{1}{$\overline{\prod_{i \in I} A_i} \subseteq \prod_{i \in I} \overline{A_i}$}
    \begin{proof}
        \step{a}{For all $i \in I$ we have $A_i \subseteq \overline{A_i}$}
        \begin{proof}
            \pf\ Lemma \ref{lemma:closure_subset}.
        \end{proof}
        \step{c}{$\prod_{i \in I} A_i \subseteq \prod_{i \in I} \overline{A_i}$}
        \qedstep
        \begin{proof}
            \pf\ Since $\prod_{i \in I} A_i$ is closed by Corollary \ref{corollary:closed_box}.
        \end{proof}
    \end{proof}
    \step{2}{$\prod_{i \in I} \overline{A_i} \subseteq \overline{\prod_{i \in I} A_i}$}
    \begin{proof}
        \step{a}{\pflet{$x \in \prod_{i \in I} \overline{A_i}$}}
        \step{b}{\pflet{$U$ be a neighbourhood of $x$}}
        \step{c}{\pick\ $V_i$ open in $X_i$ such that $x \in \prod_{i \in I} V_i \subseteq U$}
        \step{d}{For $i \in I$, pick $a_i \in V_i \cap A_i$}
        \begin{proof}
            \pf\ By Theorem \ref{theorem:closure_neighbourhood} and \stepref{a} using the Axiom of Choice.
        \end{proof}
        \step{f}{$U$ intersects $\prod_{i \in I} A_i$}
        \qedstep
        \begin{proof}
            \pf\ $a \in U \cap \prod_{i \in I} A_i$.
        \end{proof}
    \end{proof}
    \qed
\end{proof}

The following example shows that Theorem \ref{theorem:product_continuous} fails in the box topology.

\begin{example}
    Define $f : \RR \rightarrow \RR^\omega$ by $f(t) = (t, t, \ldots)$. Then $\pi_n \circ f = \id{\RR}$ is continuous for all $n$.
    But $f$ is not continuous when $\RR^\omega$ is given the box topology because the inverse image of
    \[ (-1,1) \times (-1/2, 1/2) \times (-1/3,1/3) \times \cdots \]
    is $\{0\}$ which is not open.
\end{example}

The following example shows that Proposition \ref{proposition:converge_product} fails in the box topology.

\begin{example}
    Give $\RR^\omega$ the box topology. Let $a_n = (1/n, 1/n, \ldots)$ for $n \geq 1$ and $l = (0, 0, \ldots)$. Then $\pi_i(a_n) \rightarrow \pi_i(l)$
    as $n \rightarrow \infty$ for all $i$, but $a_n \not\rightarrow l$ as $n \rightarrow \infty$ since the open set
    \[ (-1,1) \times (-1/2, 1/2) \times (-1/3,1/3) \times \cdots \]
    contains $l$ but does not contain any $a_n$.
\end{example}

\begin{example}
    The set $\RR^\infty$ is closed in $\RR^\omega$ under the box topology. For let $(a_n)$ be any sequence not in $\RR^\infty$.
    Let $U_n$ be an open interval around $a_n$ that does not contain $0$ if $a_n \neq 0$, and $U_n = \RR$ if $a_n = 0$.
    Then $\prod_{n \geq 0} U_n$ is a neighbourhood of $(a_n)$ that does not intersect $\RR^\infty$.
\end{example}

\begin{proposition}
    Let $\{ A_i \}_{i \in I}$ be a family of topological spaces. For $i \in I$, let $\BB_i$ be a basis for the topology on $A_i$. Then $\BB = \{ \prod_{i \in I} B_i \mid
    \forall i \in I. B_i \in \BB_i, B_i = A_i \text{ for all but finitely many } i \}$ is a basis for the box topology on $\prod_{i \in I} A_i$.
\end{proposition}

\begin{proof}
    \pf
    \step{1}{Every set in $\BB$ is open.}
    \step{2}{For every point $a \in \prod_{i \in I} A_i$ and every open set $U$ with $a \in U$, there exists $B \in \BB$ such that $a \in B \subseteq U$.}
    \begin{proof}
        \step{a}{\pflet{$U$ be open and $a \in U$}}
        \step{b}{\pick\ a family $\{ U_i \}_{i \in I}$ such that each $U_i$ is open in $A_i$, such that $U_i = A_i$ except for $i = i_1, \ldots, i_n$, and 
        such that $a \in \prod_{i \in I} U_i \subseteq U$.}
        \step{c}{For $j = 1, \ldots, n$, \pick\ $B_{i_j} \in \BB_{i_j}$ such that $a_{i_j} \in B_{i_j} \subseteq U_{i_j}$}
        \step{d}{\pflet{$B = \prod_{i \in I} B_i$ where $B_i = A_i$ for $i \neq i_1, \ldots, i_n$}}
        \step{e}{$B \in \BB$}
        \step{f}{$a \in B \subseteq U$}
    \end{proof}
    \qedstep
    \begin{proof}
        \pf\ Lemma \ref{lemma:basis}.
    \end{proof}
    \qed
\end{proof}

\section{$T_1$ Spaces}

\begin{definition}[$T_1$ Space]
    A topological space is $T_1$ if and only if every singleton is closed.
\end{definition}

\begin{lemma}
    A space is $T_1$ if and only if every finite set is closed.
\end{lemma}

\begin{proof}
    \pf\ From Lemma \ref{lemma:closed_union}. \qed
\end{proof}

\begin{theorem}
    \label{theorem:limit_point_T1}
    In a $T_1$ space, a point $a$ is a limit point of a set $A$ if and only if every neighbourhood of $a$
    contains infinitely many points of $A$.
\end{theorem}

\begin{proof}
    \pf
    \step{1}{If $a$ is a limit point of $A$ then every neighbourhood of $a$ contains infinitely many points
    of $A$.}
    \begin{proof}
        \step{a}{\assume{$a$ is a limit point of $A$.}}
        \step{b}{\pflet{$U$ be a neighbourhood of $a$.}}
        \step{c}{\assume{for a contradiction $U$ contains only finitely many points of $A$.}}
        \step{d}{$(U \cap A) - \{a\}$ is closed.}
        \begin{proof}
            \pf\ By the $T_1$ axiom.
        \end{proof}
        \step{e}{$(U - A) \cup \{a\}$ is open.}
        \begin{proof}
            \pf\ It is $U - ((U \cap A) - \{a\})$.
        \end{proof}
        \step{f}{$(U - A) \cup \{a\}$ intersects $A$ in a point other than $a$.}
        \begin{proof}
            \pf\ From \stepref{a}.
        \end{proof}
        \qedstep
        \qed
    \end{proof}
    \step{2}{If every neighbourhood of $a$ contains infinitely many points of $A$ then $a$ is a limit point
    of $A$.}
    \begin{proof}
        \pf\ Immediate from definitions.
    \end{proof}
    \qed
\end{proof}

(To see this does not hold in every space, see Proposition \ref{proposition:indiscrete_limit_point}.)

\begin{proposition}
    A space is $T_1$ if and only if, for any two distinct points $x$ and $y$, there exist neighbourhoods
    $U$ of $x$ and $V$ of $y$ such that $x \notin V$ and $y \notin U$.
\end{proposition}

\begin{proof}
    \pf
    \step{1}{\pflet{$X$ be a topological space.}}
    \step{2}{If $X$ is $T_1$ then, for any two distinct points $x$ and $y$, there exist neighbourhoods
    $U$ of $x$ and $V$ of $y$ such that $x \notin V$ and $y \notin U$.}
    \begin{proof}
        \pf\ This holds because $\{x\}$ and $\{y\}$ are closed.
    \end{proof}
    \step{3}{Suppose, for any two distinct points $x$ and $y$, there exist neighbourhoods
    $U$ of $x$ and $V$ of $y$ such that $x \notin V$ and $y \notin U$. Then $X$ is $T_1$.}
    \begin{proof}
        \step{a}{\assume{For any two distinct points $x$ and $y$, there exist neighbourhoods
        $U$ of $x$ and $V$ of $y$ such that $x \notin V$ and $y \notin U$.}}
        \step{b}{\pflet{$a \in X$}}
        \step{c}{$\{a\}$ is closed.}
        \begin{proof}
            \pf\ For all $b \neq a$ there exists a neighbourhood $U$ of $b$ such that $U \subseteq X - \{a\}$.
        \end{proof}
    \end{proof}
    \qed
\end{proof}

\begin{proposition}
    \label{proposition:T1_subspace}
    A subspace of a $T_1$ space is $T_1$.
\end{proposition}

\begin{proof}
    \pf\ From Proposition \ref{theorem:closed_subspace}.
\end{proof}

\section{Hausdorff Spaces}

\begin{definition}[Hausdorff Space]
    A topological space is \emph{Hausdorff} if and only if, for any points $x$, $y$ with $x \neq y$,
    there exist disjoint open sets $U$ and $V$ such that $x \in U$ and $y \in V$.
\end{definition}

\begin{theorem}
    Every Hausdorff space is $T_1$.
\end{theorem}

\begin{proof}
    \pf
    \step{0}{\pflet{$X$ be a Hausdorff space.}}
    \step{1}{\pflet{$b \in X$} \prove{$\overline{\{b\}} = \{b\}$}}
    \step{2}{\assume{$a \in \overline{\{b\}}$ and $a \neq b$}}
    \step{3}{\pick\ disjoint neighbourhoods $U$ of $a$ and $V$ of $b$.}
    \step{4}{$U$ intersects $\{b\}$}
    \begin{proof}
        \pf\ Theorem \ref{theorem:closure_neighbourhood}.
    \end{proof}
    \step{5}{$b \in U$}
    \qedstep
    \begin{proof}
        \pf\ This contradicts the fact that $U$ and $V$ are disjoint (\stepref{3}).
    \end{proof}
    \qed
\end{proof}

\begin{proposition}
    An infinite set under the finite complement topology is $T_1$ but not Hausdorff.
\end{proposition}

\begin{proof}
    \pf
    \step{0}{\pflet{$X$ be an infinite set under the finite complement topology.}}
    \step{1}{Every singleton is closed.}
    \begin{proof}
        \pf\ By definition.
    \end{proof}
    \step{2}{\pick{$a, b \in X$ with $a \neq b$}}
    \step{3}{There are no disjoint neighbourhoods $U$ of $a$ and $V$ of $b$.}
    \begin{proof}
        \step{a}{\pflet{$U$ be a neighbourhood of $a$ and $V$ a neighbourhood of $b$.}}
        \step{b}{$X - U$ and $X - V$ are finite.}
        \step{c}{\pick\ $c \in X$ that is not in $X - U$ or $X - V$.}
        \step{d}{$c \in U \cap V$}
    \end{proof}
    \qed
\end{proof}

\begin{proposition}
    The product of a family of Hausdorff spaces is Hausdorff.
\end{proposition}

\begin{proof}
    \pf
    \step{1}{\pflet{$\{ X_i \}_{i \in I}$ be a family of Hausdorff spaces.}}
    \step{2}{\pflet{$a, b \in \prod_{i \in I} X_i$ with $a \neq b$}}
    \step{3}{\pick\ $i \in I$ such that $a_i \neq b_i$}
    \step{4}{\pick\ $U$, $V$ disjoint open sets in $X_i$ with $a_i \in U$ and $b_i \in V$}
    \step{5}{$\inv{\pi_i}(U)$ and $\inv{\pi_i}(V)$ are disjoint open sets in $\prod_{i \in I} X_i$ with $a \in \inv{\pi_i}(U)$
    and $b \in \inv{\pi_i}(V)$}
    \qed
\end{proof}

\begin{theorem}
    Every linearly ordered set under the order topology is Hausdorff.
\end{theorem}

\begin{proof}
    \pf
    \step{1}{\pflet{$X$ be a linearly ordered set under the order topology.}}
    \step{2}{\pflet{$a, b \in X$ with $a \neq b$}}
    \step{3}{\assume{w.l.o.g.~$a < b$}}
    \step{4}{\case{There exists $c$ such that $a < c < b$}}
    \begin{proof}
        \pf\ The sets $(-\infty,c)$ and $(c,+\infty)$ are disjoint neighbourhoods of $a$ and $b$
        respectively.
    \end{proof}
    \step{5}{\case{There is no $c$ such that $a < c < b$}}
    \begin{proof}
        \pf\ The sets $(-\infty, b)$ and $(a,+\infty)$ are disjoint neighbourhoods of $a$ and $b$
        respectively.
    \end{proof}
    \qed
\end{proof}

\begin{example}
    In particular, $S_\Omega$ and $\overline{S_\Omega}$ are Hausdorff.
\end{example}

\begin{theorem}
    \label{theorem:subspace_Haudorff}
    A subspace of a Hausdorff space is Hausdorff.
\end{theorem}

\begin{proof}
    \pf
    \step{1}{\pflet{$X$ be a Hausdorff space and $Y$ a subspace of $X$.}}
    \step{2}{\pflet{$x, y \in Y$ with $x \neq y$}}
    \step{3}{\pick\ disjoint neighbourhoods $U$ of $x$ and $V$ of $y$ in $X$.}
    \step{4}{$U \cap Y$ and $V \cap Y$ are disjoint neighbourhoods of $x$ and $y$ respectively in $Y$.}
    \qed
\end{proof}

\begin{corollary}
    For any family of nonempty spaces $\{ X_\alpha \}_{\alpha \in J}$, if
    $\prod_{\alpha \in J} X_\alpha$ is Hausdorff then each $X_\alpha$.
\end{corollary}

\begin{proposition}
    A space $X$ is Hausdorff if and only if the diagonal $\Delta = \{ (x,x) \mid x \in X \}$ is closed in $X^2$.
\end{proposition}

\begin{proof}
    \pf
    \begin{align*}
        & X \text{ is Hausdorff} \\
        \Leftrightarrow & \forall x,y \in X. x \neq y \Rightarrow \exists V, W \text{ open}. x \in V \wedge y \in W \wedge V \cap W = \emptyset \\
        \Leftrightarrow & \forall (x,y) \in X^2 - \Delta. \exists V,W \text{ open}. (x,y) \subseteq V \times W \subseteq X^2 - \Delta \\
        \Leftrightarrow & \Delta \text{ is closed} & \qed
    \end{align*}
\end{proof}

\begin{theorem}
    In a Hausdorff space, a net has at most one limit.
\end{theorem}

\begin{proof}
    \pf
    \step{1}{\pflet{$X$ be a Hausdorff space.}}
    \step{2}{\assume{for a contradiction $(a_\alpha)_{\alpha \in J}$ is a net 
    with limits $l$ and $m$.}}
    \step{3}{\pick\ disjoint neighbourhoods $U$ of $l$ and $V$ of $m$}
    \begin{proof}
        \pf\ By the Hausdorff axiom.
    \end{proof}
    \step{4}{\pick\ $\alpha$ and $\beta$ such that 
    $a_\gamma \in U$ for $\gamma \geq \alpha$ and $a_\gamma \in V$ for 
    $\gamma \geq \beta$}
    \step{5}{\pick\ $\gamma \in J$ with $\gamma \geq \alpha$
    and $\gamma \geq \beta$}
    \step{5}{$a_\gamma \in U \cap V$}
    \qedstep
    \begin{proof}
        \pf\ This contradicts the fact that $U$ and $V$ are disjoint (\stepref{3}).
    \end{proof}
    \qed
\end{proof}

To see this is not always true in spaces that are $T_1$ but not Hausdorff:

\begin{proposition}
    Let $X$ be an infinite set under the finite complement topology. Let $(a_n)_{n \in \NN}$ be a sequence
    with all points distinct. Then for every $l \in X$ we have $a_n \rightarrow l$ as
    $n \rightarrow \infty$.
\end{proposition}

\begin{proof}
    \pf\ Let $U$ be any neighbourhood of $l$. Since $X - U$ is finite, there must exist $N$
    such that, for all $n \geq N$, we have $a_n \in U$. \qed
\end{proof}

\begin{proposition}
    \label{proposition:agree_on_closure}
    Let $X$ be a topological space. Let $Y$ a Hausdorff space. Let $A \subseteq X$. Let $f, g : \overline{A} \rightarrow Y$ be continuous.
    If $f$ and $g$ agree on $A$ then $f = g$.
\end{proposition}

\begin{proof}
    \pf
    \step{1}{\pflet{$x \in \overline{A}$}}
    \step{2}{\assume{$f(x) \neq g(x)$}}
    \step{3}{\pick\ disjoint neighbourhoods $V$ of $f(x)$ and $W$ of $g(x)$.}
    \step{4}{\pick\ $y \in \inv{f}(V) \cap \inv{g}(W) \cap A$}
    \begin{proof}
        \pf\ Since $\inv{f}(V) \cap \inv{g}(W)$ is a neighbourhood of $x$ and hence intersects $A$.
    \end{proof}
    \step{5}{$f(y) = g(y) \in V \cap W$}
    \qedstep
    \begin{proof}
        \pf\ This contradicts the fact that $V$ and $W$ are disjoint (\stepref{3}).
    \end{proof}
    \qed
\end{proof}

\begin{proposition}
    Let $\{ X_i \}_{i \in I}$ be a family of Hausdorff spaces. Then $\prod_{i \in I} X_i$ under the box topology is Hausdorff.
\end{proposition}

\begin{proof}
    \pf
    \step{1}{\pflet{$\{ X_i \}_{i \in I}$ be a family of Hausdorff spaces.}}
    \step{2}{\pflet{$a, b \in \prod_{i \in I} X_i$ with $a \neq b$}}
    \step{3}{\pick\ $i \in I$ such that $a_i \neq b_i$}
    \step{4}{\pick\ $U$, $V$ disjoint open sets in $X_i$ with $a_i \in U$ and $b_i \in V$}
    \step{5}{$\inv{\pi_i}(U)$ and $\inv{\pi_i}(V)$ are disjoint open sets in $\prod_{i \in I} X_i$ with $a \in \inv{\pi_i}(U)$
    and $b \in \inv{\pi_i}(V)$}
    \qed
\end{proof}

\begin{proposition}
    \label{proposition:Hausdorff_finer}
    Let $\TT$ and $\TT'$ be two topologies on the same set $X$ with $\TT \subseteq \TT'$
    If $\TT$ is Haudorff then $\TT'$ is Hausdorff.
\end{proposition}

\begin{proof}
    \pf\ Immediate from definitions.
\end{proof}

\begin{proposition}
    Let $X$ be a Hausdorff space. Let $\DD \subseteq \pow X$ be maximal
    with respect to the finite intersection property. Then $\bigcap_{D \in \DD}
    \overline{D}$ contains at most one point.
\end{proposition}

\begin{proof}
    \pf
    \step{1}{\pflet{$x,y \in \bigcap_{D \in \DD} \overline{D}$}}
    \step{2}{\assume{for a contradiction $x \neq y$}}
    \step{3}{\pick\ disjoint open subsets $U$ and $V$ of $x$ and $y$ respectively.}
    \step{4}{$U, V \in \DD$}
    \begin{proof}
        \pf\ Proposition \ref{proposition:every_closure_maximal_fip}.
    \end{proof}
    \qedstep
    \begin{proof}
        \pf\ This contradicts the fact that $\DD$ satisfies the finite intersection property.
    \end{proof}
    \qed
\end{proof}

\begin{proposition}
    Let $X$ be a topological space. Let $Y$ be a Hausdorff space. Let $f : X
    \rightarrow Y$ be continuous. Then $\{ x \in X \mid f(x) = g(x) \}$
    is closed in $X$.
\end{proposition}

\begin{proof}
    \pf
    \step{1}{\pflet{$A = \{ x \in X \mid f(x) = g(x) \}$}}
    \step{2}{\pflet{$x \in X - A$}}
    \step{3}{\pick\ disjoint open neighourhoods $U$ and $V$ of $f(x)$ and $g(x)$ in $Y$}
    \step{4}{$x \in \inv{f}(U) \cap \inv{g}(V) \subseteq X - A$}
    \qed
\end{proof}

\begin{proposition}
    The space $\RR_K$ is Hausdorff.
\end{proposition}

\begin{proof}
    \pf\ Since its topology is finer than $\RR$. \qed
\end{proof}

\begin{proposition}
    The image of a Hausdorff space under a continuous map is
    not necessarily Hausdorff.
\end{proposition}

\begin{proof}
    \pf\ Take the identity function from a set under the
    discrete topology to the same set under the indiscrete topology. \qed
\end{proof}

\section{Compactly Generated Spaces}

\begin{definition}[Compactly Generated]
    Let $X$ be a topological space. Then $X$ is \emph{compactly generated} if and only if, for all $A \subseteq X$,
    we have $A$ is open in $X$ if and only if, for every compact subspace $C$ of $X$, $A \cap C$ is open in $C$.
\end{definition}

\begin{lemma}
    \label{lemma:compactly_generated_continuous}
    Let $X$ be a compactly generated space. Let $Y$ be a topological space. Let $f : X \rightarrow Y$. Supposet that,
    for every compact $C \subseteq X$, we have $f \restriction C : C \rightarrow Y$ is continuous. Then $f$ is continuous.
\end{lemma}

\begin{proof}
    \pf
    \step{1}{\pflet{$V$ be open in $Y$.}}
    \step{2}{For every compact $C \subseteq X$, we have $\inv{f}(V) \cap C$ is open in $C$.}
    \begin{proof}
        \pf\ Since $\inv{f}(V) \cap C = \inv{(f \restriction C)}(V)$.
    \end{proof}
    \step{3}{$\inv{f}(V)$ is open in $X$.}
    \qed
\end{proof}
\section{The First Countability Axiom}

\begin{definition}[First Countability Axiom]
    A topological space $X$ satisfies the \emph{first countability axiom}, or is \emph{first countable}, if and only if every point has a countable local basis.
\end{definition}

\begin{example}
    The space $S_\Omega$ is first countable. For any $\alpha \in S_\Omega$,
    the set $\{ (\beta, \alpha + 1) \mid \beta < \alpha \} \cup \{ [0, \alpha + 1) \}$
    is a local basis at $\alpha$.
\end{example}

\begin{lemma}[Sequence Lemma (CC)]
    Let $X$ be a first countable space. Let $A \subseteq X$ and $l \in \overline{A}$.
    Then there exists a sequence in $A$ that converges to $l$.
\end{lemma}

\begin{proof}
    \pf
    \step{0}{\pick\ a countable local basis $\{ B_n \mid n \in \ZZ^+ \}$ at $l$ such that $B_1 \supseteq B_2 \supseteq \cdots$.}
    \begin{proof}
        \pf\ Lemma \ref{lemma:countable_local_basis}.
    \end{proof}
    \step{1}{For all $n \geq 1$, \pick\ $a_n \in A \cap B_n$.
    \prove{$a_n \rightarrow l$ as $n \rightarrow \infty$}}
    \step{2}{\pflet{$U$ be a neighbourhood of $A$}}
    \step{3}{\pick\ $N$ such that $B_N \subseteq U$}
    \step{4}{For $n \geq N$ we have $a_n \in U$}
    \begin{proof}
        \pf\ $a_n \in B_n \subseteq B_N \subseteq U$
    \end{proof}
    \qed
\end{proof}

\begin{example}
    \label{example:S_Omega_bar_not_first_countable}
    The space $\overline{S_\Omega}$ is not first countable, since $\Omega$
    is a limit point for $S_\Omega$ but there is no sequence of points in
    $S_\Omega$ that converges to $\Omega$.
\end{example}

\begin{theorem}[CC]
    Let $X$ be a first countable space and $Y$ a topological space. Let $f : X \rightarrow Y$. Suppose that, for every sequence $(x_n)$ in $X$ and $l \in X$, if $x_n \rightarrow l$
    as $n \rightarrow \infty$, then $f(x_n) \rightarrow f(l)$ as $n \rightarrow \infty$. Then $f$ is continuous.
\end{theorem}

\begin{proof}
    \pf
    \step{1}{\pflet{$A \subseteq X$}}
    \step{2}{\pflet{$a \in A$} \prove{$f(a) \in \overline{f(A)}$}}
    \step{3}{\pick\ a sequence $(x_n)$ in $A$ that converges to $a$.}
    \begin{proof}
        \pf\ By the Sequence Lemma.
    \end{proof}
    \step{4}{$f(x_n) \rightarrow f(a)$}
    \step{5}{$f(a) \in \overline{f(A)}$}
    \begin{proof}
        \pf\ By Lemma \ref{lemma:closure_net}.
    \end{proof}
    \qedstep
    \begin{proof}
        \pf\ By Theorem \ref{theorem:continuous}.
    \end{proof}
    \qed
\end{proof}

\begin{example}[CC]
    The space $\RR^\omega$ under the box topology is not first countable.
\end{example}

\begin{proof}
    \pf\ Let $\{ B_n \mid n \geq 0 \}$ be a countable set of neighbourhoods of $\vec{0}$. We will construct a neighbourhood of $\vec{0}$ that does not include any of these.

    For $n \geq 0$, pick a neighbourhood $U_n$ of 0 such that $U_n \subset \pi_n(B_p)$. Then $\prod_{n=0}^\infty U_n$ is a neighbourhood of $\vec{0}$ that does not
    include any $B_n$. \qed
\end{proof}

\begin{example}
    If $J$ is an uncountable set then $\RR^J$ is not first countable.
\end{example}

\begin{proof}
    \pf
    \step{1}{\pflet{$\{ B_n \mid n \geq 0 \}$ be any countable set of neighbourhoods of $\vec{0}$.}}
    \step{2}{For $n \geq 0$, \pick\ a basis element $\prod_{\alpha \in J} U_{n \alpha}$ that contains $\vec{0}$ and is included in $B_n$.}
    \begin{proof}
        \pf\ Using the Axiom of Countable Choice.
    \end{proof}
    \step{3}{For $n \geq 0$, \pflet{$J_n = \{ \alpha \in J \mid U_{n \alpha} \neq \RR \}$}}
    \step{4}{\pick\ $\beta \in J$ such that $\beta \notin J_n$ for any $n$.}
    \begin{proof}
        \pf\ Since each $J_n$ is finite so $\bigcup_n J_n$ is countable.
    \end{proof}
    \step{5}{$\pi_\beta((-1,1))$ is a neighbourhood of $\vec{0}$ that does not include any $B_n$.}
    \qed
\end{proof}

\begin{example}
    The space $\RR_l$ is first countable.

    \begin{proof}
        \pf\ For any real number $x$, the set $\{ [x,q) \mid q \in \QQ, q > x \}$
        is a countable local basis at $x$. \qed
    \end{proof}
\end{example}

\begin{proof}
    \pf\ For any $a \in \RR$, the set $\{ [a,a + 1/n) \mid n \geq 1 \}$ is a countable local basis.
\end{proof}

\begin{example}
    The ordered square is first countable.
\end{example}

\begin{proof}
    \pf\ For any $(a,b) \in I_o^2$ with $b \neq 0,1$, the set $\{ (\{a\} \times (b - 1/n,b + 1/n)) \cap I_o^2 \mid n \geq 1 \}$ is a countable local basis.
\end{proof}

\begin{proposition}
    A subspace of a first countable space is first countable.
\end{proposition}

\begin{proof}
    \pf
    \step{1}{\pflet{$X$ be a first countable space.}}
    \step{2}{\pflet{$Y \subseteq X$}}
    \step{3}{\pflet{$y \in Y$}}
    \step{4}{\pick\ a countable local basis $\BB$ for $y$ in $X$}
    \step{5}{$\{ B \cap Y \mid B \in \BB \}$ is a countable local basis for $y$ in $Y$}
    \qed
\end{proof}

\begin{example}
    The space $S_\Omega \times \overline{S_\Omega}$ is not first countable, because it has a subspace homeomorphic
    to $\overline{S_\Omega}$.
\end{example}

\begin{proposition}[AC]
    A countable product of first countable spaces is first countable.
\end{proposition}

\begin{proof}
    \pf
    \step{1}{\pflet{$(X_n)$ be a sequence of first countable spaces.}}
    \step{2}{\pflet{$(x_n) \in \prod_n X_n$}}
    \step{3}{For all $n$, \pick\ a countable local basis $\BB_n$ for $x_n$ in $X_n$}
    \step{4}{\pflet{$\BB$ be the set of all sets of the form $\prod_n U_n$
    where $(U_n)$ is a family such that $U_n \in \BB_n$ for finitely many $n$
    and $U_n = X_n$ for all other $n$}}
    \step{5}{$\BB$ is a countable local basis for $(x_n)$}
    \qed
\end{proof}

\begin{corollary}
    The Sorgenfrey plane is first countable.
\end{corollary}

\begin{proposition}
    The image of a first countable space under a continuous open map is first countable.
\end{proposition}

\begin{proof}
    \pf
    \step{1}{\pflet{$X$ be a first countable space.}}
    \step{2}{\pflet{$Y$ be a topological space.}}
    \step{3}{\pflet{$f : X \twoheadrightarrow Y$ be a surjective, continuous open map.}}
    \step{4}{\pflet{$y \in Y$}}
    \step{5}{\pick\ $x \in X$ such that $f(x) = y$}
    \step{6}{\pick\ a countable local basis $\BB$ at $x$. \prove{$\{ f(B) \mid B \in \BB \}$
    is a local basis at $y$.}}
    \step{7}{\pflet{$V$ be a neighbourhood of $y$}}
    \step{8}{\pick\ $B \in \BB$ such that $x \in B \subseteq \inv{f}(V)$}
    \step{9}{$y \in f(B) \subseteq V$}
    \qed
\end{proof}

\begin{example}[Choice]
    Let $\QQ^\infty$ be the set of all sequences of rationals that end in an
    infinite sequence of 0s. Then $\QQ^\infty$ as a subspace of $\RR^\omega$
    under the box topology is not first countable.

    \begin{proof}
        \pf
        \step{1}{\assume{for a contradiction $\{ B_n \mid n \in \ZZ^+ \}$ is a
        countable local basis at 0.}}
        \step{2}{For $n \in \ZZ^+$, \pick\ a sequence of open neighbourhoods $(U_{nm})$
        of 0 in $\RR$ such that $\prod_m U_{nm} \cap \QQ^\infty \subseteq U_n$}
        \step{3}{For $n \in \ZZ^+$, \pick\ a nonempty open proper subset
        $V_n$ of $U_{nn}$ that contains 0.}
        \step{4}{$\prod_n V_n$ is a nonempty open neighbourhood of 0 but
        there is no $n$ such that $B_n \subseteq \prod_n V_n$}
        \qed
    \end{proof}
\end{example}

\begin{proposition}
    Every first countable space is compactly generated.
\end{proposition}

\begin{proof}
    \pf
    \step{1}{\pflet{$X$ be a first countable space.}}
    \step{2}{\pflet{$B \subseteq X$}}
    \step{3}{\assume{for every compact $C \subseteq X$ we have $B \cap C$ is closed in $C$.} \prove{$B$ is closed in $X$.}}
    \step{4}{\pflet{$x \in \overline{B}$} \prove{$x \in B$}}
    \step{5}{\pick\ a sequence $(x_n)$ in $B$ that converges to $x$.}
    \begin{proof}
        \pf\ By the Sequence Lemma.
    \end{proof}
    \step{6}{\pflet{$C = \{ x_n \mid n \in \ZZ^+ \} \cup \{x\}$}}
    \step{7}{$C$ is compact.}
    \step{8}{$B \cap C$ is closed.}
    \step{9}{$x \in B \cap C$.}
    \qed
\end{proof}

\section{Strong Continuity}

\begin{definition}[Strongly Continuous]
    Let $X$ and $Y$ be topological spaces and $f : X \rightarrow Y$ be a function. Then $f$ is \emph{strongly continuous} if and only if, for every subset $U \subseteq Y$,
    we have $U$ is open in $Y$ if and only if $\inv{f}(U)$ is open in $X$.
\end{definition}

\begin{proposition}
    Let $X$ and $Y$ be topological spaces and $f : X \rightarrow Y$ be a function. Then $f$ is \emph{strongly continuous} if and only if, for every subset $C \subseteq Y$,
    we have $C$ is closed in $Y$ if and only if $\inv{f}(C)$ is closed in $X$.
\end{proposition}

\begin{proof}
    \pf\ Since $X - \inv{f}(C) = \inv{f}(Y - C)$. \qed
\end{proof}

\begin{proposition}
    \label{proposition:strongly_continuous_composite}
    Let $X$, $Y$ and $Z$ be topological spaces. Let $f : X \rightarrow Y$ and $g : Y \rightarrow Z$. If $f$ and $g$ are strongly continuous then so is $g \circ f$.
\end{proposition}

\begin{proof}
    \pf\ Since $\inv{(g \circ f)}(U) = \inv{f}(\inv{g}(U))$. \qed
\end{proof}

\begin{proposition}
    \label{proposition:continuous_strongly_continuous}
    Let $X$, $Y$ and $Z$ be topological spaces.
    Let $f : X \rightarrow Y$ and $g : Y \rightarrow Z$.
    If $g \circ f$ is continuous and $f$ is strongly continuous then 
    $g$ is continuous.
\end{proposition}

\begin{proof}
    \pf
    \step{b}{\pflet{$V \subseteq Z$ be open.}}
    \step{c}{$\inv{f}(\inv{g}(V))$ is open in $X$.}
    \begin{proof}
        \pf\ Since $g \circ f$ is continuous.
    \end{proof}
    \step{d}{$\inv{f}(V)$ is open in $Y$.}
    \begin{proof}
        \pf\ Since $g$ is strongly continuous.
    \end{proof}
    \qed
\end{proof}

\begin{proposition}
    \label{proposition:strongly_continuous_strongly_continuous}
    Let $X$, $Y$ and $Z$ be topological spaces.
    Let $f : X \rightarrow Y$ and $g : Y \rightarrow Z$.
    If $g \circ f$ is strongly continuous and $f$ is strongly continuous then 
    $g$ is strongly continuous.
\end{proposition}

\begin{proof}
    \pf\ For $V \subseteq Z$, we have $V$ is open iff $\inv{f}(\inv{g}(V))$ is
    open iff $\inv{g}(V)$ is open.
\end{proof}

\section{Saturated Sets}

\begin{definition}
    Let $X$ and $Y$ be sets and $p : X \twoheadrightarrow Y$ a surjective function. Let $C \subseteq X$. Then $C$ is \emph{saturated} with respect to $p$ if and only if, for all
    $x, y \in X$, if $x \in C$ and $p(x) = p(y)$ then $y \in C$.
\end{definition}

\begin{proposition}
    \label{proposition:saturated}
    Let $X$ and $Y$ be sets and $p : X \twoheadrightarrow Y$ a surjective function. Let $C \subseteq X$. Then the following are equivalent:
    \begin{enumerate}
        \item $C$ is saturated with respect to $p$.
        \item There exists $D \subseteq Y$ such that $C = \inv{p}(D)$
        \item $C = \inv{p}(p(C))$.
    \end{enumerate}
\end{proposition}

\begin{proof}
    \pf
    \step{1}{$1 \Rightarrow 3$}
    \begin{proof}
        \step{a}{\assume{$C$ is saturated with respect to $p$.}}
        \step{b}{$C \subseteq \inv{p}(p(C))$}
        \begin{proof}
            \pf\ Trivial.
        \end{proof}
        \step{c}{$\inv{p}(p(C)) \subseteq C$}
        \begin{proof}
            \step{i}{\pflet{$x \in \inv{p}(p(C))$}}
            \step{ii}{$p(x) \in p(C)$}
            \step{iii}{There exists $y \in C$ such that $p(x) = p(y)$}
            \step{iv}{$x \in C$}
            \begin{proof}
                \pf\ From \stepref{a}.
            \end{proof}
        \end{proof}
    \end{proof}
    \step{2}{$3 \Rightarrow 2$}
    \begin{proof}
        \pf\ Trivial.
    \end{proof}
    \step{3}{$2 \Rightarrow 1$}
    \begin{proof}
        \pf\ This follows because if $p(x) \in D$ and $p(x) = p(y)$ then $p(y) \in D$.
    \end{proof}
    \qed
\end{proof}

\section{Quotient Maps}

\begin{definition}[Quotient Map]
    Let $X$ and $Y$ be topological spaces and $p : X \rightarrow Y$. Then $p$ is a \emph{quotient map} if and only if $p$ is surjective and strongly continuous.
\end{definition}

\begin{proposition}
    \label{proposition:quotient_map}
    Let $X$ and $Y$ be topological spaces and $p : X \twoheadrightarrow Y$ be a surjective function. Then the following are equivalent.
    \begin{enumerate}
        \item $p$ is a quotient map.
        \item $p$ is continuous and maps saturated open sets to open sets.
        \item $p$ is continuous and maps saturated closed sets to closed sets.
    \end{enumerate}
\end{proposition}

\begin{proof}
    \pf
    \step{1}{$1 \Rightarrow 2$}
    \begin{proof}
        \step{a}{\assume{$p$ is a quotient map.}}
        \step{b}{\pflet{$U$ be a saturated open set in $X$.}}
        \step{c}{$\inv{p}(p(U))$ is open in $X$.}
        \begin{proof}
            \pf\ Since $U = \inv{p}(p(U))$ be Proposition \ref{proposition:saturated}.
        \end{proof}
        \step{d}{$p(U)$ is open in $Y$.}
        \begin{proof}
            \pf\ From \stepref{a}.
        \end{proof}
    \end{proof}
    \step{2}{$1 \Rightarrow 3$}
    \begin{proof}
        \pf\ Similar.
    \end{proof}
    \step{3}{$2 \Rightarrow 1$}
    \begin{proof}
        \step{a}{\assume{$p$ is continuous and maps saturated open sets to open sets.}}
        \step{b}{\pflet{$U \subseteq Y$}}
        \step{c}{\assume{$\inv{p}(U)$ is open in $X$}}
        \step{d}{$\inv{p}(U)$ is saturated.}
        \begin{proof}
            \pf\ Proposition \ref{proposition:saturated}.
        \end{proof}
        \step{e}{$U$ is open in $Y$.}
    \end{proof}
    \step{4}{$3 \Rightarrow 1$}
    \begin{proof}
        \pf\ Similar.
    \end{proof}
    \qed
\end{proof}

\begin{corollary}
    \label{corollary:quotient_continuous_open}
    Every surjective continuous open map is a quotient map.
\end{corollary}

\begin{corollary}
    \label{corollary:closed_map_quotient}
    Every surjective continuous closed map is a quotient map.
\end{corollary}

\begin{example}
    The converses of these corollaries do not hold.

    Let $A = \{ (x,y) \mid x \geq 0 \} \cup \{ (x,y) \mid y = 0 \}$. Then $\pi_1 : A \rightarrow \RR$ is a quotient map, but not an open map or a closed map.

    We prove that $\pi_1$ maps saturated open sets to open sets:
    \begin{proof}
        \step{1}{\pflet{$\inv{\pi_1}(U)$ be a saturated open set in $A$} \prove{$U$ is open in $\RR$}}
        \step{2}{\pflet{$x \in U$}}
        \step{3}{$(x,0) \in \inv{\pi_1(U)}$}
        \step{4}{\pick\ $W$, $V$ open in $\RR$ such that $(x,0) \subseteq W \times V \subseteq \inv{\pi_1(U)}$}
        \step{5}{$x \in W \subseteq U$}
    \end{proof}

    It is not an open map because it maps $((-1,1) \times (1,2)) \cap A$ to $[0,1)$.

    It is not a closed map because it maps $\{ (x, 1/x) \mid x > 0 \}$ to $(0, +\infty)$.
\end{example}

\begin{proposition}
    \label{proposition:quotient_restriction}
    Let $p : X \twoheadrightarrow Y$ be a quotient map. Let $A \subseteq X$ be saturated with respect to $p$. Let $q : A \twoheadrightarrow p(A)$ be the restriction of $p$.
    \begin{enumerate}
        \item If $A$ is either open or closed in $X$ then $q$ is a quotient map.
        \item If $p$ is either an open map or a closed map then $q$ is a quotient map.
    \end{enumerate}
\end{proposition}

\begin{proof}
    \pf
    \step{0a}{\pflet{$p : X \twoheadrightarrow Y$ be a quotient map.}}
    \step{0b}{\pflet{$A \subseteq X$ be saturated with respect to $p$.}}
    \step{0c}{\pflet{$q : A \twoheadrightarrow p(A)$ be the restriction of $p$.}}
    \step{b}{$q$ is continuous.}
    \begin{proof}
        \pf\ Theorem \ref{theorem:continuous_restriction}.
    \end{proof}
    \step{1}{If $A$ is open in $X$ then $q$ is a quotient map.}
    \begin{proof}
        \step{a}{\assume{$A$ is open in $X$.}}
        \step{c}{$q$ maps saturated open sets to open sets.}
        \begin{proof}
            \step{i}{\pflet{$U \subseteq A$ be saturated with respect to $q$
            and open in $A$}}
            \step{ii}{$U$ is saturated with respect to $p$}
            \begin{proof}
                \step{one}{\pflet{$x,y \in X$}}
                \step{two}{\assume{$x \in U$}}
                \step{three}{\assume{$p(x) = p(y)$}}
                \step{four}{$x \in A$}
                \begin{proof}
                    \pf\ From \stepref{i} and \stepref{two}.
                \end{proof}
                \step{five}{$y \in A$}
                \begin{proof}
                    \pf\ From \stepref{0b} and \stepref{three}
                \end{proof}
                \step{six}{$q(x) = x(y)$}
                \begin{proof}
                    \pf\ From \stepref{0c}, \stepref{three}, \stepref{four}, \stepref{five}.
                \end{proof}
                \step{seven}{$y \in U$}
                \begin{proof}
                    \pf\ From \stepref{i}, \stepref{two}, \stepref{six}
                \end{proof}
            \end{proof}
            \step{iii}{$U$ is open in $X$}
            \begin{proof}
                \pf\ Lemma \ref{lemma:subspace_open}, \stepref{a}, \stepref{i}.
            \end{proof}
            \step{iv}{$p(U)$ is open in $Y$}
            \begin{proof}
                \pf\ Proposition \ref{proposition:quotient_map}, \stepref{0a}, \stepref{ii}, \stepref{iii}
            \end{proof}
            \step{v}{$q(U)$ is open in $p(A)$}
            \begin{proof}
                \pf\ Since $q(U) = p(U) = p(U) \cap p(A)$.
            \end{proof}
        \end{proof}
        \qedstep
        \begin{proof}
            \pf\ By Proposition \ref{proposition:quotient_map}.
        \end{proof}
    \end{proof}
    \step{2}{If $A$ is closed in $X$ then $q$ is a quotient map.}
    \begin{proof}
        \pf\ Similar.
    \end{proof}
    \step{3}{If $p$ is an open map then $q$ is a quotient map.}
    \begin{proof}
        \step{a}{\assume{$p$ is an open map}}
        \step{b}{$q$ maps saturated open sets to open sets.}
        \begin{proof}
            \step{i}{\pflet{$U$ be open in $A$ and saturated with respect to $q$}}
            \step{ii}{\pick\ $V$ open in $X$ such that $U = A \cap V$}
            \step{iii}{$p(V)$ is open in $Y$}
            \step{iv}{$q(U) = p(V) \cap p(A)$}
            \begin{proof}
                \step{one}{$q(U) \subseteq p(V) \cap p(A)$}
                \begin{proof}
                    \pf\ From \stepref{ii}.
                \end{proof}
                \step{two}{$p(V) \cap p(A) \subseteq q(U)$}
                \begin{proof}
                    \step{aa}{\pflet{$y \in p(V) \cap p(A)$}}
                    \step{ab}{\pick\ $x \in V$ and $x' \in A$ such that $p(x) = p(x') = y$}
                    \step{ac}{$x \in A$}
                    \begin{proof}
                        \pf\ By \stepref{0b}. 
                    \end{proof}
                    \step{ad}{$x \in U$}
                    \begin{proof}
                        \pf\ From \stepref{ii}
                    \end{proof}
                \end{proof}
            \end{proof}
        \end{proof}
        \qedstep
        \begin{proof}
            \pf\ By Proposition \ref{proposition:quotient_map}.
        \end{proof}
    \end{proof}
    \step{3}{If $p$ is a closed map then $q$ is a quotient map.}
    \begin{proof}
        \pf\ Similar.
    \end{proof}
    \qed
\end{proof}

\begin{example}
    This example shows we cannot remove the hypotheses on $A$ and $p$.

    Define $f : [0,1] \rightarrow [2,3] \rightarrow [0,2]$ by $f(x) = x$ if $x \leq 1$, $f(x) = x-1$ if $x \geq 2$. Then $f$ is a quotient map but its restriction $f'$ to
    $[0,1) \cup [2,3]$ is not, because $\inv{f'}([1,2])$ is open but $[1,2]$ is not.
\end{example}

For a counterexample where $A$ is saturated, see Example \ref{example:quotient_restriction}.

\begin{proposition}
    \label{proposition:quotient_product}
    Let $p : A \twoheadrightarrow C$ and $q : B \twoheadrightarrow D$ be open quotient maps. Then $p \times q : A \times B \rightarrow C \times D$ is an open quotient map.
\end{proposition}

\begin{proof}
    \pf\ From Corollary \ref{corollary:quotient_continuous_open}, Proposition \ref{proposition:open_map_product}
    and Theorem \ref{theorem:product_continuous}. \qed
\end{proof}

\begin{theorem}
    \label{theorem:quotient_composite}
    Let $p : X \twoheadrightarrow Y$ be a quotient map. Let $Z$
    be a topological space and $f : Y \rightarrow Z$ be a function. Then
    \begin{enumerate}
        \item $f \circ p$ is continuous if and only if $f$ is continuous.
        \item $f \circ p$ is a quotient map if and only if $f$ is a quotient map.
    \end{enumerate}
\end{theorem}

\begin{proof}
    \pf
    \step{1}{If $f \circ p$ is continuous then $f$ is continuous.}
    \begin{proof}
        \pf\ Proposition \ref{proposition:continuous_strongly_continuous}.
    \end{proof}
    \step{2}{If $f$ is continuous then $f \circ p$ is continuous.}
    \begin{proof}
        \pf\ Theorem \ref{theorem:continuous_composite}.
    \end{proof}
    \step{3}{If $f \circ p$ is a quotient map then $f$ is a quotient map.}
    \begin{proof}
        \pf\ Proposition \ref{proposition:strongly_continuous_strongly_continuous}.
    \end{proof}
    \step{4}{If $f$ is a quotient map then $f \circ p$ is a quotient map.}
    \begin{proof}
        \pf\ From Proposition \ref{proposition:strongly_continuous_composite}.
    \end{proof}
    \qed
\end{proof}

\begin{proposition}
    \label{proposition:quotient_retraction}
    Let $X$ and $Y$ be topological spaces. Let $p : X \rightarrow Y$ and $f : Y \rightarrow X$ be continuous maps such that $p \circ f = \id{Y}$. Then $p$ is a quotient map.
\end{proposition}

\begin{proof}
    \pf
    \step{1}{\pflet{$V \subseteq Y$}}
    \step{2}{\assume{$\inv{p}(V)$ is open in $X$.}}
    \step{3}{$\inv{f}(\inv{p}(V))$ is open in $Y$.}
    \begin{proof}
        \pf\ Because $f$ is continuous.
    \end{proof}
    \step{4}{$V$ is open in $Y$.}
    \begin{proof}
        \pf\ Because $\inv{f}(\inv{p}(V)) = V$.
    \end{proof}
    \qed
\end{proof}

\section{Quotient Topology}

\begin{definition}[Quotient Topology]
    Let $X$ be a topological space, $Y$ a set and $p : X \twoheadrightarrow Y$ be a surjective function. Then the \emph{quotient topology} on $Y$ is the unique topology
    on $Y$ with respect to which $p$ is a quotient map, namely $\TT = \{ U \in \pow Y \mid \inv{p}(U) \text{ is open in } X \}$.
\end{definition}

We prove this is a topology.

\begin{proof}
    \pf
    \step{1}{$Y \in \TT$}
    \begin{proof}
        \pf\ Since $\inv{p}(Y) = X$ by surjectivity.
    \end{proof}
    \step{2}{For all $\AA \subseteq \TT$ we have $\bigcup \AA \in \TT$}
    \begin{proof}
        \pf\ Since $\inv{p}(\bigcup \AA) = \bigcup_{U \in \AA} \inv{p}(U)$
    \end{proof}
    \step{3}{For all $U, V \in \TT$ we have $U \cap V \in \TT$}
    \begin{proof}
        \pf\ Since $\inv{p}(U \cap V) = \inv{p}(U) \cap \inv{p}(V)$.
    \end{proof}
    \qed
\end{proof}

\begin{definition}[Quotient Space]
    Let $X$ be a topological space and $\sim$ an equivalence relation on $X$. Let $p : X \twoheadrightarrow X / \sim$ be the canonical surjection.
    Then $X / \sim$ under the quotient topology is called a \emph{quotient space}, \emph{identification space} or \emph{decomposition space} of $X$.
\end{definition}

Here is a counterexample showing we cannot remove all the hypotheses of
Proposition \ref{proposition:quotient_restriction} except that $A$ is saturated.

\begin{example}
    \label{example:quotient_restriction}
    Let $X=(0,1/2] \cup \{1\} \cup \{1+1/n:n \geq 2\}$ as a subspace of $\RR$. 
    Define $R$ to be the equivalence relation on $X$ where $xRy$ iff 
    ($x=y$ or $|x - y|=1$), so we identify $1/n$ with $1+1/n$ for all $n \geq 2$. 
    Let $Y$ be the resulting quotient space $X/R$ in the quotient topology
    and $p : X \twoheadrightarrow Y$ the canonical surjection.
    
    Let $A=\{1\} \cup (0 ,1/2] - \{1/n:n \geq 2\} \subseteq X$.
    Then $A$ is saturated under $p$ but the restriction $q$ of $p$ to $A$
    is not a quotient map because it maps the saturated open set $\{1\}$
    to $\{1\}$ which is not open in $p(A)$.
\end{example}

\begin{proposition}
    Let $X$, $Y$ and $Z$ be topological spaces. Let $f : X \rightarrow Y$ and $g : Y \rightarrow Z$. If $f$ and $g$ are quotient maps then so is $g \circ f$.
\end{proposition}

\begin{proof}
    \pf\ From Proposition \ref{proposition:strongly_continuous_composite}. \qed
\end{proof}

\begin{example}
    The product of two quotient maps is not necessarily a quotient map.

    Let $X = \RR$ and $X^*$ the quotient space formed by identifying all positive integers. Let $p : X \twoheadrightarrow X^*$ be the canonical surjection.

    We prove $p \times \id{\QQ} : X \times \QQ \rightarrow X^* \times \QQ$ is not a quotient map.

    \begin{proof}
        \pf
        \step{1}{For $n \geq 1$, \pflet{$c_n = \sqrt{2} / n$}}
        \step{2}{For $n \geq 1$, \pflet{$U_n = \{ (x,y) \in X \times \QQ \mid n - 1/4 < x < n + 1/4, (y + n > x + c_n \text{ and } y + n > -x + c_n) \text{ or }
        (y + n < x + c_n \text{ and } y + n < -x + c_n) \}$}}
        \step{3}{For $n \geq 1$, we have $U_n$ is open in $X \times \QQ$}
        \step{4}{For $n \geq 1$, we have $\{ n \} \times \QQ \subseteq U_n$}
        \step{5}{\pflet{$U = \bigcup_{n=1}^\infty U_n$}}
        \step{6}{$U$ is open in $X \times \QQ$}
        \step{7}{$U$ is saturated with respect to $p \times \id{\QQ}$}
        \step{7a}{\pflet{$U' = (p \times \id{\QQ})(U)$}}
        \step{7a}{\assume{for a contradiction $U'$ is open in $X^* \times \QQ$}}
        \step{8}{$(1,0) \in U'$}
        \step{9}{\pick\ a neighbourhood $W$ of $1$ in $X^*$ and $\delta > 0$ such that $W \times (- \delta, \delta) \subseteq U'$}
        \step{10}{$\inv{p}(W) \times (-\delta, \delta) \subseteq U$}
        \step{11}{\pick\ $n$ such that $c_n < \delta$}
        \step{12}{$n \in \inv{p}(W)$}
        \step{13}{\pick\ $\epsilon > 0$ such that $\epsilon < \delta - c_n$ and $\epsilon < 1/4$ and $(n - \epsilon, n + \epsilon) \subseteq \inv{p}(W)$}
        \step{14}{$(n - \epsilon, n + \epsilon) \times (- \delta, \delta) \subseteq U$}
        \step{15}{\pick\ a rational $y$ such that $c_n - \epsilon / 2 < y < c_n + \epsilon / 2$}
        \step{16}{$(n + \epsilon / 2, y) \notin U$}
        \qedstep
        \begin{proof}
            \pf\ This contradicts \stepref{14}.
        \end{proof}
        \qed
    \end{proof}
\end{example}

\begin{proposition}
    Let $X$ be a topological space and $\sim$ an equivalence relation on $X$.
    Then $X / \sim$ is $T_1$ if and only if every equivalence class is closed in $X$.
\end{proposition}

\begin{proof}
    \pf\ Immediate from definitions. \qed
\end{proof}

\section{Retractions}

\begin{definition}[Retraction]
    Let $X$ be a topological space and $A \subseteq X$. A \emph{retraction} of $X$ onto $A$ is a continuous map $r : X \rightarrow A$ such that, for all $a \in A$,
    we have $r(a) = a$. In this case we call $A$ a \emph{retract} of $X$.
\end{definition}

\begin{proposition}
    Every retraction is a quotient map.
\end{proposition}

\begin{proof}
    \pf\ Proposition \ref{proposition:quotient_retraction} with $f$ the inclusion $A \hookrightarrow X$. \qed
\end{proof}

\begin{proposition}
    If $Z$ is Hausdorff and $Y$ is a retract of $Z$ then $Y$ is closed.
\end{proposition}

\begin{proof}
    \pf
    \step{1}{\pflet{$x \in Z - Y$}}
    \step{2}{$r(x) \neq x$}
    \step{3}{\pick\ disjoint open neighbourhoods $U$ and $V$ of $x$ and $r(x)$ respectively.}
    \step{4}{$x \in U \cap \inv{r}(V) \subseteq Z - Y$}
    \qed
\end{proof}

\section{Homogeneous Spaces}

\begin{definition}[Homogeneous]
    A topological space $X$ is \emph{homogeneous} if and only if, for any
    points $a, b \in X$, there exists a homeomorphism $\phi : X \cong X$
    such that $\phi(a) = b$.
\end{definition}

\section{Regular Spaces}

\begin{definition}[Regular Space]
    A topological space $X$ is \emph{regular} if and only if it is $T_1$ and, for any closed set $A$ and point $a \notin A$, there exist disjoint open sets $U$, $V$ such that
    $A \subseteq U$ and $a \in V$.    
\end{definition}

\begin{proposition}
    Every regular space is Hausdorff.
\end{proposition}

\begin{proof}
    \pf\ Immediate from definitions. \qed
\end{proof}

\begin{proposition}
    \label{proposition:regular_neighbourhoods}
    Let $X$ be a $T_1$ space. Then $X$ is regular if and only if, for any point
    $x \in X$ and open neighbourhood $U$ of $x$, there exists a neighbourhood
    $V$ of $x$ with $\overline{V} \subseteq U$.
\end{proposition}

\begin{proof}
    \pf
    \step{1}{\pflet{$X$ be a $T_1$ space.}}
    \step{2}{If $X$ is regular then,  for any point
    $x \in X$ and open neighbourhood $U$ of $x$, there exists a neighbourhood
    $V$ of $x$ with $\overline{V} \subseteq U$.}
    \begin{proof}
        \step{a}{\assume{$X$ is regular.}}
        \step{b}{\pflet{$x \in X$}}
        \step{c}{\pflet{$U$ be an open neighbourhood of $x$.}}
        \step{d}{\pick\ disjoint open sets $V$ and $W$ such that $X - U \subseteq W$
        and $x \in V$}
        \step{e}{$\overline{V} \subseteq U$}
        \begin{proof}
            \pf\ Since $V \subseteq X - W \subseteq U$.
        \end{proof}
    \end{proof}
    \step{3}{If, for any point
    $x \in X$ and open neighbourhood $U$ of $x$, there exists a neighbourhood
    $V$ of $x$ with $\overline{V} \subseteq U$, then $X$ is regular.}
    \begin{proof}
        \step{a}{\assume{for  any point
        $x \in X$ and open neighbourhood $U$ of $x$, there exists a neighbourhood
        $V$ of $x$ with $\overline{V} \subseteq U$.}}
        \step{b}{\pflet{$A$ be a closed set with $a \notin A$}}
        \step{c}{\pick\ a neighbourhood $V$ of $a$ with $\overline{V} \subseteq X - A$}
        \step{d}{\pflet{$U = X - \overline{V}$}}
        \step{e}{$U$ and $V$ are disjoint open sets with $A \subseteq U$ and $a \in V$}
    \end{proof}
    \qed
\end{proof}

\begin{proposition}
    A subspace of a regular space is regular.
\end{proposition}

\begin{proof}
    \pf
    \step{1}{\pflet{$X$ be a regular space.}}
    \step{2}{\pflet{$Y \subseteq X$}}
    \step{3}{$Y$ is $T_1$}
    \begin{proof}
        \pf\ Proposition \ref{proposition:T1_subspace}.
    \end{proof}
    \step{4}{\pflet{$A$ be a closed set in $Y$ and $a \in Y - A$}}
    \step{5}{\pick\ a closed set $B$ in $X$ such that $A = Y \cap B$}
    \begin{proof}
        \pf\ Theorem \ref{theorem:closed_subspace}.
    \end{proof}
    \step{6}{\pick\ disjoint open sets $U$ and $V$ in $X$ such that $B \subseteq U$
    and $a \in V$}
    \step{7}{$U \cap Y$ and $V \cap Y$ are disjoint open sets in $Y$ such that
    $A \subseteq U \cap Y$ and $a \in V \cap Y$.}
    \qed
\end{proof}

\begin{corollary}
    For any family of nonempty spaces $\{X_\alpha\}_{\alpha \in J}$, if
    $\prod_{\alpha \in J} X_\alpha$ is regular, then each $X_\alpha$ is regular.
\end{corollary}

\begin{proposition}
    A product of regular spaces is regular.
\end{proposition}

\begin{proof}
    \pf
    \step{1}{\pflet{$\{ X_\alpha \}_{\alpha \in J}$ be a family of regular spaces.}}
    \step{2}{\pflet{$X = \prod_{\alpha \in J} X_\alpha$}}
    \step{3}{\pflet{$x \in X$ and $U$ be an open neighbourhood of $x$.}}
    \step{4}{\pick\ $U_\alpha$ open in $X_\alpha$ for all $\alpha$ such that
    $x \in \prod_{\alpha \in J} U_\alpha \subseteq U$ and $U_\alpha = X_\alpha$
    for all $\alpha$ except $\alpha_1$, \ldots, $\alpha_n$.}
    \step{5}{For $1 \leq i \leq n$ \pick\ an open neighbourhood $V_{\alpha_i}$ of
    $\pi_i(x)$ in $X_{\alpha_i}$ such that $\overline{V_{\alpha_i}} \subseteq U_{\alpha_i}$}
    \step{6}{For $\alpha \neq \alpha_1, \ldots, \alpha_n$, \pflet{$V_\alpha = X_\alpha$}}
    \step{7}{\pflet{$V = \prod_\alpha V_\alpha$}}
    \step{8}{$V$ is an open neighbourhood of $x$.}
    \step{9}{$\overline{V} \subseteq U$}
    \begin{proof}
        \pf\ Proposition \ref{proposition:closure_in_product_topology_without_choice}.
    \end{proof}
    \qed
\end{proof}

\begin{example}
    The space $\RR_K$ is not regular.

    \begin{proof}
        \pf
        \step{1}{$K$ is closed in $\RR_K$}
        \step{2}{\assume{for a contradiction there exist disjoint open sets $U$ and $V$ with $0 \in U$ and $K \subseteq V$}}
        \step{3}{\pick\ an open interval $(a,b)$ such that $0 \in (a,b)$ and $(a,b) - K \subseteq U$}
        \step{4}{\pick\ $n \in \ZZ^+$ such that $1/n \in (a,b)$}
        \step{5}{\pick\ an open interval $(c,d)$ such that $1/n \in (c,d) \subseteq V$}
        \step{6}{\pick\ $z \in (\max(c,1/(n+1)), 1/n)$}
        \step{7}{$z \in U$}
        \begin{proof}
            \pf
            \step{a}{$a < z$}
            \begin{proof}
                \pf\ $a < 0 < 1/(n+1) < z$ (\stepref{3}, \stepref{6})
            \end{proof}
            \step{b}{$z < b$}
            \begin{proof}
                \pf\ $z < 1/n < b$ (\stepref{4}, \stepref{6})
            \end{proof}
            \step{c}{$z \in (a,b) - K \subseteq U$}
            \begin{proof}
                \pf\ \stepref{3}
            \end{proof}
        \end{proof}
        \step{8}{$z \in V$}
        \begin{proof}
            \pf
            \step{a}{$c < z$}
            \begin{proof}
                \pf\ From \stepref{6}.
            \end{proof}
            \step{b}{$z < d$}
            \begin{proof}
                \pf\ $z < 1/n < d$ (\stepref{5}, \stepref{6})
            \end{proof}
            \step{c}{$z \in (c,d) \subseteq V$}
            \begin{proof}
                \pf\ \stepref{5}
            \end{proof}
        \end{proof}
        \qedstep
        \begin{proof}
            \pf\ This contradicts the fact that $U$ and $V$ are disjoint (\stepref{2}).
        \end{proof}
        \qed
    \end{proof}
\end{example}

\begin{proposition}
    In a regular space, every pair of distinct points have neighbourhoods whose
    closures are disjoint.
\end{proposition}

\begin{proof}
    \pf
    \step{1}{\pflet{$X$ be a regular space.}}
    \step{2}{\pflet{$a, b \in X$ with $a \neq b$.}}
    \step{3}{\pick\ a neighbourhood $U$ of $a$ such that $\overline{U} \subseteq
    X - \{b\}$.}
    \begin{proof}
        \pf\ Proposition \ref{proposition:regular_neighbourhoods}.
    \end{proof}
    \step{4}{\pick\ a neighbourhood $V$ of $b$ such that $\overline{V} \subseteq
    X - \overline{U}$.}
    \begin{proof}
        \pf\ Proposition \ref{proposition:regular_neighbourhoods}.
    \end{proof}
    \step{5}{$U$ and $V$ are disjoint neighbourhoods of $a$ and $b$ respectively
    with disjoit closures.}
    \qed
\end{proof}

\begin{proposition}
    Every linearly ordered set under the order topology is regular.
\end{proposition}

\begin{proof}
    \pf
    \step{1}{\pflet{$X$ be a linearly ordered set in the order topology.}}
    \step{2}{$X$ is $T_1$.}
    \step{3}{\pflet{$x \in X$}}
    \step{4}{\pflet{$U$ be an open neightbourhood of $x$.}}
    \step{5}{\pick\ a basic open set $B$ such that $x \in B \subseteq U$
    \prove{There exists an open neighbourhood $V$ of $x$ such that 
    $\overline{V} \subseteq B$}}
    \step{6}{\case{$B = (a,b)$}}
    \begin{proof}
        \step{a}{\case{There exists $c$, $d$ with $a < c < x < d < b$}}
        \begin{proof}
            \step{i}{\pflet{$V = (c,d)$}}
            \step{ii}{$\overline{V} \subseteq [c,d] \subseteq (a,b)$}
        \end{proof}
        \step{b}{\case{There exists $c$ such that $a < c < x$ but there is no $d$
        such that $x < d < b$}}
        \begin{proof}
            \step{i}{\pflet{$V = (a,d)$}}
            \step{ii}{$\overline{V} \subseteq [x,d] \subseteq (a,b)$}
        \end{proof}
        \step{c}{\case{There is no $c$ such that $a < c < x$ but there exists $d$
        such that $x < d < b$}}
        \begin{proof}
            \step{i}{\pflet{$V = (c,b)$}}
            \step{ii}{$\overline{V} \subseteq [c,x] \subseteq (a,b)$}
        \end{proof}
        \step{d}{\case{There is no $c$ such that $a < c < x$ and no $d$ such that
        $x < d < b$}}
        \begin{proof}
            \step{i}{\pflet{$V = (a,b) = \{x\}$}}
            \step{ii}{$\overline{V} = \{x\} = (a,b)$}
        \end{proof}
    \end{proof}
    \step{7}{\case{$B = [\bot, b)$}}
    \begin{proof}
        \pf\ Similar.
    \end{proof}
    \step{8}{\case{$B = (a,\top]$}}
    \begin{proof}
        \pf\ Similar.
    \end{proof}
    \qedstep
    \begin{proof}
        \pf\ By Proposition \ref{proposition:regular_neighbourhoods}.
    \end{proof}
    \qed
\end{proof}

\begin{example}
    In particular, $S_\Omega$ and $\overline{S_\Omega}$ are regular.
\end{example}

\begin{proposition}
    \label{proposition:regular_space_compact_closed}
    Let $X$ be a regular space. Let $A$ be compact in $X$ and $B$ be
    closed in $X$ with $A \cap B = \emptyset$. Then there exist
    disjoint open neighbourhoods of $A$ and $B$.
\end{proposition}

\begin{proof}
    \pf
    \step{1}{For all $a \in A$, there exist disjoint open neighbourhoods
    of $a$ and $B$.}
    \step{2}{$\{ U \text{ open in } X \mid \exists V \text{ open in } X.
    B \subseteq V, U \cap V = \emptyset \}$ covers $A$.}
    \step{3}{\pick\ a finite subcover $\{ U_1, \ldots, U_n \}$.}
    \step{4}{For $1 \leq i \leq n$, \pick\ $V_i$ such that
    $B \subseteq V_i$ and $U_i \cap V_i = \emptyset$}
    \step{5}{\pflet{$U = U_1 \cup \cdots \cup U_n$}}
    \step{6}{\pflet{$V = V_1 \cap \cdots \cap V_n$}}
    \step{7}{$U$ and $V$ are disjoint open neighbourhoods of $A$ and $B$
    respectively.}
    \qed
\end{proof}

\begin{proposition}
    The image of a regular space under a continuous map is
    not necessarily regular.
\end{proposition}

\begin{proof}
    \pf\ Take the identity function from a set under the
    discrete topology to the same set under the indiscrete topology. \qed
\end{proof}

\section{Dense Sets}

\begin{definition}[Dense]
    Let $X$ be a topological space and $A \subseteq X$. Then $A$ is \emph{dense}
    if and only if $\overline{A} = X$.
\end{definition}

\section{Connected Spaces}

\begin{definition}[Separation]
    A \emph{separation} of a topological space $X$ is a pair of disjoint open sets $U$, $V$ such that $U \cup V = \emptyset$.
\end{definition}

\begin{definition}[Connected]
    A topological space is \emph{connected} if and only if it has no separation; otherwise it is \emph{disconnected}.
\end{definition}

\begin{example}
    The space $S_\Omega$ is not connected, because $\{0\}$ and $(0, +\infty)$ form a separation.
\end{example}

\begin{example}
    The space $\overline{S_\Omega}$ is not connected, because $\{0\}$ and $(0, +\infty)$ form a separation.
\end{example}

\begin{example}
    The space $S_\Omega \times \overline{S_\Omega}$ is not connected because $\{(0,0)\}$
    and $S_\Omega \times \overline{S_\Omega} - \{(0,0)\}$ form a separation.
\end{example}

\begin{proposition}
    \label{proposition:open_closed_connective}
    A topological space $X$ is connected if and only if the only sets that are both open and closed are $X$ and $\emptyset$.
\end{proposition}

\begin{proof}
    Immediate from defintions.
\end{proof}

\begin{lemma}
    \label{lemma:limit_point_separation}
    If $Y$ is a subspace of $X$, a separation of $Y$ is a pair of disjoint nonempty sets $A$ and $B$ whose union is $Y$, neither of which contains a limit point of the other.
\end{lemma}

\begin{proof}
    \pf
    \step{1}{\pflet{$A, B \subseteq Y$}}
    \step{2}{If $A$ and $B$ form a separation of $Y$ then $A$ and $B$ are disjoint and nonempty, $A \cup B = Y$, and neither of $A$ and $B$ contains a limit point of the other.}
    \begin{proof}
        \step{a}{\assume{$A$ and $B$ form a separation of $Y$}}
        \step{b}{$A$ and $B$ are disjoint and nonempty and $A \cup B = Y$}
        \begin{proof}
            \pf\ From \stepref{a} and the definition of separation.
        \end{proof}
        \step{c}{$A$ does not contain a limit point of $B$}
        \begin{proof}
            \step{i}{\assume{for a contradiction $l \in A$ and $l$ is a limit point of $B$ in $X$.}}
            \step{ii}{$l$ is a limit point of $B$ in $Y$}
            \begin{proof}
                \pf\ Proposition \ref{proposition:limit_point_subspace}.
            \end{proof}
            \step{iii}{$l \in B$}
            \begin{proof}
                \step{one}{$B$ is closed in $Y$}
                \begin{proof}
                    \pf\ Since $A$ is open in $Y$ and $B = Y - A$ from \stepref{a}.
                \end{proof}
                \qedstep
                \begin{proof}
                    \pf\ Corollary \ref{corollary:closed_limit_point}.
                \end{proof}
            \end{proof}
            \qedstep
            \begin{proof}
                \pf\ This contradicts the fact that $A \cap B = \emptyset$ (\stepref{a}).
            \end{proof}
        \end{proof}
        \step{d}{$B$ does not contain a limit point of $A$}
        \begin{proof}
            \pf\ Similar.
        \end{proof}
    \end{proof}
    \step{3}{If $A$ and $B$ are disjoint and nonempty, $A \cup B = Y$, and neither of $A$ and $B$ contains a limit point of the other, then $A$ and $B$ form a separation of $Y$.}
    \begin{proof}
        \step{a}{\assume{$A$ and $B$ are disjoint and nonempty, $A \cup B = Y$, and neither of $A$ and $B$ contains a limit point of the other.}}
        \step{b}{$A$ is open in $Y$}
        \begin{proof}
            \step{i}{$B$ is closed in $Y$}
            \begin{proof}
                \step{one}{\pflet{$l$ be a limit point of $B$ in $Y$}}
                \step{two}{$l$ is a limit point of $B$ in $X$}
                \begin{proof}
                    \pf\ Proposition \ref{proposition:limit_point_subspace}.
                \end{proof}
                \step{three}{$l \notin A$}
                \begin{proof}
                    \pf\ By \stepref{a}
                \end{proof}
                \step{four}{$l \in B$}
                \begin{proof}
                    \pf\ By \stepref{a} since $A \cup B = Y$
                \end{proof}
                \qedstep
                \begin{proof}
                    \pf\ Corollary \ref{corollary:closed_limit_point}.
                \end{proof}
            \end{proof}
            \qedstep
            \begin{proof}
                \pf\ Since $A = Y - B$.
            \end{proof}
        \end{proof}
        \step{c}{$B$ is open in $Y$}
        \begin{proof}
            \pf\ Similar.
        \end{proof}
    \end{proof}
    \qed
\end{proof}

\begin{example}
    Every set under the indiscrete topology is connected.
\end{example}

\begin{example}
    The discrete topology on a set $X$ is connected if and only if $|X| \leq 1$.
\end{example}

\begin{example}
    The finite complement topology on a set $X$ is connected if and only if either $|X| \leq 1$ or $X$ is infinite.
\end{example}

\begin{example}
    The countable complement topology on a set $X$ is connected if and only if either $|X| \leq 1$ or $X$ is uncountable.
\end{example}

\begin{example}
    The rationals $\QQ$ are disconnected. For any irrational $a$, the sets $(- \infty, a) \cap \QQ$ and $(a, + \infty) \cap \QQ$ form a separation of $\QQ$.
\end{example}

\begin{lemma}
    \label{lemma:subset_connected_subspace}
    Let $X$ be a topological space. If $C$ and $D$ form a separation of $X$, and $Y$ is a connected subspace of $X$, then either $Y \subseteq C$ or $Y \subseteq D$.
\end{lemma}

\begin{proof}
    \pf\ Otherwise $Y \cap C$ and $Y \cap D$ would form a separation of $Y$. \qed
\end{proof}

\begin{theorem}
    \label{theorem:connected_union}
    The union of a set of connected subspaces of a space $X$ that have a point in common is connected.
\end{theorem}

\begin{proof}
    \pf
    \step{1}{\pflet{$\AA$ be a set of connected subspaces of the space $X$ that have the point $a$ in common.}}
    \step{2}{\assume{for a contradiction $C$ and $D$ form a separation of $\bigcup \AA$}}
    \step{3}{\assume{without loss of generality $a \in C$}}
    \step{4}{For all $A \in \AA$ we have $A \subseteq C$}
    \begin{proof}
        \pf\ Lemma \ref{lemma:subset_connected_subspace}.
    \end{proof}
    \step{5}{$D = \emptyset$}
    \qedstep
    \begin{proof}
        \pf\ This contradicts \stepref{2}.
    \end{proof}
    \qed
\end{proof}

\begin{theorem}
    \label{theorem:connected_closure}
    Let $X$ be a topological space and $A$ a connected subspace of $X$. If $A \subseteq B \subseteq \overline{A}$ then $B$ is connected.
\end{theorem}

\begin{proof}
    \pf
    \step{1}{\assume{for a contradiction $C$ and $D$ form a separation of $B$.}}
    \step{2}{\assume{without loss of generality $A \subseteq C$}}
    \begin{proof}
        \pf\ Lemma \ref{lemma:subset_connected_subspace}.
    \end{proof}
    \step{3}{$B \subseteq C$}
    \begin{proof}
        \step{3}{\pflet{$x \in B$}}
        \step{4}{$x \in \overline{A}$}
        \step{5}{Either $x \in A$ or $x$ is a limit point of $A$.}
        \begin{proof}
            \pf\ Theorem \ref{theorem:limit_point_closure}.
        \end{proof}
        \step{6}{Either $x \in A$ or $x$ is a limit point of $C$.}
        \begin{proof}
            \pf\ Lemma \ref{lemma:limit_point_subset}, \stepref{2}.
        \end{proof}
        \step{7}{$x \in C$}
        \begin{proof}
            \pf\ Lemma \ref{lemma:limit_point_separation}.
        \end{proof}
    \end{proof}
    \step{8}{$D = \emptyset$}
    \qedstep
    \begin{proof}
        \pf\ This contradicts \stepref{1}.
    \end{proof}
    \qed
\end{proof}

\begin{theorem}
    \label{theorem:connected_continuous_image}
    The image of a connected space under a continuous map is connected.
\end{theorem}

\begin{proof}
    \pf
    \step{1}{\pflet{$f : X \twoheadrightarrow Y$ be a surjective continuous map where $X$ is connected.}}
    \step{2}{\assume{for a contradiction $C$ and $D$ form a separation of $Y$.}}
    \step{3}{$\inv{f}(C)$ and $\inv{f}(D)$ form a separation of $X$.}
    \qed
\end{proof}

\begin{corollary}
    The Sorgenfrey plane is disconnected.
\end{corollary}

\begin{proof}
    \pf\ Since $\RR_l$ is the continuous image of $\RR_l^2$ under the continuous map $\pi_1$ and $\RR_l$ is disconnected. \qed
\end{proof}

\begin{theorem}
    \label{theorem:product_connected}
    The product of a family of connected spaces is connected.
\end{theorem}

\begin{proof}
    \pf
    \step{One}{The product of two connected spaces is connected.}
    \begin{proof}
        \step{1}{\pflet{$X$ and $Y$ be connected spaces.}}
        \step{2}{\pick\ $a \in X$ and $b \in Y$}
        \begin{proof}
            \pf\ We may assume $X$ and $Y$ are nonempty since otherwise $X \times Y = \emptyset$ which is connected.
        \end{proof}
        \step{3}{$X \times \{b\}$ is connected.}
        \begin{proof}
            \pf\ It is homeomorphic to $X$.
        \end{proof}
        \step{4}{For all $x \in X$ we have $\{x\} \times Y$ is connected.}
        \begin{proof}
            \pf\ It is homeomorphic to $Y$.
        \end{proof}
        \step{5}{For any $x \in X$ \pflet{$T_x = (X \times \{b\}) \cup (\{x\} \times Y)$}}
        \step{6}{For all $x \in X$, $T_x$ is connected.}
        \begin{proof}
            \pf\ Theorem \ref{theorem:connected_union} since $(x,b) \in (X \times \{b\}) \cap (\{x\} \times Y)$.
        \end{proof}
        \step{7}{$X \times Y$ is connected.}
        \begin{proof}
            \pf\ Theorem \ref{theorem:connected_union} since $X \times Y = \bigcup_{x \in X} T_x$ and $(a,b)$ is a point in every $T_x$.
        \end{proof}
    \end{proof}
    \step{Two}{The product of a finite family of connected spaces is connected.}
    \begin{proof}
        \pf\ From \stepref{One} by induction.
    \end{proof}
    \step{Three}{The product of any family of connected spaces is connected.}
    \begin{proof}
        \step{1}{\pflet{$\{ X_\alpha \}_{\alpha \in J}$ be a family of connected spaces.}}
        \step{2}{\pflet{$X = \prod_{\alpha \in J} X_\alpha$}}
        \step{3}{\pick\ $a \in X$}
        \begin{proof}
            \pf\ We may assume $X \neq \emptyset$ as the empty space is connected.
        \end{proof}
        \step{4}{For every finite subset $K$ of $J$, \pflet{$X_K = \{ x \in X \mid \forall \alpha \in J - K. x_\alpha = a_\alpha \}$}}
        \step{5}{For every finite $K \subseteq J$, we have $X_K$ is connected.}
        \begin{proof}
            \pf\ From \stepref{Two} since $X_K \cong \prod_{\alpha \in K} X_K$.
        \end{proof}
        \step{6}{\pflet{$Y = \bigcup_K X_K$}}
        \step{7}{$Y$ is connected}
        \begin{proof}
            \pf\ Theorem \ref{theorem:connected_union} since $a$ is a common point.
        \end{proof}
        \step{8}{$X = \overline{Y}$}
        \begin{proof}
            \step{a}{\pflet{$x \in X$}}
            \step{b}{\pflet{$U = \prod_{\alpha \in J} U_\alpha$ be a basic neighbourhood of $x$ where $U_\alpha = X_\alpha$ for all $\alpha$ except
            $\alpha \in K$ for some finite $K \subseteq J$}}
            \step{c}{\pflet{$y \in X$ be the point with $y_\alpha = x_\alpha$ for $\alpha \in K$ and $y_\alpha = a_\alpha$ for all other $\alpha$}}
            \step{d}{$y \in U \cap X_K$}
            \step{e}{$y \in U \cap Y$}
        \end{proof}
        \step{9}{$X$ is connected.}
        \begin{proof}
            \pf\ Theorem \ref{theorem:connected_closure}.
        \end{proof}          
    \end{proof}
    \qed
\end{proof}

\begin{corollary}
    The space $\RR^\omega$ is connected.
\end{corollary}

\begin{corollary}
    The space $\RR^I$ is connected.
\end{corollary}

\begin{example}
    The set $\RR^\omega$ is disconnected under the box topology. The set of bounded sequences and the set of unbounded sequences form
    a separation.
\end{example}

\begin{proposition}
    Let $\TT$ and $\TT'$ be two topologies on the same set $X$. If $\TT \subseteq \TT'$ and $(X, \TT')$ is connected then $(X, \TT)$ is connected.
\end{proposition}

\begin{proof}
    \pf\ If $U$ and $V$ form a separation of $(X, \TT)$ then they form a separation of $(X, \TT')$. \qed
\end{proof}

\begin{proposition}
    Let $X$ be a topological space and $(A_n)$ a sequence of connected subspaces of $X$. If $A_n \cap A_{n+1} \neq \emptyset$ for all $n$ then $\bigcup_n A_n$ is
    connected.
\end{proposition}

\begin{proof}
    \pf
    \step{1}{\assume{for a contradiction $C$ and $D$ form a separation of $\bigcup_n A_n$}}
    \step{2}{\assume{without loss of generality $A_0 \subseteq C$}}
    \begin{proof}
        \pf\ Lemma \ref{lemma:subset_connected_subspace}.
    \end{proof}
    \step{3}{For all $n$ we gave $A_n \subseteq C$}
    \begin{proof}
        \pf
        \step{i}{\assume{$A_n \subseteq C$}}
        \step{ii}{\pick\ $x \in A_n \cap A_{n+1}$}
        \step{iii}{$x \in C$}
        \step{iv}{$A_{n+1} \subseteq C$}
        \begin{proof}
            \pf\ Lemma \ref{lemma:subset_connected_subspace}.
        \end{proof}
        \qedstep
        \begin{proof}
            \pf\ The result follows by induction.
        \end{proof}
    \end{proof}
    \step{4}{$D = \emptyset$}
    \qedstep
    \begin{proof}
        \pf\ This contradicts \stepref{1}.
    \end{proof}
    \qed
\end{proof}

\begin{proposition}
    Let $X$ be a topological space. Let $A, C \subseteq X$. If $C$ is connected and intersects both $A$ and $X - A$ then $C$ intersects $\partial A$.
\end{proposition}

\begin{proof}
    \pf\ Otherwise $C \cap A^\circ$ and $C - \overline{A}$ would form a separation of $C$. \qed
\end{proof}

\begin{example}
    The space $\RR_l$ is disconnected. For any real $x$, the sets $(- \infty, x)$ and $[x, +\infty)$ form a separation.
\end{example}

\begin{proposition}
    Let $X$ and $Y$ be connected spaces. Let $A$ be a proper subset of $X$ and $B$ a proper subset of $Y$. Then $(X \times Y) - (A \times B)$ is connected.
\end{proposition}

\begin{proof}
    \pf
    \step{1}{\pick\ $a \in X - A$ and $b \in Y - B$}
    \step{2}{For $x \in X - A$ we have $(X \times \{b\}) \cup (\{x\} \times Y)$ is connected.}
    \begin{proof}
        \pf\ Theorem \ref{theorem:connected_union} since $(x,b)$ is a common point.
    \end{proof}
    \step{3}{For $y \in Y - B$ we have $(X \times \{y\}) \cup (\{a\} \times Y)$ is connected.}
    \begin{proof}
        \pf\ Theorem \ref{theorem:connected_union} since $(a,y)$ is a common point.
    \end{proof}
    \step{4}{$(X \times Y) - (A \times B)$ is connected.}
    \begin{proof}
        \pf\ Theorem \ref{theorem:connected_union} since it is the union of the sets in \stepref{2} and \stepref{3} with $(a,b)$ as a common point.
    \end{proof}
    \qed
\end{proof}

\begin{proposition}
    Let $p : X \twoheadrightarrow Y$ be a quotient map. If $Y$ is connected and $\inv{p}(y)$ is connected for all $y \in Y$, then $X$ is connected.
\end{proposition}

\begin{proof}
    \pf
    \step{1}{\assume{for a contradiction $C$ and $D$ form a separation of $X$.}}
    \step{2}{$C$ is saturated.}
    \begin{proof}
        \step{a}{\pflet{$x \in C$, $y \in X$ with $p(x) = p(y) = a$, say}}
        \step{b}{$y \notin D$}
        \begin{proof}
            \pf\ Otherwise $\inv{p}(a) \cap C$ and $\inv{p}(a) \cap D$ form a separation of $\inv{p}(a)$.
        \end{proof}
        \step{c}{$y \in C$}
    \end{proof}
    \step{3}{$D$ is saturated.}
    \begin{proof}
        \pf\ Similar.
    \end{proof}
    \step{4}{$p(C)$ and $p(D)$ form a separation of $Y$.}
    \qed
\end{proof}

\begin{proposition}
    Let $X$ be a connected space and $Y$ a connected subspace of $X$. Suppose $A$ and $B$ form a separation of $X - Y$. Then $Y \cup A$ and $Y \cup B$ are both connected.
\end{proposition}

\begin{proof}
    \pf
    \step{1}{$Y \cup A$ is connected.}
    \begin{proof}
        \step{a}{\assume{for a contradiction $C$ and $D$ form a separation of $Y \cup A$}}
        \step{b}{\assume{without loss of generality $Y \subseteq C$}}
        \step{c}{\pick\ open sets $A_1$, $B_1$, $C_1$, $D_1$ in $X$ with
        \begin{align*}
            A & = A_1 - Y \\
            B & = B_1 - Y \\
            C & = C_1 \cap (Y \cup A) \\
            D & = D_1 \cap (Y \cup A)
        \end{align*}
        }
        \step{d}{$B_1 \cup C_1$ and $A_1 \cap D_1$ form a separation of $X$}
    \end{proof}
    \step{2}{$Y \cup B$ is connected.}
    \begin{proof}
        \pf\ Similar.
    \end{proof}
    \qed
\end{proof}

\begin{theorem}
    \label{theorem:connected_continuum}
    Let $L$ be a linearly ordered set under the order topology. Then $L$ is connected if and only if $L$ is a linear continuum.
\end{theorem}

\begin{proof}
    \pf
    \step{One}{If $L$ is a linear continuum then $L$ is connected.}
    \begin{proof}
        \step{1}{\pflet{$L$ be a linear continuum under the order topology.}}
        \step{2}{\assume{for a contradiction $C$ and $D$ form a separation of $L$.}}
        \step{3}{\pick\ $a \in C$ and $b \in D$.}
        \step{4}{\assume{without loss of generality $a < b$.}}
        \step{5}{\pflet{$S = \{ x \in L \mid a < x \text{ and } [a,x) \subseteq C \}$}}
        \step{6}{$S$ is nonempty.}
        \begin{proof}
            \pf\ Since $a \in C$ and $C$ is open.
        \end{proof}
        \step{7}{$S$ is bounded above by $b$.}
        \begin{proof}
            \pf\ Since $b \notin C$.
        \end{proof}
        \step{8}{\pflet{$s = \sup S$}}
        \step{9a}{$s \in S$}
        \begin{proof}
            \step{i}{\pflet{$y \in [a,s)$} \prove{$y \in C$}}
            \step{ii}{\pick\ $z$ with $y < z \in S$}
            \begin{proof}
                \pf\ By minimality of $s$.
            \end{proof}
            \step{iii}{$y \in [a,z) \subseteq C$}
        \end{proof}
        \step{9}{\case{$s \in C$}}
        \begin{proof}
            \step{a}{\pick\ $x$ such that $s < x$ and $[s,x) \subseteq C$}
            \begin{proof}
                \pf\ Since $C$ is open and $s$ is not greatest in $L$ because $s < b$.
            \end{proof}
            \step{b}{$x \in S$}
            \begin{proof}
                \pf\ Since $[a,x) = [a,s) \cup [s,x) \subseteq C$.
            \end{proof}
            \qedstep
            \begin{proof}
                \pf\ This contradicts the fact that $s$ is an upper bound for $S$.
            \end{proof}
        \end{proof}
        \step{10}{\case{$s \in D$}}
        \begin{proof}
            \step{a}{\pick\ $x < s$ such that $(x,s] \subseteq D$}
            \step{b}{\pick\ $y$ with $x < y < s$}
            \begin{proof}
                \pf\ Since $L$ is dense.
            \end{proof}
            \step{c}{$y \in C$}
            \begin{proof}
                \pf\ From \stepref{9a}.
            \end{proof}
            \step{d}{$y \in D$}
            \begin{proof}
                \pf\ From \stepref{a}.
            \end{proof}
            \qedstep    \step{1}{\pflet{$L$ be a linear continuum under the order topology.}}
            \step{2}{\assume{for a contradiction $C$ and $D$ form a separation of $L$.}}
            \step{3}{\pick\ $a \in C$ and $b \in D$.}
            \step{4}{\assume{without loss of generality $a < b$.}}
            \step{5}{\pflet{$S = \{ x \in L \mid a < x \text{ and } [a,x) \subseteq C \}$}}
            \step{6}{$S$ is nonempty.}
            \begin{proof}
                \pf\ Since $a \in C$ and $C$ is open.
            \end{proof}
            \step{7}{$S$ is bounded above by $b$.}
            \begin{proof}
                \pf\ Since $b \notin C$.
            \end{proof}
            \step{8}{\pflet{$s = \sup S$}}
            \step{9a}{$s \in S$}
            \begin{proof}
                \step{i}{\pflet{$y \in [a,s)$} \prove{$y \in C$}}
                \step{ii}{\pick\ $z$ with $y < z \in S$}
                \begin{proof}
                    \pf\ By minimality of $s$.
                \end{proof}
                \step{iii}{$y \in [a,z) \subseteq C$}
            \end{proof}
            \step{9}{\case{$s \in C$}}
            \begin{proof}
                \step{a}{\pick\ $x$ such that $s < x$ and $[s,x) \subseteq C$}
                \begin{proof}
                    \pf\ Since $C$ is open and $s$ is not greatest in $L$ because $s < b$.
                \end{proof}
                \step{b}{$x \in S$}
                \begin{proof}
                    \pf\ Since $[a,x) = [a,s) \cup [s,x) \subseteq C$.
                \end{proof}
                \qedstep
                \begin{proof}
                    \pf\ This contradicts the fact that $s$ is an upper bound for $S$.
                \end{proof}
            \end{proof}
            \step{10}{\case{$s \in D$}}
            \begin{proof}
                \step{a}{\pick\ $x < s$ such that $(x,s] \subseteq D$}
                \step{b}{\pick\ $y$ with $x < y < s$}
                \begin{proof}
                    \pf\ Since $L$ is dense.
                \end{proof}
                \step{c}{$y \in C$}
                \begin{proof}
                    \pf\ From \stepref{9a}.
                \end{proof}
                \step{d}{$y \in D$}
                \begin{proof}
                    \pf\ From \stepref{a}.
                \end{proof}
                \qedstep
                \begin{proof}
                    \pf\ This contradicts \stepref{2}.
                \end{proof}
            \end{proof}
        \end{proof}        
    \end{proof}
    \step{Two}{If $L$ is connected then $L$ is a linear continuum.}
    \begin{proof}
        \step{1}{\assume{$L$ is connected.}}
        \step{2}{Every nonempty subset of $L$ that is bounded above has a supremum.}
        \begin{proof}
            \step{a}{\pflet{$X$ be a nonempty subset of $L$ bounded above by $b$.}}
            \step{b}{\assume{for a contradiction $X$ has no supremum.}}
            \step{b}{\pflet{$U$ be the set of upper bounds of $X$,}}
            \step{c}{$U$ is nonempty.}
            \begin{proof}
                \pf\ Since $b \in U$.
            \end{proof}
            \step{d}{$U$ is open.}
            \begin{proof}
                \step{i}{\pflet{$x \in U$}}
                \step{ii}{\pick\ an upper bound $y$ for $X$ such that $y < x$}
                \step{iii}{Either $x$ is greatest in $L$ and $(y,x] \subseteq U$, or there exists $z > x$ such that $(y,z) \subseteq U$}
            \end{proof}
            \step{e}{\pflet{$V$ be the set of lower bounds of $U$.}}
            \step{f}{$V$ is nonempty.}
            \begin{proof}
                \pf\ Since $X \subseteq V$
            \end{proof}
            \step{g}{$V$ is open.}
            \begin{proof}
                \step{i}{\pflet{$x \in V$}}
                \step{ii}{\pick\ $y \in X$ with $x < y$}
                \begin{proof}
                    \pf\ $x$ cannot be an upper bound for $X$, because it would be the supremum of $X$.
                \end{proof}
                \step{iii}{Either $x$ least in $L$ and $[x,y) \subseteq V$, or there exists $z < x$ such that $(z,y) \subseteq V$}
            \end{proof}
            \step{h}{$L = U \cup V$}
            \begin{proof}
                \step{i}{\pflet{$x \in L - U$}}
                \step{ii}{\pick\ $y \in X$ such that $x < y$}
                \step{iii}{For all $u \in U$ we have $x < y \leq u$}
                \step{iv}{$x \in V$}
            \end{proof}
            \step{i}{$U \cap V = \emptyset$}
            \begin{proof}
                \pf\ Any element of $U \cap V$ would be a supremum of $X$.
            \end{proof}
            \step{j}{$U$ and $V$ form a separation of $L$.}
            \qedstep
            \begin{proof}
                \pf\ This contradicts \stepref{1}.
            \end{proof}
        \end{proof}
        \step{3}{$L$ is dense.}
        \begin{proof}
            \step{a}{\pflet{$x, y \in L$ with $x < y$}}
            \step{b}{There exists $z \in L$ such that $x < z < y$}
            \begin{proof}
                \pf\ Otherwise $(- \infty, y)$ and $(x, +\infty)$ would form a separation of $L$.
            \end{proof}
        \end{proof}
    \end{proof}
    \qed
\end{proof}

\begin{corollary}
    The real line $\RR$ is connected.
\end{corollary}

\begin{corollary}
    \label{corollary:connected_interval}
    Every interval in $\RR$ is connected.
\end{corollary}

\begin{corollary}
    The ordered square is connected.
\end{corollary}

\begin{theorem}[Intermediate Value Theorem]
    Let $X$ be a connected space. Let $Y$ be a linearly ordered set under the order topology. Let $f : X \rightarrow Y$ be continuous.
    Let $a, b \in X$ and $r \in Y$. Suppose $f(a) < r < f(b)$. Then there exists $c \in X$ such that $f(c) = r$.
\end{theorem}

\begin{proof}
    \pf\ Otherwise $\inv{f}((- \infty, r))$ and $\inv{f}((r, + \infty))$ would form a separation of $X$. \qed
\end{proof}

\begin{proposition}
    Every function $f : [0,1] \rightarrow [0,1]$ has a fixed point.
\end{proposition}

\begin{proof}
    \pf
    \step{1}{\pflet{$g : [0,1] \rightarrow [-1,1]$ be the function $g(x) = f(x) - x$} \prove{there exists $x \in [0,1]$ such that $g(x) = 0$}}
    \step{2}{\assume{without loss of generality $g(0) \neq 0$ and $g(1) \neq 0$}}
    \step{3}{$g(0) > 0$}
    \step{4}{$g(1) < 0$}
    \step{5}{There exists $x \in (0,1)$ such that $g(x) = 0$}
    \begin{proof}
        \pf\ By the Intermediate Value Theorem.
    \end{proof}
\end{proof}

\begin{example}
    The space $\RR_K$ is connected.
\end{example}

\begin{proof}
    \pf
    \step{1}{\assume{for a contradiction $U$ and $V$ form a separation of $\RR_K$}}
    \step{2}{\assume{without loss of generality $0 \in U$}}
    \step{3}{There exists an open interval $(a,b)$ such that $(a,b) -K\subseteq U$
    and $(a,b) \nsubseteq U$}
    \begin{proof}
        \pf\ Otherwise $U$ and $V$ would form a separation of $\RR$.
    \end{proof}
    \step{4}{\pick\ $1/n \in (a,b) - U$}
    \step{5}{$1/n \in V$}
    \step{6}{There exists an open interval $(c,d)$ around $1/n$ that is included in $V$}
    \qedstep
    \begin{proof}
        \pf\ This is a contradiction since $(a,b) - K$ and $(c,d)$ must intersect.
    \end{proof}
    \qed
\end{proof}

\section{Totally Disconnected Spaces}

\begin{definition}[Totally Disconnected]
    A topological space $X$ is \emph{totally disconnected} if and only if the only connected subspaces are the singletons.
\end{definition}

\begin{example}
    Every discrete space is totally disconnected.
\end{example}

\begin{example}
    The rationals $\QQ$ are totally disconnected.
\end{example}

\begin{example}
    The Cantor set is totally disconnected.
\end{example}

\begin{proof}
    \pf
    \step{1}{\pflet{$(A_n)$ be the sequence of sets in Definition \ref{definition:Cantor_set}.}}
    \step{2}{\pflet{$C$ be the Cantor set $\bigcap_n A_n$}}
    \step{3}{\assume\ for a contradiction $D \subseteq C$ is connected and has more than one point.}
    \step{3}{\pflet{$x, y \in D$ with $x < y$}}
    \step{4}{\pick\ $n$ such that $|x - y| > 1 / 3^n$}
    \step{5}{$A_n$ is a sequence of disjoint intervals of length $1/3^n$}
    \step{6}{$x$ and $y$ are in two different intervals out of the intervals that make up $A_n$}
    \step{7}{There exists $z$ with $x < z < y$ such that $z \notin A_n$}
    \step{8}{$(-\infty, z) \cap D$ and $(z, +\infty) \cap D$ form a separation of $D$.}
    \qed
\end{proof}

\section{Paths and Path Connectedness}

\begin{definition}[Path]
    Let $X$ be a topological space and $a, b \in X$. A \emph{path} from $a$ to $b$ is a continuous function $p : [0,1] \rightarrow X$
    such that $p(0) = a$ and $p(1) = b$.
\end{definition}

\begin{definition}[Path Connected]
    A topological space is \emph{path connected} if and only if there exists a path between any two points.
\end{definition}

\begin{proposition}
    \label{proposition:connected_path_connected}
    Every path connected space is connected.
\end{proposition}

\begin{proof}
    \pf
    \step{1}{\pflet{$X$ be a path connected space.}}
    \step{2}{\assume{for a contradiction $C$ and $D$ form a separation of $X$.}}
    \step{3}{\pick\ $a \in C$ and $b \in D$.}
    \step{4}{\pick\ a path $p : [0,1] \rightarrow X$ from $a$ to $b$.}
    \step{5}{$\inv{p}(C)$ and $\inv{p}(D)$ form a separation of $[0,1]$.}
    \qedstep
    \begin{proof}
        \pf\ This contradicts Corollary \ref{corollary:connected_interval}.
    \end{proof}
    \qed
\end{proof}

\begin{example}
    The space $S_\Omega$ is not path connected, because it is not connected.
\end{example}

\begin{example}
    The space $\overline{S_\Omega}$ is not path connected, because it is not connected.
\end{example}

\begin{example}
    The space $S_\Omega \times \overline{S_\Omega}$ is not path connected, because it is not connected.
\end{example}

\begin{example}
    The space $\RR_l$ is not path connected, because it is not connected.
\end{example}

\begin{corollary}
    The Sorgenfrey plane is not path connected.
\end{corollary}

\begin{corollary}
    The space $\RR^\omega$ is not path connected.
\end{corollary}

An example that shows the converse does not hold:
\begin{example}
    The ordered square is not path connected.

    \begin{proof}
        \pf
        \step{1}{\assume{for a contradiction $p : [0,1] \rightarrow I_o^2$ is a path from $(0,0)$ to $(1,1)$.}}
        \step{2}{$p$ is surjective.}
        \begin{proof}
            \pf\ By the Intermediate Value Theorem.
        \end{proof}
        \step{3}{For $x \in [0,1]$, \pick\ a rational $q_x \in \inv{p}((x,0),(x,1))$}
        \begin{proof}
            \pf\ Since $\inv{p}((x,0),(x,1))$ is open and nonempty by \stepref{2}.
        \end{proof}
        \step{4}{For $x, y \in [0,1]$, if $x \neq y$ then $q_x \neq q_y$}
        \begin{proof}
            \pf\ We have $p(q_x) \neq p(q_y)$ because $((x,0),(x,1))$ and $((y,0),(y,1))$ are disjoint.
        \end{proof}
        \step{5}{$\{q_x \mid x \in [0,1] \}$ is an uncountable set of rationals.}
        \qedstep
        \begin{proof}
            \pf\ This contradicts the fact that the rationals are countable.
        \end{proof}
        \qed
    \end{proof}
\end{example}

\begin{proposition}
    \label{proposition:path_connected_continuous_image}
    The continuous image of a path connected space is path connected.
\end{proposition}

\begin{proof}
    \pf
    \step{1}{\pflet{$X$ be a path connected space, $Y$ a topological space, and $f : X \twoheadrightarrow Y$ be continuous and surjective.}}
    \step{2}{\pflet{$a, b \in Y$}}
    \step{3}{\pick\ $c, d \in X$ with $f(c) = a$ and $f(d) = b$}
    \step{4}{\pick\ a path $p : [0,1] \rightarrow X$ from $c$ to $d$.}
    \step{5}{$f \circ p$ is a path from $a$ to $b$ in $Y$.}
    \qed
\end{proof}

\begin{proposition}[AC]
    \label{proposition:product_path_connected}
    The product of a family of path-connected spaces is path-connected.
\end{proposition}

\begin{proof}
    \pf
    \step{1}{\pflet{$\{ X_\alpha \}_{\alpha \in J}$ be a family of path-connected spaces.}}
    \step{2}{\pflet{$a, b \in \prod_{\alpha \in J} X_\alpha$}}
    \step{3}{For $\alpha \in J$, \pick\ a path $p_\alpha : [0,1] \rightarrow X_\alpha$ from $a_\alpha$ to $b_\alpha$}
    \begin{proof}
        \pf\ Using the Axiom of Choice.
    \end{proof}
    \step{4}{Define $p : [0.1] \rightarrow \prod_{\alpha \in J} X_\alpha$ by $p(t)_\alpha = p_\alpha(t)$}
    \step{5}{$p$ is a path from $a$ to $b$.}
    \begin{proof}
        \pf\ Theorem \ref{theorem:product_continuous}.
    \end{proof}
    \qed
\end{proof}

\begin{corollary}
    The space $\RR^\omega$ is path connected.
\end{corollary}

\begin{corollary}
    The space $\RR^I$ is path connected.
\end{corollary}

\begin{proposition}
    The continuous image of a path-connected space is path-connected.
\end{proposition}

\begin{proof}
    \pf
    \step{1}{\pflet{$f : X \twoheadrightarrow Y$ be continuous and surjective
    where $X$ is path-connected.}}
    \step{2}{\pflet{$a, b \in Y$}}
    \step{3}{\pick\ $a', b' \in X$ with $f(a') = a$ and $f(b') = b$.}
    \step{4}{\pick\ a path $p : [0,1] \rightarrow X$ from $a'$ to $b'$.}
    \step{5}{$f \circ p$ is a path from $a$ to $b$.}
    \qed
\end{proof}

\begin{proposition}
    Let $X$ be a topological space. The union of a set of path-connected subspaces of $X$
    that have a point in common is path-connected.
\end{proposition}

\begin{proof}
    \pf
    \step{1}{\pflet{$\AA$ be a set of path-connected subspaces of $X$ with the point $a$ in common.}}
    \step{2}{\pflet{$b, c \in \bigcup \AA$}}
    \step{3}{\pick\ $B, C \in \AA$ with $b \in B$ and $c \in C$.}
    \step{4}{\pick\ a path $p$ in $B$ from $b$ to $a$.}
    \step{5}{\pick\ a path $q$ in $C$ from $a$ to $c$.}
    \step{6}{The concatenation of $p$ and $q$ is a path from $b$ to $c$ in $\bigcup \AA$.}
    \qed
\end{proof}

\begin{proposition}
    Let $A \subseteq \RR^2$ be countable. Then $\RR^2 - A$ is path-connected.
\end{proposition}

\begin{proof}
    \pf
    \step{1}{\pflet{$a, b \in \RR^2 - A$}}
    \step{2}{\pick\ a line $l$ in $\RR^2$ with $a$ on one side and $b$ on the other.}
    \step{3}{For every point $x$ on $l$, \pflet{$p_x$ be the path in $\RR^2$ consisting of a line from $a$ to $x$ then a line from $x$ to $b$}}
    \step{4}{For $x \neq y$ we have $p_x$ and $p_y$ have no points in common except $a$ and $b$}
    \step{5}{There are only countably many $x$ such that a point of $A$ lies on $p_x$.}
    \step{6}{There exists $x$ such that $p_x$ is a path from $a$ to $b$ in $\RR^2 - A$.}
    \qed
\end{proof}

\begin{proposition}
    Every open connected subspace of $\RR^2$ is path-connected.
\end{proposition}

\begin{proof}
    \pf
    \step{1}{\pflet{$U$ be an open connected subspace of $\RR^2$.}}
    \step{2}{For all $x_0 \in U$, \pflet{$PC(x_0) = \{ y \in U \mid \text{there exists a path from $x$ to $y$} \}$}}
    \step{2a}{For all $x_0 \in U$, the set $PC(x_0)$ is open and closed in $U$.}
    \begin{proof}
        \step{a}{\pflet{$x_0 \in U$}}
        \step{b}{$PC(x_0)$ is open in $U$}
        \begin{proof}
            \step{i}{\pflet{$y \in PC(x_0)$}}
            \step{ii}{\pick\ $\epsilon > 0$ such that $B(y,\epsilon) \subseteq U$}
            \begin{proof}
                \pf\ Since $U$ is open.
            \end{proof}
            \step{iii}{$B(y, \epsilon) \subseteq PC(x_0)$}
            \begin{proof}
                \pf\ For all $z \in B(y, \epsilon)$, pick a path from $x_0$ to $y$ then concatenate the straight line from $y$ to $z$.
            \end{proof}
        \end{proof}
        \step{c}{$PC(x_0)$ is closed in $U$}
        \begin{proof}
            \step{i}{\pflet{$y \in U$ be a limit point of $PC(x_0)$}}
            \step{ii}{\pick\ $\epsilon > 0$ such that $B(y, \epsilon) \subseteq U$}
            \step{iii}{\pick\ $z \in PC(x_0) \cap B(y, \epsilon)$}
            \step{iv}{$y \in PC(x_0)$}
            \begin{proof}
                \pf\ Pick a path from $x_0$ to $z$ then concatenate the straight line from $z$ to $y$.
            \end{proof}
        \end{proof}
    \end{proof}
    \step{3}{$PC(x_0) = U$}
    \begin{proof}
        \pf\ Proposition \ref{proposition:open_closed_connective}.
    \end{proof}
    \qed
\end{proof}

\begin{example}
    If $A$ is a connected subspace of $X$, then $A^\circ$ is not necessarily connected.

    Take two closed circles in $\RR^2$ that touch at one point. The interior of this space is
    two open circles, and these two circles form a separation.
\end{example}

\begin{example}
    If $A$ is a connected subspace of $X$ then $\partial A$ is not necessarily connected.

    We have $[0,1]$ is connected but $\partial [0,1] = \{ 0, 1 \}$ is not.
\end{example}

\begin{example}
    If $A$ is a subspace of $X$ and $A^\circ$ and $\partial A$ are connected, then $A$ is not necessarily connected.

    We have $\QQ^\circ = \emptyset$ and $\partial \QQ = \RR$ are connected but $\QQ$ is not connected.
\end{example}

\begin{proposition}
    \label{proposition:components_box_topology}
    Give $\RR^\omega$ the box topology. Let $x, y \in \RR^\omega$. Then $x$ and $y$ lie in the same comoponent if and only if $x-y$ is eventually zero, i.e.
    there exists $N$ such that, for all $n \geq N$,
    we have $x_n = y_n$.
\end{proposition}

\begin{proof}
    \pf
    \step{1}{The component containing 0 is the set of sequences that are eventually zero.}
    \begin{proof}
        \step{a}{\pflet{$B$ be the set of sequences that are eventually zero.}}
        \step{b}{$B$ is path-connected.}
        \begin{proof}
            \step{i}{\pflet{$x, y \in B$}}
            \step{o}{\pick\ $N$ such that, for all $n \geq N$, we have $x_n = y_n = 0$}
            \step{ii}{\pflet{$p : [0,1] \rightarrow \RR^\omega$, $p(t) = (1-t)x + ty$} \prove{$p$ is continuous.}}
            \step{iii}{\pflet{$t \in [0,1]$ and $\prod_j U_j$ be a basic open neighbourhood of $p(t)$, where each $U_j$ is open in $\RR$}}
            \step{iv}{\pick\ $\delta$ such that, for all $n < N$ and all $s \in [0,1]$, if $|s-t| < \delta$ then $p(s)_n \in U_n$}
            \step{v}{For all $s \in [0,1]$, if $|s-t| < \delta$ then $p(s) \in \prod_j U_j$}
        \end{proof}
        \step{bb}{$B$ is connected.}
        \begin{proof}
            \pf\ Proposition \ref{proposition:connected_path_connected}.
        \end{proof}
        \step{c}{If $C$ is connected and $B \subseteq C$ then $B = C$.}
        \begin{proof}
            \step{i}{\assume{$C$ is connected and $B \subseteq C$}}
            \step{ii}{\assume{for a contradiction $x \in C - B$}}
            \step{iii}{For $n \geq 1$, \pflet{$c_n = 1$ if $x_n = 0$, $c_n = n / x_n$ otherwise}}
            \step{iv}{\pflet{$h : \RR^\omega \rightarrow \RR^\omega$ be the function $h(x) = (c_n x_n)_{n \geq 1}$}}
            \step{v}{$h$ is a homeomorphism of $\RR^\omega$ with itself.}
            \step{vi}{$h(x)$ is unbounded.}
            \begin{proof}
                \pf\ For any $b > 0$, pick $N > b$ such that $x_N \neq 0$. Then $h(x)_N > b$.
            \end{proof}
            \step{vii}{$\inv{h}(\{ \text{bounded sequences} \}) \cap C$ and $\inv{h}(\{ \text{unbounded sequences} \}) \cap C$ form a separation of $C$}
            \qedstep
            \begin{proof}
                \pf\ This contradicts \stepref{i}.
            \end{proof}
        \end{proof}
    \end{proof}
    \qedstep
    \begin{proof}
        \pf\ Since $\lambda x.x-y$ is a homeomorphism of $\RR^\omega$ with itself.
    \end{proof}
    \qed
\end{proof}

\section{The Topologist's Sine Curve}

\begin{definition}[The Topologist's Sine Curve]
    Let $S = \{ (x, \sin 1/x) \mid 0 < x \leq 1 \}$, The \emph{topologist's sine curve} is the closure $\overline{S}$ of $S$ in $\RR^2$.
\end{definition}

\begin{proposition}
    The topologist's sine curve is connected.
\end{proposition}

\begin{proof}
    \pf
    \step{1}{\pflet{$S = \{ (x, \sin 1/x) \mid 0 < x \leq 1 \}$}}
    \step{2}{$S$ is connected.}
    \begin{proof}
        \pf\ Theorem \ref{theorem:connected_continuous_image}.
    \end{proof}
    \step{3}{$\overline{S}$ is connected.}
    \begin{proof}
        \pf\ Theorem \ref{theorem:connected_closure}.
    \end{proof}
    \qed
\end{proof}

\begin{proposition}
    The topologist's sine curve is $\{ (x, \sin 1/x) \mid 0 < x \leq 1 \}
    \cup (\{0\} \times [-1.1])$.
\end{proposition}

\begin{proof}
    \pf\ Sketch proof: Given a point $(0.y)$ with $-1 \leq y \leq 1$, pick $a$ such that $\sin a = y$. Then $((1/a,y), (1/(a+2\pi),y), (1/(a+4\pi),y), \ldots)$ is a sequence
    in $S$ that converges to $(0,y)$.

    Conversely, let $(x,y)$ be any point not in $S \cup (\{0\} \times [-1,1])$. If $x < 0$ or $y > 1$ or $y < -1$ then we can easily find a neighbourhood that does not intersect
    $S \cup (\{0\} \times [-1,1])$. If $x > 0$ and $-1 \leq y \leq 1$, then we have $y \neq \sin 1/x$. Hence pick a neighbourhood that does not intersect $S$.
\end{proof}

\begin{proposition}
    \label{proposition:greatest_element_closed}
    Every closed subset of $\RR$ that is bounded above has a greatest element.
\end{proposition}

\begin{proof}
    \pf\ It has a supremum, which is a limit point of the set and hence an element. \qed
\end{proof}

\begin{proposition}[CC]
    The topologist's sine curve is not path connected.
\end{proposition}

\begin{proof}
    \pf
    \step{1}{\assume{For a contradction $p : [0,1] \rightarrow \overline{S}$ is a path from $(0,0)$ to $(1, \sin 1)$.}}
    \step{2}{$\{ t \in [0,1] \mid p(t) \in \{ 0 \} \times [-1,1] \}$ is closed.}
    \begin{proof}
        \pf\ Since $p$ is continuous and $\{0\} \times [-1,1]$ is closed.
    \end{proof}
    \step{3}{\pflet{$b$ be the largest number in $[0,1]$ such that $p(b) \in \{ 0 \} \times [-1,1]$.}}
    \begin{proof}
        \pf\ Proposition \ref{proposition:greatest_element_closed}.
    \end{proof}
    \step{4}{\pflet{$x : [b,1] \rightarrow \overline{S}$ be the function $\pi_1 \circ p$}}
    \step{4}{\pflet{$y : [b,1] \rightarrow \overline{S}$ be the function $\pi_2 \circ p$}}
    \step{5}{\pick\ a sequence $t_n$ in $(b,1]$ such that $t_n \rightarrow b$ and $y(t_n) = (-1)^n$ for all $n$}
    \begin{proof}
        \step{a}{\pflet{$n \geq 1$}}
        \step{b}{\pick\ $u$ with $0 < u < x(1/n)$ and $\sin (1/u) = (-1)^n$}
        \step{c}{\pick\ $t_n$ with $b < t_n < 1/n$ and $x(t_n) = u$}
        \begin{proof}
            \pf\ By the Intermediate Value Theorem
        \end{proof}
    \end{proof}
    \qedstep
    \begin{proof}
        \pf\ This contradicts Proposition \ref{proposition:converge_continuous} since $y$ is continuous and $y(t_n)$ does not converge.
    \end{proof}
    \qed
\end{proof}

\begin{corollary}
    The closure of a path-connected subspace of a space is not necessarily path-connected.
\end{corollary}

\section{The Long Line}

\begin{definition}[The Long Line]
    The \emph{long line} is the space $\omega_1 \times [0,1)$ in the dictionary order
    under the order topology, where $\omega_1$ is the first uncountable ordinal.
\end{definition}

\begin{lemma}
    \label{lemma:interval_in_long_line}
    For any ordinal $\alpha$ with $0 < \alpha < \omega_1$ we have $[(0,0),(\alpha,0)) \cong
    [0,1)$
\end{lemma}

\begin{proof}
        \step{a}{$[(0,0),(1,0)) \cong [0,1)$}
        \begin{proof}
            \pf\ The map $\pi_2$ is a homeomorphism.
        \end{proof}
        \step{b}{If $[(0,0),(\alpha,0)) \cong [0,1)$ then $[(0,0),(\alpha + 1,0)) \cong [0,1)$}
        \begin{proof}
            \pf\ Proposition \ref{proposition:interval_concatenate_interval}.
        \end{proof}
        \step{c}{If $\lambda$ is a limit ordinal with $\lambda < \omega_1$ 
        and $[(0,0),(\alpha,0)) \cong [0,1)$ for all $\alpha$ with $0 < \alpha < \lambda$
        then $[(0,0),(\lambda,0)) \cong [0,1)$}
        \begin{proof}
            \step{i}{\pflet{$\lambda$ be a limit ordinal $< \omega_1$}}
            \step{ii}{\assume{$[(0,0),(\alpha,0)) \cong [0,1)$ for all $\alpha$ with $0 < \alpha < \lambda$}}
            \step{ii}{\pick\ a sequence of ordinals $\alpha_0 < \alpha_1 < \cdots$
            with limit $\lambda$}
            \begin{proof}
                \pf\ Since $\lambda$ is countable.
            \end{proof}
            \step{iii}{$[(\alpha_i,0),(\alpha_{i+1},0)) \cong [0,1)$ for all $i$}
            \begin{proof}
                \pf\ Lemma \ref{lemma:order_iso_real_intervals}.
            \end{proof}
            \qedstep
            \begin{proof}
                \pf\ By Proposition \ref{proposition:interval_concatenate_intervals}.
            \end{proof}
        \end{proof}
        \qedstep
        \begin{proof}
            \pf\ By transfinite induction.
        \end{proof}
\end{proof}

\begin{proposition}[CC]
    The long line is path-connected.
\end{proposition}

\begin{proof}
    \pf
    \step{2}{\pflet{$(\alpha,i),(\beta,j) \in \omega_1 \times [0,1)$}}
    \step{3a}{\assume{without loss of generality $(\alpha, i) < (\beta, j)$}}
    \step{3}{$[(0,0),(\beta + 1, 0)) \cong [0,1)$}
    \begin{proof}
        \pf\ By Lemma \ref{lemma:interval_in_long_line}
    \end{proof}
    \step{4}{$[(\alpha,i),(\beta,j)) \cong [0,1)$}
    \begin{proof}
        \pf\ Lemma \ref{lemma:order_iso_real_intervals}.
    \end{proof}
    \step{5}{\pick\ a homeomorphism $q : [0,1) \rightarrow [(\alpha,i),(\beta,j))$}
    \step{6}{$q \cup \{(1,(\beta,j))\}$ is a path from $(\alpha, i)$ to $(\beta, j)$}
    \qed
\end{proof}

\begin{proposition}
    Every point in the long line has a neighbourhood homeomorphic to an interval in $\RR$.
\end{proposition}

\begin{proof}
    \pf\ For any $(\alpha, i)$ in the long line, the neighbourhood $[(0,0),(\alpha + 1, 0))$
    satisfies the condition by Lemma \ref{lemma:interval_in_long_line}.
\end{proof}

\section{Components}

\begin{proposition}
    Let $X$ be a topological space. Define the relation $\sim$ on $X$ by
    $a \sim b$ if and only if there exists a connected subspace $A \subseteq X$
    with $a, b \in A$. Then $\sim$ is an equivalence relation on $X$.
\end{proposition}

\begin{proof}
    \pf
    \step{1}{$\sim$ is reflexive.}
    \begin{proof}
        \pf\ For any $a \in X$ we have $\{a\}$ is a connected subspace that
        contains $a$.
    \end{proof}
    \step{2}{$\sim$ is symmetric.}
    \begin{proof}
        \pf\ Trivial.
    \end{proof}
    \step{3}{$\sim$ is transitive.}
    \begin{proof}
        \step{i}{\pflet{$a, b, c \in X$}}
        \step{ii}{\assume{$a \sim b$ and $b \sim c$}}
        \step{iii}{\pick\ connected subspaces $A$ and $B$ with $a,b \in A$
        and $b,c \in B$}
        \step{iv}{$A \cup B$ is a connected subspace that contains $a$ and $c$}
        \begin{proof}
            \pf\ Theorem \ref{theorem:connected_union}.
        \end{proof}
    \end{proof}
    \qed
\end{proof}

\begin{definition}[(Connected) Component]
    Let $X$ be a topological space. The \emph{(connected) components} of
    $X$ are the equivalence classes under the above $\sim$.
\end{definition}

\begin{lemma}
    \label{lemma:component_includes_connected}
    Let $X$ be a topological space. If $A \subseteq X$ is connected 
    and nonempty then there
    exists a unique component $C$ of $X$ such that $A \subseteq C$.
\end{lemma}

\begin{proof}
    \pf
    \step{1}{\pick\ $a \in A$}
    \step{2}{\pflet{$C$ be the $\sim$-equivalence class of $a$.}}
    \step{3}{$A \subseteq C$}
    \begin{proof}
        \pf\ For all $x \in A$ we have $x \sim a$.
    \end{proof}
    \step{4}{If $C'$ is a component and $A \subseteq C'$ then $C = C'$}
    \begin{proof}
        \pf\ Since we have $a \in C'$.
    \end{proof}
    \qed
\end{proof}

\begin{theorem}
    Let $X$ be a topological space. The components of $X$ are connected
    disjoint subspaces of $X$ whose union is $X$ such that each nonempty
    connected subspace of $X$ intersects only one of them.
\end{theorem}

\begin{proof}
    \pf
    \step{1}{Every component of $X$ is connected.}
    \begin{proof}
        \pf\ For $a \in X$, the $\sim$-equivalence class of $a$ is
        $\bigcup \{ A \subseteq X \mid A \text{ is connected}, a \in A \}$
        which is connected by Theorem \ref{theorem:connected_union}.
    \end{proof}
    \step{2}{The components form a partition of $X$.}
    \begin{proof}
        \pf\ Immediate from the definition.
    \end{proof}
    \step{3}{Every nonempty connected subspace of $X$ intersects a unique
    component of $X$.}
    \begin{proof}
        \step{a}{\pflet{$A \subseteq X$ be connected and nonempty.}}
        \step{b}{\pflet{$C$ be the component such that $A \subseteq C$}}
        \begin{proof}
            \pf\ Lemma \ref{lemma:component_includes_connected}.
        \end{proof}
        \step{c}{$A$ intersects $C$}
        \step{d}{If $A$ intersects the component $C'$ then $C' = C$}
        \begin{proof}
            \step{i}{\pflet{$C'$ be a component that intersects $A$}}
            \step{ii}{\pick\ $b \in A \cap C'$}
            \step{iii}{$A \subseteq C'$}
            \begin{proof}
                \pf\ For all $x \in A$ we have $x \sim b$.
            \end{proof}
            \step{iv}{$C = C'$}
            \begin{proof}
                \pf\ By uniqueness in \stepref{b}.
            \end{proof}
        \end{proof}
    \end{proof}
    \qed
\end{proof}

\begin{proposition}
    Every component of a space is closed.
\end{proposition}

\begin{proof}
    \pf
    \step{1}{\pflet{$X$ be a topological space and $C$ a component of $X$.}}
    \step{2}{$\overline{C}$ is connected.}
    \begin{proof}
        \pf\ Theorem \ref{theorem:connected_closure}.
    \end{proof}
    \step{3}{$C = \overline{C}$}
    \begin{proof}
        \pf\ Lemma \ref{lemma:subset_connected_subspace}.
    \end{proof}
    \step{4}{$C$ is closed.}
    \begin{proof}
        \pf\ Lemma \ref{lemma:closure_closed}.
    \end{proof}
    \qed
\end{proof}

\begin{proposition}
    If a topological space has finitely many components then every component is open.
\end{proposition}

\begin{proof}
    \pf\ Each component is the complement of a finite union of closed sets. \qed
\end{proof}

\section{Path Components}

\begin{proposition}
    Let $X$ be a topological space. Define the relation $\sim$ on $X$ by:
    $a \sim b$ if and only if there exists a path in $X$ from $a$ to $b$.
    Then $\sim$ is an equivalence relation on $X$.
\end{proposition}

\begin{proof}
    \pf
    \step{1}{$\sim$ is reflexive.}
    \begin{proof}
        \pf\ For $a \in X$, the constant function $[0,1] \rightarrow X$
        with value $a$ is a path from $a$ to $a$.
    \end{proof}
    \step{2}{$\sim$ is symmetric.}
    \begin{proof}
        \pf\ If $p : [0,1] \rightarrow X$ is a path from $a$ to $b$,
        then $\lambda t.p(1-t)$ is a path from $b$ to $a$.
    \end{proof}
    \step{3}{$\sim$ is transitive.}
    \begin{proof}
        \pf\ Concatenate paths.
    \end{proof}
    \qed
\end{proof}

\begin{definition}[Path Component]
    Let $X$ be a topological space. The \emph{path components} of $X$ are
    the equivalence relations under $\sim$.
\end{definition}.

\begin{theorem}
    \label{theorem:path_components}
    The path components of $X$ are path-connected disjoint subspaces of $X$
    whose union is $X$ such that every nonempty path-connected subspace of $X$
    intersects exactly one path component.
\end{theorem}

\begin{proof}
    \pf
    \step{1}{Every path component is path-connected.}
    \begin{proof}
        \pf\ If $a$ and $b$ are in the same path component then $a \sim b$,
        i.e. there exists a path from $a$ to $b$.
    \end{proof}
    \step{2}{The path components are disjoint and their union is $X$.}
    \begin{proof}
        \pf\ Immediate from the definition.
    \end{proof}
    \step{3}{Every non-empty path-cönnected subspace of $X$ intersects exactly one path component.}
    \begin{proof}
        \step{a}{\pflet{$A$ be a nonempty path-connected subspace of $X$.}}
        \step{b}{\pick\ $a \in A$}
        \step{c}{$A$ intersects the $\sim$-equivalence class of $a$.}
        \step{d}{\pflet{$C$ be any path component that intersects $A$.}}
        \step{e}{\pick\ $b \in A \cap C$}
        \step{f}{$a \sim b$}
        \begin{proof}
            \pf\ Since $A$ is path-connected.
        \end{proof}
        \step{g}{$C$ is the $\sim$-equivalence class of $a$.}
    \end{proof}
    \qed
\end{proof}

\begin{proposition}
    \label{proposition:path_component_sub_component}
    Every path component is included in a component.
\end{proposition}

\begin{proof}
    \pf
    \step{1}{\pflet{$X$ be a topological space and $C$ a path component of $X$.}}
    \step{2}{$C$ is path-connected.}
    \begin{proof}
        \pf\ Theorem \ref{theorem:path_components}.
    \end{proof}
    \step{3}{$C$ is connected.}
    \begin{proof}
        \pf\ Proposition \ref{proposition:connected_path_connected}.
    \end{proof}
    \step{4}{$C$ is included in a component.}
    \begin{proof}
        \pf\ Lemma \ref{lemma:component_includes_connected}.
    \end{proof}
    \qed
\end{proof}

\section{Local Connectedness}

\begin{definition}[Locally Connected]
    Let $X$ be a topological space and $a \in X$. Then $X$ is \emph{locally
    connected} at $a$ if and only if every neighbourhood of $a$ includes a
    connected neighbourhood of $a$.

    The space $X$ is \emph{locally connected} if and only if it is locally
    connected at every point.
\end{definition}

\begin{example}
    The real line is both connected and locally connected.
\end{example}

\begin{example}
    The space $\RR - \{0\}$ is disconnected but locally connected.
\end{example}

\begin{example}
    The topologist's sine curve is connected but not locally connected.
\end{example}

\begin{example}
    The rationals $\QQ$ are neither connected nor locally connected.
\end{example}

\begin{example}
    The space $S_\Omega$ is not locally connected, because $\omega$ has no connected neighbourhood.
\end{example}

\begin{example}
    The space $\overline{S_\Omega}$ is not locally connected, because $\omega$ has no connected neighbourhood.
\end{example}

\begin{example}
    The space $S_\Omega \times \overline{S_\Omega}$ is not locally connected, because $(\omega,0)$ has no connected neighbourhood.
\end{example}

\begin{example}
    The space $\RR_l$ is not locally connected.
\end{example}

\begin{proposition}
    The space $\RR^\omega$ is locally connected.
\end{proposition}

\begin{proof}
    \pf\ Every basic open set is connected by Theorem \ref{theorem:product_connected}. \qed
\end{proof}

\begin{proposition}
    The space $\RR^I$ is locally connected.
\end{proposition}

\begin{proof}
    \pf\ Every basic open set is connected by Theorem \ref{theorem:product_connected}. \qed
\end{proof}

\begin{proposition}
    The space $\RR_K$ is not locally connected.
\end{proposition}

\begin{proof}
    \pf\ The open neighbourhood $(-1,1) - K$ of 0 contains
    no connected open neighbourhood of 0. \qed
\end{proof}

\begin{theorem}
    \label{theorem:open_component_locally_connected}
    A topological space $X$ is locally connected if and only if, for every
    open set $U$ in $X$, every component of $U$ is open in $X$.
\end{theorem}

\begin{proof}
    \pf
    \step{1}{If $X$ is locally connected then, for every open set $U$ in $X$,
    every component of $U$ is open in $X$.}
    \begin{proof}
        \step{a}{\assume{$X$ is locally connected.}}
        \step{b}{\pflet{$U$ be open in $X$.}}
        \step{c}{\pflet{$C$ be a component of $U$.}}
        \step{d}{\pflet{$a \in C$}}
        \step{e}{\pflet{$V$ be a connected neighbourhood of $a$ such that $V \subseteq U$}}
        \step{f}{$V \subseteq C$}
        \begin{proof}
            \pf\ Lemma \ref{lemma:component_includes_connected}.
        \end{proof}
        \qedstep
        \begin{proof}
            \pf\ Lemma \ref{lemma:open}.
        \end{proof}
    \end{proof}
    \step{2}{If, for every open set $U$ in $X$, every component of $U$ is
    open in $X$, then $X$ is locally connected.}
    \begin{proof}
        \step{a}{\assume{for every open set $U$ in $X$, every component of $U$ is open in $X$.}}
        \step{b}{\pflet{$a \in X$}}
        \step{c}{\pflet{$U$ be a neighbourhood of $a$}}
        \step{d}{The component of $U$ that contains $a$ is a connected neighbourhood of $a$ included in $U$.}
    \end{proof}
    \qed
\end{proof}

\begin{corollary}
    The space $\RR^\omega$ under the box topology is not locally connected.
\end{corollary}

\begin{proof}
    \pf\ Since the components are not open (Proposition \ref{proposition:components_box_topology}).
\end{proof}

\begin{example}
    The ordered square is locally connected.
\end{example}

\begin{proof}
    \pf\ Every neighbourhood of a point includes an interval around that point, which is connected 
    by Theorem \ref{theorem:connected_continuum}.
\end{proof}

\begin{example}
    Let $X$ be the set of all rational points on the line segment $[0,1] \times \{0\}$,
    and $Y$ the set of all rational points on the line segment $[0,1] \times \{1\}$.
    Let $A$ be the space consisting of all line segments joining the point $(0,1)$ to a point
    of $X$, and all line segments joining the point $(1,0)$ to a point of $Y$. Then $A$ is
    path-connected but is not locally connected at any point,
\end{example}

\begin{proposition}
    Any discrete topological space is locally connected.
\end{proposition}

\begin{proof}
    \pf\ For any point $x$, the singleton $\{x\}$ is a connected open neighbourhood. \qed
\end{proof}

\begin{proposition}
    The continuous image of a locally connected space is not necessarily locally connected.
\end{proposition}

\begin{proof}
    \pf
    \step{1}{\pick\ a topological space $X$ that is not locally connected.}
    \step{2}{\pflet{$X'$ be the same set as $X$ under the discrete topology.}}
    \step{3}{\pflet{$i : X' \rightarrow X$ be the identity function.}}
    \step{4}{$X'$ is locally connected.}
    \step{5}{$i$ is continuous.}
    \qed
\end{proof}

\begin{proposition}
    Let $X$ and $Y$ be topological spaces and $p : X \twoheadrightarrow Y$ be a quotient map. If $X$ is locally connected then so is $Y$.
\end{proposition}

\begin{proof}
    \pf
    \step{a}{\pflet{$U$ be an open set in $Y$.}}
    \step{b}{\pflet{$C$ be a component of $U$.}}
    \step{c}{$\inv{p}(C)$ is a union of components of $\inv{p}(U)$}
    \begin{proof}
        \step{i}{\pflet{$x \in \inv{p}(C)$}}
        \step{ii}{\pflet{$D$ be the component of $\inv{p}(U)$ that contains $x$.}}
        \step{iii}{$p(D)$ is connected.}
        \begin{proof}
            \pf\ Theorem \ref{theorem:connected_continuous_image}.
        \end{proof}
        \step{iv}{$p(D) \subseteq C$.}
        \begin{proof}
            \pf\ From \stepref{b} since $p(x) \in p(D) \cap C$ (\stepref{i}, \stepref{ii}).
        \end{proof}
        \step{v}{$D \subseteq \inv{p}(C)$}
    \end{proof}
    \step{d}{$\inv{p}(C)$ is open in $\inv{p}(U)$}
    \begin{proof}
        \pf\ Theorem \ref{theorem:open_component_locally_connected}.
    \end{proof}
    \step{e}{$C$ is open in $U$}
    \begin{proof}
        \pf\ Since the restriction of $p$ to $p : \inv{p}(U) \twoheadrightarrow U$
        is a quotient map by Proposition \ref{proposition:quotient_restriction}.
    \end{proof}
    \qedstep
    \begin{proof}
        \pf\ Theorem \ref{theorem:open_component_locally_connected}.
    \end{proof}
    \qed
\end{proof}

\begin{proposition}
    The Sorgenfrey plane is not locally connected.
\end{proposition}

\begin{proof}
    \pf
    \step{1}{\assume{for a contradiction that there is a connected open neighbourhood $U$ of $(0,0)$}}
    \step{2}{\pick\ $x$, $y$ such that $[0,x) \times [0,y) \subseteq U$}
    \step{3}{$[0,x/2)\times[0,y/2)$ is clopen in $U$.}
    \qed
\end{proof}

\begin{proposition}
    An open subspace of a locally connected space is locally connected.
\end{proposition}

\begin{proof}
    \pf
    \step{1}{\pflet{$X$ be locally connected.}}
    \step{2}{\pflet{$Y \subseteq X$ be open.}}
    \step{3}{\pflet{$y \in Y$}}
    \step{4}{\pflet{$U$ be an open neighbourhood of $y$ in $Y$}}
    \step{5}{$U$ is an open neighbourhood of $y$ in $X$.}
    \step{6}{\pick\ a connected open neighbourhood $V$ of $y$ in $X$
    such that $V \subseteq U$}
    \step{7}{$V$ is a connected open neightbourhood of $y$ in $Y$.}
    \qed
\end{proof}

\begin{proposition}
    The product of two locally connected spaces is locally connected.
\end{proposition}

\begin{proof}
    \pf
    \step{1}{\pflet{$X$ and $Y$ be locally connected spaces.}}
    \step{2}{\pflet{$(x,y) \in X \times Y$}}
    \step{3}{\pflet{$U$ be a neighbourhood of $(x,y)$}}
    \step{4}{\pick\ neighbourhoods $V$ of $x$ and $W$ of $y$
    such that $V \times W \subseteq U$}
    \step{5}{\pick\ connected neighbourhoods $V'$ of $x$ and
    $W'$ of $y$ with $V' \subseteq V$ and $W' \subseteq W$}
    \step{6}{$V' \times W'$ is connected.}
    \step{7}{$(x,y) \in V' \times W' \subseteq U$}
    \qed
\end{proof}

The following example shows that the product of a countable family of locally connected spaces
is not necessarily locally connected.

\begin{proposition}
    The space $(\RR - \{0\})^\omega$ is not locally connected.
\end{proposition}

\begin{proof}
    \pf
    \step{1}{\pflet{$p = (1, 1, \ldots)$}}
    \step{2}{\assume{for a contradiction $U$ is a connected open neighbourhood of $p$ in $(\RR - \{0\})^\omega$}}
    \step{3}{\pick\ a basic open set $\prod_n U_n$ with $p \in \prod_n U_n \subseteq U$
    and $U_n = \RR - \{0\}$ for all but finitely many $n$.}
    \step{4}{\pick\ $n$ such that $U_n = \RR - \{0\}$}
    \step{5}{$\pi_n(U) = \RR - \{0\}$ is connected.}
    \qedstep
    \begin{proof}
        \pf\ This is a contradiction.
    \end{proof}
    \qed
\end{proof}

\section{Local Path Connectedness}

\begin{definition}[Locally Path-Connected]
    Let $X$ be a topological space and $a \in X$. Then $X$ is \emph{locally
    path-connected} at $a$ if and only if every neighbourhood of $a$ includes a
    path-connected neighbourhood of $a$.

    The space $X$ is \emph{locally path-connected} if and only if it is locally
    path-connected at every point.
\end{definition}

\begin{proposition}
    The space $\RR^\omega$ is locally path connected.
\end{proposition}

\begin{proof}
    \pf\ Every basic open neighbourhood is path connected by
    Proposition \ref{proposition:product_path_connected}. \qed
\end{proof}

\begin{proposition}
    The space $\RR^I$ is locally path connected.
\end{proposition}

\begin{proof}
    \pf\ Every basic open neighbourhood is path connected by
    Proposition \ref{proposition:product_path_connected}. \qed
\end{proof}

\begin{proposition}
    Every locally path connected space is locally connected.
\end{proposition}

\begin{corollary}
    The space $\RR^\omega$ under the box topology is not locally path connected.
\end{corollary}

\begin{corollary}
    The space $\RR_K$ is not locally path connected.
\end{corollary}

\begin{example}
    The space $S_\Omega$ is not locally path connected, because it is not locally connected.
\end{example}

\begin{example}
    The space $\overline{S_\Omega}$ is not locally path connected, because it is not locally connected.
\end{example}

\begin{example}
    The space $S_\Omega \times \overline{S_\Omega}$ is not locally path connected, because it is not locally connected.
\end{example}

\begin{example}
    The space $\RR_l$ is not locally path connected, because it is not locally connected.
\end{example}

\begin{corollary}
    The Sorgenfrey plane is not locally path connected.
\end{corollary}

\begin{theorem}
    \label{theorem:open_path_component_locally_path_connected}
    A topological space $X$ is locally path-connected if and only if, for every
    open set $U$ in $X$, every path component of $U$ is open in $X$.
\end{theorem}

\begin{proof}
    \pf
    \step{1}{If $X$ is locally path-connected then, for every open set $U$ in $X$,
    every path component of $U$ is open in $X$.}
    \begin{proof}
        \step{a}{\assume{$X$ is locally path-connected.}}
        \step{b}{\pflet{$U$ be open in $X$.}}
        \step{c}{\pflet{$C$ be a path component of $U$.}}
        \step{d}{\pflet{$a \in C$}}
        \step{e}{\pflet{$V$ be a path-connected neighbourhood of $a$ such that $V \subseteq U$}}
        \step{f}{$V \subseteq C$}
        \begin{proof}
            \pf\ Lemma \ref{lemma:component_includes_connected}.
        \end{proof}
        \qedstep
        \begin{proof}
            \pf\ Lemma \ref{lemma:open}.
        \end{proof}
    \end{proof}
    \step{2}{If, for every open set $U$ in $X$, every component of $U$ is
    open in $X$, then $X$ is locally connected.}
    \begin{proof}
        \step{a}{\assume{for every open set $U$ in $X$, every component of $U$ is open in $X$.}}
        \step{b}{\pflet{$a \in X$}}
        \step{c}{\pflet{$U$ be a neighbourhood of $a$}}
        \step{d}{The component of $U$ that contains $a$ is a connected neighbourhood of $a$ included in $U$.}
    \end{proof}
    \qed
\end{proof}

\begin{theorem}
    If a space is locally path connected then its components and its path components are the same.
\end{theorem}

\begin{proof}
    \pf
    \step{1}{\pflet{$X$ be a locally path connected space.}}
    \step{2}{\pflet{$C$ be a component of $X$.}}
    \step{3}{\pflet{$x \in C$}}
    \step{4}{\pflet{$P$ be the path component of $x$} \prove{$P = C$}}
    \step{5}{$P \subseteq C$}
    \begin{proof}
        \pf\ Proposition \ref{proposition:path_component_sub_component}.
    \end{proof}
    \step{6}{\pflet{$Q$ be the union of the other path components included in $C$}}
    \step{7}{$C = P \cup Q$}
    \begin{proof}
        \pf\ Proposition \ref{proposition:path_component_sub_component}.
    \end{proof}
    \step{8}{$P$ and $Q$ are open in $C$}
    \begin{proof}
        \step{a}{$C$ is open.}
        \begin{proof}
            \pf\ Theorem \ref{theorem:open_component_locally_connected}.
        \end{proof}
        \qedstep
        \begin{proof}
            \pf\ Theorem \ref{theorem:open_path_component_locally_path_connected}.
        \end{proof}
    \end{proof}
    \step{9}{$Q = \emptyset$}
    \begin{proof}
        \pf\ Otherwise $P$ and $Q$ would form a separation of $C$.
    \end{proof}
    \qed
\end{proof}

\begin{example}
    The ordered square is not locally path connected, since it is connected but not path connected.
\end{example}

\begin{proposition}
    Let $X$ be a locally path-connected space. Then every connected open subspace
    of $X$ is path-connected.
\end{proposition}

\begin{proof}
    \pf
    \step{1}{\pflet{$U$ be a connected open subspace of $X$.}}
    \step{2}{\pflet{$P$ be a path component of $U$.}}
    \step{3}{\pflet{$Q$ be the union of the other path components of $U$.}}
    \step{4}{$P$ and $Q$ are open in $U$.}
    \begin{proof}
        \pf\ Theorem \ref{theorem:open_path_component_locally_path_connected}.
    \end{proof}
    \step{5}{$Q = \emptyset$}
    \begin{proof}
        \pf\ Otherwise $P$ and $Q$ form a separation of $U$.
    \end{proof}
    \qed
\end{proof}

\begin{proposition}
    An open subspace of a locally connected space is locally connected.
\end{proposition}

\begin{proof}
    \pf
    \step{1}{\pflet{$X$ be locally connected.}}
    \step{2}{\pflet{$Y \subseteq X$ be open.}}
    \step{3}{\pflet{$y \in Y$}}
    \step{4}{\pflet{$U$ be an open neighbourhood of $y$ in $Y$}}
    \step{5}{$U$ is an open neighbourhood of $y$ in $X$.}
    \step{6}{\pick\ a path-connected open neighbourhood $V$ of $y$ in $X$
    such that $V \subseteq U$}
    \step{7}{$V$ is a path-connected open neightbourhood of $y$ in $Y$.}
    \qed
\end{proof}

\begin{proposition}
    The product of two locally path-connected spaces is locally connected.
\end{proposition}

\begin{proof}
    \pf
    \step{1}{\pflet{$X$ and $Y$ be locally path-connected spaces.}}
    \step{2}{\pflet{$(x,y) \in X \times Y$}}
    \step{3}{\pflet{$U$ be a neighbourhood of $(x,y)$}}
    \step{4}{\pick\ neighbourhoods $V$ of $x$ and $W$ of $y$
    such that $V \times W \subseteq U$}
    \step{5}{\pick\ path-connected neighbourhoods $V'$ of $x$ and
    $W'$ of $y$ with $V' \subseteq V$ and $W' \subseteq W$}
    \step{6}{$V' \times W'$ is connected.}
    \step{7}{$(x,y) \in V' \times W' \subseteq U$}
    \qed
\end{proof}

The following example shows that the product of a countable family of locally path-connected spaces
is not necessarily locally path-connected.

\begin{proposition}
    The space $(\RR - \{0\})^\omega$ is not locally path-connected.
\end{proposition}

\begin{proof}
    \pf\ It is not locally connected. \qed
\end{proof}

\begin{proposition}
    The continuous image of a locally path connected space is not necessarily locally path connected.
\end{proposition}

\begin{proof}
    \pf
    \step{1}{\pick\ a topological space $X$ that is not locally path connected.}
    \step{2}{\pflet{$X'$ be the same set as $X$ under the discrete topology.}}
    \step{3}{\pflet{$i : X' \rightarrow X$ be the identity function.}}
    \step{4}{$X'$ is locally path connected.}
    \step{5}{$i$ is continuous.}
    \qed
\end{proof}

\section{Weak Local Connectedness}

\begin{definition}[Weakly Locally Connected]
    Let $X$ be a topological space and $a \in X$. Then $X$ is \emph{weakly locally connected} at $a$ if and only if
    every neighbourhood of $a$ includes a connected subspace that includes a neighbourhood of $a$.
\end{definition}

\begin{proposition}
    Let $X$ be a topological space. If $X$ is weakly locally connected at every point then $X$ is locally connected.
\end{proposition}

\begin{proof}
    \pf
    \step{1}{\assume{$X$ is weakly locally connected at every point.}}
    \step{2}{\pflet{$U$ be open in $X$.}}
    \step{3}{\pflet{$C$ be a component of $U$.}}
    \step{4}{$C$ is open in $X$.}
    \begin{proof}
        \step{a}{\pflet{$x \in C$}}
        \step{b}{\pick\ a connected subspace $D$ of $U$ that includes a neighbourhood $V$ of $x$.}
        \step{c}{$D \subseteq C$}
        \begin{proof}
            \pf\ Lemma \ref{lemma:component_includes_connected}.
        \end{proof}
        \step{d}{$x \in V \subseteq C$}
        \qedstep
        \begin{proof}
            \pf\ Lemma \ref{lemma:open}.
        \end{proof}
    \end{proof}
    \qedstep
    \begin{proof}
        \pf\ Theorem \ref{theorem:open_component_locally_connected}.
    \end{proof}
    \qed
\end{proof}

\begin{example}
    The 'infinite broom' (Munkres p.163) is an example of a space that is
    weakly locally connected at a point $p$ but not locally connected at $p$.
\end{example}

\section{Quasicomponents}

\begin{proposition}
    Let $X$ be a topological space. Define $\sim$ on $X$ by $x \sim y$ if and only if there exists no
    separation $U$ and $V$ of $X$ such that $x \in U$ and $y \in V$. Then $\sim$ is an equivalence relation on $X$.
\end{proposition}

\begin{proof}
    \pf
    \step{1}{$\sim$ is reflexive.}
    \begin{proof}
        \pf\ Since the two sets that make up a separation are disjoint.
    \end{proof}
    \step{2}{$\sim$ is symmetric.}
    \begin{proof}
        \pf\ Immediate from the defintion.
    \end{proof}
    \step{3}{$\sim$ is transitive.}
    \begin{proof}
        \step{a}{\assume{$x \sim y$ and $y \sim z$}}
        \step{b}{\assume{for a contradiction there is a separation $U$ and $V$ of $X$
        with $x \in U$ and $z \in V$}}
        \step{c}{$y \in U$ or $y \in V$}
        \qedstep
        \begin{proof}
            \pf\ Either case contradicts \stepref{a}.
        \end{proof}
    \end{proof}
    \qed
\end{proof}

\begin{definition}[Quasicomponents]
    For $X$ a topological space, the \emph{quasicomponents} of $X$ are the
    equivalence classes under $\sim$.
\end{definition}

\begin{proposition}
    Let $X$ be a topological space. Then every component of $X$ is included in a quasicomponent of $X$.
\end{proposition}

\begin{proof}
    \pf
    \step{1}{\pflet{$C$ be a component of $X$.}}
    \step{2}{\pflet{$x, y \in C$} \prove{$x \sim y$}}
    \step{3}{\assume{for a contradiction there exists a separation $U$ and $V$ of $X$ with $x \in U$ and $y \in V$}}
    \step{4}{$C \cap U$ and $C \cap V$ form a separation of $C$.}
    \qedstep
    \begin{proof}
        \pf\ This contradicts \stepref{1}.
    \end{proof}
\end{proof}

\begin{proposition}
    In a locally connected space, the components and the quasicomponents are the same.
\end{proposition}

\begin{proof}
    \pf
    \step{1}{\pflet{$X$ be a locally connected space and $Q$ a quasicomponent of $X$.}}
    \step{2}{\pick\ a component $C$ of $X$ such that $C \subseteq Q$}
    \step{3}{\pflet{$D$ be the union of the components of $X$}}
    \step{4}{$C$ and $D$ are open in $X$.}
    \begin{proof}
        \pf\ Theorem \ref{theorem:open_component_locally_connected}.
    \end{proof}
    \step{5}{$D$ cannot contain any points of $Q$.}
    \begin{proof}
        \pf\ If it did, then $C$ and $D$ would form a separation of $X$
        and there would be points $x, y \in Q$ with $x \in C$ and $y \in D$.
    \end{proof}
    \step{6}{$C = Q$}
    \qed
\end{proof}

\section{Open Coverings}

\begin{definition}[Open Covering]
    Let $X$ be a topological space. An \emph{open covering} of $X$ is a covering
    of $X$ whose elements are all open sets.
\end{definition}

\section{Lindel\"{o}f Spaces}

\begin{definition}[Lindel\"{o}f Space]
    A topological space $X$ is \emph{Lindel\"{o}f} if and only if every open covering
    has a countable subcovering.
\end{definition}

\begin{proposition}
    Let $X$ be a topological space. Then $X$ is Lindel\"{o}f if and only if 
    every set of closed sets that has the countable intersection property has
    nonempty intersection.
\end{proposition}

\begin{proof}
    \pf\ The following are equivalent.
    \begin{enumerate}
        \item $X$ is Lindel\"{o}f.
        \item Every open covering of $X$ has a countable subcovering.
        \item For any set $\CC$ of closed sets, if $\{ X - C \mid C \in \CC \}$ covers $X$ then there is a countable subset $\CC_0$ such that
        $\{ X - C \mid C \in \CC_0 \}$ covers $X$
        \item For any set $\CC$ of closed sets, if $\bigcap \CC = \emptyset$ then there is a countable subset $\CC_0$ with empty intersection.
        \item Any set of closed sets with the countable intersection property has nonempty intersection.
    \end{enumerate}
    \qed
\end{proof}

\begin{proposition}[CC]
    \label{proposition:lindelof_basis}
    Let $X$ be a topological space and $\BB$ a basis for the topology on $X$.
    Then the following are equivalent.
    \begin{enumerate}
        \item $X$ is Lindel\"{o}f.
        \item Every open covering of $X$ by elements of $\BB$ has a countable subcovering.
    \end{enumerate}
\end{proposition}

\begin{proof}
    \pf
    \step{1}{$1 \Rightarrow 2$}
    \begin{proof}
        \pf\ Immediate from definitions.
    \end{proof}
    \step{2}{$2 \Rightarrow 1$}
    \begin{proof}
        \step{a}{\assume{Every open covering of $X$ by elements of $\BB$ has a countable subcovering.}}
        \step{b}{\pflet{$\UU$ be an open covering of $X$.}}
        \step{c}{$\{ B \in \BB \mid \exists U \in \UU. B \subseteq U \}$ covers $X$.}
        \step{d}{\pick\ a finite subcovering $\BB_0$.}
        \step{e}{For $B \in BB$, \pick\ $U_B \in \UU$ such that $B \subseteq U_B$}
        \step{f}{$\{ U_B \mid B \in \BB_0 \}$ covers $X$.}
    \end{proof}
    \qed
\end{proof}

\begin{example}[AC]
    The space $\overline{S_\Omega}$ is Lindel\"{o}f.

    \begin{proof}
        \pf
        \step{1}{\pflet{$\AA$ be any open cover of $\overline{S_\Omega}$}}
        \step{2}{\pick\ $U \in \AA$ such that $\Omega \in U$}
        \step{3}{\pick\ $\alpha < \Omega$ such that $(\alpha, \Omega] \subseteq U$}
        \step{4}{For $\beta < \alpha$, \pick\ $U_\beta \in \AA$ such that 
        $\beta \in U_\beta$}
        \step{5}{$\{ U_\beta \mid \beta < \alpha \} \cup \{ U \}$ covers
        $\overline{S_\Omega}$}
        \qed
    \end{proof}
\end{example}

\begin{proposition}
    Every closed subspace of a Lindel\"{o}f space is Lindel\"{o}f.
\end{proposition}

\begin{proof}
    \pf
    \step{1}{\pflet{$X$ be a Lindel\"{o}f space.}}
    \step{2}{\pflet{$Y \subseteq X$ be closed.}}
    \step{3}{\pflet{$\AA$ be an open covering of $Y$.}}
    \step{4}{\pflet{$\BB = \{ U \text{ open in } X \mid U \cap Y \in \AA \} \cup
    \{ X - Y \}$}}
    \step{5}{$\BB$ is an open covering of $X$.}
    \step{6}{\pick\ a countable subcovering $\BB_0$}
    \step{7}{$\{ U \cap Y \mid U \in \BB_0 \} - \{ \emptyset \}$ is a countable
    subset of $\AA$ that covers $Y$.}
    \qed
\end{proof}

The following examples show that an open subspace of a Lindel\"{o}f space is not
necessarily Lindel\"{o}f.
\begin{example}
    The space $S_\Omega$ is not Lindel\"{o}f, because the open cover $\{ [0,x) \mid
    x \in S_\Omega \}$ has no countable subcover.
\end{example}

\begin{example}
    The set $[0,1] \times (0,1)$ as a subspace of the ordered square is not Lindel\"{o}f.

    The open cover $\{ \{x\} \times (0,1) \mid x \in [0,1] \}$ has no countable subcover.
\end{example}

\begin{proposition}[Choice]
    The continuous image of a Lindel\"{o}f space is Lindel\"{o}f.
\end{proposition}

\begin{proof}
    \pf
    \step{1}{\pflet{$X$ be a Lindel\"{o}f space.}}
    \step{2}{\pflet{$Y$ be a topological space.}}
    \step{3}{\pflet{$f : X \twoheadrightarrow Y$ be continuous and surjective.}}
    \step{4}{\pflet{$\AA$ be an open cover of $Y$.}}
    \step{5}{$\{ \inv{f}(V) \mid V \in \AA \}$ is an open cover of $X$.}
    \step{6}{\pick\ a countable subcover $\BB$}
    \step{7}{For $U \in \BB$, \pick\ $V_U \in \AA$ such that $U = \inv{f}(V_U)$}
    \step{8}{$\{ V_U \mid U \in \BB \}$ covers $Y$.}
    \qed
\end{proof}

\begin{example}
    The space $S_\Omega \times \overline{S_\Omega}$ is not Lindel\"{o}f, because its image $S_\Omega$
    under $\pi_1$ is not Lindel\"{o}f.
\end{example}

The following is an example of a Lindel\"{o}f space that is not first countable.
\begin{example}[Choice]
    Let $\QQ^\infty$ be the set of all sequences of rationals that end in an
    infinite sequence of 0s. Then $\QQ^\infty$ as a subspace of $\RR^\omega$
    under the box topology is Lindel\"{o}f, because it is countable.
\end{example}

\section{Separable Spaces}

\begin{definition}[Separable]
    A topological space is \emph{separable} if and only if it has a countable dense
    subset.
\end{definition}

\begin{proposition}[AC]
    A countable product of separable spaces is separable.
\end{proposition}

\begin{proof}
    \pf
    \step{1}{\pflet{$(X_n)$ be a sequence of separable spaces.}}
    \step{2}{For each $n$, \pick\ a countable dense set $D_n$ in $X_n$
    \prove{$\prod_n D_n$ is dense in $\prod_n X_n$}}
    \step{3}{\pflet{$\prod_n U_n$ be a nonempty basic open set where each $U_n$ is
    open in $X_n$.}}
    \step{4}{For each $n$, \pick\ $a_n \in D_n \cap U_n$}
    \step{5}{$(a_n) \in \prod_n D_n \cap \prod_n U_n$}
    \qed
\end{proof}

\begin{example}
    The space $\RR_l$ is separable. The set $\QQ$ is dense.
\end{example}

\begin{corollary}
    The Sorgenfrey plane is separable.
\end{corollary}

\begin{proposition}
    An open subspace of a separable space is separable.
\end{proposition}

\begin{proof}
    \pf
    \step{1}{\pflet{$X$ be a separable space.}}
    \step{2}{\pflet{$Y \subseteq X$ be open.}}
    \step{3}{\pick\ a countable set $D$ dense in $X$. \prove{$D \cap Y$ is dense in $Y$.}}
    \step{4}{\pflet{$U$ be a nonempty open set in $Y$}}
    \step{5}{$U$ is open in $X$}
    \step{6}{$U$ intersects $D$}
    \step{7}{$U$ intersects $D \cap Y$}
    \qed
\end{proof}

The following example shows that a closed subspace of a separable space is not
necessarily separable.

\begin{example}
    The space $\RR_l^2$ is separable, but $\{ (x, -x) \mid x \in \RR \}$ as a
    subspace is uncountable and discrete, and hence not separable.
\end{example}

\begin{example}
    The space $S_\Omega$ is not separable. For any countable $D \subseteq S_\Omega$,
    we have $\sup D + 1 \notin \overline{D}$.
\end{example}

\begin{example}    
    The space $\overline{S_\Omega}$ is not separable. For any countable $D
    \subseteq \overline{S_\Omega}$, we have $\sup (D - \{ \Omega \}) + 1 \notin
    \overline{D}$.
\end{example}

\begin{proposition}
    The continuous image of a separable space is separable.
\end{proposition}

\begin{proof}
    \pf
    \step{1}{\pflet{$X$ be a separable space.}}
    \step{2}{\pflet{$Y$ be a topological space.}}
    \step{3}{\pflet{$f : X \twoheadrightarrow Y$ be continuous and separable.}}
    \step{4}{\pick\ a countable dense $D$ in $X$ \prove{$f(D)$ is dense in $Y$.}}
    \step{5}{\pflet{$V$ be open in $Y$ and nonempty.}}
    \step{6}{\pick\ $a \in \inv{f}(V) \cap D$}
    \step{7}{$f(a) \in V \cap f(D)$}
    \qed
\end{proof}

\begin{example}
    The space $S_\Omega \times \overline{S_\Omega}$ is not separable, because its continuous image $S_\Omega$
    under $\pi_1$ is not separable.
\end{example}

\begin{proposition}[Choice]
    In a separable space, every set of disjoint open sets is countable.
\end{proposition}

\begin{proof}
    \pf
    \step{1}{\pflet{$X$ be a separable space.}}
    \step{2}{\pick\ a countable dense set $D$.}
    \step{3}{\pflet{$\AA$ be a set of disjoint open sets.}}
    \step{4}{For $U \in \AA$ nonempty, \pick\ $a_U \in U \cap D$}
    \step{5}{The mapping $\AA \rightarrow D$ that maps $U$ to $a_U$ is injective.}
    \step{6}{$\AA$ is countable.}
    \qed
\end{proof}

\begin{proposition}
    \label{proposition:RRI_separable}
    The product space $\RR^I$ is separable.
\end{proposition}

\begin{proof}
    \pf
    \step{1}{\pflet{$D$ be the set of all step functions with rational values
    and rational endpoints of the intervals.}}
    \step{2}{\pflet{$U = \prod_{x \in I} U_x$ be a basic open set in $\RR^I$,
    where $U_x = \RR$ except for $x = x_1, \ldots, x_n$ with $0 \leq x_1 < \cdots
    < x_n \leq 1$}}
    \step{3}{\assume{w.l.o.g. $x_1 = 0$ and $x_n = 1$}}
    \step{4}{\pick\ rationals $q_1$, \ldots, $q_{n-1}$ such that
    $x_1 < q_1 < x_2 < q_2 < \cdots < x_{n-1} < q_{n-1} < x_n$}
    \step{5}{For $1 \leq i \leq n$, \pick\ a rational $r_i \in U_{x_i}$}
    \step{6}{\pflet{$g$ be the step function with
    \[
        g(x) = \begin{cases}
            r_1 & \text{if } 0 \leq x < q_1 \\
            r_{i+1} & \text{if } q_i \leq x < q_{i+1}, 1 \leq i \leq n-2 \\
            r_n & \text{if } q_{n-1} \leq x \leq 1
                \end{cases}
    \]}}
    \step{7}{$g \in D \cap U$}
    \qed
\end{proof}

\begin{proposition}
    If $J$ is a set and $|J| > 2^{\aleph_0}$ then $\RR^J$ is not separable.
\end{proposition}

\begin{proof}
    \pf
    \step{1}{\pflet{$J$ be a set.}}
    \step{2}{\pflet{$D$ be dense in $\RR^J$.}}
    \step{3}{Define $f : J \rightarrow \pow D$ by: $f(\alpha) = D \cap
    \inv{\pi_\alpha}(I)$ where $I = (0,1)$.}
    \step{4}{$f$ is injective.}
    \begin{proof}
        \step{a}{\pflet{$\alpha, \beta \in J$}}
        \step{b}{\assume{$\alpha \neq \beta$}}
        \step{c}{\pick\ $x \in D \cap \inv{\pi_\alpha}(I) \cap \inv{\pi_\beta}((1,2))$}
        \step{d}{$x \in f(\alpha)$ and $x \notin f(\beta)$}
    \end{proof}
    \step{5}{$|J| \leq 2^|D|$}
    \step{6}{If $|J| > 2^{\aleph_0}$ then $D$ is uncountable.}
    \qed
\end{proof}


The following is an example of a separable space that is not first countable.
\begin{example}[Choice]
    Let $\QQ^\infty$ be the set of all sequences of rationals that end in an
    infinite sequence of 0s. Then $\QQ^\infty$ as a subspace of $\RR^\omega$
    under the box topology is separable, because it is countable.
\end{example}

\begin{example}
    The ordered square is not separable, because any dense subset must include
    an element from $\{x\} \times [0,1]$ for every $x \in [0,1]$.
\end{example}

\section{The Second Countability Axiom}

\begin{definition}[Second Countability Axiom]
    A topological space satisfies the \emph{second countability axiom},
    or is \emph{second countable}, if and only if it has a countable basis.    
\end{definition}

\begin{example}
    The space $\RR$ is second countable.

    The set $\{ (a,b) \mid a,b \in \QQ \}$ is a basis.
\end{example}

\begin{proposition}
    The space $\RR_K$ is second countable.
\end{proposition}

\begin{proof}
    The set $\{ (a,b) \mid a,b \in \QQ \} \cup \{ (a,b) - K \mid a,b \in \QQ \}$
    is a basis. \qed
\end{proof}

\begin{proposition}
    \label{proposition:second_countable_subspace}
    A subspace of a second countable space is second countable.
\end{proposition}

\begin{proof}
    \pf\ If $\BB$ is a countable basis for $X$ and $Y \subseteq X$
    then $\{ B \cap Y \mid B \in \BB \}$ is a countable basis for $Y$. \qed
\end{proof}

\begin{proposition}[CC]
    \label{proposition:lindelof_second_countable}
    Every second countable space is Lindel\"{o}f.
\end{proposition}

\begin{proof}
    \pf\ From Proposition \ref{proposition:lindelof_basis}.
\end{proof}

\begin{corollary}
    The Sorgenfrey plane is not second countable.
\end{corollary}

\begin{example}
    The space $S_\Omega$ is not second countable, because it is not Lindel\"{o}f.
\end{example}

\begin{proposition}
    The long line $L$ is not second countable.
\end{proposition}

\begin{proof}
    \pf\ 
    \step{1}{\pflet{$\BB$ be a basis for $L$.}}
    \step{2}{For $\alpha < \omega_1$, \pick\ $B_\alpha \in \BB$ such that
    $(\alpha, 1/2) \in B_\alpha \subseteq ((\alpha, 0), (\alpha + 1,0))$}
    \step{3}{$\BB$ is uncountable.}
    \begin{proof}
        \pf\ The mapping $\alpha \mapsto B_\alpha$ is an injection $\omega_1 \rightarrow \BB$.
    \end{proof}
\end{proof}

\begin{corollary}
    The long line cannot be imbedded into $\RR^n$ for any $n$.
\end{corollary}

\begin{proposition}
    Every second countable space is first countable.
\end{proposition}

\begin{proof}
    \pf
    \step{1}{\pflet{$X$ be a second countable space.}}
    \step{2}{\pick\ a countable bases $\BB$ for $X$.}
    \step{3}{\pflet{$x \in X$}}
    \step{4}{$\{ B \in \BB \mid x \in B \}$ is a countable local basis at $x$.}
    \qed
\end{proof}

\begin{corollary}
    The space $\RR^\omega$ under the box topology is not second countable.
\end{corollary}

\begin{example}
    The space $S_\Omega \times \overline{S_\Omega}$ is not second countable, because it is not first countable.
\end{example}

\begin{proposition}[AC]
    A countable product of second countable spaces is second countable.
\end{proposition}

\begin{proof}
    \pf
    \step{1}{\pflet{$(X_n)$ be a sequence of second countable spaces.}}
    \step{2}{For each $n$, \pick\ a countable basis $\BB_n$ of $X_n$}
    \step{3}{\pflet{$\BB = \{ \prod_i U_i \mid U_i \in \BB_i \text{ for finitely
    many } i, U_i = X_i \text{ for all other } i \}$}}
    \step{4}{$\BB$ is a countable basis for $\prod_n X_n$}
    \qed
\end{proof}

\begin{corollary}
    The space $\RR^\omega$ is second countable.
\end{corollary}

\begin{corollary}
    The space $\RR^\omega$ is Lindel\"{o}f.
\end{corollary}

\begin{corollary}
    The space $\RR^\omega$ is separable.
\end{corollary}

\begin{proposition}[AC]
    Any discrete subspace of a second countable space is countable.
\end{proposition}

\begin{proof}
    \pf
    \step{1}{\pflet{$X$ be a second countable space.}}
    \step{2}{\pflet{$A \subseteq X$ be discrete.}}
    \step{3}{\pick\ a countable basis $\BB$ for $X$.}
    \step{4}{For all $a \in A$, \pick\ $B_a \in \BB$ such that $B_a \cap A = \{ a \}$}
    \begin{proof}
        \step{a}{\pflet{$a \in A$}}
        \step{b}{\pick\ $U$ open in $X$ such that $U \cap A = \{a\}$}
        \step{c}{\pick\ $B \in \BB$ such that $a \in B \subseteq U$}
    \end{proof}
    \step{5}{The mapping $A \rightarrow \BB$ that maps $a$ to $B_a$ is injective.}
    \step{6}{$A$ is countable.}
    \qed
\end{proof}

\begin{proposition}[AC]
    Every second countable space is separable.
\end{proposition}

\begin{proof}
    \pf
    \step{1}{\pflet{$X$ be a second countable space.}}
    \step{2}{\pick\ a countable basis $\BB$ for $X$.}
    \step{3}{For all $B \in \BB$ nonempty \pick\ $a_B \in B$.}
    \step{4}{\pflet{$A = \{ a_B \mid B \in \BB, B \neq \emptyset \}$} \prove{$A$ is dense}}
    \step{5}{\pflet{$x \in X$} \prove{$x \in \overline{A}$}}
    \step{6}{\pflet{$U$ be a neighbourhood of $x$} \prove{$U$ intersects $A$}}
    \step{7}{\pick\ $B \in \BB$ such that $x \in B \subseteq U$}
    \step{8}{$a_B \in U \cap A$}
    \qed
\end{proof}

\begin{example}[AC]
    The space $\RR_l$ is not second countable.

    \begin{proof}
        \pf
        \step{1}{\pflet{$\BB$ be any basis for $\RR_l$}}
        \step{2}{For $x \in \RR$, \pick\ $B_x \in \BB$ such that $x \in B_x \subseteq [x,x+1)$}
        \step{3}{The mapping $\RR \rightarrow \BB$ that maps $x$ to $B_x$ is injective.}
        \begin{proof}
            \pf\ If $x < y$ then $x \in B_x$ and $x \notin B_y$.
        \end{proof}
        \step{4}{$\BB$ is uncountable.}
        \qed
    \end{proof}
\end{example}

\begin{example}[CC]
    The space $\RR_l$ is Lindel\"{o}f.
    
    \begin{proof}
        \step{1}{\pflet{$\AA$ be a covering of $\RR_l$ by basic open sets of the form $[a,b)$}}
        \step{2}{\pflet{$C = \bigcup \{ (a,b) \mid [a,b) \in \AA \}$}}
        \step{3}{$\RR - C$ is countable.}
        \begin{proof}
            \step{a}{For every $x \in \RR - C$, \pick\ a rational $q_x$ such that $(x,q_x) \subseteq C$}
            \begin{proof}
                \step{i}{\pflet{$x \in \RR - C$}}
                \step{ii}{\pick\ $b$ such that $[x,b) \in \AA$}
                \step{iii}{\pick\ a rational $q$ such that $q \in (x,b)$}
            \end{proof}
            \step{b}{The mapping $x \mapsto q_x$ is an injection $\RR - C \rightarrow \QQ$}
        \end{proof}
        \step{4}{\pick\ a countable $\AA' \subseteq \AA$ that covers $\RR - C$}
        \step{5}{Under the standard topology on $\RR$, $C$ is second countable.}
        \begin{proof}
            \pf\ Proposition \ref{proposition:second_countable_subspace}.
        \end{proof}
        \step{6}{\pick\ a countable $\AA'' \subseteq \AA$ such that $\{ (a,b) \mid [a,b) \in \AA'' \}$ covers $C$.}
        \begin{proof}
            \pf\ Proposition \ref{proposition:lindelof_basis}.
        \end{proof}
        \step{7}{$\AA' \cup \AA''$ covers $\RR_l$.}
        \qed
    \end{proof}
\end{example}

The following example shows that the product of two Lindel\"{o}f spaces is not
necessarily Lindel\"{o}f.
\begin{example}
    The Sorgenfrey plane is not Lindel\"{o}f.
    
    \begin{proof}
        \pf
        \step{1}{\pflet{$L = \{ (x, -x) \mid x \in \RR \}$}}
        \step{2}{$L$ is closed in $\RR_l^2$}
        \step{3}{\pflet{$\UU = \{ [a,b) \times [a,-d) \mid a,b,d \in \RR \}$}}
        \step{4}{$\UU \cup \{ \RR - L \}$ covers $\RR_l^2$}
        \step{5}{Every element of $\UU$ intersects $L$ at exactly one point.}
        \step{6}{No countable subset of $\UU$ covers $\RR_l^2$.}
        \qed
    \end{proof}
\end{example}

\begin{proposition}[AC]
    A topological space is second countable if and only if every basis includes a
    countable basis.
\end{proposition}

\begin{proof}
    \pf
    \step{1}{\pflet{$X$ be a topological space.}}
    \step{2}{If $X$ is second countable then every basis includes a countable basis.}
    \begin{proof}
        \step{a}{\assume{$X$ is second countable.}}
        \step{b}{\pflet{$\BB$ be a basis.}}
        \step{c}{\pick\ a countable basis $\CC$.}
        \step{d}{For every pair of basis elements $C, C' \in \CC$ such that there
        exists $B \in \BB$ with $C \subseteq B \subseteq C'$, \pick\ $B_{CC'} \in \BB$
        such that $C \subseteq B_{CC'} \subseteq C'$ \prove{The set of all $B_{CC'}$
        form a basis for $X$.}}
        \step{e}{\pflet{$x \in X$}}
        \step{f}{\pflet{$U$ be a neighbourhood of $x$.}}
        \step{g}{\pick\ $C' \in \CC$ such that $x \in C' \subseteq U$}
        \step{h}{\pick\ $B \in \BB$ such that $x \in B \subseteq C'$}
        \step{i}{\pick\ $C \in \CC$ such that $x \in C \subseteq B$}
        \step{j}{$x \in B_{CC'} \subseteq U$}
    \end{proof}
    \step{3}{If every basis includes a countable basis then $X$ is second countable.}
    \begin{proof}
        \pf\ The set of all open sets is a basis and so includes a countable basis.
    \end{proof}
    \qed
\end{proof}

\begin{proposition}[AC]
    Let $X$ be a second countable space. Let $A \subseteq X$ be uncountable.
    Then $A$ contains uncountably many of its own limit points.
\end{proposition}

\begin{proof}
    \pf
    \step{1}{\pflet{$X$ be a second countable space.}}
    \step{1a}{\pick\ a countable basis $\BB$ for $X$.}
    \step{2}{\pflet{$A \subseteq X$}}
    \step{3}{\assume{only countably many points of $A$
    are limit points of $A$.}}
    \step{4}{For every point $x$ of $A$ that is not a limit point of $A$,
    \pick\ $B_x \in \BB$ such that $B_x \cap A = \{x\}$.}
    \step{5}{The mapping $A - A' \rightarrow \BB$ that maps $x$ to $B_x$ is injective.}
    \step{6}{$A$ is countable.}
    \qed
\end{proof}

\begin{example}
    Thes space $\overline{S_\Omega}$ is not second countable because it is neither
    first countable nor separable.
\end{example}

\begin{proposition}
    The image of a first countable space under a continuous open map is first countable.
\end{proposition}

\begin{proof}
    \pf
    \step{1}{\pflet{$X$ be a first countable space.}}
    \step{2}{\pflet{$Y$ be a topological space.}}
    \step{3}{\pflet{$f : X \twoheadrightarrow Y$ be a surjective, continuous open map.}}
    \step{4}{\pflet{$y \in Y$}}
    \step{5}{\pick\ $x \in X$ such that $f(x) = y$}
    \step{6}{\pick\ a countable local basis $\BB$ at $x$. \prove{$\{ f(B) \mid B \in \BB \}$
    is a local basis at $y$.}}
    \step{7}{\pflet{$V$ be a neighbourhood of $y$}}
    \step{8}{\pick\ $B \in \BB$ such that $x \in B \subseteq \inv{f}(V)$}
    \step{9}{$y \in f(B) \subseteq V$}
    \qed
\end{proof}

\begin{proposition}
    The image of a second countable space under a continuous open map is second countable.
\end{proposition}

\begin{proof}
    \pf
    \step{1}{\pflet{$X$ be a second countable space.}}
    \step{2}{\pflet{$Y$ be a topological space.}}
    \step{3}{\pflet{$f : X \twoheadrightarrow Y$ be a surjective, continuous open map.}}
    \step{6}{\pick\ a countable basis $\BB$. \prove{$\{ f(B) \mid B \in \BB \}$
    is a basis.}}
    \step{7}{\pflet{$V$ be a neighbourhood of $y$}}
    \step{7a}{\pick\ $x \in X$ such that $f(x) = y$}
    \step{8}{\pick\ $B \in \BB$ such that $x \in B \subseteq \inv{f}(V)$}
    \step{9}{$y \in f(B) \subseteq V$}
    \qed
\end{proof}

\begin{example}
    The ordered square is not second countable, because it is not separable.
\end{example}

\section{Sequential Compactness}

\begin{definition}[Sequentially Compact]
    A topological space is \emph{sequentially compact} if and only if
    every sequence has a convergent subsequence.
\end{definition}

\section{Limit Point Compactness}

\begin{definition}[Limit Point Compact Space]
    A topological space is \emph{limit point compact} if and only if every
    infinite set has a limit point.
\end{definition}

\begin{proposition}
    \label{proposition:limit_point_compact_sequentially_compact}
    Every limit point compact $T_1$ space is sequentially compact.
\end{proposition}

\begin{proof}
    \pf
    \step{1}{\pflet{$X$ be a limit point compact $T_1$ space.}}
    \step{2}{\pflet{$(x_n)$ be a sequence in $X$.}}
    \step{3}{\case{$\{ x_n \mid n \geq 1 \}$ is finite.}}
    \begin{proof}
        \step{a}{\pick\ $n$ such that $x_n$ occurs infintely often in the sequence $(x_n)$}
        \step{b}{The subsequence consisting of all the terms equal to $x_n$ is convergent.}
    \end{proof}
    \step{3}{\case{$\{ x_n \mid n \geq 1 \}$ is infinite.}}
    \begin{proof}
        \step{a}{\pick\ a limit point $l$ for $\{ x_n \mid n \geq 1 \}$}
        \step{b}{\pick\ an increasing sequence $n_r$ with $x_{n_r} \in B(x, 1/r)$ for all $r$}
        \begin{proof}
            \pf\ This is always possible by Theorem \ref{theorem:limit_point_T1}.
        \end{proof}
        \step{c}{$(x_{n_r})$ converges to $l$.}
    \end{proof}
    \qed
\end{proof}

\begin{example}
    The space $[0,1]$ under the lower limit topology is not limit point compact.

    The infinite set $A = \{ 1 - 1/n \mid n \geq 1 \}$ has no limit point. 1 is not a limit point
    because the neighbourhood $\{1\}$ does not intersect $A$.
\end{example}

\begin{proposition}
    A closed subspace of a limit point compact space is limit point compact.
\end{proposition}

\begin{proof}
    \pf
    \step{1}{\pflet{$X$ be a limit point compact space.}}
    \step{2}{\pflet{$A \subseteq X$ be closed.}}
    \step{3}{\pflet{$B \subseteq A$ be infinite.}}
    \step{4}{\pick\ a limit point $l$ of $B$ in $X$.}
    \step{5}{$l \in A$}
    \step{6}{$l$ is a limit point of $B$ in $A$.}
    \qed
\end{proof}

\begin{example}
    An open subspace of a limit point compact space is not necessarily limit point compact.

    The space $[0,1]$ is limit point compact but $(0,1)$ is not.
\end{example}

\begin{example}
    The continuous image of a limit point compact space is not necessarily limit point compact.

    Let $Y$ be the indiscrete space with two points. Then $\ZZ^+ \times Y$ is limit point compact but $\ZZ^+$
    is not.
\end{example}

\begin{example}
    A limit point compact subspace of a Hausdorff space is not necessarily closed.

    The space $S_\Omega$ is limit point compact but is not closed in $\overline{S_\Omega}$.
\end{example}

For an example that shows that the product of two limit point compact spaces is not necessarily
limit point compact, see L. A. Steen and J. A. Seerbach Jr. \emph{Counterexamples in
Topology} Example 112.

\begin{example}
    The space $\RR_l$ is not limit point compact. The set $\ZZ$ has no limit point.
\end{example}

\begin{proposition}
    The Sorgenfrey plane is not limit point compact.
\end{proposition}

\begin{proof}
    \pf\ The set $\ZZ^2$ has no limit point. \qed
\end{proof}

\begin{proposition}
    The space $\RR^\omega$ under the box topology is not limit point compact.
    The set $\ZZ^\omega$ has no limit point.
\end{proposition}

\begin{proposition}
    The space $\RR^I$ is not limit point compact. The set $\ZZ^I$ has no limit point.
\end{proposition}

\begin{proposition}
    The space $\RR_K$ is not limit point compact.
\end{proposition}

\begin{proof}
    \pf\ The set $\ZZ$ has no limit point. \qed
\end{proof}

\section{Countable Compactness}

\begin{definition}[Countably Compact]
    A topological space is \emph{countably compact} if and only if every
    countable open covering has a finite subcovering.
\end{definition}

\begin{proposition}[AC]
    Every closed subspace of a countably compact space is countably compact.
\end{proposition}

\begin{proof}
    \pf
    \step{1}{\pflet{$X$ be a countably compact space.}}
    \step{2}{\pflet{$A \subseteq X$ be closed.}}
    \step{3}{\pflet{$\UU$ be a countable open cover of $A$.}}
    \step{4}{For $U \in \UU$, \pick\ an open set $V_U$ is $X$ such that
    $U = V_U \cap A$}
    \step{5}{$\{ V_U \mid U \in \UU \} \cup \{ X - A \}$ is a countable open cover of $X$}
    \step{6}{\pick\ a finite subcover $\{ V_{U_1}, \ldots, V_{U_n}, X - A \}$}
    \step{7}{$\{ U_1, \ldots, U_n \}$ covers $A$.}
    \qed
\end{proof}

\begin{proposition}[AC]
    Every countably compact space is limit point compact.
\end{proposition}

\begin{proof}
    \pf
    \step{a}{\assume{$X$ is countably compact.}}
    \step{b}{\pflet{$A \subseteq X$ be infinite.}}
    \step{d}{\assume{for a contradiction $A$ has no limit point.}}
    \step{c}{\pick\ a countably infinite $B \subseteq A$}
    \step{e}{$B$ is discrete.}
    \begin{proof}
        \pf\ For all $b \in B$, there exists $U_b$ open in $X$ such that $U_b \cap B = \{ b \}$.
    \end{proof}
    \step{f}{$\{ \{ b \} \mid b \in B \}$ is a countable cover of $B$ that has no finite subcover.}
    \step{g}{$B$ is not countably compact.}
    \step{h}{$B$ is not closed in $X$}
    \step{i}{$B$ has a limit point.}
    \step{j}{$A$ has a limit point.}
    \qedstep
    \begin{proof}
        \pf\ This contradicts \stepref{d}.
    \end{proof}
    \qed
\end{proof}

\begin{proposition}[AC]
    Every limit point compact $T_1$ space is countably compact.
\end{proposition}

\begin{proof}
    \pf
    \step{1}{\pflet{$X$ be a limit point compact $T_1$ space.}}
    \step{2}{\pflet{$\{ U_n \mid n \in \ZZ^+ \}$ be a countable open cover of $X$.}}
    \step{3}{For $n \in \ZZ^+$, \pflet{$V_n = U_1 \cup \cdots \cup V_n$}}
    \step{4}{\assume{for a contradiction none of the $V_n$ covers $X$}}
    \step{5}{For $n \in \ZZ^+$, \pick\ $a_n \in X - V_n$}
    \step{6}{\pick\ a limit point $l$ for $\{ a_n \mid n \in \ZZ^+ \}$}
    \step{7}{\pick\ $n$ such that $l \in U_n$}
    \step{8}{\case{$l = a_m$ for some $m \leq n$}}
    \begin{proof}
        \pf\ $U_n - \{ a_1, \ldots, a_{m-1}, a_{m+1}, \ldots, a_n \}$
        is a neighbourhood of $l$ that intersects $\{ a_n \mid n \in \ZZ^+ \}$
        only at $l$, contradicting \stepref{6}.
    \end{proof}
    \step{9}{\case{$l \neq a_m$ for any $m \leq n$}}
    \begin{proof}
        \pf\ $U_n - \{ a_1, \ldots, a_n \}$ is a neighbourhood of $l$ that does not intersect $\{ a_n \mid n \in \ZZ^+ \}$,
        which contradicts \stepref{6}.
    \end{proof}
    \qed
\end{proof}

The following example shows we cannot remove the hypothesis that the space is $T_1$.

\begin{example}
    Let $Y$ be the indiscrete space with two points. Then $\ZZ^+ \times Y$
    is a limit point compact space that is not countably compact,
    since $\{ \{n\} \times Y \mid n \in \ZZ^+ \}$ is a countable open
    cover that has no finite subcover.
\end{example}

\begin{proposition}
    \label{proposition:nested_sequence_nonempty_intersection}
    A topological space is countably compact if and only if every nested sequence
    $C_1 \supseteq C_2 \supseteq \cdots$ of nonempty closed sets has nonempty intersection.
\end{proposition}

\begin{proof}
    \pf
    \step{1}{\pflet{$X$ be a topological space.}}
    \step{2}{If $X$ is countably compact then every nested sequence of nonempty closed sets
    has nonempty intersection.}
    \begin{proof}
        \step{a}{\assume{$X$ is countably compact.}}
        \step{b}{\pflet{$C_1 \supseteq C_2 \supseteq \cdots$ be a nested sequence of nonempty closed sets.}}
        \step{c}{\assume{for a contradiction $\bigcap_n C_n = \emptyset$}}
        \step{d}{$\{ X - C_n \mid n \in \ZZ^+ \}$ covers $X$}
        \step{e}{\pick\ a finite subcover $\{ X - C_{n_1}, \ldots, X - C_{n_k} \}$
        where $n_1 < \cdots < n_k$}
        \step{f}{$C_{n_k} = \emptyset$}
        \qedstep
        \begin{proof}
            \pf\ This contradicts \stepref{b}.
        \end{proof}
        \qed
    \end{proof}
    \step{3}{If every nested sequence of nonempty closed sets has nonempty intersection
    then $X$ is countably compact.}
    \begin{proof}
        \step{a}{\assume{Every nested sequence of nonempty closed sets has nonempty intersection.}}
        \step{b}{\pflet{$\{ U_n \mid n \geq 1 \}$ is a countable open cover of $X$.}}
        \step{c}{$X - U_1 \supseteq X - (U_1 \cup U_2) \supseteq \cdots$ is a nested sequence of closed sets with empty intersection.}
        \step{d}{\pick\ $k$ such that $X - (U_1 \cup \cdots \cup U_k) = \emptyset$}
        \step{e}{$\{ U_1, \ldots, U_k \}$ covers $X$.}
    \end{proof}
    \qed
\end{proof}

\section{Subnets}

\begin{definition}[Subnet]
    Let $X$ be a topological space. Let $(a_\alpha)_{\alpha \in J}$ be a net
    in $X$. A \emph{subnet} of $(a_\alpha)_{\alpha \in J}$ is a net of the form
    $(a_{g(\beta)})_{\beta \in K}$ where $K$ is a directed set, $g : K \rightarrow J$
    is monotone, and $g(K)$ is cofinal in $J$.
\end{definition}

\begin{proposition}
    Let $X$ be a topological space. Let $(a_\alpha)_{\alpha \in J}$ be a net
    in $X$. Let $l \in X$. If $(a_\alpha)$ converges to $l$ then any subnet
    of $(a_\alpha)$ converges to $l$.
\end{proposition}

\begin{proof}
    \pf
    \step{1}{\pflet{$X$ be a topological space.}}
    \step{2}{\pflet{$(a_\alpha)_{\alpha \in J}$ be a net in $X$.}}
    \step{3}{\pflet{$l \in X$}}
    \step{4}{\assume{$a_\alpha \rightarrow l$ as $\alpha \rightarrow \infty$}}
    \step{5}{\pflet{$(a_{g(\beta)})_{\beta \in K}$ be a subnet of $(a_\alpha)_{\alpha \in J}$}}
    \step{6}{\pflet{$U$ be a neighbourhood of $l$.}}
    \step{7}{\pick\ $\alpha \in J$ be such that, for all $\alpha' \geq \alpha$,
    we have $a_{\alpha'} \in U$}
    \step{8}{\pick\ $\beta \in K$ such that $g(\beta) \geq \alpha$.}
    \step{9}{For all $\beta' \geq \beta$ we have $a_{g(\beta')} \in U$.}
    \qed
\end{proof}

\section{Accumulation Points}

\begin{definition}[Accumulation Point]
    Let $X$ be a topological space. Let $(a_\alpha)_{\alpha \in J}$ be a net
    in $X$. Let $l \in X$. Then $l$ is an \emph{accumulation point} of
    $(a_\alpha)_{\alpha \in J}$ if and only if, for every neighbourhood $U$
    of $l$, the set $\{ \alpha \in J \mid a_\alpha \in U \}$ is cofinal in $J$.
\end{definition}

\begin{lemma}
    \label{lemma:accumulation_point_subnet}
    Let $X$ be a topological space. Let $(a_\alpha)_{\alpha \in J}$ be a net
    in $X$. Let $l \in X$. Then $l$ is an accumulation point of
    $(a_\alpha)_{\alpha \in J}$ if and only if there exists a subnet of
    $(a_\alpha)_{\alpha \in J}$ that converges to $l$.
\end{lemma}

\begin{proof}
    \pf
    \step{1}{\pflet{$X$ be a topological space.}}
    \step{2}{\pflet{$(a_\alpha)_{\alpha \in J}$ be a net in $X$.}}
    \step{3}{\pflet{$l \in X$}}
    \step{4}{If $l$ is an accumulation point of $(a_\alpha)_{\alpha \in J}$
    then there exists a subnet of $(a_\alpha)_{\alpha \in J}$ that converges
    to $l$.}
    \begin{proof}
        \step{a}{\assume{$l$ is an accumulation point of $(a_\alpha)_{\alpha \in J}$.}}
        \step{b}{\pflet{$K = \{ (\alpha, U) \mid \alpha \in J,
        U \text{ is a neighbourhood of } l, a_\alpha \in U \}$
        with $(\alpha, U) \leq (\beta, V)$ if and only if $\alpha \leq \beta$
        and $V \subseteq U$}}
        \step{c}{$K$ is a directed set}
        \begin{proof}
            \step{i}{$\leq$ is reflexive on $K$.}
            \step{ii}{$\leq$ is transitive on $K$.}
            \step{iii}{$\leq$ is antisymmetric on $K$.}
            \step{iv}{For all $(\alpha, U), (\beta, V) \in K$, there exists
            $(\gamma, W)$ such that $(\alpha, U) \leq (\gamma, W)$
            and $(\beta, V) \leq (\gamma, W)$}
            \begin{proof}
                \step{one}{\pflet{$(\alpha, U), (\beta, V) \in K$}}
                \step{two}{\pick\ $\gamma \in J$ with $\alpha \leq \gamma$
                and $\beta \leq \gamma$}
                \step{three}{\pick\ $\delta \in J$ with $\gamma \leq \delta$
                and $a_\delta \in U \cap V$}
                \step{four}{$(\alpha, U) \leq (\delta, U \cap V)$
                and $(\beta, V) \leq (\delta, U \cap V)$}
            \end{proof}
        \end{proof}
        \step{d}{\pflet{$g : K \rightarrow J$, $g(\alpha, U) = \alpha$}}
        \step{e}{$g$ is monotone}
        \step{f}{$g(K)$ is cofinal in $J$}
        \begin{proof}
            \pf\ For all $\alpha \in J$ we have $\alpha = g(\alpha, X)$.
        \end{proof}
        \step{g}{$(a_{g(\alpha, U)})_{(\alpha, U) \in K}$ converges to $l$.}
        \begin{proof}
            \step{i}{\pflet{$U$ be a neighbourhood of $l$}}
            \step{ii}{\pick\ $\alpha \in J$ such that $a_\alpha \in U$}
            \step{iii}{For all $(\beta, V) \geq (\alpha, U)$ we have $a_\beta \in U$}
            \begin{proof}
                \pf\ Since $a_\beta \in V \subseteq U$
            \end{proof}
        \end{proof}
    \end{proof}
    \step{5}{If there exists a subnet of $(a_\alpha)_{\alpha \in J}$ that
    converges to $l$ then $l$ is an accumulation point of $(a_\alpha)_{\alpha \in J}$.}
    \begin{proof}
        \step{a}{\assume{$(a_{g(\beta)})_{\beta \in K}$ converges to $l$}}
        \step{b}{\pflet{$U$ be a neighbourhood of $l$}}
        \step{c}{\pflet{$\alpha \in J$}}
        \step{d}{\pick\ $\beta \in K$ such that, for all $\beta' \geq \beta$,
        we have $a_{g(\beta')} \in U$}
        \step{e}{\pick\ $\gamma \in K$ such that $g(\gamma) \geq \alpha$}
        \step{f}{\pick\ $\delta \in K$ with $\beta \leq \delta$ and $\gamma \leq \delta$}
        \step{g}{$\alpha \leq g(\delta)$}
        \step{h}{$a_{g(\delta)} \in U$}
    \end{proof}
    \qed
\end{proof}

\section{Compact Spaces}

\begin{definition}[Compact]
    A topological space is \emph{compact} if and only if every open covering has a finite subcovering.
\end{definition}

\begin{example}
    The space $\RR_l$ is not compact. The open covering $\{ [n,n+1) \mid n \in \ZZ \}$ has no finite subcovering.
\end{example}

\begin{proposition}
    The space $\RR$ is not compact.
\end{proposition}

\begin{proof}
    \pf\ The open covering $\{ (n,n+2) \mid n \in \ZZ \}$ has no finite subcovering. \qed
\end{proof}

\begin{corollary}
    The space $\RR_K$ is not compact.
\end{corollary}

\begin{lemma}
    \label{lemma:finite_subcovering_subspace}
    Let $X$ be a topological space and $Y$ a subspace of $X$. Then $Y$ is compact if and only if
    every covering of $Y$ by sets open in $X$ has a finite subcovering.
\end{lemma}

\begin{proof}
    \pf
    \step{1}{If $Y$ is compact then every covering of $Y$ by sets open in $X$ has a finite subcovering.}
    \begin{proof}
        \step{a}{\assume{$Y$ is compact.}}
        \step{b}{\pflet{$\UU$ be a covering of $Y$ by sets open in $X$.}}
        \step{c}{$\{ U \cap Y \mid U \in \UU \}$ is an open covering of $Y$.}
        \step{d}{\pick\ a finite subcovering $\{ U_1 \cap Y, \ldots, U_n \cap Y \}$}
        \step{e}{$\{ U_1, \ldots, U_n \}$ is a finite subcovering of $\UU$.}
    \end{proof}
    \step{2}{If every covering of $Y$ by sets open in $X$ has a finite subcovering then $Y$ is compact.}
    \begin{proof}
        \step{a}{\pflet{$\UU$ be an open covering of $Y$.}}
        \step{b}{\pflet{$\VV = \{ V \text{ open in } X \mid V \cap Y \in \UU \}$.}}
        \step{c}{$\VV$ is a covering of $Y$ by sets open in $X$.}
        \step{d}{\pick\ a finite subcovering $\{ V_1, \ldots, V_n \}$}
        \step{e}{$\{ V_1 \cap Y, \ldots, V_n \cap Y \}$ is a finite subcovering of $\UU$.}
    \end{proof}
    \qed
\end{proof}

\begin{proposition}
    \label{proposition:closed_subspace_compact}
    Every closed subspace of a compact space is compact.
\end{proposition}

\begin{proof}
    \pf
    \step{1}{\pflet{$X$ be a compact space and $Y \subseteq X$ be closed.}}
    \step{2}{\pflet{$\UU$ be a covering of $Y$ by sets open in $X$.}}
    \step{3}{$\UU \cup \{ X - Y \}$ is an open covering of $X$.}
    \step{4}{\pick\ a finite subcovering $\UU_0$}
    \step{5}{$\UU_0 \cap \UU$ is a finite subset of $\UU$ that covers $Y$.}
    \qed
\end{proof}

\begin{corollary}
    The space $\RR^\omega$ is not compact, because it has a
    closed subspace homeomorphic to $\RR$.
\end{corollary}

\begin{corollary}
    The space $\RR^\omega$ under the box topology is not compact.
\end{corollary}

\begin{theorem}
    \label{theorem:continuous_image_compact}
    The continuous image of a compact space is compact.
\end{theorem}

\begin{proof}
    \pf
    \step{1}{\pflet{$f : X \twoheadrightarrow Y$ be continuous and surjective.}}
    \step{2}{\pflet{$\VV$ be an open covering of $Y$}}
    \step{3}{$\{ \inv{p}(V) \mid V \in \VV \}$ is an open covering of $X$.}
    \step{4}{\pick\ a finite subcovering $\{ \inv{p}(V_1), \ldots, \inv{p}(V_n) \}$}
    \step{5}{$\{ V_1, \ldots, V_n \}$ covers $Y$.}
    \qed
\end{proof}

\begin{theorem}
    \label{theorem:generalized_tube_lemma}
    Let $A$ and $B$ be compact subspaces of $X$ and $Y$ respectively. Let $N$ be an open set in $X \times Y$ that includes $A \times B$.
    Then there exist open sets $U$ and $V$ in $X$ and $Y$ respectively such that $A \subseteq U$, $B \subseteq V$ and $U \times V \subseteq N$.
\end{theorem}

\begin{proof}
    \pf
    \step{1}{For all $x \in A$, there exist neighbourhoods $U$ of $x$ and $V$ of $B$ such that $U \times V \subseteq N$.}
    \begin{proof}
        \step{a}{\pflet{$x \in A$}}
        \step{b}{For all $y \in B$, there exist neighbourhoods $U$ of $x$ and $V$ of $y$ such that $U \times V \subseteq N$}
        \step{c}{$\{ V \text{ open in } Y \mid \exists \text{ neighbourhood } U \text{ of } x, U \times V \subseteq N \}$ covers $B$.}
        \step{d}{\pick\ a finite subcover $\{ V_1, \ldots, V_n \}$}
        \step{e}{For $i = 1, \ldots, n$, \pick\ a neighbourhood $U_i$ of $x$ such that $U_i \times V_i \subseteq N$}
        \step{f}{\pflet{$U = U_1 \cap \cdots \cap U_n$}}
        \step{g}{\pflet{$V = V_1 \cup \cdots \cup V_n$}}
        \step{h}{$U$ is a neighbourhood of $x$.}
        \step{i}{$V$ is a neighbourhood of $B$.}
        \step{j}{$U \times V \subseteq N$}
    \end{proof}
    \step{2}{$\{ U \text{ open in } X \mid \exists \text{ neighbourhood } V \text{ of } B. U \times V \subseteq N \}$ covers $A$.}
    \step{3}{\pick\ a finite subcover $\{ U_1, \ldots, U_n \}$}
    \step{4}{For $i = 1, \ldots, n$, \pick\ a neighbourhood $V_i$ of $B$ such that $U_i \times V_i \subseteq N$}
    \step{5}{\pflet{$U = U_1 \cup \cdots \cup U_n$}}
    \step{6}{\pflet{$V = V_1 \cap \cdots \cap V_n$}}
    \step{7}{$U$ and $V$ are open.}
    \step{8}{$A \subseteq U$}
    \step{9}{$B \subseteq V$}
    \step{10}{$U \times V \subseteq N$}
    \qed
\end{proof}

\begin{corollary}[Tube Lemma]
    Let $X$ and $Y$ be topological spaces with $Y$ compact. Let $a \in X$
    and $N$ be an open set in $X \times Y$ that includes $\{a\} \times Y$.
    Then there exists a neighbourhood $W$ of $a$ such that
    $N$ includes the \emph{tube} $W \times Y$.
\end{corollary}

\begin{theorem}
    \label{theorem:compact_closed_sets}
    Let $X$ be a topological space. Then $X$ is compact if and only if every set of closed sets that has the finite intersection property has nonempty intersection.
\end{theorem}

\begin{proof}
    \pf\ The following are equivalent.
    \begin{enumerate}
        \item $X$ is compact.
        \item Every open covering of $X$ has a finite subcovering.
        \item For any set $\CC$ of closed sets, if $\{ X - C \mid C \in \CC \}$ covers $X$ then there is a finite subset $\CC_0$ such that
        $\{ X - C \mid C \in \CC_0 \}$ covers $X$
        \item For any set $\CC$ of closed sets, if $\bigcap \CC = \emptyset$ then there is a finite subset $\CC_0$ with empty intersection.
        \item Any set of closed sets with the finite intersection property has nonempty intersection.
    \end{enumerate}
    \qed
\end{proof}

\begin{corollary}
    \label{corollary:nested_intersection}
    Let $X$ be a topological space and $C_1 \supseteq C_2 \supseteq \cdots$ a nested sequence of nonempty closed sets. Then $\bigcap_n C_n$ is nonempty.
\end{corollary}

\begin{proposition}
    Let $\TT$ and $\TT'$ be two topologies on the same set $X$ with $\TT \subseteq \TT'$. If $\TT'$ is compact then $\TT$ is compact.
\end{proposition}

\begin{proof}
    \pf
    \step{1}{\pflet{$\UU \subseteq \TT$ cover $X$}}
    \step{2}{$\UU \subseteq \TT'$}
    \step{3}{A finite subset of $\UU$ covers $X$.}
    \qed
\end{proof}

\begin{corollary}
    \label{corollary:compact_hausdorff_incomparable}
    If $\TT$ and $\TT'$ are two compact Hausdorff topologies on the same set $X$, then either $\TT = \TT'$ or $\TT$ and $\TT'$ are incomparable.
\end{corollary}

\begin{proof}
    \pf\ From the Proposition and Proposition \ref{proposition:Hausdorff_finer}. \qed
\end{proof}

\begin{example}
    Any set under the finite complement topology is compact.
\end{example}

\begin{proposition}
    \label{proposition:finite_union_compact}
    Let $X$ be a topological space. A finite union of compact subspaces of $X$ is compact.
\end{proposition}

\begin{proof}
    \pf
    \step{1}{\pflet{$A$ and $B$ be compact subspaces of $X$.}}
    \step{2}{\pflet{$\UU$ be a set of open sets in $X$ that covers $A \cup B$}}
    \step{3}{\pick\ a finite subset $\UU_1$ that covers $A$.}
    \begin{proof}
        \pf\ Lemma \ref{lemma:finite_subcovering_subspace}.
    \end{proof}
    \step{4}{\pick\ a finite subset $\UU_2$ that covers $B$.}
    \begin{proof}
        \pf\ Lemma \ref{lemma:finite_subcovering_subspace}.
    \end{proof}
    \step{5}{$\UU_1 \cup \UU_2$ is a finite subset that covers $A \cup B$.}
    \qedstep
    \begin{proof}
        \pf\ Lemma \ref{lemma:finite_subcovering_subspace}.
    \end{proof}
    \qed
\end{proof}

\begin{proposition}
    \label{proposition:disjoint_neighbourhoods_compact_Hausdorff}
    Let $A$ and $B$ be disjoint compact subspaces of the Hausdorff space $X$.
    Then there exist disjoint open sets $U$ and $V$ that include $A$ and $B$
    respectively.
\end{proposition}

\begin{proof}
    \pf\ From Theorem \ref{theorem:generalized_tube_lemma} with $N = X^2 - \{ (x,x) \mid x \in X \}$. \qed
\end{proof}

\begin{corollary}
    \label{corollary:closed_compact}
    Every compact subspace of a Hausdorff space is closed.
\end{corollary}

\begin{theorem}
    Let $f : X \rightarrow Y$ be a bijective continuous function. If $X$ is
    compact and $Y$ is Hausdorff then $f$ is a homeomorphism.
\end{theorem}

\begin{proof}
    \pf
    \step{1}{\pflet{$C \subseteq X$ be closed.}}
    \step{2}{$C$ is compact.}
    \begin{proof}
        \pf\ Proposition \ref{proposition:closed_subspace_compact}.
    \end{proof}
    \step{3}{$f(C)$ is compact.}
    \begin{proof}
        \pf\ Theorem \ref{theorem:continuous_image_compact}.
    \end{proof}
    \step{4}{$f(C)$ is closed.}
    \begin{proof}
        \pf\ Corollary \ref{corollary:closed_compact}.
    \end{proof}
    \qedstep
    \begin{proof}
        \pf\ Lemma \ref{lemma:homeomorphism}.
    \end{proof}
    \qed
\end{proof}

\begin{proposition}
    \label{proposition:closed_map_compact_Hausdorff}
    Let $X$ be a compact space, $Y$ a Hausdorff space, and $f : X \rightarrow Y$
    a continuous map. Then $f$ is a closed map.
\end{proposition}

\begin{proof}
    \pf
    \step{1}{\pflet{$C \subseteq X$ be closed.}}
    \step{2}{$C$ is compact.}
    \begin{proof}
        \pf\ Proposition \ref{proposition:closed_subspace_compact}.
    \end{proof}
    \step{3}{$f(C)$ is compact.}
    \begin{proof}
        \pf\ Theorem \ref{theorem:continuous_image_compact}.
    \end{proof}
    \step{4}{$f(C)$ is closed.}
    \begin{proof}
        \pf\ Corollary \ref{corollary:closed_compact}.
    \end{proof}
    \qed
\end{proof}

\begin{proposition}
    \label{proposition:closed_map_compact_projection}
    If $Y$ is compact then the projection $\pi_1 : X \times Y \rightarrow X$
    is a closed map.
\end{proposition}

\begin{proof}
    \pf
    \step{1}{\pflet{$A \subseteq X \times Y$ be closed.}}
    \step{2}{\pflet{$x \in X - \pi_1(A)$}}
    \step{3}{\pick\ a neighbourhood $U$ of $x$ such that $U \times Y \subseteq (X \times Y) - A$}
    \begin{proof}
        \pf\ By the Tube Lemma.
    \end{proof}
    \step{8}{$x \in U \subseteq X - \pi_1(A)$}
    \qedstep
    \begin{proof}
        \pf\ So $X - \pi_1(A)$ is open by Lemma \ref{lemma:open}.
    \end{proof}
    \qed
\end{proof}

\begin{proposition}
    \label{proposition:graph_closed}
    Let $X$ be a topological space and $Y$ a Hausdorff space.
    Let $f : X \rightarrow Y$ be continuous. Then the graph of $f$
    is closed in $X \times Y$.
\end{proposition}

\begin{proof}
    \step{a}{\assume{$f$ is continuous.}}
    \step{b}{\pflet{$(x,y) \in (X \times Y) - G_f$}}
    \step{c}{\pick\ disjoint neighbourhoods $U$ and $V$ of $y$ and $f(x)$ respectively.}
    \step{d}{$\inv{f}(V) \times U$ is a neighbourhood of $(x,y)$ disjoint from $G_f$.}
    \qed
\end{proof}

\begin{theorem}
    Let $X$ be a topological space and $Y$ a compact space. Let $f : X \rightarrow Y$ be a function. 
    If the graph of $f$ is closed in $X \times Y$ then $f$ is continuous.
\end{theorem}

\begin{proof}
    \pf
        \step{a}{\assume{$G_f$ is closed.}}
        \step{b}{\pflet{$x \in X$ and $V$ be a neighbourhood of $f(x)$.}}
        \step{c}{$G_f \cap (X \times (Y - V))$ is closed.}
        \step{d}{$\pi_1(G_f \cap (X \times (Y - V)))$ is closed.}
        \begin{proof}
            \pf\ Proposition \ref{proposition:closed_map_compact_projection}.
        \end{proof}
        \step{e}{\pflet{$U = X - \pi_1(G_f \cap (X \times (Y - V)))$}}
        \step{f}{$U$ is a neighbourhood of $x$}
        \step{g}{$f(U) \subseteq V$}
    \qed
\end{proof}

\begin{theorem}
    Let $X$ be a compact topological space. Let $(f_n : X \rightarrow \RR)$ be a monotone increasing sequence of continuous functions and $f : X \rightarrow \RR$
    a continuous function. If $(f_n)$ converges pointwise to $f$, then $(f_n)$ converges uniformly to $f$.
\end{theorem}

\begin{proof}
    \pf
    \step{1}{\pflet{$\epsilon > 0$}}
    \step{2}{For all $x \in X$, there exists $N$ such that, for all $n \geq N$, we have $|f_n(x) - f(x)| < \epsilon$}
    \step{3}{For $n \geq 1$, \pflet{$U_n = \{ x \in X \mid |f_n(x) - f(x)| < \epsilon\} $}}
    \step{4}{For $n \geq 1$, we have $U_n$ is open in $X$.}
    \begin{proof}
        \step{a}{\pflet{$x \in X$}}
        \step{b}{\pflet{$\delta = \epsilon - |f_n(x) - f(x)|$}}
        \step{b}{\pick\ a neighbourhood $U$ of $x$ such that $f(U) \subseteq B(f(x), \delta / 2)$}
        \step{c}{\pick\ a neighbourhood $V$ of $x$ such that $f_n(V) \subseteq B(f_n(x), \delta / 2)$}
        \step{d}{$f(U \cap V) \subseteq U_n$}
        \begin{proof}
            \pf\ For $y \in U \cap V$ we have
            \begin{align*}
                |f_n(y) - f(y)| & \leq |f_n(y) - f_n(x)| + |f_n(x) - f(x)| + |f(x) - y(y)| \\
                & < \delta / 2 + |f_n(x) - f(x)| + \delta / 2 \\
                & = \epsilon
            \end{align*}
        \end{proof}
    \end{proof}
    \step{5}{$ \{ U_n \mid n \geq 1 \}$ covers $X$}
    \begin{proof}
        \pf\ From \stepref{2}
    \end{proof}
    \step{6}{\pick\ $N$ such that $X = U_N$}
    \begin{proof}
        \step{a}{\pick\ $n_1$, \ldots, $n_k$ such that $U_{n_1}$, \ldots, $U_{n_k}$ cover $X$.}
        \step{b}{\pflet{$N = \max(n_1, \ldots, n_k)$}}
        \step{c}{For all $i$ we have $U_{n_i} \subseteq U_N$}
        \begin{proof}
            \pf\ Since $(f_n)$ is monotone increasing.
        \end{proof}
        \step{d}{$X = U_N$}
    \end{proof}
    \step{7}{For all $x \in X$ and $n \geq N$ we have $|f_n(x) - f(x)| < \epsilon$}
    \qed
\end{proof}

An example to show that we cannot remove the hypothesis that $X$ is compact:

\begin{example}
    Let $X = (0,1)$, $f_n(x) = -x^n$ and $f(x) = 0$ for $x \in X$ and $n \geq 1$. Then $f_n \rightarrow f$ pointwise and $(f_n)$ is monotone
    increasing but the convergence is not uniform since, for all $N \geq 1$, there exists $x \in (0,1)$ such that $-x^N < -1/2$.
\end{example}

An example to show that we cannot remove the hypothesis that $(f_n)$ is monotone increasing:

\begin{example}
    Let $X = [0,1]$, $f_n(x) = 1 / (n^3 (x - 1/n)^2 + 1)$ and $f(x) = 0$ for $x \in X$ and $n \geq 1$. Then $X$ is compact and $f_n \rightarrow f$
    pointwise but the convergence is not uniform since, for all $N \geq 1$, there exists $x \in [0,1]$ such that $f_N(x) = 1$, namely
    $x = 1/N$.
\end{example}

\begin{theorem}
    Let $X$ be a compact Hausdorff space. Let $\AA$ be a chain of closed connected subsets of $X$. Then $\bigcap \AA$ is connected.
\end{theorem}

\begin{proof}
    \pf
    \step{1}{\assume{for a contradiction $C$ and $D$ form a separation of $\bigcap \AA$.}}
    \step{2}{\pick\ disjoint open sets $U$ and $V$ that include $C$ and $D$ respectively.}
    \begin{proof}
        \pf\ Proposition \ref{proposition:disjoint_neighbourhoods_compact_Hausdorff}.
    \end{proof}
    \step{3}{$\{ A - (U \cup V) \mid A \in \AA \}$ is a set of closed sets with the finite intersection property.}
    \begin{proof}
        \step{a}{For all $A \in \AA$ we have $A - (U \cup V)$ is closed.}
        \step{b}{For all $A_1, \ldots, A_n \in \AA$ we have $(A_1 \cap \cdots \cap A_n) - (U \cup V)$ is nonempty.}
        \begin{proof}
            \pf
            \step{i}{\pflet{$A_1, \ldots, A_n \in \AA$}}
            \step{ii}{\assume{without loss of generality $A_1 \subseteq A_2, \ldots, A_n$}}
            \begin{proof}
                \pf\ Since $\AA$ is a chain.
            \end{proof}
            \step{iii}{$A_1 - (U \cup V)$ is nonempty}
            \begin{proof}
                \pf\ Otherwise $(A_1 \cap \cdots \cap A_n \cap U)$ and $(A_1 \cap \cdots \cap A_n \cap V)$
                would form a separation of $A_n$.
            \end{proof}
        \end{proof}
    \end{proof}
    \step{4}{$\bigcap \AA - (U \cup V)$ is nonempty.}
    \begin{proof}
        \pf\ Theorem \ref{theorem:compact_closed_sets}.
    \end{proof}
    \qedstep
    \begin{proof}
        \pf\ This contradicts \stepref{1} since $\bigcap AA - (U \cup V)
        = \bigcap \AA - (C \cup D)$.
    \end{proof}
    \qed
\end{proof}

\begin{theorem}[Tychonoff Theorem (AC)]
    The product of a family of compact spaces is compact.
\end{theorem}

\begin{proof}
    \pf
    \step{1}{\pflet{$\{ X_\alpha \}_{\alpha \in J}$ be a family of compact spaces.}}
    \step{2}{\pflet{$X = \prod_{\alpha \in J} X_\alpha$}}
    \step{3}{For any $\AA \subseteq \pow X$, we have $\bigcap_{A \in \AA} \overline{A} \neq \emptyset$}
    \begin{proof}
        \step{a}{\pflet{$\AA \subseteq \pow X$}}
        \step{b}{\pick\ $\DD \supseteq \AA$ that is maximal with respect to the finite intersection property.
        \prove{$\bigcap_{D \in \DD} \overline{D} \neq \emptyset$}}
        \begin{proof}
            \pf\ Lemma \ref{lemma:maximal_finite_intersection_property}.
        \end{proof}
        \step{c}{For $\alpha \in J$, \pick\ $x_\alpha \in X_\alpha$ such that
        $x_\alpha \in \bigcap_{D \in \DD} \overline{\pi_\alpha(D)}$}
        \begin{proof}
            \pf\ Theorem \ref{theorem:compact_closed_sets} since $\{ \overline{\pi_\alpha(D)} \mid D \in \DD \}$ is a set
            of closed sets in $X_\alpha$ with the finite intersection property.
        \end{proof}
        \step{d}{\pflet{$x = (x_\alpha)_{\alpha \in J}$} \prove{$x \in \bigcap_{D \in \DD} \overline{D}$}}
        \step{e}{For any $\beta \in J$ and neighbourhood $U$ of $x_\beta$ in $X_\beta$,
        we have $\inv{\pi_\beta}(U)$ intersects every element of $\DD$}
        \begin{proof}
            \step{i}{\pflet{$\beta \in J$}}
            \step{ii}{\pflet{$U$ be a neighbourhood of $x_\beta$ in $X_\beta$.}}
            \step{iii}{\pflet{$D \in \DD$}}
            \step{iv}{$x_\beta \in \overline{\pi_\beta(D)}$}
            \begin{proof}
                \pf\ From \stepref{c}
            \end{proof}
            \step{v}{$U$ intersects $\pi_\beta(D)$.}
            \step{vi}{$\inv{\pi_\beta}(U)$ intersects $D$.}
        \end{proof}
        \step{f}{For any $\beta \in J$ and neighbourhood $U$ of $x_\beta$ in $X_\beta$,
        we have $\inv{\pi_\beta}(U) \in \DD$}
        \begin{proof}
            \pf\ Lemma \ref{lemma:member_maximal_finite_intersection}.
        \end{proof}
        \step{g}{Every basic neighbourhood of $x$ is an element of $\DD$}
        \begin{proof}
            \pf\ Lemma \ref{lemma:finite_intersection_maximal}.
        \end{proof}
        \step{h}{Every basic neighbourhood of $x$ intersects every element of $\DD$}
        \begin{proof}
            \pf\ Since $\DD$ satisfies the finite intersection property.
        \end{proof}
        \step{i}{For all $D \in \DD$ we have $x \in \overline{D}$}
    \end{proof}
    \qedstep
    \begin{proof}
        \pf\ Theorem \ref{theorem:compact_closed_sets}.
    \end{proof}
    \qed
\end{proof}

\begin{example}
    The space $S_\Omega \times \overline{S_\Omega}$ is limit point compact.

\begin{proof}
    \pf
    \step{1}{\pflet{$A$ be an infinite subset of $S_\Omega \times \overline{S_\Omega}$.}}
    \step{2}{\pick\ $B \subseteq A$ that is countably infinite.}
    \begin{proof}
        \pf\ Proposition \ref{proposition:countable_subset}.
    \end{proof}
    \step{3}{\pflet{$b = \sup \pi_1(B)$ and $c = \sup \pi_2(B)$}}
    \step{4}{$B \subseteq [0,b] \times [0,c]$}
    \step{5}{$[0,b] \times [0,c]$ is compact.}
    \begin{proof}
        \pf\ Corollary \ref{corollary:closed_interval_compact}, Tychonoff Theorem.
    \end{proof}
    \step{6}{\pick\ a limit point $x$ of $B$ in $[0,b] \times [0,c]$.}
    \begin{proof}
        \pf\ Proposition \ref{proposition:limit_point_compact_compact}.
    \end{proof}
    \step{7}{$x$ is a limit point of $A$.}
    \begin{proof}
        \pf\ Lemma \ref{lemma:limit_point_subset}.
    \end{proof}
    \qed
\end{proof}

\end{example}

\begin{lemma}
    \label{lemma:preTychonoff}
    Let $X$ and $Y$ be topological spaces.
    Let $\AA$ be a set of basis elements for the product topology on $X \times Y$ such that no finite subset of $\AA$ covers $X \times Y$.
    If $X$ is compact, then there exists $x \in X$ such that no finite subset of $\AA$ covers the slice $\{x\} \times Y$.
\end{lemma}

\begin{proof}
    \pf
    \step{1}{\assume{for every $x \in X$, there exists a finite subset of $\AA$ that covers $\{x\} \times Y$}
    \prove{A finite subset of $\AA$ covers $X \times Y$}}
    \step{2}{$\{ U \mid \exists U_1 \times V_1, \ldots, U_n \times V_n \in \AA. U = U_1 \cap \cdots \cap U_n, V_1 \cup \cdots \cup V_n = Y \}$ covers $X$}
    \step{3}{\pick\ a finite subcover $U_1$, \ldots, $U_m$}
    \step{4}{\pick\ $U_{ij} \times V_{ij} \in \AA$ such that, for every $i$, we have $U_i = \bigcap_j U_{ij}$ and $Y = \bigcup_j V_{ij}$}
    \step{5}{The collection of all $U_{ij} \times V_{ij}$ covers $X \times Y$}
    \qed
\end{proof}

\begin{theorem}[AC]
    Let $X$ be a compact Hausdorff space. Then the quasicomponents and the components of $X$ are the same.
\end{theorem}

\begin{proof}
    \pf
    \step{1}{\pflet{$x, y \in X$}}
    \step{2}{\assume{$x$ and $y$ are in the same quasicomponent.} \prove{$x$ and $y$ are in the same component.}}
    \step{3}{\pflet{$\AA$ be the set of all closed subsets $A$ of $X$ such that $x$ and $y$ are in the same quasicomponent of $A$.}}
    \step{4}{For every chain $\BB \subseteq \AA$ we have $\bigcap \BB \in \AA$}
    \begin{proof}
        \step{a}{\pflet{$BB \subseteq \AA$ be a chain.}}
        \step{b}{\assume{for a contradiction $U$ and $V$ form a separation of $\bigcap \BB$ with $x \in U$ and $y \in V$}}
        \step{c}{\pick\ disjoint open sets $U'$, $V'$ in $X$ such that $U \subseteq U'$ and $V \subseteq V'$}
        \step{d}{$\{ B - (U' \cup V') \mid B \in \BB \}$ satisfies the finite intersection property.}
        \begin{proof}
            \step{i}{\pflet{$B_1, \ldots, B_n \in \BB$}}
            \step{ii}{\assume{without loss of generality $B_1 \subseteq \cdots \subseteq B_n$}}
            \begin{proof}
                \pf\ Since $\BB$ is a chain.
            \end{proof}
            \step{iii}{$\bigcap \{ B_1 - (U' \cup V'), \ldots, B_n - (U' \cup V') \} = B_1 - (U' \cup V')$}
            \step{iv}{$B_1 - (U' \cup V')$ is nonempty}
            \begin{proof}
                \pf\ Otherwise $B_1 \cap U'$ and $B_1 \cap V'$ would form a separation of $B_1$, contradicting the fact that $x$ and $y$ are in the same quasicomponent of $B_1$.
            \end{proof}
        \end{proof}
        \step{e}{$\bigcap \BB - (U \cup V)$ is nonempty}
        \begin{proof}
            \pf\ Theorem \ref{theorem:compact_closed_sets}.
        \end{proof}
        \qedstep
        \begin{proof}
            \pf\ This contradicts \stepref{b}.
        \end{proof}
    \end{proof}
    \step{5}{\pick\ a minimal element $D$ in $\AA$. \prove{$D$ is connected.}}
    \begin{proof}
        \pf\ By Zorn's Lemma.
    \end{proof}
    \step{6}{\assume{for a contradiction $U$ and $V$ form a separation of $D$.}}
    \step{7}{\assume{without loss of generality $x, y \in U$}}
    \begin{proof}
        \pf\ We cannot have that one of $x$, $y$ is in $U$ and the other in $V$ sicnce $D \in \AA$.
    \end{proof}
    \step{8}{$U \in \AA$}
    \begin{proof}
        \pf\ If $X$ and $Y$ form a separation of $U$ with $x \in X$ and $y \in Y$, then $X$ and $Y \cup V$ form a separation of $D$ with $x \in X$
        and $y \in Y \cup V$.
    \end{proof}
    \qedstep
    \begin{proof}
        \pf\ There is a connected set $D$ that contains both $x$ and $y$.
    \end{proof}
    \qed
\end{proof}

\begin{proof}
    \pf
    \step{1}{\pflet{$\{X_\alpha\}_{\alpha \in J}$ be a family of compact spaces.}}
    \step{2}{\pflet{$X = \prod_{\alpha \in J} X_\alpha$}}
    \step{3}{\pick\ a well-ordering $<$ on $J$ such that $J$ has a greatest element.}
    \step{4}{For $\alpha \in J$ and $p = \{ p_i \in X_i \}_{i \leq \alpha}$ a family of points,
    \pflet{$Y(p) = \{ x \in X \mid \forall i \leq \alpha. x_i = p_i \}$}}
    \step{5}{If $\alpha < \alpha'$ and $p$ is an $\alpha'$-indexed family of points then $Y(p) \subseteq Y(p \restriction \alpha)$}
    \begin{proof}
        \pf\ From definition.
    \end{proof}
    \step{6}{Given $\beta \in J$ and $p = \{ p_i \in X_i \}_{i < \beta}$ a family of points,
    \pflet{$Z(p) = \bigcap_{\alpha < \beta} Y(p \restriction \alpha)$}}
    \step{7}{Given $\beta \in J$ and $p = \{ p_i \in X_i \}_{i < \beta}$ a family of points, if $\AA$ is a finite set of basic open spaces for $X$ that
    covers $Z(p)$, then there exists $\alpha < \beta$ such that $\AA$ covers $Y(p \restriction \alpha)$}
    \begin{proof}
        \step{a}{\assume{without loss of generality $\beta$ has no immediate predecessor.}}
        \step{b}{For $A \in \AA$, \pflet{$J_A = \{ i < \beta \mid \pi_i(A) \neq X_i \}$}}
        \step{c}{\pflet{$\alpha = \max \bigcup_{A \in \AA} J_A$}}
        \step{d}{\pflet{$x \in Y(p \restriction \alpha)$}}
        \step{e}{\pflet{$y \in Z(p)$ be the point with $y_i = p_i$ for $i < \beta$ and $y_i = x_i$ for $i \geq \beta$}}
        \step{f}{\pick\ $A \in \AA$ such that $y \in A$}
        \begin{proof}
            \pf\ Since $\AA$ covers $Z(p)$.
        \end{proof}
        \step{g}{For $i \in J_A$ we have $x_i \in \pi_i(A)$}
        \begin{proof}
            \pf\ Since $i \leq \alpha$ so $x_i = p_i$
        \end{proof}
        \step{g}{For $i \in J - J_A$ we have $x_i \in \pi_i(A)$}
        \begin{proof}
            \pf\ Since $\pi_i(A) = X_i$
        \end{proof}
        \step{h}{$x \in A$}
    \end{proof}
    \step{8}{\assume{for a contraction $\AA$ is a set of basic open sets for $X$ that covers $X$ but such that no finite subset of $\AA$ covers $X$}}
    \step{9}{\pick\ a set of points $\{ p_i \}_{i \in J}$ such that, for all $\alpha \in J$, we have $Y(p \restriction \alpha)$ is not finitely covered by $\AA$}
    \begin{proof}
        \step{a}{\assume{as transfinite induction hypothesis $\alpha \in J$ and $\{ p_i \}_{i < \alpha}$ is a family of points such that,
        for all $\alpha' < \alpha$, we have $Y(p \restriction \alpha')$ is not finitely covered by $\AA$}}
        \step{b}{$Z(p)$ is not finitely covered by $\AA$}
        \begin{proof}
            \pf\ By \stepref{7}.
        \end{proof}
        \step{c}{\pick\ $p_\alpha \in X_\alpha$ such that $Y(p)$ is not finitely covered by $\AA$}
        \begin{proof}
            \pf\ By Lemma \ref{lemma:preTychonoff} since there is a homeomorphism $\phi : Z(p) \cong X_\alpha \times \prod_{\alpha' > \alpha} X_{\alpha'}$
            and, given $p_\alpha$, this homemorphism $\phi$ restricts to a homeomorphism $Y(p) \cong \{ p_\alpha \}  \times \prod_{\alpha' > \alpha} X_{\alpha'}$.
        \end{proof}
    \end{proof}
    \qedstep
    \begin{proof}
        \pf\ If $\omega$ is the greatest element of $J$ then $Y(p \restriction \omega)$ is a singleton.
    \end{proof}
    \qed
\end{proof}

\begin{theorem}
    Every complete linearly ordered set in the order topology is compact.
\end{theorem}

\begin{proof}
    \pf
    \step{1}{\pflet{$X$ be a complete linearly ordered set with least element $a$ and greatest element $b$.}}
    \step{2}{\pflet{$\AA$ be an open covering of $X$.}}
    \step{3}{For all $x < b$, there exists $y > x$ such that $[x,y]$ can be covered by at most two elements of $\AA$.}
    \begin{proof}
        \step{a}{\pflet{$x \in X$}}
        \step{b}{\pick\ $A \in \AA$ with $x \in A$}
        \step{c}{\pick\ $y > x$ such that $[x,y) \subseteq A$}
        \step{d}{\pick\ $B \in \AA$ with $y \in B$}
        \step{e}{$[x,y]$ is covered by $A$ and $B$}
    \end{proof}
    \step{4}{\pflet{$C = \{y \in X \mid [a,y] \text{ can be covered by finitely many elements of } \AA \}$}}
    \step{5}{\pflet{$c = \sup C$}}
    \step{6}{$c > a$}
    \begin{proof}
        \step{a}{\pick\ $x > a$ such that $[a,x]$ can be covered by at most two elements of $\AA$.}
        \begin{proof}
            \pf\ From \stepref{3}.
        \end{proof}
        \step{b}{$x \in C$}
    \end{proof}
    \step{7}{$c \in C$}
    \begin{proof}
        \step{a}{\pick\ $A \in \AA$}
        \step{b}{\pick\ $x < c$ such that $(x,c] \subseteq A$}
        \step{c}{\pick\ $y > x$ such that $y \in C$}
        \step{d}{\pick\ $\AA_0 \subseteq^{\mathrm{fin}} \AA$ that covers $[a,y]$}
        \step{e}{$\AA_0 \cup \{A\}$ covers $[a,c]$}
    \end{proof}
    \step{8}{$c = b$}
    \begin{proof}
        \step{a}{\assume{for a contradiction $c < b$}}
        \step{b}{\pick\ $x > c$ such that $[c,x]$ can be covered by at most two elements of $\AA$}
        \begin{proof}
            \pf\ From \stepref{3}.
        \end{proof}
        \step{c}{$[a,x]$ can be finitely covered by $\AA$}
        \begin{proof}
            \pf\ From \stepref{7}.
        \end{proof}
        \qedstep
        \begin{proof}
            \pf\ This contradicts the maximality of $c$.
        \end{proof}
    \end{proof}
    \qed
\end{proof}

\begin{corollary}
    \label{corollary:closed_interval_compact}
    Let $X$ be a linearly ordered set with the least upper bound property. Then every closed interval in $X$ is compact.
\end{corollary}

\begin{corollary}
    Every closed interval in $\RR$ is compact.
\end{corollary}

\begin{corollary}
    The space $\overline{S_\Omega}$ is compact.
\end{corollary}

\begin{example}
    The ordered square is compact.
\end{example}

\begin{example}
    The space $S_\Omega$ is limit point compact.

\begin{proof}
    \pf
    \step{1}{\pflet{$A$ be an infinite subset of $S_\Omega$.}}
    \step{2}{\pick\ $B \subseteq A$ that is countably infinite.}
    \begin{proof}
        \pf\ Proposition \ref{proposition:countable_subset}.
    \end{proof}
    \step{3}{\pflet{$b = \sup B$}}
    \step{4}{$B \subseteq [0,b]$}
    \step{5}{$[0,b]$ is compact.}
    \begin{proof}
        \pf\ Corollary \ref{corollary:closed_interval_compact}.
    \end{proof}
    \step{6}{\pick\ a limit point $x$ of $B$ in $[0,b]$.}
    \begin{proof}
        \pf\ Proposition \ref{proposition:limit_point_compact_compact}.
    \end{proof}
    \step{7}{$x$ is a limit point of $A$.}
    \begin{proof}
        \pf\ Lemma \ref{lemma:limit_point_subset}.
    \end{proof}
    \qed
\end{proof}
\end{example}

\begin{theorem}[Extreme Value Theorem]
    Any linearly ordered set under the order topology that is compact has
    a greatest and a least element.
\end{theorem}

\begin{proof}
    \pf
    \step{1}{\pflet{$X$ be a linearly ordered set under the order topology that is compact.}}
    \step{2}{$X$ has a greatest element.}
    \begin{proof}
        \step{a}{\assume{for a contradiction $X$ has no greatest element.}}
        \step{b}{$\{ (- \infty, a) \mid a \in X \}$ covers $X$.}
        \step{c}{\pick\ a finite subcover $\{ (- \infty, a_1), \ldots, (- \infty, a_n) \}$, say.}
        \step{d}{\assume{without loss of generality $a_1 \leq \cdots \leq a_n$}}
        \step{e}{$X \subseteq (- \infty, a_n)$}
        \step{f}{$a_n < a_n$}
    \end{proof}
    \step{3}{$X$ has a least element.}
    \begin{proof}
        \pf\ Similar.
    \end{proof}
    \qed
\end{proof}

\begin{example}
    The space $S_\Omega$ is not compact, because it has no greatest element.
\end{example}

\begin{example}
    The space $S_\Omega \times \overline{S_\Omega}$ is not compact, because it has a closed subspace $S_\Omega \times \{0\}$ homeomorphic to $S_\Omega$.
\end{example}

\begin{proposition}
    Every linearly ordered set in which every closed interval is compact
    satisfies the least upper bound property.
\end{proposition}

\begin{proof}
    \pf
    \step{1}{\pflet{$X$ be a linearly ordered set in which every closed interval is compact.}}
    \step{2}{\pflet{$A \subseteq X$ be nonempty with upper bound $u$}}
    \step{3}{\pick\ $a \in A$}
    \step{4}{The closed interval $[a,u]$ is compact.}
    \step{5}{\assume{for a contradiction $A$ has no supremum.}}
    \step{5}{$\{ (-\infty, x) \mid x \in A \} \cup \{ (x,+\infty) \mid x
    \text{ is an upper bound of } A \}$ covers $[a,u]$.}
    \begin{proof}
        \step{i}{\pflet{$x \in [a,u]$}}
        \step{ii}{\assume{for all $y \in A$ we have $x \notin (-\infty, y)$}}
        \step{iii}{$x$ is an upper bound for $A$}
        \step{iv}{\pick\ an upper bound $y$ for $A$ with $y < x$}
        \step{v}{$x \in (y, +\infty)$}
    \end{proof}
    \step{6}{\pick\ a finite subcover $\{(- \infty, x_1), \ldots,
    (-\infty, x_m), (y_1, +\infty), \ldots, (y_n, +\infty) \}$}
    \step{7}{\assume{$x_m = \max(x_1, \ldots, x_m)$ and $y_1 = \min(y_1, \ldots, y_n)$}}
    \step{8}{$x_m \notin (- \infty, x_i)$ for any $i$}
    \begin{proof}
        \pf\ Since $x_i \leq x_m$
    \end{proof}
    \step{9}{$x_m \notin (y_i, +\infty)$ for any $i$}
    \begin{proof}
        \pf\ Since $x_m \in A$ so $x_m \leq y_i$
    \end{proof}
    \step{10}{$x_m \in [a,u]$}
    \begin{proof}
        \step{a}{$a \notin (y_i, +\infty)$ for any $i$}
        \begin{proof}
            \pf\ Since $y_i$ is an upper bound for $A$ and $a \in A$.
        \end{proof}
        \step{b}{$a \in (-\infty, x_i)$ for some $i$}
        \begin{proof}
            \pf\ From \stepref{6}.
        \end{proof}
        \step{c}{$a < x_m$}
        \begin{proof}
            \pf\ Since $x_i \leq x_m$
        \end{proof}
        \step{d}{$x_m \leq u$}
        \begin{proof}
            \pf\ Since $u$ is an upper bound for $A$ and $x_m \in A$.
        \end{proof}
    \end{proof}
    \qedstep
    \begin{proof}
        \pf\ This contradicts \stepref{6}.
    \end{proof}
    \qed
\end{proof}

\begin{example}
    \label{example:zero_one_not_compact}
    The set $[0,1]$ is not compact under the $K$-topology.
\end{example}

\begin{proof}
    \pf\ For every $n \geq 1$, pick an open interval $U_n$ such that $U_n \cap K = \{ 1/n \}$.
    Then the open cover $\{ [0,1]-K \} \cup \{ U_n \mid n \in \ZZ^+ \}$ has no finite subcover. \qed
\end{proof}

\begin{proposition}[AC]
    \label{proposition:pre_Baire_category}
    Let $X$ be a compact Hausdorff space. Let $\AA$ be a countable set of closed sets in $X$.
    If every element of $\AA$ has empty interior, then $\bigcup \AA$ has empty interior.
\end{proposition}

\begin{proof}
    \pf
    \step{1}{\pflet{$X$ be a compact Hausdorff space.}}
    \step{2}{For every closed set $A$ in $X$ and open $U$ in $X$ with
    $U \nsubseteq A$, there exists a nonempty open set $V$ such that
    $\overline{V} \subseteq U - A$.}
    \begin{proof}
        \step{a}{\pflet{$A$ be a closed set in $X$}}
        \step{b}{\pflet{$U$ be an open set in $X$ with $U \nsubseteq A$}}
        \step{c}{\pick\ $x \in U - A$}
        \step{d}{\pick\ disjoint neighbourhoods $W$ and $V$ of $A \cup (X - U)$
        and $x$ respectively.}
        \begin{proof}
            \pf\ Proposition \ref{proposition:disjoint_neighbourhoods_compact_Hausdorff}.
        \end{proof}
        \step{e}{$\overline{V} \subseteq U - A$}
        \begin{proof}
            \pf
            \begin{align*}
                \overline{V} & \subseteq X - W & (\text{since } V \subseteq X - W) \\
                & \subseteq X - (A \cup (X - U)) \\
                & = (x - A) \cap U \\
                & = U - A
            \end{align*}
        \end{proof}
    \end{proof}
    \step{3}{\pick\ an enumeration $\{ A_1, A_2, \ldots \}$ of $\AA$}
    \step{4}{\pflet{$U_0$ be any nonempty open set} \prove{$U_0 \nsubseteq \bigcup \AA$}}
    \step{5}{\pick\ a sequence of nonempty open sets $U_1$, $U_2$, \ldots such that,
    for $n \geq 1$, we have $\overline{U_n} \subseteq U_{n-1} - A_n$}
    \begin{proof}
        \step{a}{\assume{we have picked $U_0$, $U_1$, \ldots, $U_n$}}
        \step{b}{$U_n \nsubseteq A_{n+1}$}
        \begin{proof}
            \pf\ Since $A_{n+1}$ has empty interior.
        \end{proof}
        \step{c}{\pick\ a nonempty open set $U_{n+1}$ such that $\overline{U_{n+1}}
        \subseteq U_n - A_{n+1}$}
        \begin{proof}
            \pf\ By \stepref{2}
        \end{proof}
    \end{proof}
    \step{6}{\pick\ $a \in \bigcap_{n=0}^\infty \overline{U_n}$}
    \begin{proof}
        \pf\ Corollary \ref{corollary:nested_intersection}.
    \end{proof}
    \step{7}{$a \in U_0$}
    \begin{proof}
        \pf\ Since $a \in \overline{U_1} \subseteq U_0$.
    \end{proof}
    \step{8}{$a \notin \bigcup \AA$}
    \begin{proof}
        \pf\ For all $n$, we have $a \in \overline{U_n} \subseteq U_{n-1} - A_n$.
    \end{proof}
    \qed
\end{proof}

\begin{example}
    The Cantor set is compact.
\end{example}

\begin{proof}
    \pf\ It is a closed subset of the compact set $[0,1]$. \qed
\end{proof}

\begin{proposition}
    \label{proposition:limit_point_compact_compact}
    Every compact space is limit point compact.
\end{proposition}

\begin{proof}
    \pf
    \step{1}{\pflet{$X$ be a compact space.}}
    \step{2}{\pflet{$A \subseteq X$ have no limit points.} \prove{$A$ is finite.}}
    \step{3}{$A$ is closed.}
    \begin{proof}
        \pf\ Corollary \ref{corollary:closed_limit_point}.
    \end{proof}
    \step{4}{$A$ is compact.}
    \begin{proof}
        \pf\ Proposition \ref{proposition:closed_subspace_compact}.
    \end{proof}
    \step{5}{$\{ U \mid U \text{ open }, |U \cap A| = 1 \}$ covers $A$.}
    \begin{proof}
        \pf\ From \stepref{2}, for all $a \in A$, there is a neighbourhood
        $U$ of $a$ that intersects $A$ in $a$ only.
    \end{proof}
    \step{6}{\pick\ a finite subcover $\{ U_1, \ldots, U_n \}$}
    \step{7}{For $i = 1, \ldots, n$, \pflet{$U_i \cap A = \{ x_i \}$.}}
    \step{8}{$A = \{ x_1, \ldots, x_n \}$}
    \qed
\end{proof}

\begin{corollary}
    The space $\RR^I$ is not compact.
\end{corollary}

The following examples show that not every limit point compact space is compact.
\begin{example}
    Let $Y$ be a set with two elements under the indiscrete topology.
    Then $\ZZ^+ \times Y$ is limit point compact, since every nonempty set has
    a limit point. It is not compact, since $\{ \{ n \} \times Y \mid n \in \ZZ^+ \}$
    has no finite subcover.
\end{example}

\begin{proposition}[AC]
    A topological space is compact if and only if every net has a convergent subnet.
\end{proposition}

\begin{proof}
    \pf
    \step{1}{\pflet{$X$ be a topological space.}}
    \step{2}{If $X$ is compact then every net has a convergent subnet.}
    \begin{proof}
        \step{a}{\assume{$X$ is compact.}}
        \step{b}{\pflet{$(a_\alpha)_{\alpha \in J}$ be a net in $X$.}}
        \step{c}{For $\alpha \in J$, \pflet{$B_\alpha = \{ a_\beta \mid \alpha \leq \beta \}$}}
        \step{d}{$\{ B_\alpha \mid \alpha \in J \}$ has the finite intersection property.}
        \step{e}{\pick\ $x \in \bigcap_{\alpha \in J} \overline{B_\alpha}$}
        \step{f}{$x$ is an accumulation point of $(a_\alpha)_{\alpha \in J}$}
        \begin{proof}
            \step{i}{\pflet{$U$ be a neighbourhood of $x$.}}
            \step{ii}{\pflet{$\alpha \in J$}}
            \step{iii}{$x \in \overline{B_\alpha}$}
            \step{iv}{There exists $\beta \geq \alpha$ such that $a_\beta \in U$}
        \end{proof}
        \qedstep
        \begin{proof}
            \pf\ Lemma \ref{lemma:accumulation_point_subnet}.
        \end{proof}
    \end{proof}
    \step{3}{If every net in $X$ has a convergent subnet then $X$ is compact.}
    \begin{proof}
        \step{a}{\assume{Every net in $X$ has a convergent subnet.}}
        \step{b}{\pflet{$\AA$ be a set of closed sets with the finite intersection property.}}
        \step{c}{\pflet{$\BB$ be the set of all finite intersections of elements of $\AA$ under $\supseteq$.}}
        \step{d}{For $B \in \BB$, \pick\ $a_B \in B$}
        \step{e}{\pick\ a convergent subnet $(a_{g(\alpha)})_{\alpha \in K}$
        with limit $l$. \prove{$l \in \bigcap \AA$}}
        \begin{proof}
            \pf\ From \stepref{a}.
        \end{proof}
        \step{f}{\pflet{$A \in \AA$}}
        \step{g}{\assume{for a contradiction $l \notin A$}}
        \step{h}{\pick\ $\alpha \in K$ such that, for all $\beta \geq \alpha$,
        we have $a_{g(\beta)} \in X - A$}
        \step{i}{\pick\ $\beta \in K$ such that $g(\beta) \geq A$}
        \step{j}{\pick\ $\gamma \in K$ with $\alpha \leq \gamma$ and
        $\beta \leq \gamma$}
        \step{k}{$a_{g(\gamma)} \in A$ and $a_{g(\gamma)} \in X - A$}
        \qedstep
        \begin{proof}
            \pf\ This is a contradiction.
        \end{proof}
    \end{proof}
    \qed
\end{proof}

\begin{example}
    The space $\RR_K$ is not path connected.
\end{example}

\begin{proof}
    \pf
    \step{1}{\assume{for a contradiction $p : [0,1] \rightarrow \RR_K$ was a path from $0$ to $1$.}}
    \step{2}{$p([0,1])$ as a subspace of $\RR_K$ is compact.}
    \begin{proof}
        \pf\ Theorem \ref{theorem:continuous_image_compact}.
    \end{proof}
    \step{3}{$p([0,1])$ as a subspace of $\RR_K$ is connected.}
    \begin{proof}
        \pf\ Theorem \ref{theorem:connected_continuous_image}.
    \end{proof}
    \step{4}{$p([0,1])$ is connected as a subspace of $\RR$.}
    \begin{proof}
        \pf\ Theorem \ref{theorem:connected_continuous_image} as
        the identity map is continuous as a map $\RR_K \rightarrow \RR$.
    \end{proof}
    \step{5}{$p([0,1])$ is convex.} %TODO Extract lemma
    \begin{proof}
        \step{a}{\pflet{$a, b \in p([0,1])$ and $a < c < b$}}
        \step{b}{\assume{for a contradiction $c \notin p([0,1])$}}
        \step{c}{$(-\infty, c) \cap p([0,1])$ and $(c, +\infty) \cap p([0,1])$
        form a separation of $p([0,1])$ as a subspace of $\RR$.}
        \qedstep
        \begin{proof}
            \pf\ This contradicts \stepref{4}.
        \end{proof}
    \end{proof}
    \step{6}{$[0,1] \subseteq p([0,1])$}
    \step{7}{$[0,1]$ as a subspace of $\RR_K$ is compact.}
    \begin{proof}
        \pf\ By Proposition \ref{proposition:closed_subspace_compact}
        and \stepref{2}.
    \end{proof}
    \qedstep
    \begin{proof}
        \pf\ This contradicts Example \ref{example:zero_one_not_compact}.
    \end{proof}
    \qed
\end{proof}

\begin{proposition}[Choice]
    The product of a Lindel\"{o}f and a compact space is Lindel\"{o}f.
\end{proposition}

\begin{proof}
    \pf
    \step{1}{\pflet{$X$ be a Lindel\"{o}f space.}}
    \step{2}{\pflet{$Y$ be a compact space.}}
    \step{3}{\pflet{$\AA$ be an open cover of $X \times Y$.}}
    \step{4}{$\{ W \text{ open in } X \mid W \times Y \text{ can be covered by
    finitely many elements of } \AA \}$ is an open cover of $X$.}
    \begin{proof}
        \step{a}{\pflet{$x \in X$}}
        \step{b}{$\{ V \text{ open in } Y \mid \exists U \text{ open in } X.
        \exists A \in \AA. x \in U \text{ and } U \times V \subseteq A \}$
        covers $Y$}
        \step{c}{\pick\ a finite subcover $V_1$, \ldots, $V_n$}
        \step{d}{$\{x\} \times Y \subseteq V_1 \cup \cdots \cup V_n$}
        \step{e}{\pick\ a neighbourhood $W$ of $x$ such that $W \times Y
        \subseteq V_1 \cup \cdots \cup V_n$}
        \begin{proof}
            \pf\ By the Tube Lemma.
        \end{proof}
    \end{proof}
    \step{5}{\pick\ a countable subcover $\BB$}
    \step{6}{For $W \in \BB$, \pick\ $U_{W1}, \ldots, U_{Wn_W} \in \AA$
    such that $W \times Y \subseteq U_{W1} \cup \cdots \cup U_{Wn_W}$}
    \step{7}{$\{ U_{Wi} \mid W \in \BB, 1 \leq i \leq n_W \}$ covers $X \times Y$}
    \qed
\end{proof}

\begin{proposition}
    Every compact space is Lindel\"{o}f.
\end{proposition}

\begin{proof}
    \pf\ Immediate from definitions. \qed
\end{proof}

\begin{corollary}
    The Sorgenfrey plane is not compact.
\end{corollary}

\begin{proposition}[Choice]
    \label{proposition:countable_compact_lindelof}
    A topological space that is a countable union of compact subspaces is Lindel\"{o}f.
\end{proposition}

\begin{proof}
    \pf
    \step{1}{\pflet{$X$ be a topoogical space.}}
    \step{2}{\pflet{$\AA$ be a countable set of compact subspaces of $X$ such that $X = \bigcup \AA$}}
    \step{3}{\pflet{$\UU$ be any open cover of $X$.}}
    \step{4}{For $A \in \AA$, \pick\ a finite subset $\UU_A$ that covers $A$.}
    \step{5}{$\bigcup_A \UU_A$ is a countable subset of $\UU$ that covers $X$.}
    \qed
\end{proof}

\section{Perfect Maps}

\begin{definition}[Perfect Map]
    Let $X$ and $Y$ be topological spaces and $f : X \rightarrow Y$. Then $f$ is a \emph{perfect map} if and only if $f$ is a closed map, continuous,
    surjective and $\inv{p}(y)$ is compact for all $y \in Y$.
\end{definition}

\begin{proposition}
    \label{proposition:compact_pre_image}
    Let $X$ be a topological space, $Y$ a compact space, and $p : X \rightarrow Y$ a closed map such that, for all $y \in Y$, we have $\inv{p}(y)$ is compact. Then $X$ is compact.
\end{proposition}

\begin{proof}
    \pf
    \step{1}{\pflet{$\AA$ be a set of closed sets in $X$ with the finite intersection property.}}
    \step{2}{$\BB = \{ p(A_1 \cap \cdots \cap A_n) \mid A_1, \ldots, A_n \in \AA \}$ is a set of closed sets in $Y$ with the finite intersection property.}
    \begin{proof}
        \pf\ Since $p$ is a closed map.
    \end{proof}
    \step{3}{\pick\ $y \in \bigcap \BB$}
    \begin{proof}
        \pf\ Theorem \ref{theorem:compact_closed_sets} since $Y$ is compact.
    \end{proof}
    \step{4}{$\{ A \cap \inv{p}(y) \mid A \in \AA\}$ is a set of closed sets in $\inv{p}(y)$ with the finite intersection property.}
    \step{5}{\pick\ $x \in \bigcap_{A \in \AA} (A \cap \inv{p}(y))$}
    \begin{proof}
        \pf\ Theorem \ref{theorem:compact_closed_sets} since $\inv{p}(y)$ is compact.
    \end{proof}
    \step{6}{$x \in \bigcap \AA$}
    \qedstep
    \begin{proof}
        \pf\ Theorem \ref{theorem:compact_closed_sets}.
    \end{proof}
    \qed
\end{proof}

\begin{proposition}
    \label{proposition:perfect_image_Hausdorff}
    The image of a Hausdorff space under a perfect map is Hausdorff.
\end{proposition}

\begin{proof}
    \pf
    \step{1}{\pflet{$X$ be a Hausdorff space.}}
    \step{2}{\pflet{$Y$ be a topological space.}}
    \step{3}{\pflet{$p : X \twoheadrightarrow Y$ be a perfect map.}}
    \step{4}{\pflet{$y, y' \in Y$ with $y \neq y'$}}
    \step{5}{\pick\ disjoint open neighbourhoods $U_1$ and $U_2$ of $\inv{p}(y)$
    and $\inv{p}(y')$}
    \begin{proof}
        \pf\ Proposition \ref{proposition:disjoint_neighbourhoods_compact_Hausdorff}.
    \end{proof}
    \step{6}{\pick\ neighbourhoods $V_1$ and $V_2$ of $y$ and $y'$ respectively
    such that $\inv{p}(V_1) \subseteq U_1$ and $\inv{p}(V_2) \subseteq U_2$}
    \begin{proof}
        \pf\ Lemma \ref{lemma:closed_map_neighbourhoods},
    \end{proof}
    \step{7}{$V_1 \cap V_2 = \emptyset$}
    \begin{proof}
        \step{a}{\assume{for a contradiction $z \in V_1 \cap V_2$}}
        \step{b}{\pick\ $x \in X$ such that $p(x) = z$}
        \step{c}{$x \in U_1 \cap U_2$}
        \begin{proof}
            \pf\ From \stepref{6}.
        \end{proof}
        \qedstep
        \begin{proof}
            \pf\ This contradicts the fact that $U_1$ and $U_2$ are disjoint (\stepref{5}).
        \end{proof}
    \end{proof}
    \qed
\end{proof}

\begin{proposition}
    \label{proposition:perfect_image_regular}
    The image of a regular space under a perfect map is regular.
\end{proposition}

\begin{proof}
    \pf
    \step{1}{\pflet{$X$ be a regular space.}}
    \step{2}{\pflet{$Y$ be a topological space.}}
    \step{3}{\pflet{$p : X \twoheadrightarrow Y$ be a perfect map.}}
    \step{4}{$Y$ is $T_1$.}
    \begin{proof}
        \pf\ Proposition \ref{proposition:perfect_image_Hausdorff}.
    \end{proof}
    \step{5}{\pflet{$A$ be a closed set in $Y$ and $a \in Y - A$.}}
    \step{6}{$\inv{p}(A)$ is closed in $X$ and $\inv{p}(a)$ is disjoint from $\inv{p}(A)$}
    \step{7}{\pick\ disjoint open neighbourhoods $U_1$ and $U_2$ of $\inv{p}(A)$
     and $\inv{p}(a)$ respectively.}
    \begin{proof}
        \pf\ Proposition \ref{proposition:regular_space_compact_closed}.
    \end{proof}
    \step{8}{\pick\ open neighbourhoods $V_1$ and $V_2$ of $A$ and $a$ respectively
    such that $\inv{p}(V_1) \subseteq U_1$ and $\inv{p}(V_2) \subseteq U_2$.}
    \begin{proof}
        \pf\ Lemma \ref{lemma:closed_map_neighbourhoods}.
    \end{proof}
    \step{9}{$V_1 \cap V_2 = \emptyset$}
    \begin{proof}
        \step{a}{\assume{for a contradiction $y \in V_1 \cap V_2$}}
        \step{b}{\pick\ $x \in X$ such that $p(x) = y$}
        \step{c}{$x \in U_1 \cap U_2$}
        \begin{proof}
            \pf\ From \stepref{8}
        \end{proof}
        \qedstep
        \begin{proof}
            \pf\ This contradicts the fact that $U_1$ and $U_2$ are disjoint
            (\stepref{7}).
        \end{proof}
    \end{proof}
    \qed
\end{proof}

\begin{proposition}
    \label{proposition:perfect_image_second_countable}
    The image of a second countable space under a perfect map is second countable.
\end{proposition}

\begin{proof}
    \pf
    \step{1}{\pflet{$X$ be a second countable space.}}
    \step{2}{\pflet{$Y$ be a topological space.}}
    \step{3}{\pflet{$p : X \twoheadrightarrow Y$ be a perfect map.}}
    \step{4}{\pick\ a countable basis $\BB$ for $X$.}
    \step{5}{For $J \subseteq^{\mathrm{fin}} \BB$, \pflet{$U_J = \bigcup
    \{ W \mid W \text{ open in } Y, \inv{p}(W) \subseteq \bigcup J \}$}
    \prove{$\{ U_J \mid J \subseteq^{\mathrm{fin}} \BB \}$ is a basis for $Y$.}}
    \step{6}{\pflet{$y \in Y$}}
    \step{7}{\pflet{$V$ be a neighbourhood of $y$.}}
    \step{8}{For all $x \in \inv{p}(y)$, there exists $B \in \BB$ such that
    $x \in B \subseteq \inv{p}(V)$}
    \step{9}{\pick\ finitely many sets $B_1, \ldots, B_n \in \BB$ that cover
    $\inv{p}(y)$ such that $B_i \subseteq \inv{p}(V)$ for all $i$.}
    \step{10}{\pflet{$J = \{ B_1, \ldots, B_n \}$} \prove{$y \in U_J \subseteq V$}}
    \step{11}{$y \in U_J$}
    \begin{proof}
        \pf\ There exists an open neighbourhood $W$ of $y$ such that $\inv{p}(W) \subseteq
        B_1 \cup \cdots \cup B_n$ by Lemma \ref{lemma:closed_map_neighbourhoods}.
    \end{proof}
    \step{12}{$U_J \subseteq V$}
    \begin{proof}
        \pf\ Since $\bigcup J \subseteq V$ by \stepref{9}.
    \end{proof}
    \qed
\end{proof}

\section{Isolated Points}

\begin{definition}[Isolated Point]
    Let $X$ be a topolgical space and $x \in X$. Then $x$ is an
    \emph{isolated point} if and only if $\{x\}$ is open.
\end{definition}

\begin{theorem}[AC]
    A nonempty compact Hausdorff space with no isolated points is
    uncountable.
\end{theorem}

\begin{proof}
    \pf
    \step{1}{\pflet{$X$ be a nonempty compact Hausdorff space with no isolated points.}}
    \step{2}{For every nonempty open set $U$ and every point $x \in X$,
    there exists a nonempty open set $V \subseteq U$ such that $x \notin \overline{V}$.}
    \begin{proof}
        \step{a}{\pflet{$U$ be a nonempty open set.}}
        \step{b}{\pflet{$x \in X$}}
        \step{c}{\pick\ $y \in U - \{x\}$}
        \begin{proof}
            \pf\ This is possible because $U$ cannot be $\{x\}$.
        \end{proof}
        \step{d}{\pick\ disjoint open neighbourhoods $W_1$ of $x$
        and $W_2$ of $y$}
        \step{e}{\pflet{$V = W_2 \cap U$}}
        \step{f}{$V$ is nonempty}
        \begin{proof}
            \pf\ Since $y \in V$
        \end{proof}
        \step{g}{$V$ is open}
        \begin{proof}
            \pf\ From \stepref{a}, \stepref{d}, \stepref{e}.
        \end{proof}
        \step{h}{$V \subseteq U$}
        \begin{proof}
            \pf\ From \stepref{e}
        \end{proof}
        \step{i}{$x \notin V$}
        \begin{proof}
            \pf\ From \stepref{d} and \stepref{e}
        \end{proof}
    \end{proof}
    \step{3}{\pflet{$(a_n)$ be any sequence of points in $X$.} \prove{The set $X - \{ a_1, a_2, \ldots \}$ is nonempty.}}
    \step{4}{\pick\ a sequence of nonempty open sets $V_1$, $V_2$, \ldots,
    such that $V_1 \supseteq V_2 \supseteq \cdots$ and $a_n \notin \overline{V_n}$
    for all $n$.}
    \begin{proof}
        \pf\ From \stepref{2}.
    \end{proof}
    \step{5}{\pick\ $a \in \bigcap_{n=1}^\infty \overline{V_n}$}
    \begin{proof}
        \pf\ Corollary \ref{corollary:nested_intersection}.
    \end{proof}
    \step{6}{$a \in X - \{ a_1, a_2, \ldots \}$}
    \begin{proof}
        \pf\ We cannot have $a = a_n$ because $a \in \overline{V_n}$.
    \end{proof}
    \qed
\end{proof}

\begin{corollary}
    \label{corollary:closed_interval_uncountable}
    For all $a, b \in \RR$ with $a < b$, the closed interval $[a,b]$
    is uncountable.
\end{corollary}

\begin{example}
    The Cantor set has no isolated points, and is therefore uncountable.
\end{example}

\begin{proof}
    \pf
    \step{0}{\pflet{$(A_n)$ be the sets in Definition \ref{definition:Cantor_set}.}}
    \step{1}{\pflet{$x \in C$}}
    \step{2}{\pflet{$A_n$ be the first set such that $x$ is an endpoint of one of the intervals that make up $A_n$}}
    \step{3}{\pflet{$(a_m)_{m \geq n}$ be the sequence of points defined by:
    $a_m$ is the point such that either $[a_m, x]$ or $[x, a_m]$ is one of
    the intervals that make up $A_m$.}}
    \step{4}{$(a_m)$ is a sequence of points of $C$ distinct from $x$ that converges to $x$.}
    \begin{proof}
        \pf\ Since $|a_m - x| = 1 / 3^m$ for all $m$.
    \end{proof}
    \step{5}{$x$ is a limit point of $C$.}
    \qed
\end{proof}

\section{Local Compactness}

\begin{definition}[Locally Compact]
    Let $X$ be a topological space and $x \in X$. Then $X$ is \emph{locally
    compact} at $x$ if and only if there exists a compact subspace of $X$
    that includes a neighbourhood of $x$.

    A space is \emph{locally compact} if and only if it is locally compact at
    every point.
\end{definition}

\begin{example}
    The real line is locally compact, because for every real number $x$
    we have $x \in (x-1,x+1) \subseteq [x-1,x+1]$.
\end{example}

\begin{example}
    For all $n \geq 1$, we have $\RR^n$ is locally compact. For any point 
    $x = (x_1, \ldots, x_n)$, we have $x \in (x_1 - 1, x_1 + 1) \times \cdots
    \times (x_n - 1, x_n + 1) \subseteq [x_1 - 1, x_1 + 1] \times \cdots
    \times [x_n - 1, x_n + 1]$.
\end{example}

\begin{proposition}
    The product of two locally compact spaces is locally compact.
\end{proposition}

\begin{proof}
    \pf
    \step{1}{\pflet{$X$ and $Y$ be locally compact.}}
    \step{2}{\pflet{$(x,y) \in X \times Y$}}
    \step{3}{\pick\ compact subspaces $C \subseteq X$ and
    $D \subseteq Y$ that include the open neighbourhoods
    $U$ of $x$ and $V$ of $y$.}
    \step{4}{$(x,y) \in U \times V \subseteq C \times D$}
    \qed
\end{proof}

The following example shows that a countable product of locally compact spaces
is not necessarily locally compact.

\begin{proposition}
    For any infinite set $J$, the space $\RR^J$ is not locally compact.
\end{proposition}

\begin{proof}
    \pf
    \step{1}{\assume{for a contradiction $0 \in U \subseteq C$ where $U$
    is open and $C$ is compact.}}
    \step{2}{\pick\ a basic open set 
    $B = \prod_{\alpha \in J} B_\alpha$ such that $0 \in B \subseteq U$,
    where $B_\alpha = (a_{\alpha_i}, b_{\alpha_i})$ for $\alpha = \alpha_1, \ldots, \alpha_n$
    and $B_\alpha = \RR$ for all other $\alpha$.}
    \step{3}{$\overline{B} = \prod_{\alpha \in J} \overline{B_\alpha}$ is compact.}
    \begin{proof}
        \pf Proposition \ref{proposition:closed_subspace_compact}.
    \end{proof}
    \qedstep
    \begin{proof}
        \pf\ This is a contradiction since $\RR$ is not compact.
    \end{proof}
    \qed
\end{proof}

\begin{example}
    Every linearly ordered set $X$ with the least upper bound property is locally
    compact under the order topology.

    For any point $x$, pick a basic open set $B$ such that $x \in B$. Then
    $x \in B \subseteq \overline{B}$ and $\overline{B}$ is a closed interval,
    hence compact (Corollary \ref{corollary:closed_interval_compact}).

    In particular, $S_\Omega$ and $\overline{S_\Omega}$ are locally compact.
\end{example}

\begin{example}
    The space $\RR_l$ is not locally compact.

    \begin{proof}
        \pf
        \step{1}{\assume{for a contradiction $C \subseteq \RR_l$ is compact and includes an open neighbourhood $U$ of 0.}}
        \step{2}{\pick\ $a > 0$ such that $[0,a) \subseteq U$}
        \step{3}{\pick\ a sequence of points $(a_n)$ such that $0 < a_1 < a_2 < \cdots < a$ such that $a = \sup_n a_n$}
        \step{4}{\pflet{$UU = \{ [0,a_1), [a_1,a_2), [a_2,a_3), \ldots \} \cup \{ \RR_l - [0,a) \}$}}
        \step{5}{$\UU$ is a set of open sets that covers $C$}
        \step{6}{No finite subset covers $C$.}
        \qedstep
        \begin{proof}
            \pf\ This is a contradiction.
        \end{proof}
        \qed
    \end{proof}
\end{example}

\begin{proposition}
    \label{proposition:closed_subspace_locally_compact}
    Any closed subspace of a locally compact space is locally compact.
\end{proposition}

\begin{proof}
    \pf
    \step{1}{\pflet{$X$ be a locally compact space and $Y \subseteq X$ be closed.}}
    \step{2}{\pflet{$y \in Y$.}}
    \step{3}{\pick\ a compact subspace $C$ of $X$ and neighbourhood $U$ of $y$ in $X$ such that 
    $U \subseteq C$}
    \step{4}{$y \in U \cap Y \subseteq C \cap Y$}
    \step{5}{$C \cap Y$ is compact.}
    \begin{proof}
        \pf Proposition \ref{proposition:closed_subspace_compact}.
    \end{proof}
    \qed
\end{proof}

\begin{corollary}
    The Sorgenfrey plane is not locally compact.
\end{corollary}

\begin{proof}
    \pf\ It has a closed subspace homeomorphic to $\RR_l$. \qed
\end{proof}

\begin{proposition}
    \label{proposition:locally_compact_neighbourhood}
    Let $X$ be a Hausdorff space. Let $x \in X$. Then $X$ is locally compact at $x$ if and only if,
    for every neighbourhood $U$ of $x$, there exists a neighbourhood $V$ of $x$ such that
    $\overline{V}$ is compact and $\overline{V} \subseteq U$.
\end{proposition}

\begin{corollary}
    Every open subspace of a locally compact Hausdorff space is locally compact Hausdorff.
\end{corollary}

This example shows that a subspace of a locally compact Hausdorff space is not necessarily locally
compact Hausdorff.

\begin{example}
    The rationals $\QQ$ are not locally compact.

    Assume for a contradiction $C \subseteq \QQ$ is compact and includes $(-\epsilon, \epsilon)
    \cap \QQ$. Pick an irrational $\xi \in (- \epsilon, \epsilon)$. Then $\{ (-\infty, q) \cap
    \QQ \mid q \in \QQ, q < \xi \} \cup \{ (q, +\infty) \cap
    \QQ \mid q \in \QQ, q > \xi \}$ covers $C$ but no finite subcover does.
\end{example}

\begin{proposition}
    Let $\{ X_\alpha \}_{\alpha \in J}$ be a family of nonempty spaces. If
    $\prod_{\alpha \in J} X_\alpha$ is locally compact under the box topology then each $X_\alpha$ is
    locally compact.
\end{proposition}

\begin{proof}
    \pf
    \step{a}{\pflet{$\alpha \in J$}}
    \step{b}{\pflet{$x_\alpha \in X_\alpha$}}
    \step{c}{Extend $x_\alpha$ to a family $(x_\beta)_{\beta \in J} \in \prod_{\beta \in J} X_\beta$}
    \step{d}{\pick\ a compact $C \subseteq \prod_{\beta \in J} X_\beta$ that includes a basic open neighbourhood
    $\prod_{\beta \in J} U_\beta$ of $(x_\beta)$ such that each $U_\beta$ is open in $X_\beta$.}
    \step{e}{$x_\alpha \in U_\alpha \subseteq \pi_\alpha(C)$}
    \step{f}{$\pi_\alpha(C)$ is compact.}
    \begin{proof}
        \pf\ Theorem \ref{theorem:continuous_image_compact}.
    \end{proof}
    \qed
\end{proof}

\begin{proposition}[AC]
    Let $\{ X_\alpha \}_{\alpha \in J}$ be a family of nonempty spaces. Then
    $\prod_{\alpha \in J} X_\alpha$ is locally compact if and only if each $X_\alpha$ is locally compact,
    and $X_\alpha$ is compact for all but finitely many $\alpha \in J$.
\end{proposition}

\begin{proof}
    \pf
    \step{1}{\pflet{$\{ X_\alpha \}_{\alpha \in J}$ be a family of nonempty spaces.}}
    \step{I}{If $\prod_{\alpha \in J} X_\alpha$ is locally compact then each $X_\alpha$ is locally
    compact.}
    \begin{proof}
        \step{2}{\assume{$\prod_{\alpha \in J} X_\alpha$ is locally compact.}}
        \step{3}{For all $\alpha \in J$ we have $X_\alpha$ is locally compact.}
        \begin{proof}
            \step{a}{\pflet{$\alpha \in J$}}
            \step{b}{\pflet{$x_\alpha \in X_\alpha$}}
            \step{c}{Extend $x_\alpha$ to a family $(x_\beta)_{\beta \in J} \in \prod_{\beta \in J} X_\beta$}
            \step{d}{\pick\ a compact $C \subseteq \prod_{\beta \in J} X_\beta$ that includes a basic open neighbourhood
            $\prod_{\beta \in J} U_\beta$ of $(x_\beta)$ such that each $U_\beta$ is open in $X_\beta$.}
            \step{e}{$x_\alpha \in U_\alpha \subseteq \pi_\alpha(C)$}
            \step{f}{$\pi_\alpha(C)$ is compact.}
            \begin{proof}
                \pf\ Theorem \ref{theorem:continuous_image_compact}.
            \end{proof}
        \end{proof}
    \end{proof}
    \step{II}{If $\prod_{\alpha \in J} X_\alpha$ is locally compact then $X_\alpha$ is compact
    for all but finitely many $\alpha \in J$.}
    \begin{proof}
        \step{2}{\assume{$\prod_{\alpha \in J} X_\alpha$ is locally compact.}}
        \step{a}{\pick\ $x_\alpha \in X_\alpha$ for all $\alpha$.}
        \step{b}{\pick\ a compact $C \subseteq \prod_{\alpha \in J} X_\alpha$ that includes a basic open neighbourhood
        $\prod_{\alpha \in J} U_\alpha$ of $(x_\alpha)$ such that each $U_\alpha$ is open in $X_\alpha$,
        and $U_\alpha = X_\alpha$ for all but finitely many $\alpha$.}
        \step{c}{For all but finitely many $\alpha \in J$, we have $X_\alpha = \pi_\alpha(C)$}
        \step{d}{For all but finitely many $\alpha \in J$, we have $X_\alpha$ is compact.}
        \begin{proof}
            \pf\ Theorem \ref{theorem:continuous_image_compact}.
        \end{proof}
    \end{proof}
    \step{III}{If each $X_\alpha$ is locally compact and $X_\alpha$ is compact for all but
    finitely many $\alpha \in J$ then $\prod_{\alpha \in J} X_\alpha$ is locally compact.}
    \begin{proof}
        \step{1}{\assume{$X_\alpha$ is compact for all $\alpha$ except $\alpha_1$, \ldots, $\alpha_n$}}
        \step{2}{\assume{$X_{\alpha_1}$, \ldots, $X_{\alpha_n}$ are locally compact.}}
        \step{3}{\pflet{$(x_\alpha) \in \prod X_\alpha$}}
        \step{4}{For $i = 1, \ldots, n$, \pick\ a compact $C_{\alpha_i} \subseteq X_{\alpha_i}$
        that includes the neighbourhood $U_{\alpha_i}$ of $x_{\alpha_i}$.}
        \step{5}{For $\alpha \neq \alpha_1, \ldots, \alpha_n$, \pflet{$C_\alpha = U_\alpha = X_\alpha$}}
        \step{6}{$\prod_{\alpha \in J} C_\alpha$ is compact.}
        \begin{proof}
            \pf\ Tychonoff's Theorem.
        \end{proof}
        \step{7}{$(x_\alpha) \in \prod U_\alpha \subseteq \prod C_\alpha$}
    \end{proof}
    \qed
\end{proof}

\begin{corollary}
    The space $\RR^\omega$ is not locally compact.
\end{corollary}

\begin{example}
    The space $S_\Omega \times \overline{S_\Omega}$ is locally compact.
\end{example}

The following example shows that the continuous image of a locally compact space is not necessarily
locally compact.

\begin{example}
    Pick an enumeration $\{ q_1, q_2, \ldots \}$ of $\QQ$. Let $X = \bigcup_{n \in \ZZ^+} (n, n+1)$.
    Define $f : X \rightarrow \QQ$ by $f(x) = q_n$ if $x \in (n, n+1)$. Then $f$ is continuous,
    $X$ is locally compact, but $f(X) = \QQ$ is not locally compact.
\end{example}

\begin{proposition}
    \label{proposition:continuous_open_image_locally_compact}
    The image of a locally compact space under a continuous open map is locally compact.
\end{proposition}

\begin{proof}
    \pf
    \step{1}{\pflet{$X$ be locally compact and $f : X \twoheadrightarrow Y$ be a surjective
    continuous open map.}}
    \step{2}{\pflet{$y \in Y$}}
    \step{3}{\pick\ $x \in X$ such that $f(x) = y$}
    \step{4}{\pick\ a compact $C \subseteq X$ that includes a neighbourhood $U$ of $x$}
    \step{5}{$y \in f(U) \subseteq f(C)$ and $f(U)$ is open, $f(C)$ is compact.}
    \qed
\end{proof}

\begin{lemma}
    \label{lemma:product_map_quotient}
    Let $X$, $Y$ and $Z$ be topological spaces and $p : X \rightarrow Y$.
    If $p$ is a quotient map and $Z$ is locally compact Hausdorff,
    then $p \times \id{Z} : X \times Z \rightarrow Y \times Z$ is a quotient
    map.
\end{lemma}

\begin{proof}
    \pf
    \step{1}{\pflet{$X$, $Y$ and $Z$ be topological spaces and $p : X \rightarrow Y$.}}
    \step{2}{\assume{$p$ is a quotient map and $Z$ is locally compact Hausdorff.}}
    \step{2a}{\pflet{$\pi = p \times \id{Z}$}}
    \step{3}{$\pi$ is sujective.}
    \step{4}{$\pi$ is continuous.}
    \step{5}{$\pi$ is strongly continuous.}
    \begin{proof}
        \step{a}{\pflet{$A \subseteq Y \times Z$}}
        \step{b}{\assume{$\inv{\pi}(A)$ is open.}}
        \step{c}{\pflet{$(y,z) \in A$}}
        \step{d}{\pick\ $x \in X$ such that $p(x) = y$}
        \step{e}{\pick\ open sets $U_1$ in $X$ and $V$ in $Z$ such that
        $x \in U_1$, $z \in V$, $\overline{V}$ is compact, and $U_1 \times
        \overline{V} \subseteq \inv{\pi}(A)$}
        \begin{proof}
            \step{i}{\pick\ open sets $U_1$ in $X$ and $V'$ in $Z$ such that
            $x \in U_1$, $z \in V'$ and $U' \times V' \subseteq \inv{\pi}(A)$}
            \step{ii}{\pick\ $V$ open in $Z$ such that $z \in V$, $\overline{V}$
            is compact and $\overline{V} \subseteq V'$}
            \begin{proof}
                \pf\ Proposition \ref{proposition:locally_compact_neighbourhood}.
            \end{proof}
        \end{proof}
        \step{f}{\pflet{$U = \bigcup \{ U' \text{ open in } X \mid 
        U' \times \overline{V} \subseteq \inv{\pi}(A) \}$}}
        \step{h}{$U$ is saturated}
        \begin{proof}
            \step{i}{\pflet{$a \in U$, $b \in X$ with $p(a) = p(b)$}}
            \step{ii}{$\{b\} \times \overline{V} \subseteq \inv{\pi}(A)$}
            \step{iii}{\pick\ $U'$ open in $X$ such that $b \in U'$
            and $U' \times \overline{V} \subseteq \inv{\pi}(A)$}
            \begin{proof}
                \pf\ By the Tube Lemma.
            \end{proof}
            \step{iv}{$b \in U' \subseteq U$}
        \end{proof}
        \step{i}{$\pi(U \times V)$ is open}
        \begin{proof}
            \pf\ Since $\pi(U \times V) = p(U) \times V$.
        \end{proof}
        \step{j}{$(y,z) \in \pi(U \times V)$}
        \step{k}{$\pi(U \times V) \subseteq A$}
    \end{proof}
    \qed
\end{proof}

\begin{theorem}
    Let $A$, $B$, $C$ and $D$ be topological spaces with $B$ and $C$ locally
    compact Hausdorff. Let $p : A \twoheadrightarrow B$ and $q : C
    \twoheadrightarrow D$ be quotient maps. Then $p \times q : A \times C
    \twoheadrightarrow B \times D$.
\end{theorem}

\begin{proof}
    \pf\ By Lemma \ref{lemma:product_map_quotient} since $p \times q =
    (\id{B} \times q) \circ (p \times \id{C})$. \qed
\end{proof}

\begin{proposition}
    \label{proposition:perfect_image_locally_compact}
    The image of a locally compact space under a perfect map is locally compact.
\end{proposition}

\begin{proof}
    \pf
    \step{1}{Let $X$ be a locally compact space and $p : X \twoheadrightarrow Y$
    be a perfect map.}
    \step{2}{\pflet{$y \in Y$}}
    \step{3}{For all $x \in \inv{p}(y)$, there exists a neighbourhood of $x$ that is included
    in a compact subspace of $X$.}
    \step{4}{$\{ U \text{ open in } X \mid \exists C \subseteq X. C \text{ is compact and }
    U \subseteq C \}$ covers $\inv{p}(y)$}
    \step{5}{\pick\ a finite subcover $\{ U_1, \ldots, U_n \}$}
    \step{6}{For $1 \leq i \leq n$, \pick\ $C_i \subseteq X$ such that
    $C_i$ is compact}
    \step{6a}{\pflet{$C = C_1 \cup \cdots \cup C_n$}}
    \step{6b}{$C$ is compact}
    \begin{proof}
        \pf\ Proposition \ref{proposition:finite_union_compact}.
    \end{proof}
    \step{7}{\pick\ an open neighbourhood $V$ of $y$ such
    that $\inv{p}(V) \subseteq U_1 \cup \cdots \cup U_n$}
    \begin{proof}
        \pf\ Lemma \ref{lemma:closed_map_neighbourhoods}.
    \end{proof}
    \step{8}{$V \subseteq p(C)$}
    \begin{proof}
        \step{a}{\pflet{$y \in V$}}
        \step{b}{\pick\ $x \in X$ such that $p(x) = y$}
        \step{c}{$x \in C$}
        \step{d}{$y \in p(C)$}
    \end{proof}
    \step{9}{$p(C)$ is compact.}
    \begin{proof}
        \pf\ Theorem \ref{theorem:continuous_image_compact}.
    \end{proof}
    \qed
\end{proof}

\begin{proposition}
    The space $\RR^\omega$ under the box topology is not locally compact.
\end{proposition}

\begin{proof}
    \pf
    \step{1}{\assume{for a contradiction $\RR^\omega$ under the box topology is locally compact.}}
    \step{2}{\pick\ an open neighbourhood $U$ of 0 such that $\overline{U}$ is compact.}
    \step{3}{$\overline{U}$ is compact under the product topology.}
    \begin{proof}
        \pf\ Since the product topology is coarser.
    \end{proof}
    \qedstep
    \begin{proof}
        \pf\ This contradicts Corollary \ref{corollary:compact_hausdorff_incomparable}.
    \end{proof}
    \qed
\end{proof}

\begin{proposition}
    The space $\RR_K$ is not locally compact.
\end{proposition}

\begin{proof}
    \pf
    \step{1}{\assume{for a contradiction there is a compact
    subspace $C$ that includes the open neighbourhood $U$ of 0.}}
    \step{2}{\pick\ $\epsilon$ with $0 < \epsilon < 1$ such that $(-\epsilon,\epsilon)-K \subseteq U$}
    \step{3}{$\overline{(-\epsilon,\epsilon)-K} = [-\epsilon,\epsilon]$ is compact under the $K$-topology.}
    \step{4}{For all $n \geq 2$, \pflet{$U_n = (1/(n+1),1/(n-1))$}}
    \step{5}{$\{(-1.1) - K, (1/2, +\infty) \} \cup \{ U_n \mid n \geq 2 \}$ covers $[-\epsilon,\epsilon]$.}
    \step{6}{There is no finite subcover.}
    \qed
\end{proof}

\begin{proposition}
    Every locally compact space is compactly generated.
\end{proposition}

\begin{proof}
    \pf
    \step{1}{\pflet{$X$ be locally compact.}}
    \step{2}{\pflet{$A \subseteq X$}}
    \step{3}{\assume{for every compact $C \subseteq X$ we have $A \cap C$ is open in $C$.} \prove{$A$ is open.}}
    \step{4}{\pflet{$x \in A$}}
    \step{5}{\pick\ $C \subseteq X$ compact and $U \subseteq C$ an open neighbourhood of $x$.}
    \step{6}{$A \cap C$ is open in $C$.}
    \step{7}{$A \cap U$ is open in $U$.}
    \step{8}{$A \cap U$ is open in $X$.}
    \step{9}{$x \in A \cap U \subseteq A$}
    \qed
\end{proof}

\section{Compactifications}

\begin{definition}[Compactification]
    Let $X$ be a topological space. A \emph{compactification} of $X$ consists of a compact Hausdorff
    space $Y$ and an imbedding $i : X \rightarrow Y$ such that $Y = \overline{i(X)}$.
\end{definition}

\begin{definition}[One-Point Compactification]
    Let $X$ be a topological space. A \emph{one-point compactification} of $X$ is a compactification
    $i : X \rightarrow Y$ such that $Y - i(X)$ consists of a single point.
\end{definition}

\begin{theorem}
    Let $X$ be a topological space. Then $X$ is locally compact Hausdorff if and only if there exists
    a one-point compactification $i : X \rightarrow Y$. In this case,
    $Y$ is unique up to unique homeomorphism that commutes with $i$.
\end{theorem}

\begin{proof}
    \pf
    \step{1}{For any compact Hausdorff space $Y$ and point $a \in Y$, the space $Y - \{a\}$ is locally compact Hausdorff.}
    \begin{proof}
        \step{a}{\pflet{$Y$ be a compact Hausdorff space.}}
        \step{b}{\pflet{$a \in Y$}}
        \step{c}{$Y - \{a\}$ is closed.}
        \step{d}{$Y - \{a\}$ is locally compact.}
        \begin{proof}
            \pf\ Proposition \ref{proposition:closed_subspace_locally_compact}.
        \end{proof}
        \step{e}{$Y - \{a\}$ is Hausdorff.}
        \begin{proof}
            \pf\ Theorem \ref{theorem:subspace_Haudorff}.
        \end{proof}
    \end{proof}
    \step{2}{For any locally compact Hausdorff space $X$, there exists a compact Hausdorff space $Y$ and imbedding
    $i : X \rightarrow Y$ such that $Y - i(X)$ is a single point.}
    \begin{proof}
        \step{a}{\pflet{$X$ be a locally compact Hausdorff space.}}
        \step{b}{\pflet{$Y = X \cup \{ \infty \}$}}
        \step{c}{Define a topology on $Y$ by: $U \subseteq Y$ is open if and only if $U$ is an open
        set in $X$ or $U = Y - C$ where $C$ is a compact subspace of $X$.}
        \begin{proof}
            \step{i}{$Y$ is open.}
            \begin{proof}
                \pf\ Since $Y = Y - \emptyset$ and $\emptyset$ is a compact subspace of $X$.
            \end{proof}
            \step{ii}{For any set of open sets $\UU$ we have $\bigcup \UU$ is open.}
            \begin{proof}
                \pf\ We have $\bigcup \UU = Y - (\bigcap \{ C \subseteq X \mid C \text{ is compact},
                    Y - C \in \UU \} - \bigcup \{ U \subseteq X \mid U \text{ is open in } X, U \in \UU \})$,
                    where we take the empty intersection to be $Y$.
            \end{proof}
            \step{iii}{For any open sets $U$ and $V$ we have $U \cap V$ is open.}
            \begin{proof}
                \step{one}{\pflet{$U$ and $V$ be open sets.}}
                \step{two}{\case{$U$ and $V$ are open sets in $X$.}}
                \begin{proof}
                    \pf\ In this case $U \cap V$ is open in $X$.
                \end{proof}
                \step{three}{\case{$C_1$ and $C_2$ are compact subspaces of $X$ and $U = X - C_1$,
                $V = X - C_2$}}
                \begin{proof}
                    \pf\ In this case $C_1 \cup C_2$ is compact and $U \cap V = X - (C_1 \cup C_2)$.
                \end{proof}
                \step{four}{\case{$U$ is open in $X$, $C$ is a compact subspace of $X$ and
                $V = X - C$}}
                \begin{proof}
                    \pf\ In this case $U \cap V = U - C$ which is open since $C$ is closed.
                \end{proof}
            \end{proof}
        \end{proof}
        \step{d}{$Y$ is compact.}
        \begin{proof}
            \step{i}{\pflet{$\AA$ be an open cover of $Y$.}}
            \step{ii}{\pick\ $C$ compact in $X$ such that $Y - C \in \AA$}
            \begin{proof}
                \pf\ There must be at least one such member of $\AA$ since $\infty \in \bigcup \AA$.
            \end{proof}
            \step{iii}{$\{ U \cap X \mid U \in \AA - \{ Y - C \} \}$ is a set of open sets in $X$
            that covers $C$.}
            \step{iv}{\pick\ a finite subcover $\{ U_1 \cap X, \ldots, U_n \cap X \}$}
            \step{v}{$\{ U_1 \cap X, \ldots, U_n \cap X, Y - C \}$ covers $Y$.}
        \end{proof}
        \step{e}{$Y$ is Hausdorff.}
        \begin{proof}
            \step{i}{\pflet{$x,y \in Y$ with $x \neq y$}}
            \step{ii}{\case{$x,y \in X$}}
            \begin{proof}
                \pf\ There are disjoint open sets $U$, $V$ in $X$ such that $x \in U$, $y \in V$.
            \end{proof}
            \step{iii}{\case{$x \in X$, $y = \infty$}}
            \begin{proof}
                \step{one}{\pick\ a compact $C$ that includes a neighbourhood $U$ of $x$}
                \begin{proof}
                    \pf\ Since $X$ is locally compact.
                \end{proof}
                \step{two}{$U$ and $Y - C$ are disjoint open sets in $Y$ with $x \in U$
                and $\infty \in Y - C$}
            \end{proof}
        \end{proof}
        \step{f}{Let $i : X \rightarrow Y$ be the inclusion.}
        \step{g}{$i$ is an imbedding.}
        \begin{proof}
            \step{i}{$i$ is continuous}
            \step{ii}{$i$ is an open map.}
        \end{proof}
        \step{h}{$Y - i(X) = \{ \infty \}$}
    \end{proof}
    \step{3}{If $X$ is locally compact Hausdorff, $Y$ and $Y'$ are compact Hausdorff, and $i : X \rightarrow Y$,
    $i' :  \rightarrow Y'$ are imbeddings such that $Y - i(X)$ and $Y' - i'(X)$ each have just one point,
    then there exists a unique homeomorphism $\theta : Y \cong Y'$ such that $\theta \circ i = i'$.}
    \begin{proof}
        \step{a}{\pflet{$Y - i(X) = \{a\}$ and $Y' - i'(X) = \{b\}$}}
        \step{b}{\pflet{$\theta : Y \rightarrow Y'$ be the function with $\theta(a) = b$
        and $\theta(i(x)) = i'(x)$}}
        \step{c}{$\theta$ is a bijection}
        \step{d}{$\theta$ is continuous.}
        \begin{proof}
            \step{one}{\pflet{$U \subseteq Y'$ be open.} \prove{$\inv{\theta}(U)$ is open.}}
            \step{two}{\case{$b \in U$}}
            \begin{proof}
                \step{x}{$Y' - U$ is compact}
                \step{xx}{$i(\inv{i'}(Y'-U))$ is compact.}
                \step{xxx}{$i(\inv{i'}(Y'-U))$ is closed.}
                \step{xxxx}{$\inv{\theta}(U) = X - i(\inv{i'}(Y'-U))$}
            \end{proof}
            \step{three}{\case{$b \notin U$}}
            \begin{proof}
                \pf\ $U = i'(V)$ for some $V$ open in $X$ and $\inv{\theta}(U) = i(V)$.
            \end{proof}
        \end{proof}
        \step{e}{$\theta$ is an open map.}
        \begin{proof}
            \pf\ Similar.
        \end{proof}
        \step{f}{$\theta$ is unique.}
    \end{proof}
    \qed
\end{proof}

\begin{example}
    $S^1$ is the one-point compactification of $\RR$.
\end{example}

\begin{example}
    $S^2$ is the one-point compactification of $\RR^2$.
\end{example}

\begin{definition}[Riemann Sphere]
    The \emph{Riemann sphere} or \emph{extended complex plane} is $\CC \cup \{ \infty \}$
    topologized as the one-point compactification of $\CC$. It is homeomorphic to $S^2$.
\end{definition}

\begin{example}
    The one-point compactification of $\ZZ^+$ is $\{0\} \cup \{ 1/n \mid n \in \ZZ^+ \}$.
\end{example}

\begin{proposition}
    Every locally compact Hausdorff space is completely regular.
\end{proposition}

\begin{proof}
    \pf\ It is a subspace of its one-point compactification, which is completely regular since
    it is compact Hausdorff. \qed
\end{proof}

\subsection{Equivalent Compactifications}

\begin{definition}[Equivalent Compactifications]
    Let $i : X \rightarrow Y$ and $j : X \rightarrow Z$ be compactifications of $X$. Then these compactifications
    are \emph{equivalent} if and only if there exists a homeomorphism $\phi : Y \cong Z$ such that $j = \phi \circ i$.
\end{definition}

\begin{lemma}
    Let $X$ be a topological space. Let $Z$ be a compact Hausdorff space. Let $h : X \rightarrow Z$ be an embedding.
    Then there exists a compactification $i : X \hookrightarrow Y$ of $X$ unique up to equivalence such that there
    exists an embedding $H : Y \hookrightarrow Z$ such that $h = H \circ i$.
\end{lemma}

\begin{proof}
    \pf
    \step{1}{\pflet{$Y = \overline{h(X)}$ and $i : X \hookrightarrow Y$ be the corestriction of $h$.}}
    \step{2}{$i$ is a compactification of $X$.}
    \step{3}{Let $H : Y \rightarrow Z$ be the inclusion.}
    \step{4}{$H$ is an embedding.}
    \step{5}{$h = H \circ i$}
    \step{6}{\pflet{$i' : X \rightarrow Y'$ be any compactification of $X$ and $H' : Y' \hookrightarrow Z$ be an embedding
    such that $h = H' \circ i'$}}
    \step{7}{Define $\phi : Y \rightarrow Y'$ by: $\phi(y)$ is the unique element in $Y'$ such that $H'(y) = H(y)$}
    \begin{proof}
        \pf\ This exists since $H(Y) = H'(Y') = \overline{h(X)}$.
    \end{proof}
    \step{8}{$\phi$ is a homeomorphism.}
    \step{9}{$i' = \phi \circ i$}
    \step{10}{$\phi$ is unique.}
    \qed
\end{proof}

\begin{definition}[Induced Compactification]
    Let $X$ be a topological space. Let $Z$ be a compact Hausdorff space. Let $h : X \rightarrowtail Z$ be an embedding.
    Then the compactification of $X$ \emph{induced} by $h$ is the compactification $i : X \rightarrowtail Y$ such that
    there exists an embedding $H : Y \rightarrowtail Z$ such that $h = H \circ i$.
\end{definition}

\begin{theorem}
    Let $X$ be a completely regular space. Then there exists a compactification $i : X \rightarrow Y$ such that, for every
    bounded continuous function $f : X \rightarrow \RR$, there exists a unique continuous function $g : Y \rightarrow \RR$
    such that $f = g \circ i$.
\end{theorem}

\begin{proof}
    \pf
    \step{1}{\pflet{$J$ be the set of all bounded continuous functions $X \rightarrow \RR$}}
    \step{2}{For $\alpha \in J$, \pflet{$I_\alpha = [\inf \alpha(X), \sup \alpha(X)]$}}
    \step{3}{Define $h : X \rightarrow \prod_{\alpha \in J} I_\alpha$ by $h(x) = (\alpha(x))_{\alpha \in J}$}
    \step{4}{$\prod_{\alpha \in J} I_\alpha$ is compact Hausdorff.}
    \begin{proof}
        \pf\ Tychonoff Theorem.
    \end{proof}
    \step{6}{$h$ is an embedding.}
    \begin{proof}
        \step{5}{$J$ separates points from closed sets in $X$.}
        \begin{proof}
            \pf\ Since $X$ is completely regular.
        \end{proof}
        \qedstep
        \begin{proof}
            \pf\ By the Embedding Theorem.
        \end{proof}
    \end{proof}
    \step{7}{\pflet{$i : X \hookrightarrow Y$ be the compactification induced by $h$ with embedding $H : Y \rightarrowtail \prod_{\alpha \in J} I_\alpha$
    such that $h = H \circ i$}}
    \step{8}{\pflet{$f : X \rightarrow \RR$ be any bounded continuous function.}}
    \step{9}{\pflet{$g : Y \rightarrow \RR$ be $\pi_f \circ H$}}
    \step{10}{$f = g \circ i$}
    \step{11}{If $g' : Y \rightarrow \RR$ and $f = g' \circ i$ then $g' = g$}
    \begin{proof}
        \pf\ Proposition \ref{proposition:agree_on_closure}.
    \end{proof}
    \qed
\end{proof}

\begin{theorem}
    Let $X$ be a completely regular space. Let $i : X \rightarrowtail Y$ be a compactification
    such that, for every
    bounded continuous function $f : X \rightarrow \RR$, there exists a unique continuous function $g : Y \rightarrow \RR$
    such that $f = g \circ i$.
    Let $C$ be a compact Hausdorff space. Let $f : X \rightarrow C$ be continuous. Then there exists a unique continuous
    function $g : Y \rightarrow C$ such that $f = g \circ i$.
\end{theorem}

\begin{proof}
    \pf
    \step{1}{\pick\ a set $J$ and an embedding $j : C \rightarrowtail [0,1]^J$}
    \begin{proof}
        \pf\ Theorem \ref{theorem:completely_regular_embedded_in_IJ}.
    \end{proof}
    \step{2}{For $\alpha \in J$, \pflet{$g_\alpha : Y \rightarrow \RR$ be the unique 
    continuous function such that $\pi_\alpha \circ j \circ f = g_\alpha \circ i$.}}
    \step{3}{\pflet{$h = \langle g_\alpha \mid \alpha \in J \rangle : Y \rightarrow \RR^J$}}
    \step{4}{$h \circ i = j \circ f$}
    \step{4}{$h(Y) \subseteq j(C)$}
    \begin{proof}
        \pf
        \begin{align*}
            h(Y) & = h(\overline{i(X)}) \\
            & \subseteq \overline{h(i(X))} \\
            & = \overline{j(f(X))} \\
            & \subseteq \overline{j(C)} \\
            & = j(C) & (\text{$C$ is closed, $j$ is an embedding})
        \end{align*}
    \end{proof}
    \qed
\end{proof}

\begin{corollary}
    Let $X$ be a completely regular space. Then the compactification $i : X \rightarrowtail Y$
    such that, for every bounded continuous $f : X \rightarrow \RR$, there exists a unique continuous
    $g : Y \rightarrow \RR$ such that $f = g \circ i$ is unique up to equivalence.
\end{corollary}

\begin{definition}[Stone-\u{C}ech Compactification]
    Let $X$ be a completely regular space. The \emph{Stone-\u{C}ech compactification} of $X$ is the compactification
    $i : X \rightarrowtail \beta(X)$ such that, for any compact Hausdorff space $C$ and continuous function
    $f : X \rightarrow C$, there exists a unique continuous function $g : \beta(X) \rightarrow C$ such that
    $f = g \circ i$.
\end{definition}

\begin{proposition}
    Let $X$ be a completely regular space. Every compactification of $X$ is a quotient of $i : X \rightarrow \beta(X)$.
    That is, if $j : X \hookrightarrow Y$ is a compactification of $X$, then the continuous map
    $q : \beta(X) \rightarrow Y$ such that $j = q \circ i$ is a quotient map.
\end{proposition}

\begin{proof}
    \pf
    \step{1}{$q$ is surjective.}
    \begin{proof}
        \step{a}{$q(\beta(X))$ is closed}
        \begin{proof}
            \pf\ Because it is compact.
        \end{proof}
        \step{b}{$Y \subseteq q(\beta(X))$}
        \begin{proof}
            \pf\ Since $Y = \overline{j(X)}$
        \end{proof}
        \step{c}{$q(\beta(X)) = Y$}
    \end{proof}
    \step{2}{$q$ is a closed map.}
    \begin{proof}
        \pf\ Proposition \ref{proposition:closed_map_compact_Hausdorff},
    \end{proof}
    \qedstep
    \begin{proof}
        \pf\ Corollary \ref{corollary:closed_map_quotient}.
    \end{proof}
    \qed
\end{proof}

\begin{proposition}
    The one-point compactification and the Stone-\u{C}ech compactification of $S_\Omega$ are equivalent.
\end{proposition}

\begin{proof}
    \pf
    \step{1}{\pflet{$i : S_\Omega \rightarrow \overline{S_\Omega}$ be the inclusion.}}
    \step{2}{For every continuous function $f : S_\Omega \rightarrow \RR$, there exists a unique continuous $g : \overline{S_\Omega}
    \rightarrow \RR$ such that $f = g \circ i$.}
    \begin{proof}
        \step{a}{\pflet{$f : S_\Omega \rightarrow \RR$ be continuous.}}
        \step{b}{\pflet{$\alpha$ be least such that, for all $\beta > \alpha$, we have $f(\beta) = f(\alpha)$}}
        \step{c}{\pflet{$l = f(\alpha)$}}
        \step{d}{$g : \overline{S_\Omega} \rightarrow \RR$ is continuous and satisfies $f = g \circ i$ iff $g(\gamma) = f(\gamma)$
        for all $\gamma < \Omega$, and $g(\Omega) = l$.}
    \end{proof}
    \step{3}{$i$ is the Stone-\u{C}ech compactification of $S_\Omega$.}
    \qed
\end{proof}

\begin{corollary}
    Every compactification of $S_\Omega$ is equivalent to $\overline{S_\Omega}$.
\end{corollary}

\begin{proof}
    \pf
    \step{1}{\pflet{$i : S_\Omega \rightarrowtail C$ be any compactification.}}
    \step{2}{\pflet{$j : S_\Omega \rightarrowtail \overline{S_\Omega}$ be the inclusion.}}
    \step{3}{\pflet{$q : \overline{S_\Omega} \twoheadrightarrow C$ be the quotient map such that $q \circ j = i$}}
    \step{4}{$q$ is injective.}
    \begin{proof}
        \pf
        \step{a}{For $x,y \in \S_\Omega$, if $q(j(x)) = q(j(y))$ then $x = y$}
        \step{b}{For $x \in S_\Omega$ we have $q(j(x)) \neq q(\Omega)$}
        \begin{proof}
            \pf\ Otherwise $C = i(S_\Omega)$ which is a contradiction since $i(S_\Omega) \cong S_\Omega$ is not compact.
        \end{proof}
    \end{proof}
    \step{5}{$q$ is a homeomorphism.}
    \qed
\end{proof}

\begin{proposition}
    Let $X$ be a completely regular space. Then $X$ is connected if and only if $\beta(X)$ is connected.
\end{proposition}

\begin{proof}
    \pf
    \step{1}{If $X$ is connected then $\beta(X)$ is connected.}
    \begin{proof}
        \pf\ Theorem \ref{theorem:connected_closure}.
    \end{proof}
    \step{2}{If $\beta(X)$ is connected then $X$ is connected.}
    \begin{proof}
        \step{a}{\assume{$\beta(X)$ is connected.}}
        \step{b}{\assume{for a contradiction $A$ and $B$ form a separation of $X$.}}
        \step{c}{Define $f : X \rightarrow \RR$ by $f(x) = 0$ for $x \in A$, and $f(x) = 1$ for $x \in B$}
        \step{d}{$f$ is continuous.}
        \step{e}{\pflet{$g : \beta(X) \rightarrow \RR$ be the unique extension of $f$ to $\beta(X)$}}
        \step{f}{$g(\beta(X)) = \{0,1\}$}
        \begin{proof}
            \pf
            \begin{align*}
                g(\beta(X)) & = g(\overline{X}) \\
                & \subseteq \overline{g(X)} \\
                & = \overline{\{0,1\}} \\
                & = \{0,1\}
            \end{align*}
        \end{proof}
        \qedstep
        \begin{proof}
            \pf\ This contradicts Theorem \ref{theorem:connected_continuous_image}.
        \end{proof}
    \end{proof}
    \qed
\end{proof}

\begin{proposition}
    \label{proposition:discrete_space_compactification}
    Let $X$ be a discrete space. Let $A \subseteq X$. Then the closures of $A$ and $X - A$ in $\beta(X)$ are disjoint.
\end{proposition}

\begin{proof}
    \pf
    \step{1}{Define $f : X \rightarrow \RR$ by $f(x) = 0$ for $x \in A$ and $f(x) = 1$ for $x \in X - A$}
    \step{2}{$f$ is continuous.}
    \step{3}{\pflet{$g : \beta(X) \rightarrow \RR$ be the continuous extension of $f$.}}
    \step{4}{$g(\overline{A}) = \{0\}$}
    \step{5}{$g(\overline{X - A}) = \{1\}$}
    \qed
\end{proof}

\begin{proposition}
    \label{proposition:discrete_open_closure_compactification}
    Let $X$ be a discrete space. Let $U$ be an open set in $\beta(X)$. Then $\overline{U}$ is open in $\beta(X)$.
\end{proposition}

\begin{proof}
    \pf
    \step{1}{$\overline{U} = \overline{U \cap X}$}
    \begin{proof}
        \step{a}{$\overline{U} \subseteq \overline{U \cap X}$}
        \begin{proof}
            \step{i}{\pflet{$x \in \overline{U}$}}
            \step{ii}{\pflet{$V$ be any neighbourhood of $x$.}}
            \step{iii}{$V \cap U \neq \emptyset$}
            \step{iv}{$V \cap U \cap X \neq \emptyset$}
            \begin{proof}
                \pf\ Since $X$ is dense in $\beta(X)$.
            \end{proof}
        \end{proof}
        \step{b}{$\overline{U \cap X} \subseteq \overline{U}$}
        \begin{proof}
            \pf\ Proposition \ref{proposition:closure_monotone}.
        \end{proof}
    \end{proof}
    \step{2}{$\overline{U}$ and $\overline{X - U}$ are disjoint.}
    \begin{proof}
        \pf\ Proposition \ref{proposition:discrete_space_compactification} gives us $\overline{U \cap X}$
        and $\overline{X - U}$ are disjoint.
    \end{proof}
    \step{3}{$\overline{U} = \beta(X) - \overline{X - U}$}
    \begin{proof}
        \pf\ Since
        \begin{align*}
            \overline{U} \cup \overline{X - U} & = \overline{X \cup U} & (\text{Proposition \ref{proposition:closure_union}}) \\
            & = \beta(X)
        \end{align*}
    \end{proof}
    \qed
\end{proof}

\begin{proposition}
    The Stone-\u{C}ech compactification of a discrete space is totally disconnected.
\end{proposition}

\begin{proof}
    \pf
    \step{1}{\pflet{$X$ be a discrete space.}}
    \step{2}{\pflet{$Y \subseteq X$}}
    \step{3}{\pflet{$x, y \in Y$}}
    \step{4}{\assume{$x \neq y$} \prove{$Y$ is not connected.}}
    \step{5}{\pick\ $U$ open in $X$ such that $x \in U$ and $y \notin U$}
    \step{6}{$Y \cap \overline{U}$ and $Y - \overline{U}$ form a separation of $Y$.}
    \begin{proof}
        \step{a}{$Y \cap \overline{U}$ is open in $Y$}
        \begin{proof}
            \pf\ Proposition \ref{proposition:discrete_open_closure_compactification}.
        \end{proof}
        \step{b}{$Y - \overline{U}$ is open in $Y$}
        \step{c}{$(Y \cap \overline{U}) \cap (Y - \overline{U}) = \emptyset$}
        \step{d}{$(Y \cap \overline{U}) \cup (Y - \overline{U}) = Y$}
    \end{proof}
    \qed
\end{proof}

\begin{proposition}
    \label{proposition:stone_cech_closed_disjoint}
    Let $X$ be a normal space. Let $A$ and $B$ be disjoint closed sets in $X$.
    Then the closures of $A$ and $B$ in $\beta(X)$ are disjoint.
\end{proposition}

\begin{proof}
    \pf
    \step{1}{\pick\ a continuous function $f : X \rightarrow [0,1]$ with $f(A) = \{0\}$
    and $f(B) = \{1\}$}
    \step{2}{\pflet{$g : \beta(X) \rightarrow [0,1]$ be the extension of $f$.}}
    \step{3}{$g(\overline{A}) = \{0\}$ and $g(\overline{B}) = \{1\}$}
    \qed
\end{proof}

\begin{proposition}
    \label{proposition:stone_cech_sequence}
    Let $X$ be a normal space. Let $y \in \beta(X)$.
    If $y$ is the limit of a sequence of points of $X$ then $y \in X$.
\end{proposition}

\begin{proof}
    \pf
    \step{1}{\pick\ a sequence $(x_n)$ in $X$ that converges to $y$.}
    \step{2}{\assume{w.l.o.g. the elements of $(x_n)$ are all distinct.}}
    \step{3}{\pflet{$A = \{x_{2n} \mid n \in \ZZ^+\}$ and $B = \{x_{2n+1} \mid n \in \ZZ^+\}$}}
    \step{4}{$y \in \overline{A} \cap \overline{B}$}
    \step{5}{$A$ and $B$ are closed in $X$}
    \begin{proof}
        \step{a}{$A$ is closed in $X$.}
        \begin{proof}
            \step{i}{\pflet{$z \in X - A$}}
            \step{ii}{\pick\ disjoint open sets $U$ and $V$ in $\beta(X)$ such that $z \in U$ and $y \in V$}
            \step{iii}{\pick\ $N$ such that for all $n \geq N$ we have $x_n \in V$.}
            \step{iv}{$z \in U - \{ x_1, x_2, \ldots, x_{N-1} \} \subseteq X - A$}
        \end{proof}
        \step{b}{$B$ is closed in $X$.}
        \begin{proof}
            \pf\ Similar.
        \end{proof}
    \end{proof}
    \qedstep
    \begin{proof}
        \pf\ This contradicts Proposition \ref{proposition:stone_cech_closed_disjoint}.
    \end{proof}
    \qed
\end{proof}

\begin{definition}
    Extend $\beta$ to a functor $\mathbf{Top} \rightarrow \mathbf{CompHaus}$ by defining,
    for $f : X \rightarrow Y$ a continuous function, $\beta(f) : \beta(X) \rightarrow \beta(Y)$
    to be the unique continuous extension of $f$.
\end{definition}

\section{$G_\delta$ Sets}

\begin{definition}[$G_\delta$ Set]
    Let $X$ be a topological space and $A \subseteq X$. Then $A$ is \emph{$G_\delta$}
    if and only if it is the intersection of a countable set of open sets.
\end{definition}

\begin{proposition}
    In a first countable $T_1$ space, every singleton is $G_\delta$.
\end{proposition}

\begin{proof}
    \pf
    \step{1}{\pflet{$X$ be a first countable $T_1$ space.}}
    \step{2}{\pflet{$a \in X$}}
    \step{3}{\pick\ a countable local basis $\BB$ at $a$.}
    \step{4}{$\bigcap \BB = \{ a \}$}
    \begin{proof}
        \step{a}{\pflet{$b \in X - \{a\}$} \prove{$b \notin \bigcap \BB$}}
        \step{b}{\pick\ $B \in \BB$ with $a \in B \subseteq X - \{b\}$}
        \step{c}{$b \notin B$}
    \end{proof}
    \qed
\end{proof}

\begin{example}
    In the space $\RR^\omega$ under the box topology, every singleton is $G_\delta$.
    However, $\RR^\omega$ is not first countable.
\end{example}

\section{Separated by a Continuous Function}

\begin{definition}[Separated by a Continuous Function]
    Let $X$ be a topolocigal space. Let $A, B \subseteq X$. Then $A$ and $B$
    can be \emph{separated by a continuous function} if and only if there exists
    a continuous $f : X \rightarrow [0,1]$ such that $f(A) = \{0\}$ and
    $f(B) = \{1\}$.
\end{definition}

\section{Separate Points from Closed Sets}

\begin{definition}[Separate Points from Closed Sets]
    Let $X$ be a topological space. Let $\mathcal{F}$ be a set of continuous
    functions $X \rightarrow \RR$. Then $\mathcal{F}$ \emph{separates points
    from closed sets} if and only if, for every point $x_0 \in X$
    and open neighbourhood $U$ of $x_0$, there exists $f \in \mathcal{F}$
    such that $f(x_0) > 0$ and $f$ vanishes outside $U$.
\end{definition}

\begin{theorem}[Imbedding Theorem]
    Let $X$ be a $T_1$ space. Let $\{ f_\alpha \}_{\alpha \in J}$ be a family of
    continuous functions $X \rightarrow \RR$ that separates points from
    closed sets. Define $F : X
    \rightarrow \RR^J$ by
    \[ F(x) = (f_\alpha(x))_{\alpha \in J} \enspace . \]
    Then $F$ is an imbedding.

    If, in addition, $f_\alpha(x) \in [0,1]$ for all $\alpha \in J$ and $x \in X$,
    then $F$ is an imbedding of $X$ into $[0,1]^J$.
\end{theorem}

\begin{proof}
    \step{a}{$F$ is continuous.}
    \begin{proof}
        \pf\ Theorem \ref{theorem:product_continuous}.
    \end{proof}
    \step{b}{$F$ is injective.}
    \begin{proof}
        \step{i}{\pflet{$x, y \in X$ with $x \neq y$}}
        \step{ii}{\pick\ $\alpha \in J$ such that $f_\alpha(x) > 0$ and
        $f_\alpha$ vanishes outside $X - \{y\}$}
        \begin{proof}
            \pf\ By \stepref{2}.
        \end{proof}
        \step{iii}{$F(x)_\alpha > 0$ and $F(y)_\alpha = 0$}
    \end{proof}
    \step{c}{$F : X \rightarrow F(X)$ is an open map.}
    \begin{proof}
        \step{i}{\pflet{$U$ be open in $X$.}}
        \step{ii}{\pflet{$z_0 \in U$} \prove{There exists $W$ open in $F(X)$
        such that $z_0 \in W \subseteq F(U)$}}
        \step{iii}{\pflet{$x_0 = \inv{F}(z_0)$}}
        \step{iv}{\pick\ $\alpha \in J$ such that $f_\alpha(x_0) > 0$ and $f_\alpha$
        vanishes outside $U$.}
        \step{v}{\pflet{$V = \inv{\pi_\alpha}((0, + \infty))$}}
        \step{vi}{\pflet{$W = V \cap F(X)$}}
        \step{vii}{$z_0 \in W$}
        \step{viii}{$W \subseteq F(U)$}
        \begin{proof}
            \step{one}{\pflet{$z \in W$}}
            \step{two}{\pflet{$x = \inv{F}(z)$}}
            \step{three}{$f_\alpha(x) > 0$}
            \step{four}{$x \in U$}
            \step{five}{$z \in F(U)$}
        \end{proof}
    \end{proof}
    \qed
\end{proof}


\section{Completely Regular Spaces}

\begin{definition}[Completely Regular]
    Let $X$ be a topological space. Then $X$ is \emph{completely regular} if
    and only if it is $T_1$ and, for every point $x_0 \in X$ and closed set
    $A$ not containing $x_0$, we have $\{x_0\}$ and $A$ can be separated by a
    continuous function.
\end{definition}

\begin{proposition}
    Every completely regular space is regular.
\end{proposition}

\begin{proof}
    \pf
    \step{1}{\pflet{$X$ be a completely regular space.}}
    \step{2}{\pflet{$x_0 \in X$ and $A$ be a closed space not containing $x_0$.}}
    \step{3}{\pick\ a continuous $f : X \rightarrow [0,1]$ with $f(x_0) = \{0\}$
    and $f(A) = \{1\}$.}
    \step{4}{$\inv{f}((-\infty,1/2))$ and $\inv{f}((1/2,+\infty))$ are disjoint
    open neighbourhoods of $x_0$ and $A$  respectively.}
    \qed
\end{proof}

\begin{corollary}
    The space $\RR_K$ is not completely regular.
\end{corollary}

\begin{proposition}
    A subspace of a completely regular space is completely regular.
\end{proposition}

\begin{proof}
    \pf
    \step{1}{\pflet{$X$ be a completely regular space.}}
    \step{2}{\pflet{$Y \subseteq X$}}
    \step{3}{$Y$ is $T_1$}
    \begin{proof}
        \pf\ Proposition \ref{proposition:T1_subspace}.
    \end{proof}
    \step{4}{\pflet{$y_0 \in Y$}}
    \step{5}{\pflet{$A$ be closed in $Y$.}}
    \step{6}{\assume{$y_0 \notin A$}}
    \step{7}{\pick\ $B$ closed in $X$ such that $A = B \cap Y$}
    \step{8}{\pick\ $f : X \rightarrow [0,1]$ continuous such that $f(y_0) = 0$
    and $f(B) = \{1\}$.}
    \step{9}{$f \restriction Y : Y \rightarrow [0,1]$ is continuous and separates $y_0$
    from $A$.}
    \qed
\end{proof}

\begin{proposition}
    The product of a family of completely regular spaces is completely regular.
\end{proposition}

\begin{proof}
    \pf
    \step{1}{\pflet{$\{X_\alpha\}_{\alpha \in J}$ be any family of completely
    regular spaces.}}
    \step{2}{\pflet{$X = \prod_{\alpha \in J} X_\alpha$}}
    \step{3}{\pflet{$b \in X$}}
    \step{4}{\pflet{$A$ be closed in $X$.}}
    \step{5}{\assume{$b \notin A$}}
    \step{6}{\pick\ a basic open neighbourhood $U = \prod_{\alpha \in J} U_\alpha$
    of $b$ that is disjoint from $A$, where for all $\alpha$ we have $U_\alpha$
    is open in $X_\alpha$, and $U_\alpha = X_\alpha$ for all $\alpha$
    except $\alpha_1$, \ldots, $\alpha_n$.}
    \step{7}{For $1 \leq i \leq n$, \pick\ a continuous function $f_i : X_i
    \rightarrow [0,1]$ such that $f_i(b_{\alpha_i}) = 1$ and $f(X_i - U_{\alpha_i}) = \{ 0 \}$}
    \step{8}{Define $f : X \rightarrow [0,1]$ by $f(x) = f_1(x_{\alpha_1}) \cdots
    f_n(x_{\alpha_n})$}
    \step{9}{$f$ is continuous.}
    \step{10}{$f(b) = 1$}
    \step{11}{For all $x \in A$ we have $f(x) = 0$}
    \qed
\end{proof}

\begin{example}
    The space $S_\Omega \times \overline{S_\Omega}$ is completely regular.
\end{example}

\begin{example}[Choice]
    Not every regular space is completely regular.
    
    For $m \in \ZZ$, let $L_m = \{ m \} \times [-1, 0]$.

    For each odd integer $n$ and each integer $k \geq 2$, let
    \[ C_{nk} = (\{ n + 1 - 1/k \} \times [-1, 0])
    \cup (\{n - 1 + 1/k\} \times [-1, 0])
    \cup \{ (x,y) \mid (x-n)^2 + y^2 = (1 - 1/k)^2 \text{ and } y \geq 0 \} \]
    and
    \[ p_{nk} = (n, 1 - 1/k) \]

    Let
    \[ X = \bigcup_{m \in \ZZ} L_m \cup \bigcup_{n,k} C_{nk} \cup \{ a, b \} \]

    Let $\mathcal{B}$ be the set consisting of all sets of one of the following
    four types:
    \begin{itemize}
        \item The intersection of $X$ with a horizontal open line segment that contains none
        of the points $p_{nk}$;
        \item A set formed from one of the $C_{nk}$ by deleting finitely many points;
        \item For each even integer $m$, the union of $\{a\}$ and the set of points
        $(x,y) \in X$ such that $x < m$;
        \item For each even integer $m$, the union of $\{b\}$ and the set of points
        $(x,y) \in X$ such that $x > m$.
    \end{itemize}

    Then $\mathcal{B}$ is a basis for a topology on $X$ that is regular but not completely regular.
    
    \begin{proof}
        \pf
        \step{1}{$\mathcal{B}$ forms a basis for a topology on $X$.}
        \begin{proof}
            \step{1}{For all $x \in X$, there exists $B \in \BB$ such that $x \in B$}
            \begin{proof}
                \step{a}{Every $x \in \RR^2$ is in a set of type 3}
                \step{b}{$a$ is in a set of type 3}
                \step{c}{$b$ is in a set of type 4}
            \end{proof}
            \step{2}{For all $B_1, B_2 \in BB$ and $p \in B_1 \cap B_2$,
            there exists $B_3 \in \BB$ such that $p \in B_3 \subseteq B_1 \cap B_2$}
            \begin{proof}
                \step{a}{\case{$B_1$ and $B_2$ are of type 1}}
                \begin{proof}
                    \pf\ Take $B_3 = B_1 \cap B_2$ which is of type 1.
                \end{proof}
                \step{b}{\case{$B_1$ is of type 1 and $B_2$ is of type 2}}
                \begin{proof}
                    \pf\ Take $B_3 = B_1 \cap B_2$ which is of type 1.
                \end{proof}
                \step{b}{\case{$B_1$ is of type 1 and $B_2$ is of type 3}}
                \begin{proof}
                    \pf\ Take $B_3 = B_1 \cap B_2$ which is of type 1.
                \end{proof}
                \step{b}{\case{$B_1$ is of type 1 and $B_2$ is of type 4}}
                \begin{proof}
                    \pf\ Take $B_3 = B_1 \cap B_2$ which is of type 1.
                \end{proof}
                \step{a}{\case{$B_1$ and $B_2$ are of type 2}}
                \begin{proof}
                    \pf\ Take $B_3 = B_1 \cap B_2$ which is of type 2.
                \end{proof}
                \step{b}{\case{$B_1$ is of type 2 and $B_2$ is of type 3}}
                \begin{proof}
                    \pf\ Take $B_3 = B_1 = B_1 \cap B_2$.
                \end{proof}
                \step{b}{\case{$B_1$ is of type 2 and $B_2$ is of type 4}}
                \begin{proof}
                    \pf\ Take $B_3 = B_1 = B_1 \cap B_2$.
                \end{proof}
                \step{a}{\case{$B_1$ and $B_2$ are of type 3}}
                \begin{proof}
                    \pf\ Take $B_3 = B_1 \cap B_2$ which is equal to either $B_1$
                    or $B_2$.
                \end{proof}
                \step{b}{\case{$B_1$ is of type 3 and $B_2$ is of type 4}}
                \begin{proof}
                    \step{i}{\pflet{$B_1 = \{ (x,y) \in X \mid x < m \} \cup \{ a \}$}}
                    \step{ii}{\pflet{$B_2 = \{ (x,y) \in X \mid x > n \} \cup \{ b \}$}}
                    \step{iii}{\pflet{$p = (x,y)$}}
                    \step{iv}{$n < x < m$}
                    \step{v}{\pflet{$B_3 = X \cap ((n,m) \times \{ y \})$}}
                    \step{vi}{$B_3$ is of type 1.}
                    \step{vii}{$p \in B_3 \subseteq B_1 \cap B_2$}
                \end{proof}
                \step{a}{\case{$B_1$ and $B_2$ are of type 4}}
                \begin{proof}
                    \pf\ Take $B_3 = B_1 \cap B_2$ which is equal to either $B_1$
                    or $B_2$.
                \end{proof}
            \end{proof}
        \end{proof}
        \step{2}{For any continuous $f : X \rightarrow \RR$ we have $f(a) = f(b)$}
        \begin{proof}
            \step{a}{\pflet{$f : X \rightarrow \RR$ be continuous.}}
            \step{b}{For all $c \in \RR$, we have $\inv{f}(c)$ is $G_\delta$ in $X$}
            \begin{proof}
                \pf\ $\inv{f}(c) = \bigcap_{q \in \QQ^+} \inv{f}((c-q,c+q))$
            \end{proof}
            \step{c}{For $n$ an odd integer and $k \geq 2$ an integer,
            \pflet{$S_{nk} = \{ p \in C_{nk} \mid f(p) \neq f(p_{nk}) \}$}}
            \step{cc}{Each $S_{nk}$ is countable.}
            \begin{proof}
                \step{i}{\pflet{$n$ be an odd integer and $k \geq 2$ an integer.}}
                \step{ii}{$\inv{f}(f(p_{nk}))$ is $G_\delta$}
                \begin{proof}
                    \pf\ By \stepref{b}
                \end{proof}
                \step{iii}{\pick\ a countable set $\UU$ of open sets such that $\inv{f}(f(p_{nk}) = \bigcap \UU$)}
                \step{iv}{$S_{nk} = \bigcup \{ A \subseteq^{\mathrm{fin}} C_{nk} \mid C_{nk} - A \in \UU \}$}
            \end{proof}
            \step{d}{\pick\ $d \in [-1,0]$ such that $\RR \times \{d\}$ intersects
            none of the sets $S_{nk}$}
            \begin{proof}
                \pf\ There exists such a $d$ because $[-1,0]$ is uncountable and
                $\bigcup_{n,k} S_{nk}$ is countable.
            \end{proof}
            \step{e}{For $n$ odd,
            \[ f(n-1,d) = \lim_{k \rightarrow \infty} f(p_{nk}) = f(n+1,d)\]}
            \begin{proof}
                \step{i}{\pflet{$n$ be an odd integer.}}
                \step{ii}{$f(p_{nk}) \rightarrow f(n-1,d)$ as $k \rightarrow \infty$}
                \begin{proof}
                    \step{one}{For every $k \geq 1$ we have $f(p_{nk}) = f(n-1+1/k,d)$}
                    \begin{proof}
                        \pf\ Since $(n-1+1/k,d) \notin S_{nk}$.
                    \end{proof}
                    \step{two}{$(n-1+1/k,d) \rightarrow (n-1,d)$ as $k \rightarrow \infty$}
                    \begin{proof}
                        \step{A}{\pflet{$U$ be an open neighbourhood of $(n-1,d)$}
                        \prove{There exists $K$ such that, for all $k \geq K$,
                        $(n-1+1/k,d) \in U$}}
                        \step{B}{\case{$U$ is of type 1}}
                        \begin{proof}
                            \step{ONE}{\pick\ $\epsilon > 0$ such that $(n-1-\epsilon,n-1+\epsilon) \times \{d\} \subseteq U$}
                            \step{TWO}{\pick\ $K$ such that $1/K < \epsilon$}
                            \step{THREE}{For $k \geq K$ we have $(n-1+1/k,d) \in K$}
                        \end{proof}
                        \step{C}{\case{$U$ is of type 3 or 4}}
                        \begin{proof}
                            \pf\ For all $k$ we have $(n-1+1/k,d) \in U$
                        \end{proof}
                    \end{proof}
                    \step{three}{$f(n-1+1/k,d) \rightarrow f(n,d)$ as $k \rightarrow \infty$}
                \end{proof}
                \step{ii}{$f(p_{nk}) \rightarrow f(n+1,d)$ as $k \rightarrow \infty$}
                \begin{proof}
                    \pf\ Similar.
                \end{proof}
            \end{proof}
            \step{f}{$f(a) = f(b)$}
            \begin{proof}
                \pf\ Let $c$ be the number such that $f(2n) = c$ for every integer $n$.
                Then $f(a) = f(b) = c$ since $2n \rightarrow b$ as $n \rightarrow \infty$
                and $-2n \rightarrow a$ as $n \rightarrow \infty$.
            \end{proof}
        \end{proof}
        \step{3}{$X$ is regular.}
        \begin{proof}
            \step{a}{$X$ is $T_1$}
            \begin{proof}
                \step{i}{For $x \in \RR^2$, $X - \{x\}$ is open.}
                \begin{proof}
                    \pf\ It is a union of one set of type 3, one set of type 4,
                    and a set of sets of type 1.
                \end{proof}
                \step{ii}{$X - \{a\}$ is open.}
                \begin{proof}
                    \pf\ It is the union of all the sets of type 4.
                \end{proof}
                \step{iii}{$X - \{b\}$ is open.}
                \begin{proof}
                    \pf\ It is the union of all the sets of type 3.
                \end{proof}
            \end{proof}
            \step{b}{For all $x \in X$ and $U$ an open neighbourhood of $x$,
            there exists a neighbourhood $V$ of $x$ such that $\overline{V} \subseteq U$.}
            \begin{proof}
                \step{i}{\pflet{$x \in X$}}
                \step{ii}{\pflet{$U$ be an open neighbourhood of $x$}}
                \step{iii}{\case{$U$ is of type 1}}
                \begin{proof}
                    \pf\ Let $x = (s,t)$ be in an open line segment with $x$-coordinates
                    the values in $(a,b)$. Pick $c$, $d$ with $a < c < s < d < b$.
                    Let $V$ be the set of type 1 with $y$-coordinate $t$ and $x$-coordinates in $(c,d)$.
                    Then $x \in V$ and $\overline{V} \subseteq U$.
                \end{proof}
                \step{iv}{\case{$U$ is of type 2}}
                \begin{proof}
                    \pf\ Take $V = U$.
                \end{proof}
                \step{v}{\case{$U$ is of type 3}}
                \begin{proof}
                    \pf\ If $U$ is $a$ plus the set of all points with $x$-coordinate $< m$,
                    take $V$ to be $a$ plus the set of all points with $x$-coordinate $< m - 1$.
                \end{proof}
                \step{vi}{\case{$U$ is of type 4}}
                \begin{proof}
                    \pf\ Similar.
                \end{proof}
            \end{proof}
        \end{proof}
        \step{4}{$X$ is not completely regular.}
        \begin{proof}
            \pf\ There is no continuous $f : X \rightarrow [0,1]$ with $f(a) = 0$
            and $f(b) = 1$.
        \end{proof}
        \qed
    \end{proof}
\end{example}

\begin{proposition}
    Let $X$ be completely regular. Let $A$ be a compact subspace of $X$ and $B$
    a closed set in $X$ disjoint from $A$. Then $A$ and $B$ can be separated by a
    continuous function.
\end{proposition}

\begin{proof}
    \pf
    \step{1}{For all $a \in A$, there exists a continuous function $f : X
    \rightarrow [0,1]$ such that $f(a) = 0$ and $f(B) = \{1\}$.}
    \step{2}{$\{ \inv{f}(0) \mid f : X \rightarrow [0,1] \text{ continuous},
    f(B) = \{1\} \}$ covers $A$.}
    \step{3}{\pick\ a finite subcover $\{ \inv{f_1}(0), \ldots, \inv{f_n}(0) \}$
    }
    \step{4}{Define $f : X \rightarrow [0,1]$ by $f(x) = f_1(x) \cdots f_n(x)$}
    \step{5}{$f$ is continuous.}
    \step{6}{$f(A) = \{0\}$}
    \step{7}{$f(B) = \{1\}$}
    \qed
\end{proof}

\begin{theorem}
    \label{theorem:completely_regular_embedded_in_IJ}
    A topological space is completely regular if and only if it is homeomorphic to
    a subspace of $[0,1]^J$ for some set $J$.
\end{theorem}

\begin{proof}
    \pf\ By the Imbedding Theorem. \qed
\end{proof}

See also Corollary \ref{corollary:order_topology_completely_regular}: Every linearly ordered set under the
order topology is completely regular.

\begin{proposition}
    The image of a completely regular space under a continuous map is
    not necessarily completely regular.
\end{proposition}

\begin{proof}
    \pf\ Take the identity function from a set under the
    discrete topology to the same set under the indiscrete topology. \qed
\end{proof}

\section{Normal Spaces}

\begin{definition}[Normal]
    A topological space $X$ is normal if and only if it is $T_1$ and, for any
    disjoint closed sets $A$ and $B$, there exist disjoint open sets $U$ and $V$
    with $A \subseteq U$ and $B \subseteq V$.
\end{definition}

\begin{proposition}
    Every normal space is regular.
\end{proposition}

\begin{proof}
    \pf\ Immediate from definitions. \qed
\end{proof}

\begin{corollary}
    The space $\RR_K$ is not normal.
\end{corollary}

\begin{proposition}
    \label{proposition:normal_neighbourhoods}
    Let $X$ be a $T_1$ space. Then $X$ is normal if and only if, for any closed
    set $A$ and open neighbourhood $U$ of $A$, there exists a neighbourhood
    $V$ of $A$ with $\overline{V} \subseteq U$.
\end{proposition}

\begin{proof}
    \pf
    \step{1}{\pflet{$X$ be a $T_1$ space.}}
    \step{2}{If $X$ is regular then,  for any closed set
    $A$ and open neighbourhood $U$ of $A$, there exists a neighbourhood
    $V$ of $A$ with $\overline{V} \subseteq U$.}
    \begin{proof}
        \step{a}{\assume{$X$ is regular.}}
        \step{b}{\pflet{$A$ be closed in $X$}}
        \step{c}{\pflet{$U$ be an open neighbourhood of $A$.}}
        \step{d}{\pick\ disjoint open sets $V$ and $W$ such that $X - U \subseteq W$
        and $A \subseteq V$}
        \step{e}{$\overline{V} \subseteq U$}
        \begin{proof}
            \pf\ Since $V \subseteq X - W \subseteq U$.
        \end{proof}
    \end{proof}
    \step{3}{If, for any closed set
    $A$ and open neighbourhood $U$ of $A$, there exists a neighbourhood
    $V$ of $A$ with $\overline{V} \subseteq U$, then $X$ is regular.}
    \begin{proof}
        \step{a}{\assume{for any closed set
        $A$ and open neighbourhood $U$ of $A$, there exists a neighbourhood
        $V$ of $A$ with $\overline{V} \subseteq U$.}}
        \step{b}{\pflet{$A$ and $B$ be disjoint closed sets.}}
        \step{c}{\pick\ a neighbourhood $V$ of $B$ with $\overline{V} \subseteq X - A$}
        \step{d}{\pflet{$U = X - \overline{V}$}}
        \step{e}{$U$ and $V$ are disjoint open sets with $A \subseteq U$ and $B \subseteq V$}
    \end{proof}
    \qed
\end{proof}

\begin{example}
    The space $\RR_l$ is normal.

    \begin{proof}
        \pf
        \step{1}{$\RR_l$ is $T_1$}
        \begin{proof}
            \pf\ Since the topology of $\RR_l$ is finer than $\RR$.
        \end{proof}
        \step{2}{\pflet{$A$ and $B$ be disjoint closed sets in $\RR_l$}}
        \step{3}{\pflet{$U$ be the union of all the basic open sets $[a,x)$
        with $a \in A$ such that $[a,x)$ does not intersect $B$.}}
        \step{4}{\pflet{$V$ be the union of all the basic open sets $[b,x)$
        with $b \in B$ such that $[b,x)$ does not intersect $A$.}}
        \step{5}{$U$ and $V$ are disjoint.}
        \begin{proof}
            \step{a}{\assume{for a contradiction $x \in U \cap V$}}
            \step{b}{\pick\ $[a,y)$ such that $x \in [a,y)$, $a \in A$ and $[a,y)$
            does not intersect $B$.}
            \step{c}{\pick\ $[b,z)$ such that $x \in [b,z)$, $b \in B$ and $[b,z)$
            does not intersect $A$.}
            \step{d}{\assume{w.l.o.g. $a < b$}}
            \step{e}{$b \in [a,y)$}
            \begin{proof}
                \pf\ Since $a < b \leq x < y$
            \end{proof}
            \qedstep
            \begin{proof}
                \pf\ This contradicts the fact that $[a,y)$ does not intersect $B$
                (\stepref{b}).
            \end{proof}
        \end{proof}
        \step{6}{$A \subseteq U$}
        \begin{proof}
            \step{a}{\pflet{$a \in A$}}
            \step{b}{$a \in X - B$}
            \step{c}{\pick\ $x$ such that $[a,x) \subseteq X - B$}
            \step{d}{$a \in [a,x) \subseteq U$}
        \end{proof}
        \step{7}{$B \subseteq V$}
        \begin{proof}
            \pf\ Similar.
        \end{proof}
        \qed
    \end{proof}
\end{example}

\begin{corollary}
    The Sorgenfrey plane is completely regular.
\end{corollary}

The following example shows that:
\begin{itemize}
    \item a completely regular space is not necessarily normal;
    \item the product of two normal spaces is not necessarily normal.
\end{itemize}

\begin{example}[Choice]
    The Sorgenfrey plane is not normal.

    \begin{proof}
        \pf
        \step{a}{\assume{for a contradiction $\RR_l^2$ is normal.}}
        \step{b}{\pflet{$L = \{ (x, -x) \mid x \in \RR \}$}}
        \step{c}{$L$ is closed in $\RR_l^2$.}
        \step{d}{$L$ is discrete as a subspace of $\RR_l^2$.}
        \step{e}{For every $A \subseteq L$, \pick\ disjoint open sets $U$ and $V$
        in $\RR_l^2$ such that $A \subseteq U$ and $L - A \subseteq V$.}
        \begin{proof}
            \pf\ $A$ and $L - A$ are closed in $\RR_l^2$ by \ref{theorem:closed_subspace}.
        \end{proof}
        \step{f}{\pflet{$D = \QQ^2$}}
        \step{g}{$D$ is dense in $\RR_l^2$}
        \step{h}{Define $\theta : \pow L \rightarrow \pow D$ by
        \begin{align*}
            \theta(A) & = D \cap U_A & (\emptyset \subset A \subset L) \\
            \theta(\emptyset) & = \emptyset \\
            \theta(L) & = D
        \end{align*}
        }
        \step{i}{$\theta$ is injective.}
        \begin{proof}
            \step{a}{\pflet{$A, B$ be nonempty proper subsets of $L$ with $A \neq B$}}
            \step{b}{\assume{w.l.o.g. there exists $x \in A$ such that $x \notin B$}}
            \step{c}{$U_A \cap V_B$ is nonempty}
            \step{d}{\pick\ $d \in D \cap U_A \cap V_B$}
            \step{e}{$d \in \theta(A)$ and $d \notin \theta(B)$}
            \step{f}{$\theta(A) \neq \theta(B)$}
        \end{proof}
        \qedstep
        \begin{proof}
            \pf\ This contradicts Cantor's Theorem, since $L$ is uncountable
            and $D$ is countable.
        \end{proof}
        \qed
    \end{proof}

    \begin{proof}
        \pf
        \step{a}{\pflet{$A = \{ (x, -x) \mid x \in \QQ \}$}}
        \step{b}{\pflet{$B = \{ (x, -x) \mid x \in \RR - \QQ \}$}}
        \step{c}{$A$ and $B$ are closed in $\RR_l^2$}
        \step{d}{\pflet{$V$ be any open set in $\RR_l^2$ that includes $B$}
        \prove{There is no open set $U$ that includes $A$ and is disjoint from $B$.}}
        \step{e}{For $n \in \ZZ^+$, \pflet{$K_n = \{ x \in [0,1] - \QQ \mid
        [x,x+1/n) \times [-x,-x+1/n) \subseteq V \}$}}
        \step{f}{$[0.1] - \QQ = \bigcup_n K_n$ is countable.}
        \begin{proof}
            \step{i}{\pflet{$x \in [0,1] - \QQ$} \prove{There exists $n$ such that
            $x \in K_n$}}
            \step{ii}{\pick\ $\epsilon$, $\delta$ such that $[x,x+\epsilon) \times
            [-x,-x+\delta) \subseteq V$}
            \step{iii}{\pick\ $n$ such that $1/n \leq \epsilon$ and $1/n \leq \delta$}
            \step{iv}{$x \in K_n$}
        \end{proof}
        \step{g}{\pick\ $n$, $a$, $b$ such that $(a,b) \subseteq \overline{K_n}$}
        \begin{proof}
            \pf\ By Proposition \ref{proposition:pre_Baire_category} it cannot be that
            every $\overline{K_n}$ has empty interior.
        \end{proof}
        \step{h}{For all $x \in (a,b)$ and $\epsilon \in (0,1/n)$ we have
        $(x, -x+\epsilon) \in V$}
        \begin{proof}
            \step{i}{\pflet{$x \in (a,b)$}}
            \step{ii}{\pflet{$\epsilon \in (0,1/n)$}}
            \step{iii}{\pick\ $y \in K_n \cap (x - \epsilon, x)$}
            \begin{proof}
                \pf\ Pick $z \in (a,b) \cap (x-\epsilon,x)$. Then $z \in \overline{K_n}$.
            \end{proof}
            \step{iv}{$(x, -x+\epsilon) \in [y.y+1/n) \times [-y,-y+1/n) \subseteq V$}
        \end{proof}
        \step{i}{For every rational $q \in (a,b)$, we have $(q, -q)$ is a limit point of $V$.}
        \step{j}{There is no open set $U$ that includes $A$ and is disjoint from $B$.}
        \qed
    \end{proof}
\end{example}

\begin{proposition}
    In a regular space, every pair of disjoint closed sets have open neighbourhoods whose
    closures are disjoint.
\end{proposition}

\begin{proof}
    \pf
    \step{1}{\pflet{$X$ be a regular space.}}
    \step{2}{\pflet{$A$ and $B$ be disjoint closed sets in $X$.}}
    \step{3}{\pick\ a neighbourhood $U$ of $A$ such that $\overline{U} \subseteq
    X - B$.}
    \begin{proof}
        \pf\ Proposition \ref{proposition:normal_neighbourhoods}.
    \end{proof}
    \step{4}{\pick\ a neighbourhood $V$ of $B$ such that $\overline{V} \subseteq
    X - \overline{U}$.}
    \begin{proof}
        \pf\ Proposition \ref{proposition:normal_neighbourhoods}.
    \end{proof}
    \step{5}{$U$ and $V$ are disjoint open neighbourhoods of $A$ and $B$ respectively
    with disjoit closures.}
    \qed
\end{proof}

\begin{proposition}
    The image of a normal space under a continuous map is
    not necessarily normal.
\end{proposition}

\begin{proof}
    \pf\ Take the identity function from a set under the
    discrete topology to the same set under the indiscrete topology. \qed
\end{proof}

\begin{proposition}
    \label{proposition:closed_continuous_image_normal}
    The image of a normal space under a closed continuous map is normal.
\end{proposition}

\begin{proof}
    \pf
    \step{1}{\pflet{$X$ be a normal space.}}
    \step{2}{\pflet{$Y$ be a topological space.}}
    \step{3}{\pflet{$f : X \twoheadrightarrow Y$ be a closed continuous surjective map.}}
    \step{4}{For every $y \in Y$ and every open neighbourhood $U$ of $\inv{p}(y)$,
    there exists an open neighbourhood $W$ of $y$ such that $\inv{p}(W) \subseteq U$}
    \begin{proof}
        \step{a}{\pflet{$y \in Y$}}
        \step{b}{\pflet{$U$ be an open neighbourhood of $\inv{p}(y)$}}
        \step{d}{\pflet{$W = Y - p(X - U)$}}
        \step{e}{$W$ is open}
        \begin{proof}
            \pf\ Since $p$ is a closed map.
        \end{proof}
        \step{f}{$y \in W$}
        \begin{proof}
            \pf\ Since $\inv{p}(y) \subseteq V$.
        \end{proof}
        \step{g}{$\inv{p}(W) \subseteq U$}
        \begin{proof}
            \pf\ If $p(x) \in W$ then $p(x) \notin p(X-U)$ so $x \in U$.
        \end{proof}
    \end{proof}
    \step{5}{\pflet{$A$ be a closed set in $Y$ and $U$ be an open neighbourhood of
    $A$.}}
    \step{6}{\pick\ an open neighbourhood $V$ of $\inv{p}(A)$ such that
    $\overline{V} \subseteq \inv{p}(U)$}
    \step{7}{\pflet{$W = \bigcup \{ W' \text{ open in } Y \mid \inv{p}(W') \subseteq
    V \}$}}
    \step{8}{$A \subseteq W$}
    \begin{proof}
        \step{a}{\pflet{$y \in A$}}
        \step{b}{$\inv{p}(y) \subseteq V$}
        \step{c}{\pick\ an open neightbourhood $W'$ of $y$ such that $\inv{p}(W')
        \subseteq V$}
        \begin{proof}
            \pf\ By \stepref{4}.
        \end{proof}
        \step{d}{$y \in W' \subseteq W$}
    \end{proof}
    \step{9}{$\overline{W} \subseteq U$}
    \begin{proof}
        \step{a}{$W \subseteq p(\overline{V})$}
        \begin{proof}
            \step{i}{\pflet{$y \in W$}}
            \step{ii}{\pick\ $W'$ open in $Y$ such that $y \in W'$ and $\inv{p}(W') \subseteq V$}
            \step{iii}{\pick\ $x \in X$ such that $p(x) = y$}
            \step{iv}{$x \in V$}
            \step{v}{$x \in \overline{V}$}
            \step{vi}{$y \in p(\overline{V})$}
        \end{proof}
        \step{b}{$\overline{W} \subseteq p(\overline{V})$}
        \begin{proof}
            \pf\ $p(\overline{V})$ is closed since $p$ is a closed map.
        \end{proof}
        \step{c}{$p(\overline{V}) \subseteq U$}
    \end{proof}
    \qed
\end{proof}

\begin{theorem}
    Every linearly ordered set under the order topology is normal.
\end{theorem}

\begin{proof}
    \pf\ See L. A. Steen and J. A. Seerbach Jr. \emph{Counterexamples in
    Topology} Example 39. \qed
\end{proof}

\begin{corollary}
    \label{corollary:order_topology_completely_regular}
    Every linearly ordered set under the order topology is completely regular.
\end{corollary}

\begin{example}
    In particular, $S_\Omega$ and $\overline{S_\Omega}$ are normal.
\end{example}

\begin{proposition}
    \label{proposition:closed_subspace_normal}
    Every closed subspace of a normal space is normal.
\end{proposition}

\begin{proof}
    \pf
    \step{1}{\pflet{$X$ be a normal space.}}
    \step{2}{\pflet{$Y$ be a closed subspace of $X$.}}
    \step{3}{\pflet{$A$ and $B$ be closed in $Y$.}}
    \step{4}{$A$ and $B$ are closed in $X$.}
    \step{5}{\pick\ $U'$ and $V'$ open in $X$ and disjoint such that
    $A \subseteq U'$ and $B \subseteq V'$}
    \step{6}{$U' \cap Y$ and $V' \cap Y$ are disjoint open neighbourhoods of $A$
    and $B$ respectively in $Y$.}
    \qed
\end{proof}

\begin{corollary}
    Let $\{ X_\alpha \}_{\alpha \in J}$ be a family of nonempty spaces. If
    $\prod_{\alpha \in J} X_\alpha$ is normal then each $X_\alpha$ is normal.
\end{corollary}

The following example shows:
\begin{itemize}
    \item A completely regular space is not necessarily normal.
    \item A subspace of a normal space is not necessarily normal.
    (Since $\RR^J \cong (0,1)^J \subseteq [0,1]^J$ and $[0,1]^J$
    is comapct Hausdorff hence normal.)
    \item A product of a family of normal spaces is not necessarily normal.
\end{itemize}

\begin{example}[Choice]
    If $J$ is an uncountable set then $\RR^J$ is not normal.
\end{example}

\begin{proof}
    \pf
    \step{1}{\pflet{$J$ be an uncountable set.}}
    \step{2}{\pflet{$X = (\ZZ^+)^J$}}
    \step{3}{$X$ is not normal.}
    \begin{proof}
        \step{a}{For $x \in X$ and $B \subseteq^{\mathrm{fin}} J$,
        \pflet{$U(x,B) = \{ y \in X \mid \forall \alpha \in B. y(\alpha) = x(\alpha) \}$}}
        \step{b}{$\{ U(x,B) \mid x \in X, B \subseteq^{\mathrm{fin}} J \}$ is a basis for $X$}
        \begin{proof}
            \step{i}{\pflet{$x \in X$ and $U = \prod_{\alpha \in J} U_\alpha$ be a basic open
            neighbourhood of $x$, where $U_\alpha = \ZZ^+$ for all $\alpha$ except
            $\alpha \in B$, where $B$ is some finite subset of $J$}}
            \step{ii}{$x \in U(x,B) \subseteq U$}
        \end{proof}
        \step{c}{For $n \in \ZZ^+$, \pflet{$P_n = \{ x \in X \mid x \text{ is injective on }
        J - \inv{x}(n) \}$}}
        \step{cc}{For every $n$, $P_n$ is closed.}
        \begin{proof}
            \step{i}{\pflet{$x \in X - P_n$}}
            \step{ii}{\pick\ $\alpha, \beta \in J$ such that $x(\alpha) = x(\beta) \neq n$}
            \step{iii}{$x \in B(x,\{\alpha, \beta\}) \subseteq X - P_n$}
        \end{proof}
        \step{d}{$P_1$ and $R_2$ are disjoint.}
        \begin{proof}
            \pf\ If $x \in X$ was in $P_1$ and $P_2$ then it would be an injective
            function $J \rightarrow \ZZ^+$, contradicting \stepref{1}.
        \end{proof}
        \step{e}{\assume{for a contradiction $U$ and $V$ are disjoint open neighbourhoods
        of $P_1$ and $P_2$ respectively.}}
        \step{f}{For every sequence $(\alpha_n)$ in $J$, sequence $0 < n_1 < n_2 < \cdots$
        of integers, and $i \geq 1$, \pflet{
            $B_i((\alpha_n),(n_k)) = \{ \alpha_1, \alpha_2, \ldots, \alpha_{n_i} \}$
            and $x_i((\alpha_n),(n_k)) \in X$ be the point
            \begin{align*}
                x_i((\alpha_n),(n_k))(\alpha_j) & = j & (1 \leq j \leq n_{i-1}) \\
                x_i((\alpha_n),(n_k))(\alpha) & = 1 & (\text{for all other } \alpha)
            \end{align*}
        }}
        \step{g}{\pick\ sequences $(\alpha_n)$ and $(n_k)$ such that, for all $i$,
        we have $U(x_i((\alpha_n),(n_k)),B_i((\alpha_n),(n_k))) \subseteq U$}
        \begin{proof}
            \step{i}{$x_1((),())(\alpha) = 1$ for all $\alpha$}
            \step{ii}{\pick\ $B_1$ such that $U(x_1,B_1) \subseteq U$}
            \step{iii}{\pflet{$p_1 = |B_1|$ and $B_1 = \{ \alpha_1, \alpha_2, \ldots,
            \alpha_{n_1} \}$}}
            \step{iv}{\assume\ we have picked $n_1$, \ldots, $n_k$ and $\alpha_1$,
            \ldots, $\alpha_{n_k}$ satisfying the conditions.}
            \step{v}{$x_{k+1}((\alpha_n),(n_k)) \in P_1$}
            \step{vi}{\pick\ $B_{k+1} \supset B_k$ such that $U(x_{k+1},B_{k+1}) \subseteq U$}
            \step{vii}{\pflet{$n_{k+1} = |B_{k+1}|$ and $B_{k+1}-B_k
            = \{ \alpha_{n_k + 1}, \alpha_{n_k + 2}, \ldots, \alpha_{n_{k+1}} \}$}}
        \end{proof}
        \step{h}{\pflet{$A = \{ \alpha_1, \alpha_2, \ldots, \}$}}
        \step{i}{Define $y : J \rightarrow \ZZ^+$ by
        \begin{align*}
            y(\alpha_j) & = j & (j \in \ZZ^+) \\
            y(\alpha) & = 2 & (\text{for all other } j)
        \end{align*}}
        \step{j}{\pick\ $B$ such that $U(y,B) \subseteq V$}
        \step{k}{\pflet{$i$ be least such that $B \cap A \subseteq B_i$}}
        \step{l}{$U(x_{i+1}((\alpha_n),(n_k)),B_{i+1}((\alpha_n),(n_k)))
        \cap U(y,B) \neq \emptyset$}
        \begin{proof}
            \pf\ $x_{i+1}$ is in both.
        \end{proof}
        \qedstep
        \begin{proof}
            \pf\ This is a contradiction since
            $U(x_{i+1}((\alpha_n),(n_k)),B_{i+1}((\alpha_n),(n_k))) \subseteq U$ and
            $U(y,B) \subseteq V$.
        \end{proof}
    \end{proof}
    \qedstep
    \begin{proof}
        \pf\ By Propositionm \ref{proposition:closed_subspace_normal}
        since $(\ZZ^+)^J$ is a closed subspace of $\RR^J$.
    \end{proof}
    \qed
\end{proof}

\begin{itemize}
    \item A completely regular space is not necessarily normal.
    \item A subspace of a normal space is not necessarily normal.
    \item A product of two normal spaces is not necessarily normal.
\end{itemize}

\begin{example}
    The space $S_\Omega \times \overline{S_\Omega}$ is not normal.

    \begin{proof}
        \pf
        \step{1}{\pflet{$A = \{ (x,x) \mid x \in S_\Omega \}$}}
        \step{2}{\pflet{$B = \{ (x, \Omega) \mid x \in S_\Omega \}$}}
        \step{3}{$A$ and $B$ are disjoint closed subsets in $S_\Omega \times
        \overline{S_\Omega}$}
        \step{4}{\assume{for a contradiction $U$ and $V$ are disjoint open
        neighbourhoods of $A$ and $B$ respectively.}}
        \step{5}{For all $x \in S_\Omega$, there exists $\beta$ with $x < \beta
        < \Omega$ such that $(x, \beta) \notin U$}
        \begin{proof}
            \step{a}{\pflet{$x \in S_\Omega$}}
            \step{b}{\pick\ $U_1$ open in $S_\Omega$ and $U_2$ open in
            $\overline{S_\Omega}$ such that $(x, \Omega) \in U_1 \times U_2 \subseteq
            V$}
            \step{c}{\pick\ $\gamma$ such that $(\gamma, \Omega] \subseteq U_2$}
            \step{d}{\pflet{$\beta = \gamma + 1$}}
            \step{e}{$(x, \beta) \in V$}
            \step{f}{$(x, \beta) \notin U$}
        \end{proof}
        \step{6}{For $x \in S_\Omega$, \pflet{$\beta(x)$ be the least element
        such that $x < \beta(x)$ and $(x, \beta(x)) \notin U$}}
        \step{7}{\pflet{$b = \lim_{n \rightarrow \infty} \beta^n(0)$}}
        \step{8}{$(beta^n(0), \beta^{n+1}(0)) \rightarrow (b,b)$ as $n \rightarrow \infty$}
        \step{9}{$(b,b) \in A \subseteq U$}
        \step{10}{For all $n$ we have $(beta^n(0), \beta^{n+1}(0)) \notin U$}
        \qedstep
        \qed
    \end{proof}
\end{example}

\begin{proposition}
    Every regular Lindel\"{o}f space is normal.
\end{proposition}

\begin{proof}
    \pf
    \step{1}{\pflet{$X$ be a regular Lindel\"{o}f space.}}
    \step{3}{\pflet{$A$ and $B$ be disjoint closed sets in $X$.}}
    \step{4}{\pick\ a sequence $(U_n)$ of open sets disjoint from $B$ that cover $A$}
    \step{5}{\pick\ a sequence $(V_n)$ of open sets disjoint from $A$ that cover $B$}
    \step{6}{For $n \in \ZZ^+$, \pflet{$U_n' = U_n - \bigcup_{i=1}^n \overline{V_i}$
    and $V_n' = V_n - \bigcup_{i=1}^n \overline{U_i}$}}
    \step{7}{\pflet{$U' = \bigcup_n U_n'$ and $V' = \bigcup_n V_n'$}}
    \step{8}{$U'$ and $V'$ are disjoint open neighbourhoods of $A$ and $B$
    respectively.}
    \qed
\end{proof}

\begin{corollary}
    The space $\RR^\omega$ is normal.
\end{corollary}

\begin{corollary}
    The space $\RR^I$ is not Lindel\"{o}f.
\end{corollary}

\begin{theorem}[Urysohn Lemma]
    In a normal space, any two disjoint closed sets can be separated by a
    continuous function.
\end{theorem}

\begin{proof}
    \pf
    \step{one}{\pflet{$X$ be a normal space.}}
    \step{two}{\pflet{$A$ and $B$ be disjoint closed sets in $X$.}}
    \step{1}{\pflet{$P = [0,1] \cap \QQ$}}
    \step{2}{\pick\ a family of open sets $\{U_p\}_{p \in P}$ in $X$
    such that $A \subseteq U_0$, $U_1 = X - B$ and, for all $p, q \in P$, if $p < q$ then $\overline{U_p} \subseteq U_q$.}
    \begin{proof}
        \step{a}{\pick\ an enumeration $\{ p_1, p_2, p_3, \ldots \}$ of $P$
        such that $p_1 = 1$ and $p_2 = 0$.}
        \step{b}{\pflet{$U_1 = X - B$}}
        \step{c}{\pick\ $U_0$ open in $X$ such that $A \subseteq U_0$
        and $\overline{U_0} \subseteq X$}
        \begin{proof}
            \pf\ Proposition \ref{proposition:normal_neighbourhoods}.
        \end{proof}
        \step{d}{For all $n$, given open sets $U_{p_1}$, \ldots, $U_{p_n}$
        that satisfies the condition, there exists $U_{p_{n+1}}$ that satisfies
        the condition.}
        \begin{proof}
            \step{i}{Let $q$ be the greatest element out of $\{ p_1, \ldots, p_n \}$
            such that $q < p_{n+1}$ and $r$ the least element such that $p_{n+1} < r$}
            \step{ii}{\pick\ $U_{p_{n+1}}$ open in $X$ such that
            $\overline{U_1} \subseteq U_{p_{n+1}}$ and $U_{p_{n+1}} \subseteq U_r$}
            \begin{proof}
                \pf\ Proposition \ref{proposition:normal_neighbourhoods}.
            \end{proof}
        \end{proof}
    \end{proof}
    \step{3}{Extend the family to $\{U_p\}_{p \in \QQ}$ by defining $U_p = \emptyset$
    if $p < 0$, and $U_p = X$ for $p > 1$.}
    \step{4}{For $x \in X$, \pflet{$\QQ(x) = \{ p \in \QQ \mid x \in U_p \}$}}
    \step{5}{Define $f : X \rightarrow [0,1]$ by $f(x) = \inf \QQ(x)$}
    \begin{proof}
        \step{a}{For all $x \in X$, 0 is a lower bound for $\QQ(x)$}
        \step{b}{For all $x \in X$, we have $1 \in \QQ(x)$}
    \end{proof}
    \step{6}{$f$ is continuous.}
    \begin{proof}
        \step{a}{For all $r \in P$ and $x \in \overline{U_r}$ we have $f(x) \leq r$}
        \begin{proof}
            \step{i}{\pflet{$r \in P$}}
            \step{ii}{\pflet{$x \in \overline{U_r}$}}
            \step{iii}{For all $s > r$ we have $x \in U_s$}
            \step{iv}{For all $s > r$ we have $s \in \QQ(x)$}
            \step{v}{For all $s > r$ we have $f(x) \leq s$}
            \step{vi}{$f(x) \leq r$}
        \end{proof}
        \step{b}{For all $r \in P$ and $x \in U_r$ we have $f(x) \geq r$}
        \begin{proof}
            \step{i}{\pflet{$r \in P$}}
            \step{ii}{\pflet{$x \in X - U_r$}}
            \step{iii}{For all $s < r$ we have $x \notin U_s$}
            \step{iv}{$r$ is a lower bound for $\QQ(x)$}
            \step{v}{$r \leq f(x)$}
        \end{proof}
        \step{c}{\pflet{$x_0 \in X$}}
        \step{d}{\pflet{$(c,d)$ be an open interval that contains $f(x_0)$}
        \prove{There exists an open neighbourhood $U$ of $x_0$ such that
        $f(U) \subseteq (c,d)$}}
        \step{e}{\pick\ rationals $q$, $r$ with $c < q < f(x_0) < r < d$}
        \step{f}{\pflet{$U = U_r - \overline{U_q}$}}
        \step{g}{$U$ is open}
        \step{h}{$x_0 \in U$}
        \begin{proof}
            \step{i}{$x_0 \in U_r$}
            \begin{proof}
                \pf\ From \stepref{b}.
            \end{proof}
            \step{ii}{$x_0 \notin \overline{U_q}$}
            \begin{proof}
                \pf\ From \stepref{a}.
            \end{proof}
        \end{proof}
        \step{i}{$f(U) \subseteq (c,d)$}
        \begin{proof}
            \step{i}{\pflet{$x \in U$}}
            \step{ii}{$r \leq f(x) \leq s$}
            \begin{proof}
                \pf\ From \stepref{a} and \stepref{b}.
            \end{proof}
            \step{iii}{$c < f(x) < d$}
        \end{proof}
    \end{proof}
    \step{7}{$f(x) = 0$ for $x \in A$.}
    \begin{proof}
        \pf\ Since for $x \in A$ we have $\QQ(x) = \{ q \in \QQ \mid q \geq 0 \}$.
    \end{proof}
    \step{8}{$f(x) = 1$ for $x \in B$.}
    \begin{proof}
        \pf\ Since for $x \in B$ we have $\QQ(x) = \{ q \in \QQ \mid q > 1 \}$.
    \end{proof}
    \qed
\end{proof}

\begin{corollary}
    Every normal space is completely regular.
\end{corollary}

\begin{proposition}
    A connected normal space with more than one point is uncountable.
\end{proposition}

\begin{proof}
    \pf
    \step{1}{\pflet{$X$ be a connected normal space with more than one point.}}
    \step{2}{\pick\ two distinct points $a$ and $b$ in $X$}
    \step{3}{\pick\ a continuous $f : X \rightarrow [0,1]$ with $f(a) = 0$
    and $f(b) = 1$}
    \step{4}{$f$ is surjective.}
    \begin{proof}
        \pf\ Since $f(X)$ is connected therefore $f(X) = [0,1]$.
    \end{proof}
    \step{5}{$X$ is uncountable.}
    \qed
\end{proof}

\begin{corollary}
    A connected regular space with more than one point is uncountable.
\end{corollary}

\begin{proof}
    \pf\ Since a countable regular space is Lindel\"{o}f, hence normal. \qed
\end{proof}

\begin{example}
    There exists a connected Hausdorff space that is countably infinite.

    See L. A. Steen and J. A. Seerbach Jr. \emph{Counterexamples in
    Topology} Example 75.
\end{example}

\begin{lemma}[Shrinking Lemma (Choice)]
    Let $X$ be a normal space. Let $(U_n)$ be a countable, point-finite open covering of $X$.
    Then there exists an open covering $(V_n)$ such that $\forall n. \overline{V_n} \subseteq U_n$.
\end{lemma}

\begin{proof}
    \pf
    \step{1}{\pick\ a sequence $(V_n)$ such that $\forall n. \overline{V_n} \subseteq U_n$
    and $\forall n. (V_1, \ldots, V_n, U_{n+1}, U_{n+2}, \ldots)$ covers $X$.}
    \begin{proof}
        \step{a}{\assume{as induction hypothesis we have chosen $V_1$, \ldots, $V_n$ satisfying these conditions.}}
        \step{b}{\pflet{$A = X - V_1 - \cdots - V_n - \bigcup_{k=n+2}^\infty U_k$}}
        \step{c}{\pick\ $V_{n+1}$ open such that $A \subseteq V_{n+1}$ and $\overline{V_{n+1}} \subseteq U_{n+1}$}
    \end{proof}
    \step{2}{$(V_n)$ covers $X$.}
    \begin{proof}
        \step{a}{\pflet{$x \in X$}}
        \step{b}{\pflet\ $n$ be largest such that $x \in U_n$}
        \step{c}{$x \in V_1 \cup \cdots \cup V_n$}
    \end{proof}
    \qed
\end{proof}

\section{Universal Extension Property}

\begin{definition}[Universal Extension Property]
    Let $Y$ be a topological space. Then $Y$ has the \emph{universal extension property}
    if and only if, for every normal space $X$, every closed set $A$ in $X$, and every
    continuous function $f : A \rightarrow Y$, there exists a continuous extension of
    $f$ to $X$.
\end{definition}

\begin{theorem}[Tietze Extension Theorem (Choice)]
    \label{theorem:tietze}
    The space $I$ has the universal extension property.
\end{theorem}

\begin{proof}
    \pf
    \step{1}{For any $r > 0$ and any continuous function $f : A \rightarrow [-r,r]$,
    there exists a continuous function $g : X \rightarrow [-r/3,r/3]$ such that,
    for all $a \in A$, we have $|g(a) - f(a)| \leq 2r/3$.}
    \begin{proof}
        \step{a}{\pflet{$I_1 = [-r,-r/3]$}}
        \step{b}{\pflet{$I_2 = [-r/3,r/3]$}}
        \step{c}{\pflet{$I_3 = [r/3,r]$}}
        \step{d}{\pflet{$B = \inv{f}(I_1)$}}
        \step{e}{\pflet{$C = \inv{f}(I_3)$}}
        \step{f}{\pick\ a continuous $g : X \rightarrow [-r/3,r/3]$ such that
        $g(x) = -r/3$ for all $x \in B$ and $g(x) = r/3$ for all $x \in C$.}
        \begin{proof}
            \pf\ By the Urysohn Lemma.
        \end{proof}
        \step{g}{For all $a \in A$, $|g(a) - f(a)| \leq 2r/3$}
        \begin{proof}
            \pf\ $f(a)$ and $g(a)$ are both in the same interval out of $I_1$,
            $I_2$ and $I_3$.
        \end{proof}
    \end{proof}
    \step{2}{\pflet{$f : A \rightarrow [-1,1]$ be continuous.}}
    \step{3}{\pick\ a sequence $(g_n)_{n \geq 1}$ of continuous functions
    $g_n : X \rightarrow [-2^n/3^[n+1],2^n/3^{n+1}]$ such that, for all $a \in A$,
    we have $|f(a) - g_1(a) - \cdots - g_n(a)| \leq (2/3)^n$}
    \begin{proof}
        \pf\ Applying \stepref{1} repeatedly to $f - g_1 - \cdots - g_n$.
    \end{proof}
    \step{4}{Define $g : X \rightarrow [-1,1]$ by $g(x) = \sum_{n=1}^\infty g_n(x)$}
    \begin{proof}
        \pf\ The sum always converges to a value in $[-1,1]$ by the Comparison Test.
    \end{proof}
    \step{5}{$g$ is continuous.}
    \begin{proof}
        \pf\ By the Weierstrass $M$-test.
    \end{proof}
    \step{6}{For all $a \in A$, $g(a) = f(a)$}
    \begin{proof}
        \pf\ Since
        $|f(a) - g_1(a) - \cdots - g_n(a)| \rightarrow 0$ as $n \rightarrow \infty$.
    \end{proof}
    \qed
\end{proof}

\begin{theorem}[Tietze Extension Theorem (Choice)]
    The space $\RR$ has the universal extension property.
\end{theorem}

\begin{proof}
    \pf
    \step{1}{Any continuous function
    $f : A \rightarrow (-1,1)$ can be extended to a continuous function $X \rightarrow (-1,1)$.}
    \begin{proof}
        \step{a}{\pflet{$f : A \rightarrow (-1,1)$ be continuous}}
        \step{b}{\pick\ a continuous extension $g : X \rightarrow [-1,1]$ of $f$.}
        \begin{proof}
            \pf\ Theorem \ref{theorem:tietze}.
        \end{proof}
        \step{c}{\pflet{$D = \inv{g}(-1) \cup \inv{g}(1)$}}
        \step{d}{$D$ is closed}
        \step{e}{$D \cap A = \emptyset$}
        \step{f}{\pick\ a continuous $\phi : X \rightarrow [0,1]$ such that
        $\phi(D) = \{0\}$ and $\phi(A) = \{1\}$.}
        \begin{proof}
            \pf\ By the Urysohn Lemma.
        \end{proof}
        \step{g}{Define $h : X \rightarrow [-1,1]$ by $h(x) = \phi(x)g(x)$}
        \step{h}{$h$ is continuous.}
        \step{i}{$h$ is an extension of $f$.}
        \step{j}{$h(X) \subseteq (-1,1)$}
    \end{proof}
    \qedstep
    \begin{proof}
        \pf\ Since $(-1,1) \cong \RR$.
    \end{proof}
    \qed
\end{proof}

\begin{proposition}[Choice]
    The product of a family of spaces with the universal extension property has the universal extension property.
\end{proposition}

\begin{proof}
    \pf
    \step{0}{\pflet{$\{ Y_\alpha \}_{\alpha \in J}$ be a family of spaces with the universal extension property.}}
    \step{00}{\pflet{$Y = \prod_{\alpha \in J} Y_\alpha$}}
    \step{1}{\pflet{$X$ be a normal space.}}
    \step{2}{\pflet{$A \subseteq X$ be closed.}}
    \step{3}{\pflet{$f : A \rightarrow Y$ be continuous.}}
    \step{4}{For $\alpha \in J$, \pick\ a continuous extension $g_\alpha : X \rightarrow Y_\alpha$ of $\pi_\alpha \circ f : A \rightarrow Y_\alpha$.}
    \step{5}{Define $g : X \rightarrow Y$ by $g(x) = (g_\alpha(x))_\alpha$}
    \step{6}{$g$ is continuous.}
    \step{7}{$g \restriction A = f$}
    \qed
\end{proof}

\begin{proposition}
    \label{proposition:retract_universal_extension}
    A retract of a space with the universal extension property has the universal extension property.
\end{proposition}

\begin{proof}
    \pf
    \step{1}{\pflet{$Y$ be a space with the universal extension property.}}
    \step{2}{\pflet{$Z \subseteq Y$ and $r : Y \rightarrow Z$ be a retraction.}}
    \step{3}{\pflet{$X$ be a normal space.}}
    \step{4}{\pflet{$A \subseteq X$ be closed.}}
    \step{5}{\pflet{$f : A \rightarrow Z$ be continuous.}}
    \step{6}{\pick\ a continuous extension $g : X \rightarrow Y$ of $f : A \rightarrow Y$}
    \step{7}{$r \circ g : X \rightarrow Z$ is continuous}
    \step{8}{$(r \circ g) \restriction A = f$}
    \qed
\end{proof}

\section{Absolute Retracts}

\begin{definition}[Absolute Retract]
    Let $Y$ be a topological space. Then $Y$ is an \emph{absolute retract} if and only if $Y$ is normal and,
    for every normal space $Z$ and imbedding $i : Y \rightarrow Z$ such that $i(Y)$ is closed in $Z$,
    we have $i(Y)$ is a retract of $Z$.
\end{definition}

\begin{proposition}
    Every normal space that has the universal extension property is an absolute retract.
\end{proposition}

\begin{proof}
    \pf
    \step{1}{\pflet{$Y$ be a space with the universal extension property.}}
    \step{2}{\pflet{$Z$ be a normal space.}}
    \step{3}{\pflet{$i : Y \rightarrow Z$ be an imbedding such that $i(Y)$ is closed.}}
    \step{4}{\pick\ a continuous extension $r : Z \rightarrow Y$ of $\inv{i} : i(Y) \rightarrow Y$}
    \step{5}{$r \circ i = \id{Y}$}
    \qed
\end{proof}

\begin{proposition}[Choice]
    Every absolute retract has the universal extension property.
\end{proposition}

\begin{proof}
    \pf
    \step{1}{\pflet{$Y$ be an absolute retract.}}
    \step{2}{\pflet{$X$ be a normal space.}}
    \step{3}{\pflet{$A$ be closed in $X$.}}
    \step{4}{\pflet{$f : A \rightarrow Y$ be continuous.}}
    \step{5}{\assume{w.l.o.g. $X \cap Y = \emptyset$}}
    \step{6}{\pflet{$\sim$ be the equivalence relation on $X \cup Y$ generated by $a \sim f(a)$ for all $a \in A$.}}
    \step{7}{\pflet{$Z_f$ be the quotient space $Z_f = (X \cup Y) / \sim$ with canonical map $\pi : X \cup Y \twoheadrightarrow Z_f$}}
    \step{8}{$Z_f$ is normal.}
    \begin{proof}
        \step{a}{$Z_f$ is $T_1$}
        \begin{proof}
            \step{i}{\pflet{$p \in Z_f$} \prove{$\inv{\pi}(p)$ is closed.}}
            \step{ii}{\case{$p = \pi(a)$ for some $a \in A$}}
            \begin{proof}
                \pf\ Then $\inv{\pi}(p) = \{ f(a) \} \cup \inv{f}(f(a))$.
            \end{proof}
            \step{iii}{\case{$p = \pi(x)$ for some $x \in X - A$}}
            \begin{proof}
                \pf\ Then $\inv{\pi}(p) = \{x\}$
            \end{proof}
            \step{iii}{\case{$p = \pi(y)$ for some $y \in Y - f(A)$}}
            \begin{proof}
                \pf\ Then $\inv{\pi}(p) = \{ y \}$.
            \end{proof}
        \end{proof}
        \step{b}{Any two disjoint closed sets in $Z_f$ can be separated by a continuous function.}
        \begin{proof}
            \step{i}{\pflet{$C$ and $D$ be disjoint closed sets in $Z_f$}}
            \step{ii}{\pick\ $g : Y \rightarrow I$ that separates $\inv{\pi}(C)$ from $\inv{\pi}(D)$}
            \begin{proof}
                \pf\ Urysohn Lemma.
            \end{proof}
            \step{iii}{Define $h : X \rightarrow I$ by
            \begin{align*}
                h(a) & = g(f(a)) & (a \in A) \\
                h(x) & = 0 & (x \in \inv{\pi}(C)) \\
                h(x) & = 1 & (x \in \inv{\pi}(D))
            \end{align*}}
            \step{iv}{$h$ is continuous}
            \begin{proof}
                \pf\ Pasting Lemma.
            \end{proof}
            \step{v}{\pick\ a continuous extension $k : X \rightarrow I$ of $h$ to $X$}
            \begin{proof}
                \pf\ Tietze Extension Theorem.
            \end{proof}
            \step{vi}{\pflet{$l : Z_f \rightarrow I$ be the function such that $l \circ \pi = [k,g] : X \cup Y \rightarrow I$}}
            \begin{proof}
                \pf\ Since $k(a) = g(f(a))$ for all $a \in A$.
            \end{proof}
            \step{vii}{$l$ separactes $C$ from $D$.}
        \end{proof}
    \end{proof}
    \step{9}{$p \restriction Y : Y \rightarrow Z_f$ is an imbedding.}
    \step{10}{$p(Y)$ is closed in $Z_f$.}
    \step{11}{\pick\ a retraction $r : Z_f \rightarrow Y$ such that $r \circ (p \restriction Y) = \id{Y}$.}
    \step{12}{$r \circ (p \restriction X) : X \rightarrow Y$ extends $f$.}
    \qed
\end{proof}

\section{Completely Normal Spaces}

\begin{definition}[Completely Normal]
    A topological space is \emph{completely normal} if and only if every subspace is
    normal.
\end{definition}

\begin{proposition}
    Every subspace of a completely normal space is completely normal.
\end{proposition}

\begin{proof}
    \pf\ From Proposition \ref{proposition:subspace_transitive}. \qed
\end{proof}

\section{Separated Sets}

\begin{definition}[Separated]
    Let $X$ be a topological space and $A, B \subseteq X$. Then $A$ and $B$ are
    \emph{separated} if and only if $\overline{A} \cap B = A \cap \overline{B} = \emptyset$.
\end{definition}

\begin{proposition}
    \label{proposition:completely_normal_separated}
    Let $X$ be a topological space. Then $X$ is completely normal if and only if,
    for all $A, B \subseteq X$, if $A$ and $B$ are separated then there exist
    disjoint open neighbourhoods $U$ and $V$ of $A$ and $B$ respectively.
\end{proposition}

\begin{proof}
    \pf
    \step{1}{\pflet{$X$ be a topological space.}}
    \step{2}{If $X$ is completely normal then every pair of separated sets in $X$ have disjoint
    open neighbourhoods.}
    \begin{proof}
        \step{a}{\assume{$X$ is completely normal.}}
        \step{b}{\pflet{$A$ and $B$ be separated sets in $X$.}}
        \step{c}{$A$ and $B$ are disjoint closed sets in $X - (\overline{A} \cap \overline{B})$.}
        \step{d}{\pick\ disjoint open neighbourhoods $U$ and $V$ of $A$ and $B$ in $X - (\overline{A} \cap \overline{B})$.}
        \step{e}{$U$ and $V$ are disjoint open neighbourhoods of $A$ and $B$ in $X$.}
    \end{proof}
    \step{3}{If every pair of separated sets in $X$ have disjoint open neighbourhoods then
    $X$ is completely normal.}
    \begin{proof}
        \step{a}{\assume{Every pair of separated sets in $X$ have disjoint open neighbourhoods.}}
        \step{b}{\pflet{$Y \subseteq X$}}
        \step{c}{\pflet{$A$ and $B$ be disjoint closed sets in $Y$.}}
        \step{d}{$A$ and $B$ are separated in $X$.}
        \step{e}{\pick\ disjoint open neighbourhoods $U$ and $V$ of $A$ and $B$ in $X$.}
        \step{f}{$U \cap Y$ and $V \cap Y$ are disjoint open neighbourhoods of $A$ and $B$ in $Y$.}
    \end{proof}
    \qed
\end{proof}

\begin{example}
    The space $\RR_l$ is completely normal.

    \begin{proof}
        \pf
        \step{1}{\pflet{$A$ and $B$ be separated sets in $\RR_l$.}}
        \step{2}{For all $a \in A$, \pick\ $x_a \in X$ such that $[a,x_a) \cap B = \emptyset$}
        \step{3}{For all $b \in B$, \pick\ $y_b \in X$ such that $[b,y_b) \cap A = \emptyset$}
        \step{4}{\pflet{$U = \bigcup_{a \in A} [a,x_a)$}}
        \step{5}{\pflet{$V = \bigcup_{b \in B} [b,y_b)$}}
        \step{6}{$U$ and $V$ are open neightbourhoods of $A$ and $B$ respectively.}
        \step{7}{$U \cap V = \emptyset$}
        \begin{proof}
            \step{a}{\assume{for a contradiction $x \in U \cap V$}}
            \step{b}{\pick\ $a \in A$ and $b \in B$ such that $x \in [a,x_a)$ and
            $x \in [b,y_b)$}
            \step{c}{\assume{w.l.o.g. $a < b$}}
            \step{d}{$b \in [a,x_a)$}
            \begin{proof}
                \pf\ Since $a < b \leq x < x_a$.
            \end{proof}
            \qedstep
            \begin{proof}
                \pf\ This contradicts the fact that $[a,x_a) \cap B = \emptyset$ (\stepref{2}).
            \end{proof}
        \end{proof}
        \qedstep
        \begin{proof}
            \pf\ Proposition \ref{proposition:completely_normal_separated}.
        \end{proof}
        \qed
    \end{proof}
\end{example}

\begin{corollary}
    The product of two completely normal spaces is not necessarily completely normal.
\end{corollary}

\begin{proof}
    \pf\ Since $\RR_l^2$ is not normal. \qed
\end{proof}

\begin{proposition}
    Every linearly ordered set in the order topology is completely normal.
\end{proposition}

\begin{proof}
    \pf\ ???? \qed
\end{proof}

\begin{example}
    Not every compact Hausdorff space is completely normal.
\end{example}

\begin{proof}
    \pf\ The space $\overline{S_\Omega}^2$ is compact Hausdorff, but its subspace
    $S_\Omega \times \overline{S_\Omega}$ is not normal. \qed
\end{proof}

\begin{proposition}
    Every second countable regular space is completely normal.
\end{proposition}

\begin{proof}
    \pf\ Since a subspace of a second countable regular space is second countable
    and regular, hence normal. \qed
\end{proof}

\section{Vanish Precisely}

\begin{definition}[Vanish Precisely]
Let $X$ be a set and $f : X \rightarrow [0,1]$. Let $A \subseteq X$. Then $f$
\emph{vanishes precisely} on $A$ if and only if $A = \inv{f}(0)$.    
\end{definition}

\begin{theorem}
    \label{theorem:vanish_precisely_closed_gdelta}
    Let $X$ be a normal space. Let $A \subseteq X$. Then there exists a 
    continuous function
    $f : X \rightarrow [0,1]$ that vanishes precisely on $A$ if and only if
    $A$ is closed and $G_\delta$.
\end{theorem}

\begin{proof}
    \pf
    \step{1}{If there exists a continuous $f : X \rightarrow [0,1]$
    that vanishes precisely on $A$ then $A$ is closed.}
    \begin{proof}
        \pf\ Since $A = \inv{f}(\{0\})$. \qed
    \end{proof}
    \step{2}{If there exists a continuous $f : X \rightarrow [0,1]$
    that vanishes precisely on $A$ then $A$ is $G_\delta$.}
    \begin{proof}
        \pf\ Since $A = \bigcap_{n \in \ZZ^+} \inv{f}((-1/n,1/n))$.
    \end{proof}
    \step{3}{If $A$ is closed and $G_\delta$ then there exists a continuous
    $f : X \rightarrow [0,1]$ that vanishes precisely on $A$.}
    \begin{proof}
        \step{a}{\assume{$A$ is closed and $G_\delta$.}}
        \step{b}{\pick\ a sequence $U_n$ of open sets such that $A = \bigcap_n U_n$}
        \step{c}{For $n \in \ZZ^+$ \pick\ a continuous $\phi_n : X \rightarrow [0,1]$
        such that $\phi_n(x) = 0$ for $x \in A$ and $\phi_n(x) = 1$
        for $x \in X - U_n$}
        \begin{proof}
            \pf\ By the Urysohn Lemma.
        \end{proof}
        \step{d}{Define $f : X \rightarrow [0,1]$ by $f(x) = \sum_{n=1}^\infty
        1/2^n f_n(x)$}
        \step{e}{$f$ is continuous.}
        \begin{proof}
            \pf\ By the Weierstrass $M$-test.
        \end{proof}
        \step{f}{$f$ vanishes precisely on $A$.}
    \end{proof}
    \qed
\end{proof}

\begin{theorem}[Strong Form of the Urysohn Lemma]
    Let $X$ be a normal space. Let $A, B \subseteq X$. Then there exists
    a continuous function $f : X \rightarrow [0,1]$ such that
    $A = \inv{f}(0)$ and $B = \inv{f}(1)$ if and only if
    $A$ and $B$ are disjoint closed $G_\delta$ sets.
\end{theorem}

\begin{proof}
    \pf
    \step{1}{For any continuous $f : X \rightarrow [0,1]$, we have
    $\inv{f}(0)$ and $\inv{f}(1)$ are disjoint closed $G_\delta$ sets.}
    \begin{proof}
        \pf\ By Theorem \ref{theorem:vanish_precisely_closed_gdelta}.
    \end{proof}
    \step{2}{If $A$ and $B$ are disjoint closed $G_\delta$ sets, then there exists
    a continuous $f : X \rightarrow [0,1]$ such that $A = \inv{f}(0)$
    and $B = \inv{f}(1)$.}
    \begin{proof}
        \step{a}{\assume{$A$ and $B$ are disjoint closed $G_\delta$ sets.}}
        \step{b}{\pick\ a continuous function $g : X \rightarrow [0,1]$ that vanishes
        precisely on $A$.}
        \begin{proof}
            \pf\ By Theorem \ref{theorem:vanish_precisely_closed_gdelta}.
        \end{proof}
        \step{c}{\pick\ a continuous function $h : X \rightarrow [0,1]$ that vanishes
        precisely on $B$.}
        \begin{proof}
            \pf\ By Theorem \ref{theorem:vanish_precisely_closed_gdelta}.
        \end{proof}
        \step{d}{\pflet{$f : X \rightarrow [0,1]$ by $f(x) = f(x)/(f(x) + g(x))$}}
        \step{e}{$f$ is continuous}
        \step{f}{$A = \inv{f}(0)$}
        \step{g}{$B = \inv{f}(1)$}
    \end{proof}
    \qed
\end{proof}

\section{Perfectly Normal Spaces}

\begin{definition}[Perfectly Normal]
    A topological space is \emph{perfectly normal} if and only if it is normal
    and every closed set is $G_\delta$.
\end{definition}

\begin{proposition}
    Every perfectly normal space is completely normal.
\end{proposition}

\begin{proof}
    \pf
    \step{1}{\pflet{$X$ be a perfectly normal space.}}
    \step{2}{$A$ and $B$ be separated sets in $X$.}
    \step{3}{\pick\ $f, g : X \rightarrow [0,1]$ such that $f$ vanishes
    precisely on $\overline{A}$ and $g$ vanishes precisely on $\overline{B}$}
    \begin{proof}
        \pf\ Theorem \ref{theorem:vanish_precisely_closed_gdelta}.
    \end{proof}
    \step{4}{\pflet{$h = f - g : X \rightarrow [-1,1]$}}
    \step{5}{$\inv{h}([-1,0))$ and $\inv{h}((0,1])$ are disjoint open neighbourhoods
    of $B$ and $A$ respectively.}
    \qedstep
    \begin{proof}
        \pf\ Proposition \ref{proposition:completely_normal_separated}.
    \end{proof}
    \qed
\end{proof}

\begin{example}
    The space $\overline{S_\Omega}$ is completely normal but not perfectly normal,
    because $\{\Omega\}$ is not $G_\delta$.
\end{example}

\section{Coherent Topology}

\begin{definition}[Coherent Topology]
    Let $(X_n)$ be a sequence of topological spaces such that $X_n$ is a closed subspace of $X_{n+1}$ for all $n$.
    Let $X = \bigcup_n X_n$. The topology on $X$ \emph{coherent} with the subspaces $X_i$ is $\{ U \in \pow X \mid \forall n.
    U \cap X_n \text{ is open in } X_n \}$.

    We prove this is a topology.
\end{definition}

\begin{proof}
    \pf
    \step{1}{$X$ is open.}
    \begin{proof}
        \pf\ For all $n$, we have $X \cap X_n = X_n$ is open in $X_n$.
    \end{proof}
    \step{2}{For any set $\UU$ of open sets, we have $\bigcup \UU$ is open.}
    \begin{proof}
        \pf\ For all $n$, we have $(\bigcup \UU) \cap X_n = \bigcup \{ U \cap X_n \mid U \in \UU \}$ is open in $X_n$.
    \end{proof}
    \step{3}{For any open sets $U$ and $V$, we have $U \cap V$ is open.}
    \begin{proof}
        \pf\ For all $n$, we have $U \cap V \cap X_n = (U \cap X_n) \cap (V \cap X_n)$ is open in $X_n$.
    \end{proof}
    \qed
\end{proof}

\begin{proposition}
    \label{proposition:coherent_continuous}
    Let $(X_n)$ be a sequence of topological spaces such that $X_n$ is a closed subspace of $X_{n+1}$ for all $n$.
    Let $X = \bigcup_n X_n$ with the coherent topology. Let $Y$ be a topological space and $f : X \rightarrow Y$.
    Then $f$ is continuous if and only if, for all $n$, we have $f \restriction X_n$ is continuous.
\end{proposition}

\begin{proof}
    \pf\ Both are equivalent to: for all $V$ open in $Y$ and all $n$, we have $\inv{f}(V) \cap X_n$ is open in $X_n$. \qed
\end{proof}

\begin{proposition}[Choice]
    Let $(X_n)$ be a sequence of normal spaces such that $X_n$ is a closed subspace of $X_{n+1}$ for all $n$.
    Let $X = \bigcup_n X_n$ under the coherent topology. Then $X$ is normal.
\end{proposition}

\begin{proof}
    \pf
    \step{1}{\pflet{$A$ and $B$ be disjoint closed sets in $X$.}}
    \step{2}{\pick\ a sequence $(f_n)$ of continuous functions $f_n : A \cup B \cup X_n \rightarrow I$
    such that $f_n(A) = \{0\}$ and $f_n(B)=\{1\}$}
    \begin{proof}
        \pf\ Given such a function on $A \cup B \cup X_n$, extend it to $A \cup B \cup X_{n+1}$ by the Tietze Extension Theorem,
        where we take $X_0 = \emptyset$.
    \end{proof}
    \step{3}{Define $f : X \rightarrow I$ by $f(x) = f_n(x)$ for $x \in X_n$}
    \step{4}{$f$ is continuous.}
    \begin{proof}
        \pf\ Proposition \ref{proposition:coherent_continuous}.
    \end{proof}
    \step{5}{$f$ separates $A$ from $B$.}
    \qed
\end{proof}

\section{Support}

\begin{definition}[Support]
    Let $X$ be a topological space. Let $\phi : X \rightarrow \RR$. The \emph{support} of $\phi$ is
    \[ \support \phi = \overline{\inv{\phi}(\RR - \{0\}} \enspace . \]
\end{definition}

\section{Partitions of Unity}

\begin{definition}[Partition of Unity]
    Let $X$ be a topological space.
    Let $(U_1, \ldots, U_n)$ be a finite open covering of $X$. A \emph{partition of unity dominated by $(U_1, \ldots, U_n)$}
    is an $n$-tuple of continuous functions $\phi_1, \ldots, \phi_n : X \rightarrow I$ such that:
    \begin{itemize}
        \item For each $i$ we have $\support \phi_i \subseteq U_i$
        \item For $x \in X$ we have $\sum_{i=1}^n \phi_i(x) = 1$.
    \end{itemize}
\end{definition}

\begin{theorem}
    \label{theorem:partition_of_unity}
    Let $X$ be a normal space. Let $(U_1, \ldots, U_n)$ be a finite open covering of $X$. Then there exists a partition of unity
    dominated by $(U_1, \ldots, U_n)$.
\end{theorem}

\begin{proof}
    \pf
    \step{1}{\pick\ a finite open covering $(V_1, \ldots, V_n)$ such that, for each $i$, $\overline{V_i} \subseteq U_i$}
    \begin{proof}
        \pf\ Shrinking Lemma.
    \end{proof}
    \step{2}{\pick\ a finite open covering $(W_1, \ldots, W_n)$ such that, for each $i$, $\overline{W_i} \subseteq V_i$}
    \begin{proof}
        \pf\ Shrinking Lemma.
    \end{proof}
    \step{3}{For each $i$, \pick\ a continuous $\psi_i : X \rightarrow I$ such that $\psi_i(\overline{W_i}) = \{1\}$ and $\psi_i(X-V_i) = \{0\}$}
    \begin{proof}
        \pf\ Urysohn Lemma.
    \end{proof}
    \step{4}{For each $i$, $\support \psi_i \subseteq U_i$}
    \step{5}{For all $x \in X$, we have $\sum_{i=1}^n \psi_i(x) > 0$}
    \step{6}{For each $i$, define $\phi_i : X \rightarrow I$ by $\phi_i(x) = \psi_i(x) / \sum_{j=1}^n \psi_j(x)$}
    \step{7}{$(\phi_1, \ldots, \phi_n)$ is a partition of unity dominated by $(U_1, \ldots, U_n)$.}
    \qed
\end{proof}

\begin{theorem}
    \label{theorem:imbed_in_RN}
    Let $X$ be a compact Hausdorff space such that, for every $x \in X$, there exists an open neighbourhood $U$ of $x$ and a positive integer $k$
    such that $U$ can be imbedded in $\RR^k$. Then there exists $N$ such that $X$ can be imbedded in $\RR^N$.
\end{theorem}

\begin{proof}
    \pf
    \step{2}{\pick\ a finite open covering $(U_1, \ldots, U_n)$ of $X$ such that each $U_i$ is homeomorphic to an open subset of $\RR^{k_i}$
    via $g_i : U_i \rightarrow \RR^{k_i}$.}
    \step{3}{$X$ is normal.}
    \begin{proof}
        \pf\ Since it is compact Hausdorff.
    \end{proof}
    \step{4}{\pick\ a partition of unity $(\phi_1, \ldots, \phi_n)$ dominated by $(U_1, \ldots, U_n)$.}
    \begin{proof}
        \pf\ Theorem \ref{theorem:partition_of_unity}.
    \end{proof}
    \step{5}{For all $i$, \pflet{$A_i = \support \phi_i$}}
    \step{6}{For all $i$, define $h_i : X \rightarrow \RR^{k_i}$ by
    \[ h_i(x) = \begin{cases}
        \phi_i(x) g_i(x) & \text{if } x \in U_i \\
        0 & \text{if } x \in X - A_i
    \end{cases} \]}
    \step{7}{For all $i$, $h_i$ is continuous.}
    \begin{proof}
        \pf\ Pasting Lemma.
    \end{proof}
    \step{8}{Define $F : X \rightarrow \RR^{n + k_1 + \cdots + k_n}$ by $F(x)$ is the concatenation of $(\phi_1(x), \ldots, \phi_n(x))$ and $h_1(x)$, \ldots,
    $h_n(x)$}
    \step{9}{$F$ is continuous.}
    \step{10}{$F$ is injective.}
    \begin{proof}
        \step{a}{\pflet{$x, y \in X$}}
        \step{b}{\assume{$F(x) = F(y)$}}
        \step{c}{\pick\ $i$ such that $\phi_i(x) > 0$}
        \step{d}{$\phi_i(y) > 0$}
        \step{e}{$x, y \in U_i$}
        \step{f}{$h_i(x) = h_i(y)$}
        \step{g}{$g_i(x) = g_i(y)$}
        \step{h}{$x = y$}
    \end{proof}
    \step{11}{$F$ is an imbedding.}
    \begin{proof}
        \pf\ Since $X$ is compact.
    \end{proof}
    \qed
\end{proof}

\section{The Line With Two Origins}

The line with two origins is an example of a space that satisfies all the conditions in the definition of a 1-manifold
except the Hausdorff condition.

\begin{definition}[Line With Two Origins]
    The \emph{line with two origins} $L$ is $(\RR - \{0\}) \cup \{a,b\}$ under the topology
    generated by the basis consisting of all open intervals in $\RR$ that do not contain $0$,
    and all sets of the form $(-x,0) \cup \{ a \} \cup (0,x)$ and $(-x,0) \cup \{b\} \cup (0,x)$
    where $x > 0$.
\end{definition}

\begin{proposition}
    The line with two origins is $T_1$.
\end{proposition}

\begin{proof}
    \pf\ Easy.
\end{proof}

\begin{proposition}
    The line with two origins is not Hausdorff.
\end{proposition}

\begin{proof}
    \pf\ There are no disjoint open neighbourhoods of $a$ and $b$. \qed
\end{proof}

\begin{proposition}
    The line with two origins is second countable.
\end{proposition}

\begin{proof}
    \pf\ Easy. \qed
\end{proof}

\begin{proposition}
    Every point in the line with two origins has an open neighbourhood that is homemorphic to an open subpace of $\RR$.
\end{proposition}

\begin{proof}
    \pf\ Easy. \qed
\end{proof}

\section{Countably Locally Finite Sets}

\begin{definition}[Countably Locally Finite]
    Let $X$ be a topological space and $\BB \subseteq \pow X$. Then $\BB$ is \emph{countably locally finite} if and only if
    $\BB$ is the union of countably many sets, each locally finite.
\end{definition}

\begin{proposition}
    Let $X$ be a second countable space and $\AA \subseteq \pow X$. Then $\AA$ is countably locally finite if and only if it is countable.
\end{proposition}

\begin{proof}
    \pf
    \step{0}{Every locally finite set is countable.}
    \begin{proof}
        \step{a}{\pflet{$\mathcal{C}$ be locally finite.}}
        \step{aa}{\assume{w.l.o.g. $\emptyset \notin \mathcal{C}$}}
        \step{b}{\pick\ a countable basis $\BB$ of $X$.}
        \step{c}{For $B \in \BB$ that intersects only finitely many elements of $\mathcal{C}$, let those elements of $\mathcal{C}$
        be $C_{B1}$, \ldots, $C_{Bn_B}$.}
        \step{d}{$\mathcal{C} = \{ C_{Bk} \mid B \in \BB, B \text{ intersects only finitely many elements of } \mathcal{C},
        1 \leq k \leq n_B \}$}
        \begin{proof}
            \step{i}{\pflet{$C \in \mathcal{C}$}}
            \step{ii}{\pick\ $x \in C$}
            \step{iii}{\pick\ $B \in \BB$ such that $x \in B$ and $B$ intersects only finitely many elements of $\mathcal{C}$}
            \step{iv}{$B$ intersects $C$}
            \step{v}{$C = C_{Bk}$ for some $k$.}
        \end{proof}
    \end{proof}
    \step{1}{If $\AA$ is countably locally finite then it is countable.}
    \step{2}{If $\AA$ is countable then it is countably locally finite.}
    \begin{proof}
        \pf\ Since it is a countable union of finite sets.
    \end{proof}
    \qed
\end{proof}

\begin{lemma}
    \label{lemma:regular_countably_locally_finite_basis_perfectly_normal}
    Every regular space with a countably locally finite basis is perfectly normal.
\end{lemma}

\begin{proof}
    \pf
    \step{1}{\pflet{$X$ be a regular space with a countably locally finite basis.}}
    \step{2}{\pick\ a countably locally finite basis $\BB$ for $X$.}
    \step{3}{\pick\ a sequence $(\BB_n)$ of locally finite sets such that $\BB = \bigcup_n \BB_n$}
    \step{3}{For every open set $W$, there exists a countable set $\UU$ of open sets such that $W = \bigcup \UU = \bigcup \{ \overline{U}
    \mid U \in \UU \}$}
    \begin{proof}
        \step{a}{\pflet{$W$ be open in $X$.}}
        \step{b}{For $n \in \ZZ^+$, \pflet{$\mathcal{C}_n = \{ B \in \BB_n \mid \overline{B} \subseteq W \}$}}
        \step{c}{For $n \in \ZZ^+$, $\mathcal{C}_n$ is locally finite.}
        \step{d}{For $n \in \ZZ^+$, \pflet{$U_n = \bigcup \mathcal{C}_n$}}
        \step{e}{For $n \in \ZZ^+$, $U_n$ is open.}
        \step{f}{For $n \in \ZZ^+$, $\overline{U_n} = \bigcup_{B \in \mathcal{C}_n} \overline{B}$}
        \step{g}{For $n \in \ZZ^+$, $\overline{U_n} \subseteq W$}
        \step{h}{$\bigcup_n U_n \subseteq \bigcup_n \overline{U_n} \subseteq W$}
        \step{i}{$W \subseteq \bigcup_n U_n$}
        \begin{proof}
            \step{i}{\pflet{$x \in W$}}
            \step{ii}{\pick\ $B \in \BB$ such that $x \in B$ and $\overline{B} \in W$}
            \step{iii}{\pick\ $n$ such that $B \in \BB_n$}
            \step{iv}{$B \in \mathcal{C}_n$}
            \step{v}{$x \in U_n$}
        \end{proof}
    \end{proof}
    \step{4}{Every closed set is $G_\delta$.}
    \begin{proof}
        \step{a}{\pflet{$C$ be closed in $X$.}}
        \step{b}{\pick\ a countable set $\UU$ of open sets such that $X - C = \bigcup_{U \in \UU} \overline{U}$}
        \step{c}{$C = \bigcap_{U \in \UU} (X - \overline{U})$}
    \end{proof}
    \step{5}{$X$ is normal.}
    \begin{proof}
        \step{a}{\pflet{$C$ and $D$ be disjoint closed sets.}}
        \step{b}{\pick\ a countable sequence $(U_n)$ of open sets such that $X - D = \bigcup_n U_n = \bigcup_n \overline{U_n}$}
        \step{e}{\pick\ a countable sequence $(V_n)$ of open sets such that $X - C = \bigcup_n V_n = \bigcup_n \overline{V_n}$}
        \step{f}{For $n \in \ZZ^+$, \pflet{$U_n' = U_n - \bigcup_{i=1}^n \overline{V_i}$ and $V_n' = V_n - \bigcup_{i=1}^n \overline{U_i}$}}
        \step{g}{\pflet{$U' = \bigcup_n U_n'$ and $V' = \bigcup_n V_n'$}}
        \step{h}{$U'$ and $V'$ are disjoint open neighbourhoods of $C$ and $D$ respectively.}
    \end{proof}
    \qed
\end{proof}

\section{Open Refinements}

\begin{definition}[Open Refinement]
    Let $X$ be a topological space. Let $\AA, \BB \subseteq \pow X$. Then $\BB$ is an \emph{open refinement} of
    $\AA$ if and only if $\BB$ refines $\AA$ and every element of $\BB$ is open.
\end{definition}

\section{Closed Refinements}

\begin{definition}[Closed Refinement]
    Let $X$ be a topological space. Let $\AA, \BB \subseteq \pow X$. Then $\BB$ is an \emph{closed refinement} of
    $\AA$ if and only if $\BB$ refines $\AA$ and every element of $\BB$ is closed.
\end{definition}

\section{$F_\sigma$ Sets}

\begin{definition}[$F_\sigma$ Set]
    Let $X$ be a topological space. Let $W \subseteq X$. Then $W$ is \emph{$F_\sigma$} if and only if
    it is a countable union of closed sets.
\end{definition}

\begin{proposition}
    Let $X$ be a topological space. Let $W \subseteq X$. Then $W$ is $F_\sigma$ if and only if $X - W$
    is $G_\delta$.
\end{proposition}

\begin{proof}
    \pf\ Immediate from definitions. \qed
\end{proof}

\section{Locally Discrete Sets}

\begin{definition}[Locally Discrete Set]
    Let $X$ be a topological space. Let $\AA \subseteq \pow X$. Then $\AA$ is \emph{locally discrete}
    if and only if every point has a neighbourhood that intersects exactly one element of $\AA$.
\end{definition}

\section{Countably Locally Discrete Sets}

\begin{definition}[Countably Locally Discrete Set]
    Let $X$ be a topological space. Let $\AA \subseteq \pow X$. Then $\AA$ is \emph{countably locally discrete}
    if and only if $\AA$ is a countable union of locally discrete sets.
\end{definition}

\section{Paracompactness}

\begin{definition}[Paracompact Space]
    A topological space $X$ is \emph{paracompact} if and only if every open covering of $X$ has a locally finite open
    refinement that covers $X$.
\end{definition}

\begin{proposition}
    Every paracompact Hausdorff space is normal.
\end{proposition}

\begin{proof}
    \pf
    \step{1}{\pflet{$X$ be a paracompact Hausdorff space.}}
    \step{2}{$X$ is regular.}
    \begin{proof}
        \step{a}{\pflet{$a \in X$}}
        \step{b}{\pflet{$B$ be a closed set in $X$ disjoint from $a$.}}
        \step{c}{\pflet{$\AA = \{ U \text{ open in } X \mid a \notin \overline{U} \} \cup \{ X - B \}$}}
        \step{d}{$\AA$ is an open cover of $X$.}
        \begin{proof}
            \pf\ For all $x \in B$ there exists an open neighbourhood $U$ of $x$ such that $a \notin \overline{U}$
            since $X$ is Hausdorff.
        \end{proof}
        \step{e}{\pick\ a localy finite open refinement $\mathcal{C}$ that covers $X$.}
        \step{f}{\pflet{$\DD = \{ C \in \mathcal{C} \mid C \text{ intersects } B \}$}}
        \step{g}{$\DD$ covers $B$.}
        \step{h}{For all $D \in \DD$, $a \notin \overline{D}$}
        \step{i}{\pflet{$V = \bigcup \DD$}}
        \step{j}{$V$ is an open neighbourhood of $B$.}
        \step{k}{$\overline{V} = \bigcup_{D \in \DD} \overline{D}$}
        \step{l}{$a \notin \overline{V}$}
    \end{proof}
    \step{3}{Any two disjoint closed sets in $X$ have disjoint open neighbourhoods.}
    \begin{proof}
        \step{a}{\pflet{$A$ and $B$ be disjoint closed sets in $X$.}}
        \step{c}{\pflet{$\AA = \{ U \text{ open in } X \mid A \cap \overline{U} = \emptyset \} \cup \{ X - B \}$}}
        \step{d}{$\AA$ is an open cover of $X$.}
        \begin{proof}
            \pf\ For all $x \in B$, there exists an open neighbourhood $U$ of $x$ such that $A \cap \overline{U} = \emptyset$
            by \stepref{2}.
        \end{proof}
        \step{e}{\pick\ a localy finite open refinement $\mathcal{C}$ that covers $X$.}
        \step{f}{\pflet{$\DD = \{ C \in \mathcal{C} \mid C \text{ intersects } B \}$}}
        \step{g}{$\DD$ covers $B$.}
        \step{h}{For all $D \in \DD$, $A \cap \overline{D} = \emptyset$}
        \step{i}{\pflet{$V = \bigcup \DD$}}
        \step{j}{$V$ is an open neighbourhood of $B$.}
        \step{k}{$\overline{V} = \bigcup_{D \in \DD} \overline{D}$}
        \step{l}{$A \cap \overline{V} = \emptyset$}
    \end{proof}
    \qed
\end{proof}

\begin{corollary}
    Every compact Hausdorff space is normal.
\end{corollary}

\begin{corollary}
    Every locally compact Hausdorff space is completely regular.
\end{corollary}

\begin{theorem}
    Every closed subspace of a paracompact space is paracompact.
\end{theorem}

\begin{proof}
    \pf
    \step{1}{\pflet{$X$ be a paracompact space.}}
    \step{2}{\pflet{$Y \subseteq X$ be closed.}}
    \step{3}{\pflet{$\AA$ be an open covering of $Y$.}}
    \step{4}{\pflet{$\BB = \{ U \text{ open in } X \mid U \cap Y \in \AA \} \cup \{ X - Y \}$}}
    \step{5}{$\BB$ is an open covering of $X$.}
    \step{6}{\pick\ a locally finite open refinement $\CC$ that covers $X$.}
    \step{7}{\pflet{$\DD = \{ C \cap Y \mid C \in \CC \}$}
    \prove{$\DD$ is a locally finite open refinement of $\AA$ that covers $Y$.}}
    \step{8}{$\DD$ is locally finite.}
    \step{9}{$\DD$ refines $\AA$}
    \step{10}{$\DD$ covers $Y$.}
    \qed
\end{proof}

The following example shows that an open subspace of a paracompact space is not necessarily paracompact.

\begin{proposition}
    The space $S_\Omega \times \overline{S_\Omega}$ is not paracompact.
\end{proposition}

\begin{proof}
    \pf\ Since it is Hausdorff but not normal. \qed
\end{proof}

\begin{lemma}[E. Michael (Choice)]
    \label{lemma:paracompactness}
    Let $X$ be a regular space. Then the following are equivalent.
    \begin{enumerate}
        \item Every open covering of $X$ has a countably locally finite open refinement that covers $X$.
        \item Every open covering of $X$ has a locally finite refinement that covers $X$.
        \item Every open covering of $X$ has a locally finite closed refinement that covers $X$.
        \item $X$ is paracompact.
    \end{enumerate}
\end{lemma}

\begin{proof}
    \pf
    \step{1}{$1 \Rightarrow 2$}
    \begin{proof}
        \step{a}{\assume{1}}
        \step{b}{\pflet{$\AA$ be an open covering of $X$.}}
        \step{c}{\pick\ a countably locally finite open refinement $\BB$ of $\AA$ that covers $X$.}
        \step{d}{\pick\ a sequence $(\BB_n)$ of locally finite sets such thath $\BB = \bigcup_n \BB_n$}
        \step{e}{For $i \in \ZZ^+$, \pflet{$V_i = \bigcup \BB_i$}}
        \step{f}{For $n \in \ZZ^+$ and $U \in \BB_n$, \pflet{$S_n(U) = U - \bigcup_{i < n} V_i$}}
        \step{f}{For $n \in \ZZ^+$, \pflet{$\mathcal{C}_n = \{ S_n(U) \mid U \in \BB_n \}$}}
        \step{g}{For all $n \in \ZZ^+$, $\mathcal{C}_n$ refines $\mathcal{B}_n$.}
        \step{h}{\pflet{$\mathcal{C} = \bigcup_n \mathcal{C}_n$} \prove{$\mathcal{C}$ is a locally finite refinement of $\AA$
        that covers $X$.}}
        \step{i}{$\mathcal{C}$ covers $X$.}
        \begin{proof}
            \step{i}{\pflet{$x \in X$}}
            \step{ii}{\pflet{$N$ be least such that $x \in \bigcup \BB_N$}}
            \step{iii}{\pick\ $U \in \BB_N$ such that $x \in U$}
            \step{iv}{$x \in S_N(U)$}
        \end{proof}
        \step{j}{$\mathcal{C}$ is locally finite.}
        \begin{proof}
            \step{i}{\pflet{$x \in X$}}
            \step{ii}{\pflet{$N$ be least such that $x \in \bigcup \BB_N$}}
            \step{iii}{\pick\ $U \in \BB_N$ such that $x \in U$}
            \step{iv}{For $n = 1, \ldots, N$, \pick\ a neighbourhood $W_n$ of $x$ that intersects only finitely many
            elements of $\BB_n$}
            \step{v}{For $n = 1, \ldots, N$ and $V \in \BB_n$, if $W_n$ intersects $S_n(V)$ then $W_n$ intersects $V$.}
            \step{vi}{For $n = 1, \ldots, N$, $W_n$ intersects only finitely many elements of $\mathcal{C}_n$.}
            \step{vii}{For $n = 1, \ldots, N$, $U$ intersects no elements of $\mathcal{C}_n$.}
            \step{viii}{$W_1 \cap \cdots \cap W_N \cap U$ is an open neighbourhood of $x$ that intersects only finitely many elements of
            $\mathcal{C}$.}
        \end{proof}
    \end{proof}
    \step{2}{$2 \Rightarrow 3$}
    \begin{proof}
        \step{a}{\assume{2}}
        \step{b}{\pflet{$\AA$ be an open covering of $X$.}}
        \step{c}{\pflet{$\BB = \{ U \text{ open in } X \mid \exists V \in \AA. \overline{U} \subseteq V \}$}}
        \step{d}{$\BB$ covers $X$.}
        \begin{proof}
            \pf\ Since $X$ is regular.
        \end{proof}
        \step{e}{\pick\ a locally finite refinement $\CC$ of $\BB$ that covers $X$.}
        \begin{proof}
            \pf\ By \stepref{a}.
        \end{proof}
        \step{f}{\pflet{$\DD = \{ \overline{C} \mid C \in \CC \}$}}
        \step{g}{$\DD$ covers $X$.}
        \step{g}{$\DD$ is locally finite.}
        \begin{proof}
            \pf\ Lemma \ref{lemma:closures_locally_finite}.
        \end{proof}
        \step{h}{$\DD$ refines $\AA$.}
    \end{proof}
    \step{3}{$3 \Rightarrow 4$}
    \begin{proof}
        \step{a}{\assume{3}}
        \step{b}{\pflet{$\AA$ be an open covering of $X$.}}
        \step{c}{\pick\ a locally finite refinement $\BB$ of $\AA$ that covers $X$.}
        \begin{proof}
            \pf\ By \stepref{a}.
        \end{proof}
        \step{d}{$\{ U \text{ open in } X \mid U \text{ intersects only finitely many elements of } \BB \}$ is an open cover of $X$.}
        \step{e}{\pick\ a locally finite closed refinement $\mathcal{C}$ of 
        $\{ U \text{ open in } X \mid U \text{ intersects only finitely many elements of } \BB \}$
        that covers $X$.}
        \begin{proof}
            \pf\ By \stepref{a}.
        \end{proof}
        \step{f}{Every element of $\mathcal{C}$ intersects only finitely many elements of $\BB$.}
        \step{g}{For $B \in \BB$, \pflet{$\mathcal{C}(B) = \{ C \in \mathcal{C} \mid C \subseteq X - B \}$}}
        \step{h}{For $B \in \BB$, \pflet{$E(B) = X - \bigcup \mathcal{C}(B)$}}
        \step{i}{The union of any subset of $\mathcal{C}$ is closed.}
        \begin{proof}
            \pf\ Corollary \ref{corollary:union_locally_finite_closed}.
        \end{proof}
        \step{j}{For all $B \in \BB$, $E(B)$ is open.}
        \step{k}{For all $B \in \BB$, $B \subseteq E(B)$}
        \step{l}{For all $B \in \BB$, \pick\ $F(B) \in \AA$ such that $B \subseteq F(B)$}
        \begin{proof}
            \pf\ Since $\BB$ refines $\AA$ (\stepref{c}).
        \end{proof}
        \step{m}{\pflet{$\DD = \{ E(B) \cap F(B) \mid B \in \BB \}$}}
        \step{n}{$\DD$ refines $\AA$.}
        \step{o}{$\DD$ covers $X$.}
        \step{p}{$\DD$ is locally finite.}
        \begin{proof}
            \step{i}{\pflet{$x \in X$}}
            \step{ii}{\pick\ a neighbourhood $W$ of $x$ that intersects only finitely many elements of $\mathcal{C}$
            \prove{$W$ intersects only finitely many elements of $\DD$}}
            \step{iii}{\pflet{$C_1$, \ldots, $C_k$ be the elements of $\mathcal{C}$ that intersect $W$.}}
            \step{iv}{$W \subseteq C_1 \cup \cdots \cup C_k$}
            \step{v}{Every element of $\mathcal{C}$ intersects only finitely many elements of $\DD$.}
            \begin{proof}
                \step{one}{\pflet{$C \in \mathcal{C}$}}
                \step{two}{For $B \in \BB$, if $C$ intersects $E(B) \cap F(B)$ then $C$ intersects $B$.}
                \qedstep
                \begin{proof}
                    \pf\ By \stepref{f}.
                \end{proof}
            \end{proof}
            \step{vi}{$W$ intersects only finitely many elements of $\DD$.}
        \end{proof}
    \end{proof}
    \step{4}{$4 \Rightarrow 1$}
    \begin{proof}
        \pf\ Trivial.
    \end{proof}
    \qed
\end{proof}

\begin{corollary}
    \label{corollary:regular_Lindelof_paracompact}
    Every regular Lindel\"{o}f space is paracompact.
\end{corollary}

\begin{corollary}
    The space $\RR_l$ is paracompact.
\end{corollary}

The following example shows that the product of two paracompact spaces is not necessarily paracompact.

\begin{proposition}
    The Sorgenfrey plane is not paracompact.
\end{proposition}

\begin{proof}
    \pf\ It is Hausdorff but not normal. \qed
\end{proof}

\begin{theorem}
    If the Continuum Hypothesis is true then $\RR^\omega$ under the box topology
    is paracompact.
\end{theorem}

\begin{proof}
    \pf\ See M. E. Rudin. The box product of countably many compact metric spaces.
    \emph{General Topology and its Applications}, 2:293--298, 1972. \qed
\end{proof}

\begin{corollary}
    If the Continuum Hypothesis is true then $\RR^\omega$ under the box topology is normal.
\end{corollary}

\begin{proposition}
    For $J$ uncountable, the space $\RR^J$ is not paracompact.
\end{proposition}

\begin{proof}
    \pf\ It is Hausdorff but not normal. \qed
\end{proof}

\section{Precise Refinements}

\begin{definition}[Precise Refinement]
    Let $X$ be a topological space. Let $\{ U_\alpha \}_{\alpha \in J}$ and $\{ V_\alpha \}_{\alpha \in J}$
    be two families of subsets of $X$ indexed by the same set $J$. Then $\{ U_\alpha \}_{\alpha \in J}$
    is a \emph{precise refinement} of $\{ V_\alpha \}_{\alpha \in J}$ if and only if
    $\forall \alpha \in J. \overline{U_\alpha} \subseteq V_\alpha$.
\end{definition}

\begin{lemma}[Shrinking Lemma (Choice]
    Let $X$ be a paracompact Hausdorff space. Let $\{ U_\alpha \}_{\alpha \in J}$ be an open covering of $X$.
    Then there exists a locally finite open precise refinement $\{ V_\alpha \}_{\alpha \in J}$ 
    that covers $X$.
\end{lemma}

\begin{proof}
    \pf
    \step{1}{\pflet{$\AA = \{ A \text{ open in } X \mid \exists \alpha \in J. \overline{A} \subseteq U_\alpha \}$}}
    \step{2}{$\AA$ covers $X$.}
    \begin{proof}
        \pf\ Since $X$ is regular.
    \end{proof}
    \step{3}{\pick\ a locally finite open refinement $\BB$ of $\AA$ that covers $X$.}
    \step{4}{\pick\ a function $f : \BB \rightarrow J$ such that $\forall B \in \BB. \overline{B} \subseteq U_{f(B)}$}
    \step{5}{For $\alpha \in J$, \pflet{$V_\alpha = \bigcup \inv{f}(\alpha)$}}
    \step{6}{$\forall \alpha \in J. \overline{V_\alpha} = \bigcup_{f(B) = \alpha} \overline{B}$}
    \begin{proof}
        \pf\ Lemma \ref{lemma:closures_locally_finite}.
    \end{proof}
    \step{7}{$\forall \alpha \in J. \overline{V_\alpha} \subseteq U_\alpha$}
    \step{8}{$\{ V_\alpha \}_{\alpha \in J}$ is locally finite.}
    \begin{proof}
        \step{a}{\pflet{$x \in X$}}
        \step{b}{\pick\ a neighbourhood $W$ of $x$ that intersects only finitely many elements of $\BB$.}
        \step{c}{If $W$ intersects $V_\alpha$ then $W$ intersects $B$ for some $B \in \BB$ such that $f(B) = \alpha$}
        \step{d}{$W$ intersects $V_\alpha$ for only finitely many $\alpha$}
    \end{proof}
    \qed
\end{proof}

\begin{theorem}[Choice]
    Let $X$ be a paracompact Hausdorff space. Let $\{ U_\alpha \}_{\alpha \in J}$ be an open cover of $X$.
    Then there exists a partition of unity dominated by $\{ U_\alpha \}_{\alpha \in J}$.
\end{theorem}

\begin{proof}
    \pf
    \step{1}{\pick\ a locally finite open precise refinement $\{ V_\alpha \}_{\alpha \in J}$ of $\{ U_\alpha \}_{\alpha \in J}$
    that covers $X$.}
    \begin{proof}
        \pf\ Shrinking Lemma.
    \end{proof}
    \step{2}{\pick\ a locally finite open precise refinement $\{ W_\alpha \}_{\alpha \in J}$ of $\{ V_\alpha \}_{\alpha \in J}$
    that covers $X$.}
    \begin{proof}
        \pf\ Shrinking Lemma.
    \end{proof}
    \step{3}{For $\alpha \in J$, \pick\ a continuous function $\psi_\alpha : X \rightarrow [0,1]$ such that
    $\psi_\alpha(\overline{W_\alpha}) = \{1\}$ and $\psi_\alpha(X - V_\alpha) = \{0\}$.}
    \begin{proof}
        \pf\ Urysohn Lemma.
    \end{proof}
    \step{4}{$\forall \alpha \in J. \support \psi_\alpha \subseteq \overline{V_\alpha}$}
    \step{5}{$\{ \overline{V_\alpha} \}_{\alpha \in J}$ is locally finite.}
    \begin{proof}
        \pf\ Lemma \ref{lemma:closures_locally_finite}.
    \end{proof}
    \step{6}{$\{ \support \psi_\alpha \}_{\alpha \in J}$ is locally finite.}
    \step{7}{$\forall x \in X. \exists \alpha \in J. \psi_\alpha(x) > 0$}
    \step{8}{Define $\Psi : X \rightarrow \RR$ by $\Psi(x) = \sum_{\alpha \in J} \psi_\alpha(x)$.}
    \step{9}{$\Psi$ is continuous.}
    \begin{proof}
        \pf\ Since $\Psi \restriction W_\alpha$ is continuous for all $\alpha \in J$.
    \end{proof}
    \step{10}{$\forall x \in X. \Psi(x) > 0$}
    \step{11}{For $\alpha \in J$, define $\phi_\alpha : X \rightarrow \RR$ by $\phi_\alpha(x) = \psi_\alpha(x) / \Psi(x)$.}
    \step{12}{$\{ \phi_\alpha \}_{\alpha \in J}$ is a partition of unity dominated by $\{ U_\alpha \}_{\alpha \in J}$.}
    \qed
\end{proof}

\begin{theorem}
    Let $X$ be a paracompact Hausdorff space. Let $\mathcal{C} \subseteq \pow X$ be locally finite. Let $\{ \epsilon_C \}_{C \in \mathcal{C}}$
    be a family of positive reals. Then there exists a continuous function $f : X \rightarrow \RR$ such that $\forall x \in X. f(x) > 0$,
    and $\forall C \in \mathcal{C}. \forall x \in C. f(x) < \epsilon_C$.
\end{theorem}

\begin{proof}
    \pf
    \step{1}{\pflet{$J$ be the set of all open sets $U$ that intersect only finitely many elements of $\mathcal{C}$.}}
    \step{2}{\pick\ a partition of unity $\{ \phi_U \}_{U \in J}$ dominated by $\{ U \}_{U \in J}$.}
    \step{3}{For $U \in J$, \pflet{$\delta_U = \min \{ \epsilon_C \mid C \in \mathcal{C}, C \cap U \neq \emptyset \}$,
    or $\delta_U = 1$ if $U$ intersects no elements of $\mathcal{C}$}}
    \step{4}{Define $f : X \rightarrow \RR$ by $f(x) = \sum_{U \in J} \delta_U \phi_U(x)$}
    \step{5}{$\forall x \in X. f(x) > 0$}
    \step{6}{$\forall C \in \mathcal{C}. \forall x \in C. f(x) < \epsilon_C$}
    \begin{proof}
        \step{a}{\pflet{$C \in \mathcal{C}$}}
        \step{b}{\pflet{$x \in C$}}
        \step{c}{$\forall U \in J. \delta_U \phi_U(x) \leq \epsilon_C \phi_U(x)$}
        \begin{proof}
            \step{i}{\pflet{$U \in J$}}
            \step{ii}{\case{$x \in \support \phi_U$}}
            \begin{proof}
                \pf\ In this case $\delta_U \leq \epsilon_U$.
            \end{proof}
            \step{iii}{\case{$x \notin \support \phi_U$}}
            \begin{proof}
                \pf\ In this case $\phi_U(x) = 0$
            \end{proof}
        \end{proof}
    \end{proof}
    \qed
\end{proof}

\begin{proposition}[Choice]
    The product of a paracompact space and a compact space is paracompact.
\end{proposition}

\begin{proof}
    \pf
    \step{1}{\pflet{$X$ be a compact space and $Y$ a paracompact space.}}
    \step{2}{\pflet{$\AA$ be an open cover of $X \times Y$.}}
    \step{3}{For all $y \in Y$, there exists a finite subset of $\AA$ that covers $X \times \{y\}$}
    \begin{proof}
        \pf\ Since $X \times \{y\}$ is compact.
    \end{proof}
    \step{4}{For all $y \in Y$, there exists an open neighbourhood $V$ of $y$ such that $X \times V$ can be covered by finitely
    many elements of $\AA$}
    \begin{proof}
        \pf\ By the Tube Lemma.
    \end{proof}
    \step{5}{$\{ V \text{ open in } Y \mid X \times V \text{ can be covered by finitely many elements of } \AA \}$ is an open
    cover of $Y$.}
    \step{6}{\pick\ a locally finite open refinement $\PP$}
    \step{7}{For all $P \in \PP$, \pick\ a finite subset $\BB_P$ of $\AA$ such that $X \times P \subseteq \bigcup \BB_P$}
    \step{8}{\pflet{$\BB = \{ (X \times P) \cap A \mid P \in \PP, A \in \BB_P \}$}}
    \step{8}{$\BB$ is locally finite.}
    \begin{proof}
        \step{a}{\pflet{$(x,y) \in X \times Y$}}
        \step{b}{\pick\ an open neighbourhood $V$ of $y$ that intersects only finitely many elements of $\PP$.}
        \step{c}{$X \times V$ is an open neighbourhood of $(x,y)$ that intersects only finitely many elements of $\BB$.}
    \end{proof}
    \step{9}{$\BB$ refines $\AA$.}
    \step{10}{$\BB$ covers $X \times Y$.}
    \begin{proof}
        \step{a}{\pflet{$(x,y) \in X \times Y$}}
        \step{b}{\pick\ $P \in \PP$ such that $y \in P$}
        \step{c}{\pick\ $A \in \BB_P$ such that $(x,y) \in A$}
        \step{d}{$(x,y) \in (X \times P) \cap A \in \BB$}
    \end{proof}
    \qed
\end{proof}

\begin{corollary}
    The space $S_\Omega$ is not paracompact.
\end{corollary}

\begin{corollary}
    Not every locally compact Hausdorff space is paracompact.
\end{corollary}

\begin{proposition}
    Every discrete space is paracompact.
\end{proposition}

\begin{proof}
    \pf
    \step{1}{\pflet{$X$ be a discrete space.}}
    \step{2}{\pflet{$\AA$ be an open cover of $X$.}}
    \step{3}{$\{ \{x\} \mid x \in X \}$ is a locally finite open refinement of $\AA$ that covers $X$.}
    \qed
\end{proof}

\begin{proposition}
    The continuous image of a paracompact space is not necessarily paracompact.
\end{proposition}

\begin{proof}
    \pf\ Pick any non-paracompact space $X$. Let $X'$ be the same set under the discrete topology.
    The identity function is continuous as a map $X' \rightarrow X$. \qed
\end{proof}

\begin{lemma}[Expansion Lemma]
    Let $X$ be a paracompact Hausdorff space. Let $\{B_\alpha \}_{\alpha \in J}$ be a locally finite family of subsets of $X$.
    Then there exists a locally finite family of open sets $\{ U_\alpha \}_{\alpha \in J}$ such that $\forall \alpha \in J. B_\alpha
    \subseteq U_\alpha$.
\end{lemma}

\begin{proof}
    \pf
    \step{1}{\pick\ a locally finite closed refinement $\mathcal{C}$ of $\{ U \text{ open in } X \mid
    \text{for only finitely many } \alpha \in J. U \cap B_\alpha \neq \emptyset \}$}
    \step{2}{For $\alpha \in J$, \pflet{$\mathcal{C}_\alpha = \{ C \in \mathcal{C} \mid C \subseteq X - B_\alpha \}$}}
    \step{3}{For $\alpha \in J$, \pflet{$U_\alpha = X - \bigcup \mathcal{C}_\alpha$}}
    \step{4}{$\{ U_\alpha \}_{\alpha \in J}$ is locally finite.}
    \begin{proof}
        \step{a}{\pflet{$x \in X$}}
        \step{b}{\pick\ a neighbourhood $W$ of $x$ that intersects only finitely many elements of $\mathcal{C}$.}
        \step{c}{Each element of $\mathcal{C}$ intersects $U_\alpha$ for only finitely many $\alpha$.}
        \begin{proof}
            \step{i}{\pflet{$C \in \mathcal{C}$}}
            \step{ii}{$C$ intersects $B_\alpha$ for only finitely many $\alpha$.}
            \step{iii}{For all $\alpha \in J$, if $C$ intersects $U_\alpha$ then $C$ intersects $B_\alpha$.}
        \end{proof}
        \step{d}{$W$ is covered by the finitely many elements of $\mathcal{C}$ that it intersects.}
        \step{e}{$W$ intersects $U_\alpha$ for only finitely many $\alpha$.}
    \end{proof}
    \step{5}{For all $\alpha \in J$, $U_\alpha$ is open.}
    \step{6}{$\forall \alpha \in J. B_\alpha \subseteq U_\alpha$}
    \qed
\end{proof}

\begin{proposition}
    A regular space that is a countable union of closed paracompact subspaces whose interiors cover $X$ is paracompact.
\end{proposition}

\begin{proof}
    \pf
    \step{1}{\pflet{$X$ be a regular space that is a countable union of closed paracompact subspaces whose interiors cover $X$.}}
    \step{2}{\pick\ a sequence $(X_n)$ of closed paracompact subspaces such that $X = \bigcup_n (X_n)^\circ$}
    \step{3}{\pflet{$\UU$ be an open covering of $X$.}}
    \step{4}{For $n \in \ZZ^+$, \pick\ a set $\VV_n'$ of subsets of $X$ such that $\{ V \cap X_n \mid V \in \VV_n' \}$
    is a locally finite open refinement in $X_n$ of $\{ U \cap X_n \mid U \in \UU \}$}
    \step{5}{For $n \in \ZZ^+$, \pflet{$\VV_n = \{ V \cap (X_n)^\circ \mid V \in \VV_n' \}$}}
    \step{6}{\pflet{$\VV = \bigcup_n \VV_n$}}
    \step{7}{$\VV$ is countably locally finite.}
    \begin{proof}
        \step{a}{\pflet{$n \in \ZZ^+$} \prove{$\VV_n$ is locally finite.}}
        \step{b}{\pflet{$x \in X$}}
        \step{c}{\case{$x \in X_n$}}
        \begin{proof}
            \step{i}{\pick\ an open neighbourhood $W$ of $x$ such that $W \cap X_n$ intersects only finitely many elements of $\VV_n'$}
            \step{ii}{$W$ intersects only finitely many elements of $\VV_n$}
        \end{proof}
        \step{d}{\case{$x \notin X_n$}}
        \begin{proof}
            \pf\ $X - X_n$ is an open neighbourhood of $x$ that intersects no elements of $\VV_n$.
        \end{proof}
    \end{proof}
    \step{8}{$\VV$ refines $\UU$.}
    \step{9}{$\VV$ covers $X$.}
    \qed
\end{proof}

\begin{corollary}
    The space $\RR^\infty$ as a subspace of $\RR^\omega$ under the box topology is paracompact.
\end{corollary}

\begin{proposition}
    Let $p : X \twoheadrightarrow Y$ be a perfect map. If $Y$ is paracompact then so is $X$.
\end{proposition}

\begin{proof}
    \pf
    \step{1}{\pflet{$\AA$ be an open covering of $X$.}}
    \step{2}{$\{ V \text{ open in } Y \mid \inv{p}(V) \text{ can be covered by finitely many elements of } \AA \}$ is an open cover
    of $Y$.}
    \begin{proof}
        \pf\ Lemma \ref{lemma:closed_map_neighbourhoods}.
    \end{proof}
    \step{3}{\pick\ a locally finite open refinement $\BB$ of $\{ V \text{ open in } Y \mid \inv{p}(V) \text{ can be covered by finitely many elements of } \AA \}$}
    \step{4}{\pflet{$\mathcal{C} = \{ \inv{p}(V) \cap A \mid V \in \BB, A \in \AA \}$}}
    \step{5}{$\mathcal{C}$ is locally finite.}
    \begin{proof}
        \step{a}{\pflet{$x \in X$}}
        \step{b}{\pick\ an open neighbourhood $W$ of $p(x)$
        that intersects only finitely many $V \in \BB$,
        say $V_1$, \ldots, $V_m$.}
        \step{c}{There are only finitely many elements of $A$
        that intersect one of the $\inv{p}(V_i)$}
        \step{d}{$\inv{p}(W)$ intersects only finitely many elements
        of $\mathcal{C}$}
    \end{proof}
    \step{6}{$\mathcal{C}$ refines $\AA$}
    \step{7}{$\mathcal{C}$ covers $X$.}
    \begin{proof}
        \step{a}{\pflet{$x \in X$}}
        \step{b}{\pick\ $V \in \BB$ such that $p(x) \in V$}
        \step{c}{\pick\ $A \in \AA$ such that $x \in A$}
        \step{d}{$x \in \inv{p}(V) \cap A$}
    \end{proof}
    \qed
\end{proof}

\begin{proposition}
    Let $p : X \twoheadrightarrow Y$ be a perfect map.
    If $X$ is paracompact and Hausdorff then so is $Y$.
\end{proposition}

\begin{proof}
    \pf
    \step{1}{$Y$ is Hausdorff.}
    \begin{proof}
        \pf\ Proposition \ref{proposition:perfect_image_Hausdorff}.
    \end{proof}
    \step{2}{$Y$ is paracompact.}
    \begin{proof}
        \step{a}{\pflet{$\AA$ be an open covering of $Y$.}}
        \step{b}{$\{ \inv{p}(A) \mid A \in \AA \}$ is an open covering of $X$.}
        \step{c}{\pick\ a locally finite refinement $\BB$ that covers $X$.}
        \step{d}{\pflet{$\mathcal{C} = \{ p(B) \mid B \in \BB \}$}}
        \step{e}{$\mathcal{C}$ is locally finite.}
        \begin{proof}
            \step{i}{\pflet{$y \in Y$}}
            \step{ii}{\pick\ an open neighbourhood $U$ of $\inv{p}(y)$ that intersects only finitely many elements of $\BB$.}
            \begin{proof}
                \step{one}{For all $x \in \inv{p}(y)$, there exists an open neighbourhood $U$ of $x$ that intersects only finitely many elements of $\BB$.}
                \step{two}{\pick\ finitely many open sets $U_1$, \ldots, $U_n$, each of which intersects only finitely many elements of $\mathcal{C}$, that together cover $\inv{p}(y)$}
                \step{three}{\pflet{$U = U_1 \cup \cdots \cup U_n$}}
            \end{proof}
            \step{iii}{\pick\ an open neighbourhood $V$ of $y$ such that $\inv{p}(V) \subseteq U$}
            \step{iv}{$V$ intersects only finitely many elements of $\mathcal{C}$.}
        \end{proof}
        \step{f}{$\mathcal{C}$ refines $\AA$.}
        \step{g}{$\mathcal{C}$ covers $Y$.}
    \end{proof}
    \qed
\end{proof}

\section{Evaluation Map}

\begin{definition}[Evaluation Map]
    Let $X$ and $Y$ be topological spaces. The \emph{evaluation map} $e : X \times \mathcal{C}(X,Y) \rightarrow Y$
    is defined by $e(x,f) = f(x)$.
\end{definition}

\chapter{Topological Groups}

\section{Topological Groups}

\begin{definition}[Topological Group]
    A \emph{topological group} $G$ consists of a $T_1$ space $G$ and continuous maps
    $\cdot : G^2 \rightarrow G$ and $\inv{(\ )} : G \rightarrow G$ such that $(G, \cdot, \inv{(\ )})$
    is a group.
\end{definition}

\begin{example}
    \begin{enumerate}
        \item The integers $\ZZ$ under addition are a topological group.
        \item The real numbers $\RR$ under addition are a topological group.
        \item The positive reals under multiplication are a topological group.
        \item The set $\{ z \in \mathbb{C} \mid |z| = 1 \}$ under multiplication
        and given the topology of $S^1$ is a topological group.
        \item For any $n \geq 0$, the general linear group $GL_n(\RR)$ is a topological
        group under matrix multiplication, considered as a subspace of $\RR^{n^2}$.
    \end{enumerate}
\end{example}

\begin{lemma}
    Let $G$ be a $T_1$ space and $\cdot : G^2 \rightarrow G$, $\inv{(\ )} : G \rightarrow G$
    be functions such that $(G, \cdot, \inv{(\ )})$ is a group. Then $G$ is a
    topological group if and only if the function $f : G^2 \rightarrow G$ that maps
    $(x,y)$ to $x \inv{y}$ is continuous.
\end{lemma}

\begin{proof}
    \pf
    \step{1}{If $G$ is a topological group then $f$ is continuous.}
    \begin{proof}
        \pf\ From Theorem \ref{theorem:continuous_composite}.
    \end{proof}
    \step{2}{If $f$ is continuous then $G$ is a topological group.}
    \begin{proof}
        \step{a}{\assume{$f$ is continuous.}}
        \step{b}{$\inv{(\ )}$ is continuous.}
        \begin{proof}
            \pf\ Since $\inv{x} = f(e,x)$.
        \end{proof}
        \step{c}{$\cdot$ is continuous.}
        \begin{proof}
            \pf\ Since $xy = f(x,\inv{y})$.
        \end{proof}
    \end{proof}
    \qed
\end{proof}

\begin{lemma}
    Let $G$ be a topological group and $H$ a subgroup of $G$. Then $H$ is a topological group
    under the subspace topology.
\end{lemma}

\begin{proof}
    \pf
    \step{1}{$H$ is $T_1$.}
    \begin{proof}
        \pf\ From Proposition \ref{proposition:T1_subspace}.
    \end{proof}
    \step{2}{multiplication and inverse on $H$ are continuous.}
    \begin{proof}
        \pf\ From Theorem \ref{theorem:continuous_restriction}.
    \end{proof}
    \qed
\end{proof}

\begin{lemma}
    Let $G$ be a topological group and $H$ a subgroup of $G$. Then 
    $\overline{H}$ is a subgroup of $G$.
\end{lemma}

\begin{proof}
    \pf
    \step{1}{\pflet{$x, y \in \overline{H}$} \prove{$x\inv{y} \in \overline{H}$}}
    \step{2}{\pflet{$U$ be any neighbourhood of $x \inv{y}$}}
    \step{3}{\pflet{$f : G^2 \rightarrow G$, $f(a,b) = a \inv{b}$}}
    \step{4}{$\inv{f}(U)$ is a neighbourhood of $(x,y)$}
    \step{5}{\pick\ neighbourhoods $V$, $W$ of $x$ and $y$ respectively
    such that $f(V \times W) \subseteq U$.}
    \step{6}{\pick\ $a \in V \cap H$ and $b \in W \cap H$}
    \begin{proof}
        \pf\ Theorem \ref{theorem:closure_neighbourhood}.
    \end{proof}
    \step{7}{$a \inv{b} \in U \cap H$}
    \qedstep
    \begin{proof}
        \pf\ By Theorem \ref{theorem:closure_neighbourhood}.
    \end{proof}
    \qed
\end{proof}

\begin{proposition}
    \label{proposition:translation_homeomorphism}
    Let $G$ be a topological group and $\alpha \in G$. Then the maps
    $l_\alpha, r_\alpha : G \rightarrow G$ defined by $l_\alpha(x) = \alpha x$,
    $r_\alpha(x) = x\alpha$ are homeomorphisms of $G$ with itself.
\end{proposition}

\begin{proof}
    \pf\ They are continuous with continuous inverses $l_{\inv{\alpha}}$ and
    $r_{\inv{\alpha}}$. \qed
\end{proof}

\begin{corollary}
    Every topological group is homogeneous.
\end{corollary}

\begin{proof}
    \pf\ Given a topological group $G$ and $a, b \in G$, we have $l_{b\inv{a}}$
    is a homeomorphism that maps $a$ to $b$. \qed
\end{proof}

\begin{proposition}
    Let $G$ be a topological group and $H$ a subgroup of $G$. Give $G / H$
    the quotient topology. For all $\alpha \in G$, the map $\overline{f_\alpha}$ that
    sends $xH$ to $\alpha xH$ is a homeomorphism $G / H \cong G / H$.    
\end{proposition}

\begin{proof}
    \pf
    \step{1}{$\overline{f_\alpha}$ is well-defined.}
    \begin{proof}
        \pf\ If $x\inv{y} \in H$ then $(\alpha x) \inv{(\alpha y)} = x \inv{y}
        \in H$.
    \end{proof}
    \step{2}{$\overline{f_\alpha}$ is continuous.}
    \begin{proof}
        \pf\ Theorem \ref{theorem:quotient_composite} since $\overline{f_\alpha} \circ p
        = p \circ f_\alpha$ is continuous, where $p : G \twoheadrightarrow G / H$
        is the canonical surjection.
    \end{proof}
    \step{3}{$\inv{\overline{f_\alpha}}$ is continuous.}
    \begin{proof}
        \pf\ Similar since $\inv{\overline{f_\alpha}} = \overline{f_{\inv{\alpha}}}$.
    \end{proof}
    \qed
\end{proof}

\begin{corollary}
    Let $G$ be a topological group and $H$ a subgroup of $G$. Give $G / H$
    the quotient topology. Then $G / H$ is homogeneous.
\end{corollary}

\begin{proposition}
    \label{proposition:T1_quotient}
    Let $G$ be a topological group and $H$ a subgroup of $G$. Give $G / H$
    the quotient topology. If $H$ is closed in $G$ then $G / H$ is $T_1$.
\end{proposition}

\begin{proof}
    \pf
    \step{1}{\pflet{$p : G \twoheadrightarrow G / H$ be the canonical surjection}}
    \step{2}{\pflet{$x \in G$}}
    \step{3}{$\inv{p}(xH) = f_x(H)$}
    \step{4}{$\inv{p}(xH)$ is closed in $G$}
    \begin{proof}
        \pf\ Since $H$ is closed and $f_x$ is a homemorphism of $G$ with
        itself.
    \end{proof}
    \step{5}{$\{xH\}$ is closed in $G / H$}
    \qed
\end{proof}

\begin{proposition}
    \label{proposition:quotient_group_open_map}
    Let $G$ be a topological group and $H$ a subgroup of $G$. Give $G / H$ the
    quotient topology. Then the canonical surjection $p : G \twoheadrightarrow G / H$
    is an open map.
\end{proposition}

\begin{proof}
    \pf
    \step{1}{\pflet{$U \subseteq G$ be open.}}
    \step{2}{$\inv{p}(p(U)) = \bigcup_{h \in H} r_h(U)$}
    \step{3}{$\inv{p}(p(U))$ is open.}
    \step{4}{$p(U)$ is open.}
    \qed
\end{proof}

\begin{proposition}
    Let $G$ be a topological group and $H$ a closed normal subgroup of $G$.
    Then $G / H$ is a topological group under the quotient topology.
\end{proposition}

\begin{proof}
    \pf
    \step{1}{$G / H$ is $T_1$} 
    \begin{proof}
        \pf\ Proposition \ref{proposition:T1_quotient}.
    \end{proof}
    \step{2}{The map $\overline{m} : (xH,yH) \mapsto x\inv{y}H$ is continuous.}
    \begin{proof}
        \step{a}{$p^2 : G^2 \rightarrow (G / H)^2$ is a quotient map.}
        \begin{proof}
            \pf\ Propositions \ref{proposition:quotient_product}, \ref{proposition:quotient_group_open_map}.
        \end{proof}
        \step{b}{$\overline{m} \circ p^2$ is continuous.}
        \begin{proof}
            \pf\ As it is $p^2 \circ m$ where $m : G^2 \rightarrow G$ with
            $m(x,y) = x\inv{y}$
        \end{proof}
    \end{proof}
    \qed
\end{proof}

\begin{lemma}
    \label{lemma:open_subset_mult}
    Let $G$ be a topological group and $A, B \subseteq G$. If either $A$ or $B$ is open then $AB$ is open.
\end{lemma}

\begin{proof}
    \pf\ If $A$ is open we have $AB = \bigcup_{b \in B} r_b(A)$. Similarly if $B$ is open. \qed
\end{proof}

\begin{definition}[Symmetric Neighbourhood]
    Let $G$ be a topological group. A neighbourhood $V$ of $e$ is \emph{symmetric}
    if and only if $V = \inv{V}$.
\end{definition}

\begin{lemma}
    \label{lemma:symmetric_neighbourhood}
    Let $G$ be a topological group. Let $V$ be a neighbourhood of $e$. Then $V$ is symmetric if and only if, for all $x \in V$, we have $\inv{x} \in V$.
\end{lemma}

\begin{proof}
    \pf
    \step{1}{If $V$ is symmetric then, for all $x \in V$, we have $\inv{x} \in V$}
    \begin{proof}
        \pf\ Immediate from defintions.
    \end{proof}
    \step{2}{If, for all $x \in V$, we have $\inv{x} \in V$, then $V$ is symmetric.}
    \begin{proof}
        \step{a}{\assume{for all $x \in V$ we have $\inv{x} \in V$}}
        \step{b}{$V \subseteq \inv{V}$}
        \begin{proof}
            \pf\ If $x \in V$ then there exists $y \in V$ such that $x = \inv{y}$, namely $y = \inv{x}$
        \end{proof}
        \step{c}{$\inv{V} \subseteq V$}
        \begin{proof}
            \pf\ Immediate from \stepref{a}.
        \end{proof}
    \end{proof}
    \qed
\end{proof}

\begin{lemma}
    \label{lemma:symmetric_subneighbourhood}
    Let $G$ be a topological group. For every neighbourhood $U$ of $e$, there exists a symmetric neighbourhood $V$ of $e$ such that $V^2 \subseteq U$.
\end{lemma}

\begin{proof}
    \pf
    \step{1}{\pflet{$U$ be a neighbourhood of $e$.}}
    \step{2}{\pick\ a neighbourhood $V'$ of $e$ such that $V'V' \subseteq U$}
    \begin{proof}
        \pf\ Such a neighbourhood exists because multiplication in $G$ is continuous.
    \end{proof}
    \step{3}{\pick\ a neighbourhood $W$ of $e$ such that $W \inv{W} \subseteq V'$}
    \begin{proof}
        \pf\ Such a neighbourhood exists because the function that maps $(x,y)$ to $x \inv{y}$ is continuous.
    \end{proof}
    \step{4}{\pflet{$V = W \inv{W}$}}
    \step{5}{$V$ is a neighbourhood of $e$}
    \begin{proof}
        \step{a}{$e \in V$}
        \begin{proof}
            \pf\ Since $e \in W$ so $e = e \inv{e} \in V$.
        \end{proof}
        \step{b}{$V$ is open}
        \begin{proof}
            \pf\ Lemma \ref{lemma:open_subset_mult}.
        \end{proof}
    \end{proof}
    \step{6}{$V$ is symmetric}
    \begin{proof}
        \step{a}{For all $x \in V$ we have $\inv{x} \in V$}
        \begin{proof}
            \step{i}{\pflet{$x \in V$}}
            \step{ii}{\pick $y, z \in W$ such that $x = y \inv{z}$}
            \step{iii}{$\inv{x} = z\inv{y}$}
            \step{iv}{$\inv{x} \in V$}
            \step{v}{$x \in \inv{V}$}
        \end{proof}
        \qedstep
        \begin{proof}
            \pf\ Lemma \ref{lemma:symmetric_neighbourhood}
        \end{proof}
    \end{proof}
    \step{7}{$V^2 \subseteq U$}
    \begin{proof}
        \pf\ We have $V^2 \subseteq (V')^2 \subseteq U$
    \end{proof}
    \qed
\end{proof}

\begin{proposition}
    Every topological group is Hausdorff.
\end{proposition}

\begin{proof}
    \pf
    \step{1}{\pflet{$G$ be a topological group.}}
    \step{2}{\pflet{$x, y \in G$ with $x \neq y$}}
    \step{3}{\pflet{$U = G - \{x\inv[y]\}$}}
    \step{4}{\pick\ a symmetric neighbourhood $V$ of $e$ such that $VV \subseteq U$}
    \begin{proof}
        \step{a}{$U$ is open}
        \begin{proof}
            \pf\ Since $G$ is $T_1$.
        \end{proof}
        \step{b}{$e \in U$}
        \begin{proof}
            \pf\ Since $x \neq y$
        \end{proof}
        \qedstep
        \begin{proof}
            \pf\ Lemma \ref{lemma:symmetric_subneighbourhood}.
        \end{proof}
    \end{proof}
    \step{5}{$Vx$ and $Vy$ are disjoint neighbourhoods of $x$ and $y$ respectively.}
    \begin{proof}
        \step{a}{$Vx$ is open}
        \begin{proof}
            \pf\ Since $Vx = r_x(V)$
        \end{proof}
        \step{b}{$Vy$ is open}
        \begin{proof}
            \pf\ Similar.
        \end{proof}
        \step{c}{$Vx \cap Vy = \emptyset$}
        \begin{proof}
            \step{a}{\assume{for a contradiction $z \in Vx \cap Vy$}}
            \step{b}{\pick\ $a, b \in V$ such that $z = ax = by$}
            \step{c}{$x\inv{y} \in VV$}
            \begin{proof}
                \pf\ Since $x \inv{y} = \inv{a} b$
            \end{proof}
            \step{d}{$x \inv{y} \in U$}
            \qedstep
            \begin{proof}
                \pf\ From \stepref{3}.
            \end{proof}
        \end{proof}
    \end{proof}
    \qed
\end{proof}

\begin{proposition}
    \label{proposition:regular_topological_group}
    Every topological group is regular.
\end{proposition}

\begin{proof}
    \pf
    \step{1}{\pflet{$G$ be a topological group.}}
    \step{2}{\pflet{$A \subseteq G$ be a closed set and $a \notin A$.}}
    \step{3}{\pflet{$U = G - A \inv{a}$}}
    \step{4}{\pick\ a symmetric neighbourhood $V$ of $e$ such that $VV \subseteq U$}
    \begin{proof}
        \step{a}{$U$ is open}
        \begin{proof}
            \pf\ Since $A \inv{a} = r_{\inv{a}}(A)$ is closed.
        \end{proof}
        \step{b}{$e \in U$}
        \begin{proof}
            \pf\ Since $a \notin A$.
        \end{proof}
        \qedstep
        \begin{proof}
            \pf\ Lemma \ref{lemma:symmetric_subneighbourhood}.
        \end{proof}
    \end{proof}
    \step{5}{$VA$ and $Va$ are disjoint open sets with $A \subseteq VA$ and $a \in Va$}
    \begin{proof}
        \step{a}{$VA$ is open}
        \begin{proof}
            \pf\ Lemma \ref{lemma:open_subset_mult}
        \end{proof}
        \step{b}{$Va$ is open}
        \begin{proof}
            \pf\ Lemma \ref{lemma:open_subset_mult}
        \end{proof}
        \step{c}{$VA \cap Va = \emptyset$}
        \begin{proof}
            \step{i}{\assume{for a contradiction $z \in VA \cap Va$}}
            \step{ii}{\pick\ $b, c \in V$ and $d \in A$ with $z = bd = ca$}
            \step{iii}{$d \inv{a} \in U$}
            \begin{proof}
                \pf\ Since $d \inv{a} = \inv{b} c \in VV \subseteq U$
            \end{proof}
            \qedstep
            \begin{proof}
                \pf\ This contradicts \stepref{3}
            \end{proof}
        \end{proof}
    \end{proof}
    \qed
\end{proof}

\begin{example}
    Not every topological group is normal.
\end{example}

\begin{proof}
    \pf\ For $J$ an uncountable set, we have $\RR^J$ is a topological group under
    pointwise addition, but is not normal. \qed
\end{proof}

\begin{proposition}
    \label{proposition:regular_quotient}
    Let $G$ be a topological group and $H$ a subgroup of $G$. Give $G / H$
    the quotient topology. If $H$ is closed in $G$ then $G / H$ is regular.
\end{proposition}

\begin{proof}
    \pf
    \step{0}{\pflet{$p : G \twoheadrightarrow G / H$ be the canonical surjection.}}
    \step{1}{\pflet{$A$ be a closed set in $G / H$ and $aH \in (G / H) - A$.}}
    \step{2}{\pflet{$B = \inv{p}(A)$}}
    \step{3}{$B$ is a closed saturated set in $G$.}
    \step{4}{$B \cap aH = \emptyset$}
    \step{5}{$B = BH$}
    \step{6}{\pick\ a symmetric neighbourhood $V$ of $e$ such that $VB$ does not intersect $Va$}
    \begin{proof}
        \step{a}{\pflet{$U = G - B \inv{a}$}}
        \step{b}{\pick\ a symmetric neighbourhood $V$ of $e$ such that $VV \subseteq U$}
        \begin{proof}
            \step{i}{$U$ is open}
            \begin{proof}
                \pf\ Since $B \inv{a} = r_{\inv{a}}(B)$ is closed.
            \end{proof}
            \step{ii}{$e \in U$}
            \begin{proof}
                \pf\ If $e \in B \inv{a}$ then $a \in B$
            \end{proof}
            \qedstep
            \begin{proof}
                \pf\ Lemma \ref{lemma:symmetric_subneighbourhood}
            \end{proof}
        \end{proof}
        \step{c}{$VB \cap Va = \emptyset$}
        \begin{proof}
            \pf\ If $vb = v'a$ for $v, v' \in V$ and $b \in B$ then we have $b\inv{a} = \inv{v}v' \in Ba \cap VV \subseteq Ba \cap U$.
        \end{proof}
    \end{proof}
    \step{7}{$p(VB)$ and $p(Va)$ are disjoint open sets}
    \begin{proof}
        \step{a}{$p(VB)$ and $p(Va)$ are open.}
        \begin{proof}
            \pf\ Proposition \ref{proposition:quotient_group_open_map}.
        \end{proof}
        \step{b}{$p(VB) \cap p(Va) = \emptyset$}
        \begin{proof}
            \pf\ If $vbH = v'aH$ for $v,v' \in V$, $b \in B$ then $v'a = vbh$ for some $h \in H$. Hence $v'a \in Va \cap VBH = Va \cap VB$.
        \end{proof}
    \end{proof}
    \step{8}{$A \subseteq p(VB)$}
    \step{9}{$aH \in p(Va)$}
    \qed
\end{proof}

\begin{proposition}
    Let $G$ be a topological group. The component of $G$ that contains $e$ is
    a normal subgroup of $G$.
\end{proposition}

\begin{proof}
    \pf
    \step{1}{\pflet{$C$ be the component of $G$ that contains $e$.}}
    \step{2}{For all $x \in G$, $xC$ is the component of $G$ that contains $x$.}
    \begin{proof}
        \step{a}{\pflet{$x \in G$}}
        \step{b}{\pflet{$D$ be the component of $G$ that contains $x$.}}
        \step{b}{$xC \subseteq D$}
        \begin{proof}
            \pf\ Since $xC$ is connected by Theorem \ref{theorem:connected_continuous_image}.
        \end{proof}
        \step{c}{$D \subseteq xC$}
        \begin{proof}
            \pf\ Since $\inv{x}D \subseteq C$ similarly.
        \end{proof}
    \end{proof}
    \step{3}{For all $x \in G$, $Cx$ is the component of $G$ that contains $x$.}
    \begin{proof}
        \pf\ Similar.
    \end{proof}
    \step{2a}{For all $x \in C$ we have $xC = Cx = C$}
    \step{2b}{For all $x \in C$ we have $\inv{x}C = C$}
    \step{2c}{For all $x \in C$ we have $\inv{x} \in C$}
    \step{3}{For all $x, y \in C$ we have $xy \in C$}
    \begin{proof}
        \pf\ Since $xyC = xC = x$.
    \end{proof}
    \step{4}{For all $x \in G$ we have $xC = Cx$.}
    \begin{proof}
        \pf\ From \stepref{2} and \stepref{3}.
    \end{proof}
    \qed
\end{proof}

\begin{lemma}
    Let $G$ be a topological group. Let $A$ be a closed set in $G$ and $B$ a compact subspace of $G$
    such that $A \cap B = \emptyset$. Then there exists a symmetric neighbourhood $U$ of $e$
    such that $AU \cap BU = \emptyset$.
\end{lemma}

\begin{proof}
    \pf
    \step{1}{For all $b \in B$ there exists a symmetric neighbourhood $V$ of $e$ such that $bV^2 \cap A = \emptyset$}
    \begin{proof}
        \step{a}{\pflet{$b \in B$}}
        \step{c}{\pflet{$W = \inv{b} (G - A)$}}
        \step{d}{$W$ is a neighbourhood of $e$ and $bW \cap A = \emptyset$}
        \step{e}{\pick\ a symmetric neighbourhood $V$ of $e$ such that $V^2 \subseteq W$}
    \end{proof}
    \step{2}{$\{bV^2 \mid b \in B, V \text{ is a symmetric neighbourhood of } e, bV^2 \cap A = \emptyset \}$ is an open cover of $B$}
    \step{3}{\pick\ a finite subcover $b_1 V_1^2$, \ldots, $b_n V_n^2$, say.}
    \step{4}{\pflet{$U = V_1 \cap \cdots \cap V_n$}}
    \step{5}{$BU^2 \cap A = \emptyset$}
    \step{7}{$AU \cap BU = \emptyset$}
    \begin{proof}
        \pf\ If $av \in BU$ where $a \in A$ and $v \in V$ then $a = a v \inv{v} \in BU^2 \cap A$.
    \end{proof}
    \qed
\end{proof}

\begin{proposition}[AC]
    Let $G$ be a topological group. Let $A$ be a closed set in $G$, and $B$ a compact subspace of $G$. Then $AB$ is closed.
\end{proposition}

\begin{proof}
    \pf
    \step{1}{\pflet{$x \in G - AB$}}
    \step{2}{$\inv{A} x \cap B = \emptyset$}
    \step{3}{$\inv{A} x$ is closed.}
    \step{4}{\pick\ a symmetric neighbourhood $U$ of $e$ such that $\inv{A} x U \cap BU = \emptyset$}
    \step{5}{$x U^2$ is open}
    \begin{proof}
        \pf\ Lemma \ref{lemma:open_subset_mult}.
    \end{proof}
    \step{6}{$x \in x U^2 \subseteq G - AB$}
    \qed
\end{proof}

\begin{proof}
    \pf
    \step{1}{\pflet{$x \in \overline{AB}$} \prove{$x \in AB$}}
    \step{2}{\pick\ a net $(a_\alpha b_\alpha)_{\alpha \in J}$ in $AB$ that converges
    to $x$.}
    \step{3}{\pick\ a convergent subnet $(b_{g(\beta)})_{\beta \in K}$ of $(b_\alpha)_{\alpha \in J}$
    with limit $l$.}
    \step{4}{$a_{g(\beta)} \rightarrow x \inv{l}$ as $\beta \rightarrow \infty$}
    \begin{proof}
        \pf
        \begin{align*}
            a_{g(\beta)} & = a_{g(\beta)} b_{g(\beta)} \inv{b_{g(\beta)}} \\
            & \rightarrow x \inv{l}
        \end{align*}
    \end{proof}
    \step{5}{$x \inv{l} \in A$}
    \step{6}{$l \in B$}
    \begin{proof}
        \pf\ $B$ is closed because it is compact.
    \end{proof}
    \step{7}{$x \in AB$}
    \qed
\end{proof}
\begin{corollary}
    Let $G$ be a topological group and $H \leq G$. Let $p : G \twoheadrightarrow G / H$ be the quotient map.
    If $H$ is compact then $p$ is a closed map.
\end{corollary}

\begin{proof}
    \pf\ For $A$ closed in $G$, we have $\inv{p}(p(A)) = AH$ is closed, and so
    $p(A)$ is closed. \qed
\end{proof}

\begin{corollary}
    Let $G$ be a topological group and $H \leq G$. If $H$ and $G / H$ are compact then $G$ is compact.
\end{corollary}

\begin{proof}
    \pf\ From Proposition \ref{proposition:compact_pre_image} since, for all $aH \in G / H$,
    we have $\inv{p}(aH) = aH$ is compact because it is homemorphic to $H$. \qed
\end{proof}

\begin{proposition}
    Let $G$ be a locally compact topological group. Let $H \leq G$. Then $G/H$ is locally compact.
\end{proposition}

\begin{proof}
    \pf\ From Propositions \ref{proposition:continuous_open_image_locally_compact}
    and \ref{proposition:quotient_group_open_map}. \qed
\end{proof}

\begin{proposition}
    Every Lindel\"{o}f first countable topological group is second countable.
\end{proposition}

\begin{proof}
    \pf
    \step{1}{\pflet{$G$ be a topological group that is Lindel\"{o}f and first countable.}}
    \step{2}{\pick\ a countable local basis $\{ B_n \mid n \in \ZZ^+ \}$ at $e$.}
    \step{3}{For $n \in \ZZ^+$, \pick\ a countable set $C_n$ such that $\{ c B_n \mid c \in C \}$
    covers $G$. \prove{$\{ c B_n \mid n \in \ZZ^+, c \in C_n \}$ is a basis for $G$.}}
    \step{4}{\pflet{$x \in G$ and $U$ be an open neighbourhood of $x$}}
    \step{5}{\pick\ a symmetric open neighbourhood $V$ of $e$ such that $V V \subseteq \inv{x} U$}
    \step{6}{\pick\ $n$ such that $B_n \subseteq V$}
    \step{7}{\pick\ $c \in C_n$ such that $x \in c B_n$}
    \step{8}{$cB_n \subseteq U$}
    \begin{proof}
        \step{a}{\pflet{$b \in B_n$} \prove{$cb \in U$}}
        \step{b}{\pick\ $b' \in B_n$ such that $x = c b'$}
        \step{c}{$cb = x \inv{b'} b$}
        \step{d}{$cb \in U$}
        \begin{proof}
            \pf\ Since $\inv{b'} b \in VV \subseteq \inv{x} U$.
        \end{proof}
    \end{proof}
    \qed
\end{proof}

\begin{proposition}
    Every separable first countable topological group is second countable.
\end{proposition}

\begin{proof}
    \pf
    \step{1}{\pflet{$G$ be a topological group that is separable and first countable.}}
    \step{2}{\pick\ a countable local basis $\{ B_n \mid n \in \ZZ^+ \}$ at $e$.}
    \step{3}{\pick\ a countable dense set $D$ in $G$. 
    \prove{$\{ d B_n \mid n \in \ZZ^+, d \in D \}$ is a basis for $G$.}}
    \step{4}{\pflet{$x \in G$ and $U$ be an open neighbourhood of $x$}}
    \step{5}{\pick\ a symmetric open neighbourhood $V$ of $e$ such that $V V \subseteq \inv{x} U$}
    \step{6}{\pick\ $n$ such that $B_n \subseteq V$}
    \step{7}{\pick\ $d \in D \cap x \inv{B_n}$}
    \step{8}{$x \in d B_n$}
    \step{8}{$dB_n \subseteq U$}
    \begin{proof}
        \step{a}{\pflet{$b \in B_n$} \prove{$db \in U$}}
        \step{b}{\pick\ $b' \in B_n$ such that $d = x \inv{b'}$}
        \step{c}{$db = x \inv{b'} b$}
        \step{d}{$db \in U$}
        \begin{proof}
            \pf\ Since $\inv{b'} b \in VV \subseteq \inv{x} U$.
        \end{proof}
    \end{proof}
    \qed
\end{proof}

\begin{lemma}
    \label{lemma:open_subgroup_closed}
    Every open subgroup of a topological group is closed.
\end{lemma}

\begin{proof}
    \pf
    \step{1}{\pflet{$G$ be a topological group.}}
    \step{2}{\pflet{$H$ be an open subgroup of $G$.}}
    \step{3}{$H = G - \bigcup_{x \in G - H} xH$}
    \qed
\end{proof}

\begin{proposition}[Choice]
    Every locally compact topological group is paracompact.
\end{proposition}

\begin{proof}
    \pf
    \step{1}{\pflet{$G$ be a locally compact topological group.}}
    \step{2}{\pick\ a symmetric neighbourhood $U$ of $e$ with compact closure.}
    \begin{proof}
        \pf\ Lemma \ref{lemma:symmetric_subneighbourhood}.
    \end{proof}
    \step{3}{\pflet{$L = \bigcup_{n=1}^\infty U^n$}}
    \step{4}{$L$ is open.}
    \step{6}{$L$ is a subgroup of $G$.}
    \step{5}{$L$ is closed.}
    \begin{proof}
        \pf\ Lemma \ref{lemma:open_subgroup_closed}.
    \end{proof}
    \step{7}{$\overline{U} \subseteq U^2$} %TODO Extract lemma?
    \begin{proof}
        \step{a}{\pflet{$x \in \overline{U}$}}
        \step{b}{$xU$ is an open neighbourhood of $x$}
        \step{c}{\pick\ $y \in xU \cap U$}
        \step{d}{\pick\ $u \in U$ such that $y = xu$}
        \step{e}{$x = y \inv{u} \in U^2$}
    \end{proof}
    \step{8}{$L = \bigcup_{n=1}^\infty \overline{U}^n$}
    \step{9}{$L$ is Lindel\"{o}f.}
    \begin{proof}
        \pf\ Proposition \ref{proposition:countable_compact_lindelof}.
    \end{proof}
    \step{10}{For all $x \in G$, $xL$ is Lindel\"{o}f.}
    \begin{proof}
        \pf\ Proposition \ref{proposition:translation_homeomorphism}.
    \end{proof}
    \step{11}{\pflet{$\VV$ be any open cover of $G$.}}
    \step{12}{For all $x \in G$, \pick\ a countable subset $\VV_x$ of $\VV$ that covers $xL$.}
    \step{13}{\pflet{$\WW = \{ xL \cap V \mid x \in G, V \in \VV_x \}$}}
    \step{14}{$\WW$ is countably locally finite.}
    \begin{proof}
        \step{a}{For every coset $xL$ of $L$, \pick\ an enumeration $\VV_x = \{ V_{xL,n} \}_{n \in \ZZ^+}$.}
        \step{b}{\pflet{$\WW_n = \{ xL \cap V_{xL,n} \mid xL \in G / L \}$ }}
        \step{c}{Each $\WW_n$ is locally discrete.}
        \begin{proof}
            \pf\ For $x \in G$, the only member of $\WW_n$ that contains $x$ is $xL \cap V_{xL,n}$.
        \end{proof}
        \step{d}{$\WW = \bigcup_n \WW_n$}
    \end{proof}
    \step{15}{Every element of $\WW$ is open.}
    \step{16}{$\WW$ refines $\VV$.}
    \step{17}{$\WW$ covers $G$.}
    \qed
\end{proof}

\begin{corollary}
    Every locally compact topological group is normal.
\end{corollary}
\section{Actions}

\begin{definition}[Action]
    Let $X$ be a topological space. Let $G$ be a topological group.
    An \emph{action} of $G$ on $X$ is a continuous map $\cdot : G \times X \rightarrow X$
    such that:
    \begin{itemize}
        \item For all $x \in X$, $ex = x$
        \item For all $g, h \in G$ and $x \in X$, $g(hx) = (gh)x$
    \end{itemize}
\end{definition}

\begin{definition}[Orbit Space]
    Let $X$ be a topological space. Let $G$ be a topological group.
    Let $\cdot$ be an action of $G$ on $X$. The \emph{orbit space}
    of the action, $X / G$, is the quotient space of $X$ by $\sim$,
    where $x \sim y$ iff there exists $g \in G$ such that $y = gx$.
\end{definition}

\begin{proposition}
    Let $G$ be a compact topological group. Let $X$ be a topological space.
    Let $\cdot$ be an action of $G$ on $X$. The canonical map $\pi : X \twoheadrightarrow X / G$
    is a perfect map.
\end{proposition}

\begin{proof}
    \pf
    \step{1}{$\pi$ is a closed map.}
    \begin{proof}
        \step{a}{\pflet{$A \subseteq X$ be closed.} \prove{$\inv{\pi}(\pi(A)) = GA$ is closed.}}
        \step{b}{\pflet{$x \in X - GA$}}
        \step{c}{For all $g \in G$ we have $gx \notin A$}
        \step{d}{For all $g \in G$, we have $(g,x) \in \inv{\alpha}(X-A)$, where $\alpha = \cdot : G \times X \rightarrow X$}
        \step{e}{For all $g \in G$, there exist open neighbourhoods $U$ of $g$ and $V$ of $x$ such that $U \times V \subseteq \inv{\alpha}(X - A)$}
        \step{f}{$\{ U \text{ open in } G \mid \exists V \text{ open in } X. 
        x \in V \text{ and } U \times V \subseteq \inv{\alpha}(X - A) \}$ covers $G$}
        \step{g}{\pick\ a finite subcover $\{ U_1, \ldots, U_n \}$}
        \step{h}{For $1 \leq i \leq n$, \pick\ $V_i$ open in $X$ such that $x \in V_i$ and
        $U_i \times V_i \subseteq \inv{\alpha}(X - A)$}
        \step{i}{\pflet{$V = V_1 \cap \cdots \cap V_n$}}
        \step{j}{$x \in V \subseteq X - GA$}
    \end{proof}
    \step{2}{$\pi$ is continuous.}
    \begin{proof}
        \pf\ By construction it is a quotient map.
    \end{proof}
    \step{3}{For all $y \in X / G$, we have $\inv{\pi}(y)$ is compact.}
    \begin{proof}
        \pf\ By Theorem \ref{theorem:continuous_image_compact}
        since it is the image of the map $G \rightarrow X$ that maps $g$ to $gy$.
    \end{proof}
    \qed
\end{proof}

\begin{corollary}
    Let $G$ be a compact topological group. Let $X$ be a Hausdorff space.
    Let $\cdot$ be an action of $G$ on $X$. Then $X / G$ is Hausdorff.
\end{corollary}

\begin{proof}
    \pf\ Proposition \ref{proposition:perfect_image_Hausdorff}. \qed
\end{proof}

\begin{corollary}
    Let $G$ be a compact topological group. Let $X$ be a regular space.
    Let $\cdot$ be an action of $G$ on $X$. Then $X / G$ is regular.
\end{corollary}

\begin{proof}
    \pf\ Proposition \ref{proposition:perfect_image_regular}. \qed
\end{proof}

\begin{corollary}
    Let $G$ be a compact topological group. Let $X$ be a normal space.
    Let $\cdot$ be an action of $G$ on $X$. Then $X / G$ is normal.
\end{corollary}

\begin{proof}
    \pf\ Proposition \ref{proposition:closed_continuous_image_normal}. \qed
\end{proof}

\begin{corollary}
    Let $G$ be a compact topological group. Let $X$ be a locally compact space.
    Let $\cdot$ be an action of $G$ on $X$. Then $X / G$ is locally compact.
\end{corollary}

\begin{proof}
    \pf\ Proposition \ref{proposition:perfect_image_locally_compact}. \qed
\end{proof}

\begin{corollary}
    Let $G$ be a compact topological group. Let $X$ be a second countable space.
    Let $\cdot$ be an action of $G$ on $X$. Then $X / G$ is second countable.
\end{corollary}

\begin{proof}
    \pf\ Proposition \ref{proposition:perfect_image_second_countable}. \qed
\end{proof}

\begin{proposition}
    Every topological group is completely regular.
\end{proposition}

\begin{proof}
    \pf
    \step{1}{\pflet{$G$ be a topological group.}}
    \step{1a}{\pflet{$A$ be a closed set that does not contain $e$.}}
    \step{1b}{\pflet{$V_0 = G - A$}}
    \step{2}{\pick\ a sequence $(V_n)_{n \geq 1}$ of symmetric neighbourhoods of $e$ such that, for all $n \geq 1$,
    $V_n V_n \subseteq V_{n-1}$.}
    \step{3}{Define an open set $U(p)$ for every dyadic rational $p$ by:
    \begin{align*}
        U(1/2^n) & = V_n & (n \geq 0) \\
        U((2k+1)/2^n) & = V_n U(k/2^{n-1}) & (n \geq 0, 0 < k < 2^n) \\
        U(p) & = \emptyset & (p \leq 0) \\
        U(p) & = G & (p > 1)
    \end{align*}
    }
    \step{4}{For all $n \geq 0$ and $0 < k < 2^n$, we have
    \[ V_n U(k / 2^n) \subseteq U((k+1)/2^n) \]}
    \begin{proof}
        \pf\ The proof is by induction on $n$.
        \step{a}{\assume{For all $0 < k < 2^n$, we have $V_n U(k/2^n) \subseteq
        U((k+1)/2^n)$}}
        \step{b}{$V_{n+1} U(1/2^{n+1}) \subseteq U(1/2^n)$}
        \begin{proof}
            \pf
            \begin{align*}
                V_{n+1} U(1/2^{n+1}) & = V_{n+1} V_{n+1} & (\text{\stepref{3}}) \\
                & \subseteq V_n & (\text{\stepref{2}}) \\
                & = U(1/2^n) & (\text{\stepref{3}})
            \end{align*}
        \end{proof}
        \step{c}{For $0 < k < 2^{n+1}$ even we have $V_{n+1} U(k/2^{n+1}) \subseteq
        U((k+1)/2^{n+1})$}
        \begin{proof}
            \step{i}{\pflet{$0 < k < 2^{n+1}$ be even}}
            \step{ii}{\pflet{$k = 2l$}}
            \step{iii}{$V_{n+1} U(k/2^{n+1}) \subseteq
            U((k+1)/2^{n+1})$}
            \begin{proof}
                \pf
                \begin{align*}
                    V_{n+1} U(k/2^{n+1}) & = V_{n+1} U(l/2^n) \\
                    & = U((2l+1)/2^{n+1}) & (\text{\stepref{3}}) \\
                    & = U((k+1)/2^{n+1})
                \end{align*}
            \end{proof}
        \end{proof}
        \step{d}{For $0 < k < 2^{n+1}$ odd we have $V_{n+1} U(k/2^{n+1}) \subseteq
        U((k+1)/2^{n+1})$}
        \begin{proof}
            \pf
            \step{i}{\pflet{$0 < k < 2^{n+1}$ be odd}}
            \step{ii}{\pflet{$k = 2l+1$}}
            \step{iii}{$V_{n+1} U(k/2^{n+1}) \subseteq
            U((k+1)/2^{n+1})$}
            \begin{proof}
                \pf
                \begin{align*}
                    V_{n+1} U(k/2^{n+1}) & = V_{n+1} U(2l+1/2^{n+1}) \\
                    & = V_{n+1} V_{n+1} U(l/2^n) & (\text{\stepref{3}}) \\
                    & \subseteq V_n U(l/2^n) & (\text{\stepref{2}}) \\
                    & \subseteq U(l+1/2^n) & (\text{\stepref{a}}) \\
                    & = U((k+1)/2^{n+1})
                \end{align*}
            \end{proof}
        \end{proof}
    \end{proof}
    \step{4a}{For any dyadic rationals $p$, $q$, if $p < q$ then $\overline{U(p)}
    \subseteq U(q)$}
    \begin{proof}
        \step{a}{\pflet{$p$ and $q$ be dyadic rationals such that $p < q$}}
        \step{b}{\pick\ $n>0$, $k$, $l$ such that $p = k / 2^n$ and $q = l / 2^n$ and $k < l$}
        \step{c}{$V_n U(p) \subseteq U(q)$}
        \step{d}{$\overline{U(p)} \subseteq V_n U(p)$}
        \begin{proof}
            \step{i}{\pflet{$x \in \overline{U(p)}$}}
            \step{ii}{$V_n x$ is an open neighbourhood of $x$}
            \step{iii}{\pick\ $y \in U(p) \cap V_n x$}
            \step{iv}{\pick\ $v \in V_n$ such that $y = v x$}
            \step{v}{$x = \inv{v} y \in V_n U(p)$}
            \begin{proof}
                \pf\ Since $V_n$ is symmetric.
            \end{proof}
        \end{proof}
    \end{proof}
    \step{5}{For $x \in G$, \pflet{$\QQ(x) = \{ p \mid x \in U(p) \}$}}
    \step{6}{Define $f : G \rightarrow [0,1]$ by $f(x) = \inf \QQ(x)$}
    \step{7}{$f$ is continuous.}
    \step{6}{$f$ is continuous.}
    \begin{proof}
        \step{a}{For all $r \in P$ and $x \in \overline{U_r}$ we have $f(x) \leq r$}
        \begin{proof}
            \step{i}{\pflet{$r \in P$}}
            \step{ii}{\pflet{$x \in \overline{U_r}$}}
            \step{iii}{For all $s > r$ we have $x \in U_s$}
            \step{iv}{For all $s > r$ we have $s \in \QQ(x)$}
            \step{v}{For all $s > r$ we have $f(x) \leq s$}
            \step{vi}{$f(x) \leq r$}
        \end{proof}
        \step{b}{For all $r \in P$ and $x \in U_r$ we have $f(x) \geq r$}
        \begin{proof}
            \step{i}{\pflet{$r \in P$}}
            \step{ii}{\pflet{$x \in X - U_r$}}
            \step{iii}{For all $s < r$ we have $x \notin U_s$}
            \step{iv}{$r$ is a lower bound for $\QQ(x)$}
            \step{v}{$r \leq f(x)$}
        \end{proof}
        \step{c}{\pflet{$x_0 \in X$}}
        \step{d}{\pflet{$(c,d)$ be an open interval that contains $f(x_0)$}
        \prove{There exists an open neighbourhood $U$ of $x_0$ such that
        $f(U) \subseteq (c,d)$}}
        \step{e}{\pick\ rationals $q$, $r$ with $c < q < f(x_0) < r < d$}
        \step{f}{\pflet{$U = U_r - \overline{U_q}$}}
        \step{g}{$U$ is open}
        \step{h}{$x_0 \in U$}
        \begin{proof}
            \step{i}{$x_0 \in U_r$}
            \begin{proof}
                \pf\ From \stepref{b}.
            \end{proof}
            \step{ii}{$x_0 \notin \overline{U_q}$}
            \begin{proof}
                \pf\ From \stepref{a}.
            \end{proof}
        \end{proof}
        \step{i}{$f(U) \subseteq (c,d)$}
        \begin{proof}
            \step{i}{\pflet{$x \in U$}}
            \step{ii}{$r \leq f(x) \leq s$}
            \begin{proof}
                \pf\ From \stepref{a} and \stepref{b}.
            \end{proof}
            \step{iii}{$c < f(x) < d$}
        \end{proof}
    \end{proof}
    \step{8}{$f(e) = 0$}
    \step{9}{$f(A) = \{1\}$}
    \qedstep
    \begin{proof}
        \pf\ The result follows for any $a \in G$ by taking the homeomorphism
        that maps $a$ to $e$.
    \end{proof}
    \qed
\end{proof}