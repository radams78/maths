\section{Order Theory}

\begin{definition}[Preorder]
    Let $X$ be a set. A \emph{preorder} on $X$ is a binary relation $\leq$ on $X$ such that:
    \begin{description}
        \item[Reflexivity] For all $x \in X$, we have $x \leq x$
        \item[Transitivity] For all $x, y, z \in X$, if $x \leq y$ and $y \leq z$ then $x \leq z$.  
    \end{description}
\end{definition}

\begin{definition}[Preordered Set]
    A \emph{preordered set} consists of a set $X$ and a preorder $\leq$ on $X$.
\end{definition}

\begin{proposition}
    \label{proposition:iso_strictly_monotone_sujective}
    Let $X$ and $Y$ be linearly ordered sets. Let $f : X \twoheadrightarrow Y$ be strictly monotone and surjective.
    Then $f$ is a poset isomorphism.
\end{proposition}

\begin{proof}
    \pf
    \step{3}{$f$ is injective.}
    \begin{proof}
        \step{a}{\pflet{$x, y \in X$}}
        \step{b}{\assume{$f(x) = f(y)$}}
        \step{c}{$x \nless y$}
        \begin{proof}
            \pf\ By strong motonicity.
        \end{proof}
        \step{d}{$y \nless x$}
        \begin{proof}
            \pf\ By strong motonicity.
        \end{proof}
        \step{e}{$x = y$}
        \begin{proof}
            \pf\ By trichotomy.
        \end{proof}
    \end{proof}
    \step{4}{$\inv{f}$ is monotone.}
    \begin{proof}
        \step{a}{\pflet{$x, y \in X$}}
        \step{b}{\assume{$x \leq y$}}
        \step{c}{$\inv{f}(x) \ngtr \inv{f}(y)$}
        \begin{proof}
            \pf\ By strong motonicity.
        \end{proof}
        \step{d}{$\inv{f}(x) < \inv{f}(y)$}
        \begin{proof}
            \pf\ By trichotomy.
        \end{proof}
    \end{proof}
    \qed
\end{proof}

\begin{definition}[Interval]
    Let $X$ be a preordered set and $Y \subseteq X$. Then $Y$ is an \emph{interval} if and only if, for
    all $a, b \in Y$ and $c \in X$, if $a \leq c \leq b$ then $c \in Y$.
\end{definition}

\begin{definition}[Linear Continuum]
    A linearly ordered set $L$ is a \emph{linear continuum} if and only if:
    \begin{enumerate}
        \item every nonempty subset of $L$ that is bounded above has a supremum
        \item $L$ is dense
    \end{enumerate}
\end{definition}

\begin{proposition}
    Every interval in a linear continuum is a linear continuum.
\end{proposition}

\begin{proof}
    \pf
    \step{1}{\pflet{$L$ be a linear continuum and $I$ an interval in $L$.}}
    \step{2}{Every nonempty subset of $I$ that is bounded above has a supremum in $I$.}
    \begin{proof}
        \step{a}{\pflet{$X \subseteq I$ be nonempty and bounded above by $b \in I$.}}
        \step{b}{\pflet{$s$ be the supremum of $X$ in $L$.}}
        \begin{proof}
            \pf\ Since $L$ is a linear continuum.
        \end{proof}
        \step{c}{$s \in I$}
        \begin{proof}
            \step{i}{\pick\ $a \in X$}
            \begin{proof}
                \pf\ Since $X$ is nonempty (\stepref{a}).
            \end{proof}
            \step{ii}{$a \leq s \leq b$}
            \step{iii}{$s \in I$}
            \begin{proof}
                \pf\ Since $I$ is an interval (\stepref{1}).
            \end{proof}
        \end{proof}
        \step{d}{$s$ is the supremum of $X$ in $I$}
    \end{proof}
    \step{3}{$I$ is dense.}
    \begin{proof}
        \step{a}{\pflet{$x, y \in I$ with $x < y$}}
        \step{b}{\pick\ $z \in L$ with $x < z < y$}
        \begin{proof}
            \pf\ Since $L$ is dense.
        \end{proof}
        \step{c}{$z \in I$}
        \begin{proof}
            \pf\ Since $I$ is an interval.
        \end{proof}
    \end{proof}
    \qed
\end{proof}

\begin{definition}[Ordered Square]
    The \emph{ordered square} $I_o^2$ is the set $[0,1]^2$ under the dictionary order.
\end{definition}

\begin{proposition}
    The ordered square is a linear continuum.
\end{proposition}

\begin{proof}
    \pf
    \step{2}{Every nonempty subset of $I_o^2$ bounded above has a supremum.}
    \begin{proof}
        \step{a}{\pflet{$X \subseteq I_o^2$ be nonempty and bounded above by $(b,c)$}}
        \step{b}{\pflet{$s = \sup \pi_1(X)$}}
        \begin{proof}
            \pf\ The set $\pi_1(X)$ is nonempty and bounded above by $b$.
        \end{proof}
        \step{c}{\case{$s \in \pi_1(X)$}}
        \begin{proof}
            \step{i}{\pflet{$t = \sup \{ y \in [0,1] \mid (s,y) \in X \}$}}
            \begin{proof}
                \pf\ This set is nonempty and bounded above by $c$.
            \end{proof}
            \step{ii}{$(s,t)$ is the supremum of $X$.}
        \end{proof}
        \step{d}{\case{$s \notin \pi_1(X)$}}
        \begin{proof}
            \pf\ In this case $(s,0)$ is the supremum of $X$.
        \end{proof}
    \end{proof}
    \step{3}{$I_o^2$ is dense.}
    \begin{proof}
        \step{a}{\pflet{$(x_1,y_1),(x_2,y_2) \in I_o^2$ with $(x_1,y_1) < (x_2,y_2)$}}
        \step{b}{\case{$x_1 < x_2$}}
        \begin{proof}
            \step{i}{\pick\ $x_3$ with $x_1 < x_3 < x_2$}
            \step{ii}{$(x_1,y_1) < (x_3,y_1) < (x_2,y_2)$}
        \end{proof}
        \step{c}{\case{$x_1 = x_2$ and $y_1 < y_2$}}
        \begin{proof}
            \step{i}{\pick\ $y_3$ with $y_1 < y_3 < y_2$}
            \step{ii}{$(x_1,y_1) < (x_1,y_3) < (x_2,y_2)$}
        \end{proof}
    \end{proof}
    \qed
\end{proof}

\begin{proposition}
    If $X$ is a well-ordered set then $X \times [0,1)$ under the dictionary order is a linear continuum.
\end{proposition}

\begin{proof}
    \pf
    \step{1}{Every nonempty set $A \subseteq X \times [0,1)$ bounded above has a supremum.}
    \begin{proof}
        \step{a}{\pflet{$A \subseteq X \times [0,1)$ be nonempty and bounded above}}
        \step{b}{\pflet{$x_0$ be the supremum of $\pi_1(A)$}}
        \step{c}{\case{$x_0 \in \pi_1(A)$}}
        \begin{proof}
            \step{i}{\pflet{$y_0$ be the supremum of $\{ y \in [0,1) \mid (x_0,y) \in A \}$}}
            \step{ii}{$(x_0,y_0)$ is the supremum of $A$.}
        \end{proof}
        \step{d}{\case{$x_0 \notin \pi_1(A)$}}
        \begin{proof}
            \pf\ In this case $(x_0,0)$ is the supremum of $A$.
        \end{proof}
    \end{proof}
    \step{2}{$X \times [0,1)$ is dense.}
    \begin{proof}
        \step{a}{\pflet{$(x_1,y_1), (x_2,y_2) \in X \times [0,1)$ with $(x_1,y_1) < (x_2,y_2)$}}
        \step{b}{\case{$x_1 < x_2$}}
        \begin{proof}
            \step{i}{\pick\ $y_3$ such that $y_1 < y_3 < 1$}
            \step{ii}{$(x_1,y_1) < (x_1,y_3) < (x_2,y_2)$}
        \end{proof}
        \step{c}{\case{$x_1 = x_2$ and $y_1 < y_2$}}
        \begin{proof}
            \step{i}{\pick\ $y_3$ such that $y_1 < y_3 < y_2$}
            \step{ii}{$(x_1,y_1) < (x_1,y_3) < (x_2,y_2)$}
        \end{proof}
    \end{proof}
    \qed
\end{proof}

\begin{lemma}
    \label{lemma:order_iso_real_intervals}
    For all $a,b,c,d \in \RR$ with $a < b$ and $c < d$, we have 
    $[a,b) \cong [c,d)$
\end{lemma}

\begin{proof}
    \pf\ The map $\lambda t. c + (t-a)(d-c)/(b-a)$ is an order isomorphism.
\end{proof}

\begin{proposition}
    \label{proposition:interval_concatenate_interval}
    Let $X$ be a linearly ordered set. Let $a < b < c$ in $X$. Then $[a,c) \cong [0,1)$
    if and only if $[a,b) \cong [b,c) \cong [0,1)$.
\end{proposition}

\begin{proof}
    \pf
    \step{1}{If $[a,c) \cong [0,1)$ then $[a,b) \cong [b,c) \cong [0,1)$}
    \begin{proof}
        \step{a}{\assume{$f : [a,c) \cong [0,1)$ is an order isomorphism}}
        \step{b}{$[a,b) \cong [0,1)$}
        \begin{proof}
            \pf
            \begin{align*}
                [a,b) & \cong [0,f(b)) & (\text{by the restriction of } f) \\
                & \cong [0,1) & (\text{Lemma \ref{lemma:order_iso_real_intervals}})
            \end{align*}
        \end{proof}
        \step{c}{$[b,c) \cong [0,1)$}
        \begin{proof}
            \pf\ Similar.
        \end{proof}
    \end{proof}
    \step{2}{If $[a,b) \cong [b,c) \cong [0,1)$ then $[a,c) \cong [0,1)$}
    \begin{proof}
        \pf
        \begin{align*}
            [a,c) & = [a,b) * [b,c) \\
            & \cong [0,1) * [0,1) \\
            & \cong [0,1/2) * [1/2,1) & (\text{Lemma \ref{lemma:order_iso_real_intervals}})\\
            & = 1
        \end{align*}
    \end{proof}
    \qed
\end{proof}

\begin{proposition}[CC]
    \label{proposition:interval_concatenate_intervals}
    Let $X$ be a linearly ordered set. Let $x_0 < x_1 < \cdots$ be a
    strictly increasing sequence in $X$ with supremum $b$. Then $[x_0,b)
    \cong [0,1)$ if and only if $[x_i,x_{i+1}) \cong [0,1)$ for all $i$.
\end{proposition}

\begin{proof}
    \pf
    \step{1}{If $[x_0,b) \cong [0,1)$ then $[x_i,x_{i+1}) \cong [0,1)$
    for all $i$.}
    \begin{proof}
        \pf\ By Lemma \ref{lemma:order_iso_real_intervals}
    \end{proof}
    \step{2}{If $[x_i,x_{i+1}) \cong [0,1)$ for all $i$ then $[x_0,b) \cong [0,1)$}
    \begin{proof}
        \step{a}{\assume{$[x_i,x_{i+1}) \cong [0,1)$ for all $i$}}
        \step{b}{\pick\ an order isomorphism $f_i : [x_i,x_{i+1}) \cong [1/2^i,2/2^{i+1})$
        for each $i$.}
        \begin{proof}
            \pf\ By Lemma \ref{lemma:order_iso_real_intervals}
        \end{proof}
        \step{c}{The union of the $f_i$s is an order isomorphism $[x_0,b) \cong [0,1)$}
    \end{proof}
    \qed
\end{proof}

\section{Partially Ordered Sets}

\begin{definition}[Partial Order]
    A \emph{partial order} on a set $X$ is a preorder $\leq$ that is \emph{anti-symmetric}, i.e. whenever $x \leq y$ and $y \leq x$
    then $x = y$.    
\end{definition}

\begin{definition}[Linear Order]
    A \emph{linear order} on a set $X$ is a partial order such that, for any $x, y \in X$, either $x \leq y$ or $y \leq x$.
\end{definition}

\section{Strict Linear Orders}

\begin{definition}[Strict Linear Order (Extensionality, Pairing)]
    Let $A$ be a set. A \emph{strict linear order} on $A$ is a binary relation
    $R$ on $A$ that is transitive and satisfies \emph{trichotomy}: for any
    $x, y \in A$, exactly one of $xRy$, $x=y$, $yRx$ holds.
\end{definition}

\begin{theorem}
    Let $R$ be a strict linear order on $A$. Then there is no $x \in A$
    such that $xRx$.
\end{theorem}

\begin{proof}
    \pf\ Immediate from trichotomy.
\end{proof}

\begin{definition}[Well-ordering]
    A \emph{well-order} on a set $X$ is a linear order such that every nonempty set has a least element.
\end{definition}

\begin{definition}[Section]
    Given a well-ordered set $X$ and $\alpha \in X$, the \emph{section} of $X$ by $\alpha$ is $S_\alpha = \{ x \in X \mid x < \alpha \}$.
\end{definition}

\begin{theorem}[Transfinite Induction]
    Let $J$ be a well-ordered set and $J_0 \subseteq J$. Suppose that, for all $\alpha \in J$, if $S_\alpha \subseteq J_0$ then $\alpha \in J_0$.
    Then $J_0 = J$.
\end{theorem}

\begin{proof}
    \pf
    \step{1}{\assume{for a contradiction $J_0 \neq J$}}
    \step{2}{\pflet{$\alpha$ be the least element of $J \setminus J_0$}}
    \step{3}{$S_\alpha \subseteq J_0$}
    \step{4}{$\alpha \in J_0$}
    \qedstep
    \begin{proof}
        \pf\ This contradicts \stepref{2}.
    \end{proof}
    \qed
\end{proof}

\begin{theorem}[Transfinite Recursion]
    Let $J$ be a well-ordered set and $C$ a set. Let $\FF$ be the set of all functions from a section of $J$ to $C$.
    Let $\rho : \FF \rightarrow C$. Then there exists a unique function $h : J \rightarrow C$ such that, for all $\alpha \in J$,
    we have $h(\alpha) = \rho(h \restriction S_\alpha)$.
\end{theorem}

\begin{proof}
    \pf
    \step{0}{For every $\beta \in J$, there exists a unique $h_\beta : S_\beta \rightarrow J$ such that, for all $\alpha < \beta$, we have $h_\beta(\alpha) = \rho(h_\beta \restriction \alpha)$}
    \begin{proof}
        \step{a}{\pflet{$\beta \in J$}}
        \step{b}{\assume{for all $\gamma < \beta$ there exists a unique $h : S_\gamma \rightarrow J$ such that, for all $\alpha < \gamma$, we have $h(\alpha)
        = \rho(h \restriction S_\alpha)$}}
        \step{c}{For $\gamma < \beta$, \pflet{$h_\gamma : S_\gamma \rightarrow J$ be the function such that, for all $\alpha < \gamma$, we have
        $h_\gamma(\alpha) = \rho(h_\gamma \restriction S_\alpha)$}}
        \step{d}{\pflet{$h : S_\beta \rightarrow J$ be the function $h(\gamma) = \rho(h_\gamma)$ for $\gamma < \beta$}}
        \step{dd}{For $\gamma < \beta$ we have $h \restriction S_\gamma = h_\gamma$}
        \begin{proof}
            \step{i}{\pflet{$\gamma < \beta$}}
            \step{ii}{\assume{For all $\alpha < \gamma$ we have $h \restriction S_\alpha = h_\alpha$}}
            \step{iii}{For all $\alpha < \gamma$ we have $(h \restriction S_\gamma)(\alpha) = \rho((h \restriction S_\gamma) \restriction S_\alpha)$}
            \begin{proof}
                \pf
                \begin{align*}
                    (h \restriction S_\gamma)(\alpha) & = h(\alpha) \\
                    & = \rho(h_\alpha) & (\text{\stepref{d}}) \\
                    & = \rho(h \restriction S_\alpha) & (\text{\stepref{ii}}) \\
                    & = \rho((h \restriction S_\gamma) \restriction S_\alpha)
                \end{align*}
            \end{proof}
            \step{iv}{$h \restriction S_\gamma = h_\gamma$}
            \begin{proof}
                \pf\ From \stepref{d}
            \end{proof}
            \qedstep
            \begin{proof}
                \pf\ By transfinite induction.
            \end{proof}
        \end{proof}
        \step{e}{For $\alpha < \beta$ we have $h(\alpha) = \rho(h \restriction S_\alpha)$}
        \step{f}{If $h' : S_\beta \rightarrow J$ and $h'(\alpha) = \rho(h' \restriction S_\alpha)$ for all $\alpha < \beta$, then $h' = h$)}
        \begin{proof}
            \step{i}{\pflet{$h' : S_\beta \rightarrow J$ and $h'(\alpha) = \rho(h' \restriction S_\alpha)$ for all $\alpha < \beta$}}
            \step{ii}{For all $\gamma < \beta$ we have $h' \restriction S_\gamma = h_\gamma$}
            \begin{proof}
                \step{three}{For all $\alpha < \gamma$ we have $(h' \restriction S_\gamma)(\alpha) = \rho((h' \restriction S_\gamma) \restriction S_\alpha)$}
                \begin{proof}
                    \pf
                    \begin{align*}
                        (h' \restriction S_\gamma)(\alpha) & = h'(\alpha) \\
                        & = \rho(h' \restriction S_\alpha) & (\text{\stepref{i}}) \\
                        & = \rho((h' \restriction S_\gamma) \restriction S_\alpha)
                    \end{align*}
                \end{proof}
                \qedstep
                \begin{proof}
                    \pf\ From \stepref{d}
                \end{proof}
            \end{proof}
            \step{iii}{For all $\alpha < \beta$ we have $h'(\alpha) = \rho(h_\alpha)$}
        \end{proof}
    \end{proof}
    \step{1}{There exists $h : J \rightarrow C$ such that, for all $\alpha \in J$, we have $h(\alpha) = \rho(h \restriction S_\alpha)$}
    \begin{proof}
        \step{a}{For $\alpha \in J$, \pflet{$h(\alpha) = \rho(h_\alpha)$}}
        \step{b}{For $\alpha \in J$ we have $h \restriction S_\alpha = h_\alpha$}
        \begin{proof}
            \step{i}{\pflet{$\alpha \in J$}}
            \step{ii}{\assume{For all $\beta < \alpha$ we have $h \restriction S_\beta = h_\beta$}}
            \step{iii}{For all $\beta < \alpha$ we have $(h \restriction S_\alpha)(\beta) = \rho((h \restriction S_\alpha) \restriction S_\beta)$}
            \begin{proof}
                \pf
                \begin{align*}
                    (h \restriction S_\alpha)(\beta) & = h(\beta) \\
                    & = \rho(h_\beta) & (\text{\stepref{a}}) \\
                    & = \rho(h \restriction S_\beta) & (\text{\stepref{ii}}) \\
                    & = \rho((h \restriction S_\alpha) \restriction S_\beta)
                \end{align*}
            \end{proof}
            \step{iv}{$h \restriction S_\alpha = h_\alpha$}
            \begin{proof}
                \pf\ From \stepref{0}
            \end{proof}
            \qedstep
            \begin{proof}
                \pf\ By transfinite induction.
            \end{proof}
        \end{proof}
        \step{c}{For $\alpha \in J$ we have $h(\alpha) = \rho(h \restriction S_\alpha)$}
    \end{proof}
    \step{2}{If $h, h' : J \rightarrow C$ and, for all $\alpha \in J$, we have $h(\alpha) = \rho(h \restriction S_\alpha)$ and $h'(\alpha) = \rho(h' \restriction S_\alpha)$,
    then $h = h'$}
    \begin{proof}
        \step{a}{\assume{ $h, h' : J \rightarrow C$ and, for all $\alpha \in J$, we have $h(\alpha) = \rho(h \restriction S_\alpha)$ and $h'(\alpha) = \rho(h' \restriction S_\alpha)$}}
        \step{b}{\pflet{$\alpha \in J$}}
        \step{c}{\assume{for all $\beta < \alpha$ we have $h(\beta) = h'(\beta)$}}
        \step{d}{$h(\alpha) = h'(\alpha)$}
        \begin{proof}
            \pf
            \begin{align*}
                h(\alpha) & = \rho(h \restriction S_\alpha) \\
                & = \rho(h' \restriction S_\alpha) & (\text{\stepref{c}}) \\
                & = h'(\alpha)
            \end{align*}
        \end{proof}
        \qedstep
        \begin{proof}
            \pf\ By transfinite induction.
        \end{proof}
    \end{proof}
    \qed
\end{proof}

\begin{theorem}[Well-Ordering Theorem (AC)]
    Every set has a well-ordering.    
\end{theorem}

\begin{proof} %TODO
    \pf
    \step{1}{\pflet{$X$ be a set.}}
    \step{2}{\pick\ a choice function for $\pow X \setminus \{ \emptyset \}$}
    \begin{proof}
        \pf\ Lemma \ref{lemma:choice_function}.
    \end{proof}
    \step{3}{\pflet{a \emph{tower} in $X$ be a pair $(T, <)$ where $T \subseteq X$, $<$ is a well-ordering of $T$, and $x = c(X \setminus \{ y \in T \mid y < x \})$.}}
    \step{4}{For any two towers $(T_1, <_1)$ and $(T_2, <_2)$, either these two posets are equal or one is a section of the other.}
    \begin{proof}
        \step{a}{}
    \end{proof}
    \step{5}{For any tower $(T, <)$ in $X$ with $T \neq X$, there exists a tower in $X$ of which $(T, <)$ is a section.}
    \step{6}{\pflet{$T = \bigcup \{ T' \subseteq X \mid \exists R. (T,R) \text{ is a tower in } X \}$}}
    \step{7}{Define $<$ on $T$ by: $x < y$ iff there exists a tower $(T, R)$ in $X$ such that $x, y \in T$ and $xRy$.}
    \step{8}{$(T, <)$ is a tower in $X$.}
    \step{9}{$T = X$}
    \step{10}{$<$ is a well-ordering of $X$.}
    \qed
\end{proof}

\begin{theorem}[Maximum Principle (AC)]
    Every poset has a maximal chain.
\end{theorem}


\begin{lemma}[Zorn's Lemma (AC)]
    Let $A$ be a poset. If every chain in $A$ has an upper bound in $A$, then $A$ has a maximal element.
\end{lemma}

