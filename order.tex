\section{Order Theory}

\begin{definition}[Preorder]
    Let $X$ be a set. A \emph{preorder} on $X$ is a binary relation $\leq$ on $X$ such that:
    \begin{description}
        \item[Reflexivity] For all $x \in X$, we have $x \leq x$
        \item[Transitivity] For all $x, y, z \in X$, if $x \leq y$ and $y \leq z$ then $x \leq z$.  
    \end{description}
\end{definition}

\begin{definition}[Preordered Set]
    A \emph{preordered set} consists of a set $X$ and a preorder $\leq$ on $X$.
\end{definition}

\begin{proposition}
    \label{proposition:iso_strictly_monotone_sujective}
    Let $X$ and $Y$ be linearly ordered sets. Let $f : X \twoheadrightarrow Y$ be strictly monotone and surjective.
    Then $f$ is a poset isomorphism.
\end{proposition}

\begin{proof}
    \pf
    \step{3}{$f$ is injective.}
    \begin{proof}
        \step{a}{\pflet{$x, y \in X$}}
        \step{b}{\assume{$f(x) = f(y)$}}
        \step{c}{$x \nless y$}
        \begin{proof}
            \pf\ By strong motonicity.
        \end{proof}
        \step{d}{$y \nless x$}
        \begin{proof}
            \pf\ By strong motonicity.
        \end{proof}
        \step{e}{$x = y$}
        \begin{proof}
            \pf\ By trichotomy.
        \end{proof}
    \end{proof}
    \step{4}{$\inv{f}$ is monotone.}
    \begin{proof}
        \step{a}{\pflet{$x, y \in X$}}
        \step{b}{\assume{$x \leq y$}}
        \step{c}{$\inv{f}(x) \ngtr \inv{f}(y)$}
        \begin{proof}
            \pf\ By strong motonicity.
        \end{proof}
        \step{d}{$\inv{f}(x) < \inv{f}(y)$}
        \begin{proof}
            \pf\ By trichotomy.
        \end{proof}
    \end{proof}
    \qed
\end{proof}

\begin{definition}[Interval]
    Let $X$ be a preordered set and $Y \subseteq X$. Then $Y$ is an \emph{interval} if and only if, for
    all $a, b \in Y$ and $c \in X$, if $a \leq c \leq b$ then $c \in Y$.
\end{definition}

\begin{definition}[Linear Continuum]
    A linearly ordered set $L$ is a \emph{linear continuum} if and only if:
    \begin{enumerate}
        \item every nonempty subset of $L$ that is bounded above has a supremum
        \item $L$ is dense
    \end{enumerate}
\end{definition}

\begin{proposition}
    Every interval in a linear continuum is a linear continuum.
\end{proposition}

\begin{proof}
    \pf
    \step{1}{\pflet{$L$ be a linear continuum and $I$ an interval in $L$.}}
    \step{2}{Every nonempty subset of $I$ that is bounded above has a supremum in $I$.}
    \begin{proof}
        \step{a}{\pflet{$X \subseteq I$ be nonempty and bounded above by $b \in I$.}}
        \step{b}{\pflet{$s$ be the supremum of $X$ in $L$.}}
        \begin{proof}
            \pf\ Since $L$ is a linear continuum.
        \end{proof}
        \step{c}{$s \in I$}
        \begin{proof}
            \step{i}{\pick\ $a \in X$}
            \begin{proof}
                \pf\ Since $X$ is nonempty (\stepref{a}).
            \end{proof}
            \step{ii}{$a \leq s \leq b$}
            \step{iii}{$s \in I$}
            \begin{proof}
                \pf\ Since $I$ is an interval (\stepref{1}).
            \end{proof}
        \end{proof}
        \step{d}{$s$ is the supremum of $X$ in $I$}
    \end{proof}
    \step{3}{$I$ is dense.}
    \begin{proof}
        \step{a}{\pflet{$x, y \in I$ with $x < y$}}
        \step{b}{\pick\ $z \in L$ with $x < z < y$}
        \begin{proof}
            \pf\ Since $L$ is dense.
        \end{proof}
        \step{c}{$z \in I$}
        \begin{proof}
            \pf\ Since $I$ is an interval.
        \end{proof}
    \end{proof}
    \qed
\end{proof}

\begin{definition}[Ordered Square]
    The \emph{ordered square} $I_o^2$ is the set $[0,1]^2$ under the dictionary order.
\end{definition}

\begin{proposition}
    The ordered square is a linear continuum.
\end{proposition}

\begin{proof}
    \pf
    \step{2}{Every nonempty subset of $I_o^2$ bounded above has a supremum.}
    \begin{proof}
        \step{a}{\pflet{$X \subseteq I_o^2$ be nonempty and bounded above by $(b,c)$}}
        \step{b}{\pflet{$s = \sup \pi_1(X)$}}
        \begin{proof}
            \pf\ The set $\pi_1(X)$ is nonempty and bounded above by $b$.
        \end{proof}
        \step{c}{\case{$s \in \pi_1(X)$}}
        \begin{proof}
            \step{i}{\pflet{$t = \sup \{ y \in [0,1] \mid (s,y) \in X \}$}}
            \begin{proof}
                \pf\ This set is nonempty and bounded above by $c$.
            \end{proof}
            \step{ii}{$(s,t)$ is the supremum of $X$.}
        \end{proof}
        \step{d}{\case{$s \notin \pi_1(X)$}}
        \begin{proof}
            \pf\ In this case $(s,0)$ is the supremum of $X$.
        \end{proof}
    \end{proof}
    \step{3}{$I_o^2$ is dense.}
    \begin{proof}
        \step{a}{\pflet{$(x_1,y_1),(x_2,y_2) \in I_o^2$ with $(x_1,y_1) < (x_2,y_2)$}}
        \step{b}{\case{$x_1 < x_2$}}
        \begin{proof}
            \step{i}{\pick\ $x_3$ with $x_1 < x_3 < x_2$}
            \step{ii}{$(x_1,y_1) < (x_3,y_1) < (x_2,y_2)$}
        \end{proof}
        \step{c}{\case{$x_1 = x_2$ and $y_1 < y_2$}}
        \begin{proof}
            \step{i}{\pick\ $y_3$ with $y_1 < y_3 < y_2$}
            \step{ii}{$(x_1,y_1) < (x_1,y_3) < (x_2,y_2)$}
        \end{proof}
    \end{proof}
    \qed
\end{proof}

\begin{proposition}
    If $X$ is a well-ordered set then $X \times [0,1)$ under the dictionary order is a linear continuum.
\end{proposition}

\begin{proof}
    \pf
    \step{1}{Every nonempty set $A \subseteq X \times [0,1)$ bounded above has a supremum.}
    \begin{proof}
        \step{a}{\pflet{$A \subseteq X \times [0,1)$ be nonempty and bounded above}}
        \step{b}{\pflet{$x_0$ be the supremum of $\pi_1(A)$}}
        \step{c}{\case{$x_0 \in \pi_1(A)$}}
        \begin{proof}
            \step{i}{\pflet{$y_0$ be the supremum of $\{ y \in [0,1) \mid (x_0,y) \in A \}$}}
            \step{ii}{$(x_0,y_0)$ is the supremum of $A$.}
        \end{proof}
        \step{d}{\case{$x_0 \notin \pi_1(A)$}}
        \begin{proof}
            \pf\ In this case $(x_0,0)$ is the supremum of $A$.
        \end{proof}
    \end{proof}
    \step{2}{$X \times [0,1)$ is dense.}
    \begin{proof}
        \step{a}{\pflet{$(x_1,y_1), (x_2,y_2) \in X \times [0,1)$ with $(x_1,y_1) < (x_2,y_2)$}}
        \step{b}{\case{$x_1 < x_2$}}
        \begin{proof}
            \step{i}{\pick\ $y_3$ such that $y_1 < y_3 < 1$}
            \step{ii}{$(x_1,y_1) < (x_1,y_3) < (x_2,y_2)$}
        \end{proof}
        \step{c}{\case{$x_1 = x_2$ and $y_1 < y_2$}}
        \begin{proof}
            \step{i}{\pick\ $y_3$ such that $y_1 < y_3 < y_2$}
            \step{ii}{$(x_1,y_1) < (x_1,y_3) < (x_2,y_2)$}
        \end{proof}
    \end{proof}
    \qed
\end{proof}

\begin{lemma}
    \label{lemma:order_iso_real_intervals}
    For all $a,b,c,d \in \RR$ with $a < b$ and $c < d$, we have 
    $[a,b) \cong [c,d)$
\end{lemma}

\begin{proof}
    \pf\ The map $\lambda t. c + (t-a)(d-c)/(b-a)$ is an order isomorphism.
\end{proof}

\begin{proposition}
    \label{proposition:interval_concatenate_interval}
    Let $X$ be a linearly ordered set. Let $a < b < c$ in $X$. Then $[a,c) \cong [0,1)$
    if and only if $[a,b) \cong [b,c) \cong [0,1)$.
\end{proposition}

\begin{proof}
    \pf
    \step{1}{If $[a,c) \cong [0,1)$ then $[a,b) \cong [b,c) \cong [0,1)$}
    \begin{proof}
        \step{a}{\assume{$f : [a,c) \cong [0,1)$ is an order isomorphism}}
        \step{b}{$[a,b) \cong [0,1)$}
        \begin{proof}
            \pf
            \begin{align*}
                [a,b) & \cong [0,f(b)) & (\text{by the restriction of } f) \\
                & \cong [0,1) & (\text{Lemma \ref{lemma:order_iso_real_intervals}})
            \end{align*}
        \end{proof}
        \step{c}{$[b,c) \cong [0,1)$}
        \begin{proof}
            \pf\ Similar.
        \end{proof}
    \end{proof}
    \step{2}{If $[a,b) \cong [b,c) \cong [0,1)$ then $[a,c) \cong [0,1)$}
    \begin{proof}
        \pf
        \begin{align*}
            [a,c) & = [a,b) * [b,c) \\
            & \cong [0,1) * [0,1) \\
            & \cong [0,1/2) * [1/2,1) & (\text{Lemma \ref{lemma:order_iso_real_intervals}})\\
            & = 1
        \end{align*}
    \end{proof}
    \qed
\end{proof}

\begin{proposition}[CC]
    \label{proposition:interval_concatenate_intervals}
    Let $X$ be a linearly ordered set. Let $x_0 < x_1 < \cdots$ be a
    strictly increasing sequence in $X$ with supremum $b$. Then $[x_0,b)
    \cong [0,1)$ if and only if $[x_i,x_{i+1}) \cong [0,1)$ for all $i$.
\end{proposition}

\begin{proof}
    \pf
    \step{1}{If $[x_0,b) \cong [0,1)$ then $[x_i,x_{i+1}) \cong [0,1)$
    for all $i$.}
    \begin{proof}
        \pf\ By Lemma \ref{lemma:order_iso_real_intervals}
    \end{proof}
    \step{2}{If $[x_i,x_{i+1}) \cong [0,1)$ for all $i$ then $[x_0,b) \cong [0,1)$}
    \begin{proof}
        \step{a}{\assume{$[x_i,x_{i+1}) \cong [0,1)$ for all $i$}}
        \step{b}{\pick\ an order isomorphism $f_i : [x_i,x_{i+1}) \cong [1/2^i,2/2^{i+1})$
        for each $i$.}
        \begin{proof}
            \pf\ By Lemma \ref{lemma:order_iso_real_intervals}
        \end{proof}
        \step{c}{The union of the $f_i$s is an order isomorphism $[x_0,b) \cong [0,1)$}
    \end{proof}
    \qed
\end{proof}

\section{Partially Ordered Sets}

\begin{definition}[Partial Order]
    A \emph{partial order} on a set $X$ is a preorder $\leq$ that is \emph{anti-symmetric}, i.e. whenever $x \leq y$ and $y \leq x$
    then $x = y$.    
\end{definition}

\section{Strict Partial Order}

\begin{definition}[Strict Partial Order]
    A \emph{strict partial order} on a set $X$ is a relation on $X$ that is
    transitive and irreflexive.
\end{definition}

\begin{proposition}
    If $<$ is a strict partial order on $X$ and $x,y \in X$, then at most
    one of $x < y$, $y < x$, $x = y$ holds.
\end{proposition}

\begin{proposition}
    If $<$ is a strict partial order then the relation $\leq$ defined by:
    $x \leq y$ iff $x < y$ or $x = y$, is a partial order.
\end{proposition}

\begin{theorem}
    If $R$ is a well-founded relation then its transitive
    closure is a partial order.
\end{theorem}

\begin{definition}[Linear Order]
    A \emph{linear order} on a set $X$ is a partial order such that, for any $x, y \in X$, either $x \leq y$ or $y \leq x$.
\end{definition}

\section{Strict Linear Orders}

\begin{definition}[Strict Linear Order (Extensionality, Pairing)]
    Let $A$ be a set. A \emph{strict linear order} on $A$ is a binary relation
    $R$ on $A$ that is transitive and satisfies \emph{trichotomy}: for any
    $x, y \in A$, exactly one of $xRy$, $x=y$, $yRx$ holds.
\end{definition}

\begin{theorem}
    Let $R$ be a strict linear order on $A$. Then there is no $x \in A$
    such that $xRx$.
\end{theorem}

\begin{proof}
    \pf\ Immediate from trichotomy.
\end{proof}

\section{Well Orderings}

\begin{definition}[Well-ordering]
    A \emph{well-order} on a set $X$ is a linear order such that every nonempty set has a least element.
\end{definition}

\begin{proposition}
    Let $\leq$ be a linear order on $X$. Then $\leq$ is a well-order iff there is no
    function $f : \NN \rightarrow X$ such that $f(n+1) < f(n)$ for all $n$.    
\end{proposition}

\begin{definition}[Initial Segment]
    Given a well-ordered set $X$ and $\alpha \in X$, the \emph{initial segment}
    of $X$ up to $\alpha$ is $\seg \alpha = \{ x \in X \mid x < \alpha \}$.
\end{definition}

\begin{theorem}[Transfinite Induction]
    Let $\leq$ be a linear order on $J$. Then the following are equivalent:
    \begin{enumerate}
        \item $\leq$ is a well-order on $J$.
        \item For every subset $J_0 \subseteq J$, if the following condition holds:
        \begin{itemize}
            \item For every $\alpha \in J$, if $\seg \alpha \subseteq J_0$ then $\alpha \in J$.
        \end{itemize}
        then $J_0 = J$.
    \end{enumerate}
\end{theorem}

\begin{axioms}[Replacement]
    Let $\mathbf{H}$ be a class term. If $\dom \mathbf{H}$ is a set
    then $\mathbf{H}$ is a set.
\end{axioms}

\begin{theorem}[Transfinite Recursion]
    Let $J$ be a well-ordered set and $C$ a set. Let $\FF$ be the set of all functions from a section of $J$ to $C$.
    Let $G$ be a function with domain $\FF$. Then there exists a unique function $h$ with domain $J$
    such that, for all $\alpha \in J$,
    we have $h(\alpha) = \rho(h \restriction \seg \alpha)$.
\end{theorem}

\begin{proof}
    \pf
    \step{1}{If $v$ is a function and $t \in J$, we say $v$ is \emph{$\rho$-constructed up to $t$}
    iff $\dom v = \{ x \in J \mid x \leq t \}$ and, for all $x \in \dom v$,
    we have $v(x) = \rho(v \restriction \seg x)$}
    \step{2}{If $t_1 \leq t_2$, $v_1$ is $\rho$-constructed up to $t_1$,
    and $v_2$ is $\rho$-constructed up to $t_2$, then $v_1 = v_2 \restriction \{ x \in J \mid x \leq t_1 \}$}
    \step{3}{\pflet{$\mathcal{K}$ be the set of all functions that are
    $\rho$-constructed up to some $t \in J$}}
    \begin{proof}
        \pf\ $\mathcal{K}$ is a set by a Replacement Axiom.
    \end{proof}
    \step{4}{\pflet{$F = \bigcup \mathcal{K}$}}
    \step{5}{$F$ is a function}
    \step{6}{For all $x \in \dom F$ we have $F(x) = \rho(F \restriction \seg x)$}
    \step{7}{$\dom F = J$}
    \step{8}{$F$ is unique}
    \qed
\end{proof}

\begin{theorem}
    The following are equivalent.
    \begin{enumerate}
        \item The Axiom of Choice
        \item (\emph{Well-Ordering Theorem}) Every set has a well-ordering.
        \item (\emph{Zorn's Lemma}) Let $X$ be a poset. If every chain in $X$ has 
        an upper bound in $X$, then $X$ has a maximal element.
    \end{enumerate}
\end{theorem}

\begin{proof}
    \pf
    \step{1}{$1 \Rightarrow 2$}
    \begin{proof}
        \pf
        \step{0}{\assume{The Axiom of Choice}}
        \step{1}{\pflet{$X$ be a set.}}
        \step{2}{\pick\ a choice function for $\pow X \setminus \{ \emptyset \}$}
        \begin{proof}
            \pf\ Lemma \ref{lemma:choice_function}.
        \end{proof}
        \step{3}{\pflet{a \emph{tower} in $X$ be a pair $(T, <)$ where $T \subseteq X$, $<$ is a well-ordering of $T$, and $x = c(X \setminus \{ y \in T \mid y < x \})$.}}
        \step{4}{For any two towers $(T_1, <_1)$ and $(T_2, <_2)$, either these two posets are equal or one is a section of the other.}
        \begin{proof}
            \step{a}{}
        \end{proof}
        \step{5}{For any tower $(T, <)$ in $X$ with $T \neq X$, there exists a tower in $X$ of which $(T, <)$ is a section.}
        \step{6}{\pflet{$T = \bigcup \{ T' \subseteq X \mid \exists R. (T,R) \text{ is a tower in } X \}$}}
        \step{7}{Define $<$ on $T$ by: $x < y$ iff there exists a tower $(T, R)$ in $X$ such that $x, y \in T$ and $xRy$.}
        \step{8}{$(T, <)$ is a tower in $X$.}
        \step{9}{$T = X$}
        \step{10}{$<$ is a well-ordering of $X$.}
    \end{proof}
    \step{2}{$2 \Rightarrow 3$}
    \begin{proof}
        \step{a}{\assume{The Well-Ordering Theorem}}
        \step{b}{\pflet{$X$ be a poset in which every chain has an upper bound.}}
        \step{c}{\pick\ a well-ordering $R$ of $X$}
        \step{d}{Define $F : X \rightarrow \{ 0,1 \}$ by transfinite $R$-recursion by:
        \[ F(a) = \begin{cases}
            1 & \text{if } b < a \text{ for all } b \text{ such that } b R a \text{ and } f(b) = 1 \\
            0 & \text{otherwise}
        \end{cases} \]}
        \step{e}{\pflet{$C = \{ a \in X \mid f(a) = 1 \}$}}
        \step{f}{$C$ is a chain in $X$}
        \begin{proof}
            \step{i}{\pflet{$x, y \in C$}}
            \step{ii}{\assume{without loss of generality $x R y$}}
            \step{iii}{$f(y) = 1$}
            \step{iv}{for all $z$ such that $zRy$ and $f(z) = 1$ we have $z < y$}
            \step{v}{$x < y$}
        \end{proof}
        \step{g}{\pick\ an upper bound $u$ for $C$}
        \step{h}{$u$ is maximal in $X$}
        \begin{proof}
            \step{i}{\pflet{$x \in X$ with $u \leq x$}}
            \step{ii}{for all $b$ such that $bRx$ and $f(b) = 1$ we have $b < x$}
            \begin{proof}
                \pf\ Since $b \in C$ so $b \leq u \leq x$
            \end{proof}
            \step{iii}{$f(u) = 1$}
            \step{iv}{$u \leq x$}
            \step{v}{$u = x$}
        \end{proof}
        \step{3}{$3 \Rightarrow 1$}
        \begin{proof}
            \step{a}{\assume{Zorn's Lemma}}
            \step{b}{\pflet{$R$ be a relation}}
            \step{c}{\pflet{$\mathcal{A}$ be the poset of functions that are subsets of $R$ under $\subseteq$}}
            \step{d}{Every chain in $\mathcal{A}$ has an upper bound}
            \begin{proof}
                \step{i}{\pflet{$\mathcal{C} \subseteq \mathcal{A}$ be a chain.} \prove{$\bigcup \mathcal{C} \in \mathcal{A}$}}
                \step{ii}{\assume{$(x,y), (x,z) \in \bigcup \mathcal{C}$}}
                \step{iii}{\pick\ $f, g \in \mathcal{C}$ such that $f(x) = y$ and $g(x) = z$}
                \step{iv}{\assume{without loss of generality $f \subseteq g$}}
                \step{v}{$g(x) = y$}
                \step{vi}{$y = z$}
            \end{proof}
            \step{e}{\pick\ $F$ maximal in $\mathcal{A}$}
            \step{f}{$\dom F = \dom R$}
            \begin{proof}
                \step{i}{\assume{for a contradiction $x \in \dom R - \dom F$}}
                \step{ii}{\pick\ $y$ such that $xRy$}
                \step{iii}{\pflet{$G = F \cup \{ (x,y) \}$}}
                \step{iv}{$G \in \mathcal{A}$}
                \step{v}{$F \subset G$}
                \qedstep
                \begin{proof}
                    \pf\ This contradicts the maximality of $F$.
                \end{proof}
            \end{proof}
        \end{proof}
    \end{proof}
    \qed
\end{proof}

\begin{theorem}[Cardinal Comparability]
    The Axiom of Choice is equivalent to the \emph{Cardinal Comparability Theorem}: for any two sets $A$ and $B$,
    either $A \preccurlyeq B$ or $B \preccurlyeq A$.
\end{theorem}

\begin{proof}
    \pf
    \step{1}{Zorn's Lemma implies Cardinal Comparability}
    \begin{proof}
        \step{a}{\assume{Zorn's Lemma}}
        \step{b}{\pflet{$A$ and $B$ be sets.}}
        \step{c}{\pflet{$\mathcal{A}$ be the poset of all injective functions $f$
        such that $\dom f \subseteq C$ and $\ran f \subseteq D$ under $\subseteq$}}
        \step{d}{Every chain in $\mathcal{A}$ has an upper bound.}
        \begin{proof}
            \step{i}{\pflet{$\mathcal{C} \subseteq \mathcal{A}$ be a chain.}
            \prove{$\bigcup \mathcal{C} \in \mathcal{A}$}}
            \step{ii}{$\bigcup \mathcal{C}$ is a function.}
            \begin{proof} % TODO Extract lemma
                \step{one}{\pflet{$(x,y),(x,z) \in \bigcup \mathcal{C}$}}
                \step{two}{\pick\ $f, g \in \mathcal{C}$ such that $f(x) = y$ and $g(x) = z$}
                \step{three}{\assume{without loss of generality $f \subseteq g$}}
                \step{four}{$g(x) = y$}
                \step{five}{$y = z$}
            \end{proof}
            \step{iii}{$\bigcup \mathcal{C}$ is injective.}
            \begin{proof}
                \pf\ Similar.
            \end{proof}
        \end{proof}
        \step{e}{\pick\ $\hat{f}$ maximal in $\mathcal{A}$}
        \begin{proof}
            \pf\ By Zorn's Lemma.
        \end{proof}
        \step{f}{Either $\dom \hat{f} = C$ or $\ran \hat{f} = D$}
        \begin{proof}
            \step{i}{\assume{for a contradiction $\dom \hat{f} \subset C$ and $\ran \hat{f} \subset D$}}
            \step{ii}{\pick\ $x \in C - \dom \hat{f}$ and $y \in D - \ran \hat{f}$}
            \step{iii}{\pflet{$g = \hat{f} \cup \{(x,y)\}$}}
            \step{iv}{$g \in \mathcal{A}$}
            \step{v}{$\hat{f} \subset g$}
            \qedstep
            \begin{proof}
                \pf\ This contradicts the maximality of $\hat{f}$.
            \end{proof}
        \end{proof}
        \step{g}{If $\dom \hat{f} = C$ then $C \preccurlyeq D$}
        \step{h}{If $\ran \hat{f} = D$ then $D \preccurlyeq C$}
    \end{proof}
    \step{2}{Cardinal Comparability implies the Well-Ordering Theorem}
    \begin{proof}
        \step{a}{\assume{Cardinal Comparability}}
        \step{b}{\pflet{$A$ be a set}}
        \step{c}{\pick\ an ordinal $\alpha$ such that $\alpha \not\preccurlyeq A$}
        \step{d}{$A \preccurlyeq \alpha$}
        \begin{proof}
            \pf\ By Cardinal Comparability.
        \end{proof}
        \step{e}{\pick\ an injection $f : A \rightarrow \alpha$}
        \step{f}{Define $<$ on $A$ by $x < y$ iff $f(x) \in f(y)$}
        \step{g}{$<$ is a well-ordering on $A$.}
    \end{proof}
    \qed
\end{proof}

\begin{theorem}
    Given two well-ordered sets $A$ and $B$, either $A \cong B$ or one of $A$, $B$
    is isomorphic to an initial segment of the other.
\end{theorem}

\section{Ordinal Numbers}

\begin{definition}
    Let $(A, \leq)$ be a well-ordered set. The \emph{ordinal number} of $(A, \leq)$
    is the range of $E$, where $E$ is the unique function with domain $A$
    such that $E(t) = \ran (E \restriction \seg t)$ for all $t \in A$.
\end{definition}

\begin{theorem}
    Let $(A, \leq)$ be a well-ordered set and $E : A \rightarrow \alpha$ be the
    canonical function onto the ordinal of $A$. Then:
    \begin{enumerate}
        \item For all $t \in A$ we have $E(t) \notin E(t)$.
        \item $E$ is a bijection.
        \item For any $s, t \in A$, we have $s < t$ if and only if $E(s) \in E(t)$.
        \item $\alpha$ is a transitive set.
        \item $\alpha$ is well-ordered by $\in$
        \item $E$ is an order isomorphism between $(A, \leq)$ and $(\alpha, \underline{\in})$.
    \end{enumerate}
\end{theorem}

\begin{theorem}
    Two well-ordered sets are isomorphic if and only if they have the same ordinal number.
\end{theorem}

\begin{theorem}
    A set is an ordinal number if and only if it is a transitive set well-ordered
    by $\in$.
\end{theorem}

\begin{theorem}
    Every member of an ordinal number is an ordinal number.
\end{theorem}

\begin{theorem}
    Any transitive set of ordinal numbers is an ordinal number.
\end{theorem}

\begin{theorem}
    The empty set is an ordinal number.
\end{theorem}

\begin{theorem}
    The successor of an ordinal number is an ordinal number.
\end{theorem}

\begin{theorem}
    If $A$ is a set of ordinal numbers then $\bigcup A$ is an ordinal number.
\end{theorem}

\begin{theorem}
    Any nonempty set of ordinal numbers has a least element.
\end{theorem}

\begin{theorem}[Burali-Forti Paradox]
    The class of ordinal numbers is a proper class.
\end{theorem}

\begin{theorem}[Hartogs' Theorem]
    For any set $A$, there exists an ordinal that is not dominated by $A$.
\end{theorem}

\begin{proof}
    \pf
    \step{1}{\pflet{$\alpha$ be the class of all ordinals $\beta$ such that
    $\beta \preccurlyeq A$}}
    \step{2}{$\alpha$ is a set.}
    \begin{proof}
        \step{a}{\pflet{$W$ be the set of all pairs $(B, \leq)$ such that
        $B \subseteq A$ and $\leq$ is a well-ordering on $B$.}}
        \step{b}{Every member of $\alpha$ is the ordinal number of a member
        of $W$}
        \qedstep
        \begin{proof}
            \pf\ By a Replacement Axiom.
        \end{proof}
    \end{proof}
    \step{3}{$\alpha$ is an ordinal.}
    \step{4}{$\alpha$ is not dominated by $A$.}
    \qed
\end{proof}

\begin{definition}
    A class term $\mathbf{F} : \On \rightarrow \On$ is
    \emph{continuous} iff, for every limit ordinal $\lambda$,
    we have $\mathbf{F}(\lambda) = \sup_{\alpha < \lambda} F(\alpha)$.
\end{definition}

\begin{theorem}
    Let $\mathbf{F} : \On \rightarrow \On$. If $\mathbf{F}$ is continuous
    and $\mathbf{F}(\alpha) < \mathbf{F}(\alpha + 1)$ for every ordinal $\alpha$,
    then $\mathbf{F}$ is strictly monotone.
\end{theorem}

\begin{definition}
    A class term $\mathbf{F} : \On \rightarrow \On$ is \emph{normal}
    iff it is strictly monotone and continuous.
\end{definition}

\begin{theorem}
    Let $\mathbf{F} : \On \rightarrow \On$ be normal. For every
    ordinal $\beta \geq \mathbf{F}(0)$, there exists a greatest
    ordinal $\alpha$ such that $\mathbf{F}(\alpha) \leq \beta$.
\end{theorem}

\begin{theorem}
    Let $\mathbf{F} : \On \rightarrow \On$ be normal.
    Let $S$ be a set of ordinals. Then $\mathbf{F}(\sup S) = \sup_{\alpha \in S} \mathbf{F}(\alpha)$.
\end{theorem}

\begin{theorem}[Veblen Fixed-Point Theorem]
    Let $\mathbf{F} : \On \rightarrow \On$ be normal.
    For every ordinal $\alpha$, there exists $\beta \geq \alpha$
    such that $\mathbf{F}(\beta) = \beta$.
\end{theorem}

\begin{proof}
    \pf\ Let $\beta$ be the supremum of $\alpha$, $\mathbf{F}(\alpha)$, $\mathbf{F}^2(\alpha)$, \ldots. \qed
\end{proof}

\begin{lemma}
    Let $\alpha$ be an ordinal. Let $(f(\gamma))_{\gamma < \alpha}$
    be an $\alpha$-sequence of ordinals. Then there exists $\beta \leq \alpha$
    and an increasing sequence of ordinals $(g(\gamma))_{\gamma < \beta}$
    such that $\sup_{\gamma < \alpha} f(\gamma) = \sup_{\gamma < \beta} g(\gamma)$.
\end{lemma}

\section{Cardinal Numbers}

\begin{definition}[Cardinal Number (AC)]
    For any set $A$, the \emph{cardinal number} of $A$, $\card A$, is the least ordinal
    equinumerous with $A$.

    There exists some ordinal equinumerous with $A$ by the Well-Ordering Theorem.
\end{definition}

\begin{theorem}
    For any sets $A$ and $B$, we have $A \equiv B$ if and only if $\card A = \card B$.
\end{theorem}

\begin{theorem}
    A set $A$ is finite if and only if $\card A$ is a natural number.
\end{theorem}

\begin{theorem}
    The supremum of a set of cardinal numbers is a cardinal number.
\end{theorem}

\section{Cardinal Arithmetic}

\begin{definition}
    For cardinal numbers $\kappa$ and $\lambda$, the \emph{sum} $\kappa + \lambda$
    is the cardinal number of $A \cup B$, where $A$ and $B$ are disjoint sets
    of cardinality $\kappa$ and $\lambda$ respectively.
\end{definition}

\begin{theorem}
    $\kappa + \lambda = \lambda + \kappa$
\end{theorem}

\begin{theorem}
    $\kappa + (\lambda + \mu) = (\kappa + \lambda) + \mu$
\end{theorem}

\begin{theorem}
    The definition of addition agrees with the definition on natural numbers.
\end{theorem}

\begin{definition}
    For cardinal numbers $\kappa$ and $\lambda$, the \emph{product}
    $\kappa \lambda$ is the cardinality of $\kappa \times \lambda$.
\end{definition}

\begin{theorem}
    $\kappa \lambda = \lambda \kappa$
\end{theorem}

\begin{theorem}
    $\kappa (\lambda \mu) = (\kappa \lambda) \mu$
\end{theorem}

\begin{theorem}
    $\kappa (\lambda + \mu) = \kappa \lambda + \kappa \mu$
\end{theorem}

\begin{theorem}
    The definition of multiplication agrees with the definition on natural numbers.
\end{theorem}

\begin{theorem}[AC]
    For any infinite cardinal $\kappa$ we have $\kappa \kappa = \kappa$.
\end{theorem}

\begin{proof}
    \pf
    \step{1}{\pflet{$B$ be a set with cardinality $\kappa$}}
    \step{2}{\pflet{$\mathcal{H} = \{ \emptyset \} \cup \{ f \mid
    \exists A \subseteq B. A \text{ is infinite and $f$ is a bijection
    between $A \times A$ and $A$} \}$}}
    \step{3}{For every chain $\mathcal{C} \subseteq \mathcal{H}$
    we have $\bigcup \mathcal{C} \in \mathcal{H}$}
    \step{4}{\pick\ a maximal $f_0$ in $\mathcal{H}$}
    \step{5}{$f_0 \neq \emptyset$}
    \begin{proof}
        \pf\ $B$ has a subset of cardinality $\aleph_0$ and $\aleph_0 \aleph_0 = \aleph_0$.
    \end{proof}
    \step{6}{\pflet{$A_0$ be the set such that $f_0$ is a bijection between
    $A_0 \times A_0$ and $A_0$}}
    \step{7}{\pflet{$\lambda = \card A_0$}}
    \step{8}{$\card (B - A_0) < \lambda$}
    \step{9}{$\kappa = \lambda$}
    \begin{proof}
        \pf
        \begin{align*}
            \kappa & = \card A_0 + \card (B - A_0) \\
            & \leq \lambda + \lambda \\
            & = 2 \lambda \\
            & \leq \lambda \lambda \\
            & = \lambda & (\text{\stepref{6}}) \\
            & \leq \kappa & \qed
        \end{align*}
    \end{proof}
\end{proof}

\begin{theorem}[Absorption Law]
    Let $\kappa$ and $\lambda$ be cardinal numbers such that
    $0 < \kappa \leq \lambda$ and $\lambda$ is infinite. Then
    \[ \kappa + \lambda = \lambda \enspace . \]
\end{theorem}

\begin{theorem}[Absorption Law]
    Let $\kappa$ and $\lambda$ be cardinal numbers such that
    $0 < \kappa \leq \lambda$ and $\lambda$ is infinite. Then
    \[ \kappa \lambda = \lambda \enspace . \]
\end{theorem}

\begin{definition}
    For cardinal numbers $\kappa$ and $\lambda$, we write $\kappa^\lambda$
    for the cardinality of the set of functions from $\lambda$ to $\kappa$.
\end{definition}

\begin{theorem}
    $\kappa^{\lambda + \mu} = \kappa^\lambda + \kappa^\mu$
\end{theorem}

\begin{theorem}
    $(\kappa \lambda)^\mu = \kappa^\mu \lambda^\mu$
\end{theorem}

\begin{theorem}
    $(\kappa^\lambda)^\mu = \kappa^{\lambda \mu}$
\end{theorem}

\begin{theorem}
    The definition of exponentiation agrees with the definition on natural numbers.
\end{theorem}

\begin{theorem}
    Given sets $A$ and $B$, we have $\card A \leq \card B$ if and only if
    $A \preccurlyeq B$.
\end{theorem}

\begin{definition}
    Let $\aleph_0 = \card \NN$.
\end{definition}

\begin{theorem}[AC]
    For any infinite cardinal $\kappa$ we have $\aleph_0 \leq \kappa$.
\end{theorem}

\begin{theorem}[Maximum Principle (AC)]
    Every poset has a maximal chain.
\end{theorem}

\section{Rank of a Set}

\begin{definition}[Cumulative Hierarchy of Sets]
    For every ordinal $\alpha$, define the \emph{rank} $V_\alpha$
    by transfinite recursion thus:
    \begin{align*}
        V_0 & = \emptyset \\
        V_{\alpha + 1} & = \pow V_\alpha \\
        V_\lambda & = \bigcup_{\alpha < \lambda} V_\alpha
    \end{align*}
    for $\lambda$ a limit ordinal.

    The \emph{von Neumann universe} is the class $\mathbf{V} = \bigcup_{\alpha \in \On} V_\alpha$.
\end{definition}

\begin{theorem}
    If $\lambda$ is a limit ordinal and $\lambda > \omega$
    then $V_\lambda$ is a model of Zermelo set theory.
\end{theorem}

\begin{lemma}[AC]
    There exists a well-ordered set in $V_{\omega 2}$ whose
    ordinal is not in $V_{\omega 2}$.
\end{lemma}

\begin{proof}
    \pf\ Pick a well-ordering $<$ of $\pow \NN$. Then $(\pow \NN, <) \in
    V_{\omega 2}$ but its ordinal is not because its ordinal is uncountable. \qed
\end{proof}

\begin{theorem}
    The set $V_{\omega 2}$ is not a model of Zermelo-Fraenkel set theory.
\end{theorem}

Thus, the Replacement Axioms cannot be proven from the other axioms.

\begin{definition}[Well-Founded Set]
    A set $A$ is \emph{well-founded} iff $A \in V_\alpha$ for some $\alpha \in \On$.
\end{definition}

\begin{definition}[Rank]
    The \emph{rank} of a well-founded set $A$, $\rank A$, is the least
    ordinal $\alpha$ such that $A \in V_\alpha$.    
\end{definition}

\begin{theorem}
    If $A \in B$ and $B$ is well-founded then $A$ is well-founded
    and $\rank A < \rank B$.
\end{theorem}

\begin{theorem}
    If $A$ is a set and every member of $A$ is well-founded then $A$ is well-founded
    and $\rank A = \sup_{B \in A} (\rank B + 1)$.
\end{theorem}

\begin{theorem}
    The Axiom of Regularity is equivalent to the statement that every set is well-founded.
\end{theorem}

\section{Transfinite Recursion Again}

\begin{theorem}
    Let $\mathbf{A}$ be a class. Let $\mathbf{B}$
    be the class of all functions $f : \alpha \rightarrow \mathbf{A}$
    for some ordinal $\alpha$. Let $\mathbf{F}
    : \mathbf{B} \rightarrow \mathbf{A}$ be a class term.
    Then there exists a unique class term $\mathbf{G} : \On
    \rightarrow \mathbf{A}$ such that, for all $\alpha \in \On$,
    we have $\mathbf{G}(\alpha) = \mathbf{F}(\mathbf{G} \restriction \alpha)$.
\end{theorem}

\section{Alephs}

\begin{definition}
    Define the cardinal number $\aleph_\alpha$ for every ordinal
    $\alpha$ by transfinite recursion on $\alpha$ thus:
    $\aleph_\alpha$ is the least infinite cardinal different from
    $\aleph_\beta$ for all $\beta < \alpha$.
\end{definition}

\begin{theorem}
    If $\alpha < \beta$ then $\aleph_\alpha < \aleph_\beta$.
\end{theorem}

\begin{theorem}
    Every infinite cardinal has the form $\aleph_\alpha$
    for some ordinal $\alpha$.
\end{theorem}

\section{Ordinal Arithmetic}

\begin{definition}[Sum]
    Let $\alpha$ and $\beta$ be ordinals. The \emph{sum}
$\alpha + \beta$ is the ordinal of the concatenation of $A$
followed by $B$, where $A$ is a well-ordered set of ordinal $\alpha$
and $B$ a well-ordered set of ordinal $\beta$.
\end{definition}

\begin{theorem}
    Addition is associative.
\end{theorem}

\begin{theorem}
    $\alpha + 0 = \alpha$
\end{theorem}

\begin{theorem}
    $0 + \alpha = \alpha$
\end{theorem}

\begin{theorem}
    For $\lambda$ a limit ordinal we have $\alpha + \lambda = \sup_{\beta < \lambda} (\alpha + \beta)$
\end{theorem}

\begin{theorem}
    For any $\alpha$, the class term that maps $\beta$ to $\alpha + \beta$ is normal.
\end{theorem}

\begin{theorem}
    $\beta < \gamma$ iff $\alpha + \beta < \alpha + \gamma$.
\end{theorem}

\begin{theorem}
    If $\beta \leq \gamma$ then $\beta + \alpha \leq \gamma + \alpha$.
\end{theorem}

\begin{theorem}[Subtraction Theorem]
    If $\alpha < \beta$ then there exists a unique $\delta$ such that $\alpha + \delta < \beta$.
\end{theorem}

\begin{definition}[Product]
    Let $\alpha$ and $\beta$ be ordinals. The \emph{sum}
$\alpha + \beta$ is the ordinal of $A \times B$ ordered under
the Hebrew
lexicographic order, where $A$ is a well-ordered set of ordinal $\alpha$
and $B$ a well-ordered set of ordinal $\beta$.
\end{definition}

\begin{theorem}
    Multiplication is associative.
\end{theorem}

\begin{theorem}
    Multiplication distributes over addition on the left.
\end{theorem}

\begin{theorem}
    $\alpha 1 = \alpha$
\end{theorem}

\begin{theorem}
    $1 \alpha = \alpha$
\end{theorem}

\begin{theorem}
    $\alpha 0 = 0$
\end{theorem}

\begin{theorem}
    $0 \alpha = 0$
\end{theorem}

\begin{theorem}
    For $\lambda$ a limit ordinal, we have $\alpha \lambda = \sup_{\beta < \lambda} (\alpha \beta)$.
\end{theorem}

\begin{theorem}
    For $\alpha > 0$, the class term that maps $\beta$ to $\alpha \beta$ is normal.
\end{theorem}

\begin{theorem}
    If $\alpha > 0$, then $\beta < \gamma$ iff $\alpha \beta < \alpha \gamma$.
\end{theorem}

\begin{theorem}
    If $\beta \leq \gamma$ then $\beta \alpha \leq \gamma \alpha$.
\end{theorem}

\begin{theorem}[Division Theorem]
For any ordinals $\alpha$ and $\delta$ with $\delta \neq 0$,
there exist unique ordinals $\beta$ and $\gamma$ with
$\gamma < \delta$ and $\alpha = \delta \beta + \gamma$.
\end{theorem}

\begin{definition}[Exponentiation]
    For ordinals $\alpha$ and $\beta$, define the ordinal $\alpha^\beta$
    by transfinite recursion on $\beta$ by:
    \begin{align*}
        \alpha^0 & = 1 \\
        \alpha^{\beta + 1} & = \alpha^\beta + \alpha \\
        \alpha^\lambda & = \sup_{\beta < \lambda} \alpha^\beta
    \end{align*}
    for $\lambda$ a limit ordinal.
\end{definition}

\begin{theorem}
    For $\alpha > 1$, the class term that maps $\beta$ to $\alpha^\beta$ is normal.
\end{theorem}

\begin{theorem}
    If $\alpha > 1$, then $\beta < \gamma$ iff $\alpha^\beta < \alpha^\gamma$.
\end{theorem}

\begin{theorem}
    If $\beta \leq \gamma$ then $\beta^\alpha \leq \gamma^\alpha$.
\end{theorem}

\begin{theorem}[Logarithm Theorem]
    Let $\alpha$ and $\beta$ be ordinals with $\alpha \neq 0$
    and $\beta > 1$. Then there exist unique ordinals $\gamma$,
    $\delta$ and $\rho$ such that $\delta \neq 0$, $\delta < \beta$,
    $\rho < \beta^\gamma$, and $\alpha = \beta^\gamma \delta + \rho$.
\end{theorem}

\begin{theorem}
    \[ \alpha^{\beta + \gamma} = \alpha^\beta \alpha^\gamma \]
\end{theorem}

\begin{theorem}
    \[ (\alpha^\beta)^\gamma = \alpha^{\beta \gamma} \]
\end{theorem}

\section{Beth Cardinals}

\begin{definition}
    Define the cardinal $\beth_\alpha$ for every ordinal $\alpha$ by:
    \begin{align*}
        \beth_0 & = \aleph_0 \\
        \beth_{\alpha + 1} & = 2^{\beth_\alpha} \\
        \beth_\lambda & = \sup_{\alpha < \lambda} \beth_\alpha
    \end{align*}
    for $\lambda$ a limit ordinal.
\end{definition}

\begin{lemma}
    For any ordinal $\alpha$ we have $\card V_{\omega + \alpha} = \beth_\alpha$.
\end{lemma}

\section{Cofinality}

\begin{definition}[Cofinality]
    For $\lambda$ a limit ordinal, the \emph{cofinality} of $\lambda$,
    $\cf \lambda$, is the least cardinal $\kappa$ such that $\lambda$ is the supremum
    of a set of $\kappa$ smaller ordinals.

    We extend $\cf$ to all the ordinals by setting $\cf 0 = 0$ and
    $\cf (\alpha + 1) = 1$.
\end{definition}

\begin{theorem}
    For any limit ordinal $\lambda$ we have $\cf \aleph_\lambda = \cf \lambda$.
\end{theorem}

\begin{lemma}
    Let $\lambda$ be a limit ordinal. Then $\cf \lambda$
    is the least ordinal $\alpha$ such that there exists an
    increasing $\alpha$-sequence of ordinals with limit $\lambda$.
\end{lemma}

\begin{theorem}
    Let $\lambda$ be an infinite cardinal. Then $\cf \lambda$
    is the least cardinal number $\kappa$ such that $\lambda$
    can be partitioned into $\kappa$ sets each of cardinality $< \lambda$.
\end{theorem}

\begin{theorem}[K\"{o}nig's Theorem]
    Let $\kappa$ be an infinite cardinal. Then $\kappa < 2^{\cf \kappa}$.
\end{theorem}

\begin{corollary}
    $2^{\aleph_0} \neq \aleph_\omega$.
\end{corollary}
    
\begin{definition}[Regular]
    A cardinal $\kappa$ is \emph{regular} iff $\cf \kappa = \kappa$.
\end{definition}

\begin{theorem}
    For any ordinal $\lambda$, we have $\cf \lambda$ is a regular cardinal.
\end{theorem}

\begin{definition}[Singular]
    A cardinal $\kappa$ is \emph{singular} iff $\cf \kappa < \kappa$.
\end{definition}

\begin{theorem}
    For any ordinal $\alpha$ we have $\aleph_{\alpha + 1}$ is a regular cardinal.
\end{theorem}

\section{Inaccessible Cardinals}

\begin{definition}[Inaccessible]
    A cardinal number $\kappa$ is \emph{inaccessible}
    iff:
    \begin{itemize}
        \item $\kappa > \aleph_0$
        \item For every cardinal $\lambda < \kappa$ we have $2^\lambda < \kappa$
        \item $\kappa$ is regular.   
    \end{itemize}
\end{definition}

\begin{lemma}
    If $\kappa$ is inaccessible and $\alpha < \kappa$ then $\beth_\alpha < \kappa$.
\end{lemma}

\begin{lemma}
    If $\kappa$ is inaccessible and $A \in V_\kappa$ then $\card A < \kappa$.
\end{lemma}

\begin{theorem}
    If $\kappa$ is inaccessible then $V_\kappa$ is a model of ZF.
\end{theorem}

\section{Directed Set}

\begin{definition}[Directed Set]
    A preodered set $P$ is \emph{directed} iff, for all $a, b \in P$,
    there exists $c \in P$ such that $a \leq c$ and $b \leq c$.
\end{definition}

\begin{proposition}
    Every linearly ordered set is directed.
\end{proposition}

\begin{proposition}
    For any set $A$, the $\pow A$ under $\subseteq$ is directed.
\end{proposition}

\section{Cofinal Set}

\begin{definition}[Cofinal]
    Let $A$ be a preordered set and $B \subseteq A$. Then $B$ is \emph{cofinal}
    if and only if, for every $x \in A$, there exists $y \in B$ such that
    $x \leq y$.
\end{definition}

\begin{proposition}
    If $A$ is a directed preordered set and $B \subseteq A$ is cofinal then $B$ is
    directed.
\end{proposition}

\begin{proof}
    \pf
    \step{1}{\pflet{$x, y \in B$}}
    \step{2}{\pick\ $z \in A$ such that $x \leq z$ and $y \leq z$}
    \step{3}{\pick\ $z' \in B$ such that $z \leq z'$}
    \step{4}{$x \leq z'$ and $y \leq z'$}
    \qed
\end{proof}
