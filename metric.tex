
\section{The Metric Topology}

\begin{definition}[Metric]
    Let $X$ be a set. A \emph{metric} on $X$ is a function $d : X^2 \rightarrow \RR$ such that:
    \begin{enumerate}
        \item For all $x,y \in X$, $d(x,y) \geq 0$
        \item For all $x,y \in X$, $d(x,y) = 0$ if and only if $x = y$
        \item For all $x,y \in X$, $d(x,y) = d(y,x)$
        \item (\emph{Triangle Inequality}) For all $x, y, z \in X$, $d(x,z) \leq d(x,y) + d(y,z)$
    \end{enumerate}
    We call $d(x,y)$ the \emph{distance} between $x$ and $y$.
\end{definition}

\begin{definition}[Open Ball]
    Let $X$ be a metric space. Let $a \in X$ and $\epsilon > 0$. The \emph{open ball} with \emph{centre} $a$ and \emph{radius} $\epsilon$
    is
    \[ B(a, \epsilon) = \{ x \in X \mid d(a,x) < \epsilon \} \enspace . \]
\end{definition}

\begin{definition}[Metric Topology]
    Let $X$ be a metric space. The \emph{metric topology} on $X$ is the topology generated by the basis consisting of all the open balls.
\end{definition}

We prove this is a basis for a topology.

\begin{proof}
    \pf
    \step{1}{For every point $a$, there exists a ball $B$ such that $a \in B$}
    \begin{proof}
        \pf\ We have $a \in B(a,1)$.
    \end{proof}
    \step{2}{For any balls $B_1$, $B_2$ and point $a \in B_1 \cap B_2$, there exists a ball $B_3$ such that $a \in B_3 \subseteq B_1 \cap B_2$}
    \begin{proof}
        \step{a}{\pflet{$B_1 = B(c_1, \epsilon_1)$ and $B_2 = B(c_2, \epsilon_2)$}}
        \step{b}{\pflet{$\delta = \min(\epsilon_1 - d(c_1,a), \epsilon_2 - d(c_2,a))$} \prove{$B(a,\delta) \subseteq B_1 \cap B_2$}}
        \step{c}{\pflet{$x \in B(a, \delta)$}}
        \step{d}{$x \in B_1$}
        \begin{proof}
            \pf
            \begin{align*}
                d(x,c_1) & = d(x,a) + d(a,c_1) \\
                & < \delta + d(a,c_1) \\
                & \leq \epsilon_1
            \end{align*}
        \end{proof}
        \step{e}{$x \in B_2$}
        \begin{proof}
            \pf\ Similar.
        \end{proof}
    \end{proof}
    \qed
\end{proof}

\begin{proposition}
    \label{proposition:open_in_metric_space}
    Let $X$ be a metric space and $U \subseteq X$. Then $U$ is open if and only if, for every $x \in U$, there exists $\epsilon > 0$ such that $B(x, \epsilon)
    \subseteq U$.
\end{proposition}

\begin{proof}
    \pf
    \step{1}{If $U$ is open then, for all $x \in U$, there exists $\epsilon > 0$ such that $B(x, \epsilon) \subseteq U$.}
    \begin{proof}
        \step{a}{\assume{$U$ is open.}}
        \step{b}{\pflet{$x \in U$}}
        \step{c}{\pick\ $a \in X$ and $\delta > 0$ such that $x \in B(a, \delta) \subseteq U$}
        \step{d}{\pflet{$\epsilon = \delta - d(a,x)$} \prove{$B(x, \epsilon) \subseteq U$}}
        \step{e}{\pflet{$y \in B(x, \epsilon)$}}
        \step{f}{$d(y,a) < \delta$}
        \begin{proof}
            \pf
            \begin{align*}
                d(y,a) & \leq d(a,x) + d(x,y) \\
                & < \delta + d(x,y) \\
                & = \epsilon
            \end{align*}
        \end{proof}
        \step{g}{$y \in U$}
    \end{proof}
    \step{2}{If, for all $x \in U$, there exists $\epsilon > 0$ such that $B(x, \epsilon) \subseteq U$, then $U$ is open.}
    \begin{proof}
        \pf\ Immediate from definitions.
    \end{proof}
    \qed
\end{proof}

\begin{definition}[Discrete Metric]
    Let $X$ be a set. The \emph{discrete metric} on $X$ is defined by
    \[ d(x,y) = \begin{cases}
        0 & \text{if } x = y \\
        1 & \text{if } x \neq y
    \end{cases} \]
\end{definition}

\begin{proposition}
    The discrete metric induces the discrete topology.
\end{proposition}

\begin{proof}
    \pf\ For any (open) set $U$ and point $a \in U$, we have $a \in B(a,1) \subseteq U$. \qed
\end{proof}

\begin{definition}[Standard Metric on $\RR$]
    The \emph{standard metric} on $\RR$ is defined by $d(x,y) = |x-y|$.
\end{definition}

\begin{proposition}
    The standard metric on $\RR$ induces the standard topology on $\RR$.
\end{proposition}

\begin{proof}
    \pf
    \step{1}{Every open ball is open in the standard topology on $\RR$.}
    \begin{proof}
        \pf\ $B(a, \epsilon) = (a - \epsilon, a + \epsilon)$
    \end{proof}
    \step{2}{For every open set $U$ and point $a \in U$, there exists $\epsilon > 0$ such that $B(a, \epsilon) \subseteq U$}
    \begin{proof}
        \step{a}{\pflet{$U$ be an open set and $a \in U$}}
        \step{b}{\pick\ an open interval $b$, $c$ such that $a \in (b,c) \subseteq U$}
        \step{c}{\pflet{$\epsilon = \min(a-b,c-a)$}}
        \step{d}{$B(a, \epsilon) \subseteq U$}
    \end{proof}
    \qed
\end{proof}

\begin{definition}[Metrizable]
    A topological space $X$ is \emph{metrizable} if and only if there exists a metric on $X$ that induces the topology.
\end{definition}

\begin{definition}[Bounded]
    Let $X$ be a metric space and $A \subseteq X$. Then $A$ is \emph{bounded} if and only if there exists $M$ such that,
    for all $x, y \in A$, we have $d(x,y) \leq M$.
\end{definition}

\begin{definition}[Diameter]
    Let $X$ be a metric space and $A \subseteq X$. The \emph{diameter} of $A$ is
    \[ \diam A = \sup_{x,y \in A} d(x,y) \enspace . \]
\end{definition}

\begin{definition}[Standard Bounded Metric]
    Let $d$ be a metric on $X$. The \emph{standard bounded metric} corresponding to $d$ is the metric $\overline{d}$ defined by
    \[ \overline{d}(x,y) = \min(d(x,y),1) \enspace . \]
\end{definition}

We prove this is a metric.

\begin{proof}
    \pf
    \step{1}{$\overline{d}(x,y) \geq 0$}
    \begin{proof}
        \pf\ Since $d(x,y) \geq 0$
    \end{proof}
    \step{2}{$\overline{d}(x,y) = 0$ if and only if $x = y$}
    \begin{proof}
        \pf\ $\overline{d}(x,y) = 0$ if and only if $d(x,y) = 0$ if and only if $x = y$
    \end{proof}
    \step{3}{$\overline{d}(x,y) = \overline{d}(y,x)$}
    \begin{proof}
        \pf\ Since $d(x,y) = d(y,x)$
    \end{proof}
    \step{4}{$\overline{d}(x,z) \leq \overline{d}(x,y) + \overline{d}(y,z)$}
    \begin{proof}
        \pf
        \begin{align*}
            \overline{d}(x,y) + \overline{d}(y,z) & = \min(d(x,y),1) + \min(d(y,z),1) \\
            & = \min(d(x,y) + d(y,z), d(x,y) + 1, d(y,z) + 1, 2) \\
            & \geq \min(d(x,z), 1) \\
            & = \overline{d}(x,z)
        \end{align*}
    \end{proof}
    \qed
\end{proof}

\begin{lemma}
    \label{lemma:basis_radius_less_than_one}
    In any metric space $X$, the set $\BB = \{ B(a, \epsilon) \mid a \in X, \epsilon < 1 \}$ is a basis for the metric topology.
\end{lemma}

\begin{proof}
    \pf
    \step{1}{Every element of $\BB$ is open.}
    \begin{proof}
        \pf\ From Lemma \ref{lemma:basis_unions}.
    \end{proof}
    \step{2}{For every open set $U$ and point $a \in U$, there exists $B \in \BB$ such that $a \in B \subseteq U$}
    \begin{proof}
        \step{a}{\pflet{$U$ be an open set and $a \in U$}}
        \step{b}{\pick $\epsilon > 0$ such that $B(a, \epsilon) \subseteq U$}
        \step{c}{$B(a,\min(\epsilon,1/2)) \subseteq U$}
    \end{proof}
    \qedstep
    \begin{proof}
        \pf\ Lemma \ref{lemma:basis}.
    \end{proof}
    \qed
\end{proof}

\begin{proposition}
    \label{proposition:standard_bounded_metric}
    Let $d$ be a metric on the set $X$. Then the standard bounded metric $\overline{d}$ induces the same metric as $d$.
\end{proposition}

\begin{proof}
    \pf\ This follows from Lemma \ref{lemma:basis_radius_less_than_one} since the open balls with radius $< 1$ are the same under both metrics. \qed
\end{proof}

\begin{lemma}
    \label{lemma:metrics_same_topology}
    Let $d$ and $d'$ be two metrics on the same set $X$. Let $\TT$ and $\TT'$ be the topologies they induce. Then $\TT \subseteq \TT'$ if and only if,
    for all $x \in X$ and $\epsilon > 0$, there exists $\delta > 0$ such that
    \[ B_{d'}(x, \delta) \subseteq B_d(x, \epsilon) \enspace . \]
\end{lemma}

\begin{proof}
    \pf
    \step{1}{If $\TT \subseteq \TT'$ then,
    for all $x \in X$ and $\epsilon > 0$, there exists $\delta > 0$ such that
    $B_{d'}(x, \delta) \subseteq B_d(x, \epsilon)$}
    \begin{proof}
        \pf\ From Proposition \ref{proposition:open_in_metric_space} since $x \in B_d(x, \epsilon) \in \TT'$.
    \end{proof}
    \step{2}{If, for all $x \in X$ and $\epsilon > 0$, there exists $\delta > 0$ such that
    $B_{d'}(x, \delta) \subseteq B_d(x, \epsilon)$, then $\TT \subseteq \TT'$}
    \begin{proof}
        \step{a}{\assume{For all $x \in X$ and $\epsilon > 0$, there exists $\delta > 0$ such that
        $B_{d'}(x, \delta) \subseteq B_d(x, \epsilon)$}}
        \step{b}{\pflet{$U \in \TT$}}
        \step{c}{For all $x \in U$ there exists $\delta > 0$ such that $B_{d'}(x, \delta) \subseteq U$.}
        \begin{proof}
            \step{i}{\pflet{$x \in U$}}
            \step{ii}{\pick\ $\epsilon > 0$ such that $B_d(x, \epsilon) \subseteq U$}
            \begin{proof}
                \pf\ Proposition \ref{proposition:open_in_metric_space}
            \end{proof}
            \step{iii}{\pick\ $\delta > 0$ such that $B_{d'}(x, \delta) \subseteq B_d(x, \epsilon)$}
            \begin{proof}
                \pf\ By \stepref{a}
            \end{proof}
            \step{iv}{$B_{d'}(x, \delta) \subseteq U$}
        \end{proof}
        \step{d}{$U \in \TT'$}
        \begin{proof}
            \pf\ Proposition \ref{proposition:open_in_metric_space}.
        \end{proof}
    \end{proof}
    \qed
\end{proof}

\begin{proposition}
    $\RR^2$ under the dictionary order topology is metrizable.
\end{proposition}

\begin{proof}
    \pf\ Define $d : \RR^2 \rightarrow \RR$ by
    \begin{align*}
        d((x,y),(x,z)) & = \max(|y-z|,1) \\
        d((x,y),(x',y')) & = 1 & \text{if } x \neq x' \qed
    \end{align*}        \step{c}{$x \in \bigcap_{i=1}^N \inv{\pi_i}() \subseteq B_D(a, \epsilon)$}
\end{proof}

\begin{proposition}
    Let $d : X^2 \rightarrow \RR$ be a metric on $X$. Then the metric topology on $X$ is the coarsest topology such that $d$ is continuous.
\end{proposition}

\begin{proof}
    \pf
    \step{1}{$d$ is continuous.}
    \begin{proof}
        \step{a}{\pflet{$a, b \in X$}}
        \step{b}{\pflet{$\epsilon > 0$}}
        \step{c}{\pflet{$\delta = \epsilon / 2$}}
        \step{d}{\pflet{$x, y \in X$}}
        \step{e}{\assume{$\rho((a,b),(x,y)) < \delta$}}
        \step{f}{$|d(a,b) - d(x,y)| < \epsilon$}
        \begin{proof}
            \step{i}{$d(a,b) - d(x,y) < \epsilon$}
            \begin{proof}
                \pf
                \begin{align*}
                    d(a,b) & \leq d(a,x) + d(x,y) + d(y,b) \\
                    & \leq d(x,y) + 2 \rho((a,b),(x,y)) \\
                    & < d(x,y) + 2 \delta \\
                    & = d(x,y) + \epsilon
                \end{align*}
            \end{proof}
            \step{ii}{$d(a.b) - d(x,y) > - \epsilon$}
            \begin{proof}
                \pf\ Similar.
            \end{proof}
        \end{proof}
        \qedstep
    \end{proof}
    \step{2}{If $\TT$ is any topology under which $d$ is continuous then $\TT$ is finer than the metric topology.}
    \begin{proof}
        \pf\ Since $B(a, \epsilon) = \inv{d_a}((-\infty, \epsilon))$
    \end{proof}
    \qed    
\end{proof}

\begin{proposition}
    Let $X$ be a metric space with metric $d$ and $A \subseteq X$. The restriction of $d$ to $A$ is a metric on $A$ that induces the subspace topology.
\end{proposition}

\begin{proof}
    \pf
    \step{1}{The restriction of $d$ to $A$ is a metric on $A$.}
    \step{2}{Every open ball under $d \restriction A$ is open under the subspace topology.}
    \begin{proof}
        \pf\ $B_{d \restriction A}(a, \epsilon) = B_d(a, \epsilon) \cap A$.
    \end{proof}
    \step{3}{If $U$ is open in the subspace topology and $x \in U$, then there exists a $d \restriction A$-ball $B$ such that $x \in B \subseteq U$.}
    \begin{proof}
        \step{a}{\pick\ $V$ open in $X$ such that $U = V \cap A$}
        \step{b}{\pick\ $\epsilon > 0$ such that $B_d(x, \epsilon) \subseteq V$}
        \step{c}{Take $B = B_{d \restriction A}(x, \epsilon)$}
    \end{proof}
    \qed
\end{proof}

\begin{corollary}
    A subspace of a metrizable space is metrizable.
\end{corollary}

\begin{proposition}
    Every metrizable space is Hausdorff.
\end{proposition}

\begin{proof}
    \pf
    \step{1}{\pflet{$X$ be a metric space}}
    \step{2}{\pflet{$a, b \in X$ with $a \neq b$}}
    \step{3}{\pflet{$\epsilon = d(a,b) / 2$}}
    \step{4}{\pflet{$U = B(a, \epsilon)$ and $V = B(b, \epsilon)$}}
    \step{5}{$U$ and $V$ are disjoint neighbourhoods of $a$ and $b$ respectively.}
    \qed
\end{proof}

\begin{proposition}[CC]
    The product of a countable family of metrizable spaces is metrizable.
\end{proposition}

\begin{proof}
    \pf
    \step{0}{\pflet{$(X_n, d_n)$ be a sequence of metric spaces.}}
    \step{0a}{\assume{w.l.o.g.~each $d_n$ is bounded above by 1.}}
    \begin{proof}
        \pf\ By Proposition \ref{proposition:standard_bounded_metric}.
    \end{proof}
    \step{1}{\pflet{$D$ be the metric on $\RR^\omega$ defined by $D(x,y) = \sup_i (d_i(x_i, y_i) / i)$.}}
    \begin{proof}
        \step{a}{$D(x,y) \geq 0$}
        \step{b}{$D(x,y) = 0$ if and only if $x = y$}
        \step{c}{$D(x,y) = D(y,x)$}
        \step{d}{$D(x,z) \leq D(x,y) + D(y,z)$}
        \begin{proof}
            \pf
            \begin{align*}
                D(x,z) & = \sup_i \frac{d_i(x_i, z_i)}{i} \\
                & \leq \sup_i \frac{d_i(x_i, y_i) + d_i(y_i, z_i)}{i} \\
                & \leq \sup_i \frac{d_i(x_i, y_i)}{i} + \sup_i \frac{d_i(y_i, z_i)}{i} \\
                & = D(x,y) + D(y,z)
            \end{align*}
        \end{proof}
    \end{proof}
    \step{2}{Every open ball $B_D(a, \epsilon)$ is open in the product topology.}
    \begin{proof}
        \step{b}{\pick\ $N$ such that $1 / \epsilon < N$}
        \step{c}{$B_D(a,\epsilon) = \prod_{i=1}^\infty U_i$ where $U_i = B(a_i, i \epsilon)$ if $i \leq N$, and $U_i = X_i$ if $i > N$}
    \end{proof}
    \step{3}{For any open set $U$ and $a \in U$, there exists $\epsilon > 0$ such that $B_D(a, \epsilon) \subseteq U$.}
    \begin{proof}
        \step{a}{\pflet{$n \geq 1$, $V$ be an open set in $\RR$ and $a \in \inv{\pi_n}(V)$}}
        \step{b}{\pick\ $\epsilon > 0$ such that $B_{d_n}(a, \epsilon) \subseteq V$}
        \step{c}{$B_D(a, \epsilon / n) \subseteq \inv{\pi_n}(V)$}
    \end{proof}
    \qed
\end{proof}

\begin{theorem}
    \label{theorem:continuous_metric}
    Let $X$ and $Y$ be metric spaces and $f : X \rightarrow Y$. Then $f$ is continuous if and only if, for all $x \in X$ and $\epsilon > 0$,
    there exists $\delta > 0$ such that, for all $y \in X$, if $d(x,y) < \delta$ then $d(f(x), f(y)) < \epsilon$.
\end{theorem}

\begin{proof}
    \pf
    \step{1}{If $f$ is continuous then, for all $x \in X$ and $\epsilon > 0$,
    there exists $\delta > 0$ such that, for all $y \in X$, if $d(x,y) < \delta$ then $d(f(x), f(y)) < \epsilon$}
    \begin{proof}
        \step{a}{\assume{$f$ is continuous.}}
        \step{b}{\pflet{$x \in X$ and $\epsilon > 0$}}
        \step{c}{\pick\ a neighbourhood $U$ of $x$ such that $f(U) \subseteq B(f(x), \epsilon)$}
        \begin{proof}
            \pf\ Theorem \ref{theorem:continuous}.
        \end{proof}
        \step{d}{\pick\ $\delta > 0$ such that $B(x, \delta) \subseteq U$}
        \begin{proof}
            \pf\ Proposition \ref{proposition:open_in_metric_space}.
        \end{proof}
        \step{e}{For all $y \in X$, if $d(x,y) < \delta$ then $d(f(x), f(y)) < \epsilon$}
    \end{proof}
    \step{2}{If for all $x \in X$ and $\epsilon > 0$,
    there exists $\delta > 0$ such that, for all $y \in X$, if $d(x,y) < \delta$ then $d(f(x), f(y)) < \epsilon$, then $f$ is continuous.}
    \begin{proof}
        \step{a}{\assume{for all $x \in X$ and $\epsilon > 0$, there exists
        $\delta > 0$ such that, for all $y \in X$, if $d(x,y) < \delta$
        then $d(f(x),f(y)) < \epsilon$}}
        \step{b}{\pflet{$x \in X$ and $V$ be a neighbourhood of $f(x)$}}
        \step{c}{\pick\ $\epsilon > 0$ such that $B(f(x),\epsilon) \subseteq V$}
        \begin{proof}
            \pf\ Proposition \ref{proposition:open_in_metric_space}.
        \end{proof}
        \step{d}{\pick\ $\delta > 0$ such that, for all $y \in X$, if $d(x,y) < \delta$
        then $d(f(x),f(y)) < \epsilon$}
        \begin{proof}
            \pf\ By \stepref{a}
        \end{proof}
        \step{e}{\pflet{$U = B(x,\delta)$}}
        \step{f}{$U$ is a neighbourhood of $x$ with $f(U) \subseteq V$}
        \qedstep
        \begin{proof}
            \pf\ Theorem \ref{theorem:continuous}.
        \end{proof}
    \end{proof}
    \qed
\end{proof}

\begin{proposition}
    \label{proposition:convergence_metric}
    Let $X$ be a metric space. Let $(a_n)$ be a sequence in $X$ and $l \in X$.
    Then $a_n \rightarrow l$ as $n \rightarrow \infty$ if and only if,
    for all $\epsilon > 0$, there exists $N$ such that, for all $n \geq N$,
    we have $d(a_n, l) < \epsilon$.
\end{proposition}

\begin{proof}
    \pf\ From Proposition \ref{proposition:convergence_basis}. \qed
\end{proof}

\begin{proposition}
    Every metrizable space is first countable.
\end{proposition}

\begin{proof}
    \pf\ In any metric space $X$, the open balls $B(a,1/n)$ for $n \geq 1$ form a local basis at $a$.
\end{proof}

\begin{example}
    $\RR^\omega$ under the box topology is not metrizable.
\end{example}

\begin{example}
    If $J$ is uncountable then $\RR^J$ under the product topology is not metrizable.
\end{example}

\section{Real Linear Algebra}

\begin{definition}[Square Metric]
    The \emph{square metric} $\rho$ on $\RR^n$ is defined by
    \[ \rho(\vec{x}, \vec{y}) = \max(|x_1 - y_1|, \ldots, |x_n - y_n|) \]
\end{definition}

We prove this is a metric.

\begin{proof}
    \pf
    \step{1}{$\rho(\vec{x}, \vec{y}) \geq 0$}
    \begin{proof}
        \pf\ Immediate from definition.
    \end{proof}
    \step{2}{$\rho(\vec{x}, \vec{y}) = 0$ if and only if $\vec{x} = \vec{y}$}
    \begin{proof}
        \pf\ Immediate from definition.
    \end{proof}
    \step{3}{$\rho(\vec{x}, \vec{y}) = \rho(\vec{y}, \vec{x})$}
    \begin{proof}
        \pf\ Immediate from definition.
    \end{proof}
    \step{4}{$\rho(\vec{x}, \vec{z}) \leq \rho(\vec{x}, \vec{y}) + \rho(\vec{y}, \vec{z})$}
    \begin{proof}
        \pf\ Since $|x_i - z_i| \leq |x_i - y_i| + |y_i - z_i|$.
    \end{proof}
    \qed
\end{proof}

\begin{proposition}
    The square metric induces the standard topology on $\RR^n$.
\end{proposition}

\begin{proof}
    \pf
    \step{1}{For every $a \in X$ and $\epsilon > 0$, we have $B_\rho(a, \epsilon)$ is open in the standard product topology.}
    \begin{proof}
        \pf
        \[ B_\rho(a, \epsilon) = (a_1 - \epsilon, a_1 + \epsilon) \times \cdots \times (a_n - \epsilon, a_n + \epsilon) \]
    \end{proof}
    \step{2}{For any open sets $U_1$, \ldots, $U_n$ in $\RR$, we have $U_1 \times \cdots \times U_n$ is open in the square metric topology.}
    \begin{proof}
        \step{a}{\pflet{$\vec{a} \in U_1 \times \cdots \times U_n$}}
        \step{b}{For $i = 1, \ldots, n$, \pick\ $\epsilon_i > 0$ such that $(a_i - \epsilon_i, a_i + \epsilon_i) \subseteq U_i$}
        \step{c}{\pflet{$\epsilon = \min(\epsilon_1, \ldots, \epsilon_n)$}}
        \step{d}{$B_\rho(\vec{a}, \epsilon) \subseteq U$}
    \end{proof}
    \qed
\end{proof}

\begin{definition}
    Given $\vec{x}, \vec{y} \in \RR^n$, define the \emph{sum} $\vec{x} + \vec{y}$ by
    \[ (x_1, \ldots, x_n) + (y_1, \ldots, y_n) = (x_1 + y_1, \ldots, x_n + y_n) \enspace . \]
\end{definition}

\begin{definition}
    Given $\lambda \in \RR$ and $\vec{x} \in \RR^n$, define the \emph{scalar product} $\lambda \vec{x} \in \RR^n$ by
    \[ \lambda (x_1, \ldots, x_n) = (\lambda x_1, \ldots, \lambda x_n) \]
\end{definition}

\begin{definition}[Inner Product]
    Given $\vec{x}, \vec{y} \in \RR^n$, define the \emph{inner product} $\vec{x} \cdot \vec{y} \in \RR$ by
    \[ (x_1, \ldots, x_n) \cdot (y_1, \ldots, y_n) = x_1 y_1 + \cdots + x_n y_n \enspace . \]
    We write $\vec{x}^2$ for $\vec{x} \cdot \vec{x}$.
\end{definition}

\begin{definition}[Norm]
    Let $n \geq 1$. The \emph{norm} on $\RR^n$ is the function $\| \ \| : \RR^n \rightarrow \RR$ defined by
    \[ \| (x_1, \ldots, x_n) \| = \sqrt{x_1^2 + \cdots + x_n^2} \]
\end{definition}

\begin{lemma}
    \[ \| \vec{x} \|^2 = \vec{x}^2 \]
\end{lemma}

\begin{proof}
    \pf\ Immediate from definitions. \qed
\end{proof}

\begin{lemma}
    \[ \vec{x} \cdot (\vec{y} + \vec{z}) = \vec{x} \cdot \vec{y} + \vec{x} \cdot \vec{z} \]
\end{lemma}

\begin{proof}
    \pf\ Each is equal to $(x_1 y_1 + x_1 z_1, \ldots, x_n y_n + x_n z_n)$. \qed
\end{proof}

\begin{lemma}
    \label{lemma:Cauchy-Schwarz}
    \[ |\vec{x} \cdot \vec{y}| \leq \| \vec{x} \| \| \vec{y} \| \]
\end{lemma}

\begin{proof}
    \pf
    \step{1}{\assume{$\vec{x} \neq \vec{0} \neq \vec{y}$}}
    \begin{proof}
        \pf\ Otherwise both sides are 0.
    \end{proof}
    \step{2}{\pflet{$a = 1 / \| \vec{x} \|$}}
    \step{3}{\pflet{$b = 1 / \| \vec{y} \|$}}
    \step{5}{$(a \vec{x} + b \vec{y})^2 \geq 0$ and $(a \vec{x} - b \vec{y})^2 \geq 0$}
    \step{6}{$a^2 \| \vec{x} \|^2 + 2 a b \vec{x} \cdot \vec{y} + b^2 \| \vec{y} \|^2 \geq 0$ and $a^2 \| \vec{x} \|^2 - 2 a b \vec{x} \cdot \vec{y} + b^2 \| \vec{y} \|^2 \geq 0$}
    \step{7}{$2ab \vec{x} \cdot \vec{y} + 2 \geq 0$ and $-2ab \vec{x} \cdot \vec{y} + 2 \geq 0$}
    \step{8}{$\vec{x} \cdot \vec{y} \geq - 1/ab$ and $\vec{x} \cdot \vec{y} \leq 1/ab$}
    \step{9}{$\vec{x} \cdot \vec{y} \geq - \| \vec{x} \| \| \vec{y} \|$ and $\vec{x} \cdot \vec{y} \leq \| \vec{x} \| \| \vec{y} \|$}
    \qed
\end{proof}

\begin{lemma}[Triangle Inequality]
    \label{lemma:triangle_inequality}
    \[ \| \vec{x} + \vec{y} \| \leq \| \vec{x} \| + \| \vec{y} \| \]
\end{lemma}

\begin{proof}
    \pf
    \begin{align*}
        \| \vec{x} + \vec{y} \|^2 & = \| \vec{x} \|^2 + 2 \vec{x} \cdot \vec{y} + \| \vec{y} \|^2 \\
        & \leq \| \vec{x} \|^2 + 2 \| \vec{x} \| \| \vec{y} \| + \| \vec{y} \|^2 & (\text{Lemma \ref{lemma:Cauchy-Schwarz}}) \\
        & = (\| \vec{x} \| + \| \vec{y} \|)^2 & \qed
    \end{align*}
\end{proof}

\begin{definition}[Euclidean Metric]
    Let $n \geq 1$. The \emph{Euclidean metric} on $\RR^n$ is defined by
    \[ d(\vec{x}, \vec{y}) = \| \vec{x} - \vec{y} \| \enspace . \]
\end{definition}

We prove this is a metric.

\begin{proof}
    \step{1}{$d(\vec{x}, \vec{y}) \geq 0$}
    \begin{proof}
        \pf\ Immediate from definition.
    \end{proof}
    \step{2}{$d(\vec{x}, \vec{y}) = 0$ if and only if $\vec{x} = \vec{y}$}
    \begin{proof}
        \pf\ $d(\vec{x}, \vec{y}) = 0$ if and only if $\vec{x} - \vec{y} = \vec{0}$.
    \end{proof}
    \step{3}{$d(\vec{x}, \vec{y}) = d(\vec{y}, \vec{x})$}
    \begin{proof}
        \pf\ Immediate from definition.
    \end{proof}
    \step{4}{$d(\vec{x}, \vec{z}) \leq d(\vec{x}, \vec{y}) + d(\vec{y}, \vec{z})$}
    \begin{proof}
        \pf
        \begin{align*}
            \| \vec{x} - \vec{z} \| & = \| (\vec{x} - \vec{y}) + (\vec{y} - \vec{z}) \| \\
            & \leq \| \vec{x} - \vec{y} \| + \| \vec{y} - \vec{z} \| & (\text{Lemma \ref{lemma:triangle_inequality}})
        \end{align*}
    \end{proof}
    \qed
\end{proof}

\begin{proposition}
    The Euclidean metric induces the standard topology on $\RR^n$.
\end{proposition}

\begin{proof}
    \pf
    \step{1}{\pflet{$\rho$ be the square metric.}}
    \step{2}{For all $\vec{a} \in \RR^n$ and $\epsilon > 0$, we have $B_d(\vec{a}, \epsilon) \subseteq B_\rho(\vec{a}, \epsilon)$}
    \begin{proof}
        \step{a}{\pflet{$\vec{x} \in B_d(\vec{a}, \epsilon)$}}
        \step{b}{$\sqrt{(x_1 - a_1)^2 + \cdots + (x_n - a_n)^2} < \epsilon$}
        \step{c}{$(x_1 - a_1)^2 + \cdots + (x_n - a_n)^2 < \epsilon^2$}
        \step{d}{For all $i$ we have $(x_i - a_i)^2 < \epsilon^2$}
        \step{e}{For all $i$ we have $|x_i - a_i| < \epsilon$}
        \step{f}{$\rho(\vec{x}, \vec{a}) < \epsilon$}
    \end{proof}
    \step{3}{For all $\vec{a} \in \RR^n$ and $\epsilon > 0$, we have $B_\rho(\vec{a}, \epsilon / \sqrt{n}) \subseteq B_d(\vec{a}, \epsilon)$}
    \begin{proof}
        \step{a}{\pflet{$\vec{x} \in B_\rho(\vec{a}, \epsilon / \sqrt{n})$}}
        \step{b}{$\rho(\vec{x}, \vec{a}) < \epsilon / \sqrt{n}$}
        \step{c}{For all $i$ we have $|x_i - x_a| < \epsilon / \sqrt{n}$}
        \step{d}{For all $i$ we have $(x_i - x_a)^2 < \epsilon^2 / n$}
        \step{e}{$d(\vec{x}, \vec{a}) < \epsilon$}
    \end{proof}
    \qedstep
    \begin{proof}
        \pf\ By Lemma \ref{lemma:metrics_same_topology}.
    \end{proof}
    \qed
\end{proof}

\begin{proposition}
    Let $n \geq 0$. For all $c \in \RR^n$ and $\epsilon > 0$, the open ball $B(c,\epsilon)$ is path connected.
\end{proposition}

\begin{proof}
    \pf
    \step{1}{\pflet{$a, b \in B(c,\epsilon)$}}
    \step{2}{\pflet{$p : [0,1] \rightarrow B(c,\epsilon)$ be the function $p(t) = (1-t)a + tb$}}
    \begin{proof}
        \pf\ We have $p(t) \in B(c,\epsilon)$ for all $t$ because
        \begin{align*}
            d(p(t),c) & = \| (1-t)a + tb - c \| \\
        & = \| (1-t)(a-c) + t(b-c) \| \\
        & \leq (1-t) \| a-c \| + t \| b-c \| \\
        & < (1-t) \epsilon + t \epsilon \\
        & = \epsilon
        \end{align*}
    \end{proof}
    \step{3}{$p$ is a path from $a$ to $b$.}
    \qed
\end{proof}

\begin{proposition}
    Let $n \geq 0$. For all $c \in \RR^n$ and $\epsilon > 0$, the closed ball $\overline{B(c,\epsilon)}$ is path connected.
\end{proposition}

\begin{proof}
    \pf
    \step{1}{\pflet{$a, b \in \overline{B(c,\epsilon)}$}}
    \step{2}{\pflet{$p : [0,1] \rightarrow \overline{B(c,\epsilon)}$ be the function $p(t) = (1-t)a + tb$}}
    \begin{proof}
        \pf\ We have $p(t) \in \overline{B(c,\epsilon)}$ for all $t$ because
        \begin{align*}
            d(p(t),c) & = \| (1-t)a + tb - c \| \\
        & = \| (1-t)(a-c) + t(b-c) \| \\
        & \leq (1-t) \| a-c \| + t \| b-c \| \\
        & \leq (1-t) \epsilon + t \epsilon \\
        & = \epsilon
        \end{align*}
    \end{proof}
    \step{3}{$p$ is a path from $a$ to $b$.}
    \qed
\end{proof}

\begin{lemma}
    If $\sum_{i=0}^\infty x_i^2$ and $\sum_{i=0}^\infty y_i^2$ converge then $\sum_{i=0}^\infty |x_i y_i|$ converges.
\end{lemma}

\begin{proof}
    \pf
    \step{1}{For all $N \geq 0$ we have
        $\sum_{i=0}^N |x_i y_i| \leq \sqrt{\sum_{i=0}^N |x_i|^2} \sqrt{\sum_{i=0}^N |y_i|^2}$}
    \begin{proof}
        \pf\ By the Cauchy-Schwarz inequality
    \end{proof}
    \qedstep
    \begin{proof}
        \pf\ Since $\sum_{i=0}^N |x_i y_i|$ is an increasing sequence bounded above by \\ $(\sum_{i=0}^\infty x_i^2) (\sum_{i=0}^\infty y_i^2)$.
    \end{proof}
    \qed
\end{proof}

\begin{corollary}
    \label{corollary:l2_sum_converge}
    If $\sum_{i=0}^\infty x_i^2$ and $\sum_{i=0}^\infty y_i^2$ converge then $\sum_{i=0}^\infty (x_i + y_i)^2$ converges.
\end{corollary}

\begin{proof}
    \pf\ Since $\sum_{i=0}^\infty x_i^ 2$, $\sum_{i=0}^\infty y_i^2$ and $2 \sum_{i=0}^\infty x_i y_i$ all converge.
\end{proof}

\begin{definition}[$l^2$-metric]
    The \emph{$l^2$-metric} on 
    \[ \left\{ (x_n) \in \RR^\omega \mid \sum_{n=0}^\infty x_n^2 \text{ converges} \right\} \] is defined by
    \[ d(x,y) = \left( \sum_{n=0}^\infty (x_n - y_n)^2 \right)^{1/2} \]
\end{definition}

We prove this is a metric.

\begin{proof}
    \pf
    \step{1}{$d$ is well-defined.}
    \begin{proof}
        \pf\ By Corollary \ref{corollary:l2_sum_converge}.
    \end{proof}
    \step{2}{$d(x,y) \geq 0$}
    \step{3}{$d(x,y) = 0$ if and only if $x = y$}
    \step{4}{$d(x,y) = d(y,x)$}
    \step{5}{$d(x,z) \leq d(x,y) + d(y,z)$}
    \begin{proof}
        \pf\ By Lemma \ref{lemma:triangle_inequality}.
    \end{proof}
    \qed
\end{proof}

\begin{theorem}
    Addition is a continuous function $\RR^2 \rightarrow \RR$.
\end{theorem}

\begin{proof}
    \pf
    \step{1}{\pflet{$a, b \in \RR$}}
    \step{2}{\pflet{$\epsilon > 0$}}
    \step{3}{\pflet{$\delta = \epsilon / 2$}}
    \step{4}{\pflet{$x, y \in \RR$}}
    \step{5}{\assume{$\rho((a,b),(x,y)) < \delta$}}
    \step{6}{$|(a+b)-(x+y)| < \epsilon$}
    \begin{proof}
        \pf
        \begin{align*}
            |(a+b)-(x+y)| & = |a-x| + |b-y| \\
            & \leq 2 \rho((a,b),(x,y)) \\
            & < 2 \delta \\
            & = \epsilon
        \end{align*}
    \end{proof}
    \qedstep
    \begin{proof}
        \pf\ Theorem \ref{theorem:continuous_metric}
    \end{proof}
    \qed
\end{proof}

\begin{theorem}
    Multiplication is a continuous function $\RR^2 \rightarrow \RR$.
\end{theorem}

\begin{proof}
    \pf
    \step{1}{\pflet{$a, b \in \RR$}}
    \step{2}{\pflet{$\epsilon > 0$}}
    \step{3}{\pflet{$\delta = \min(\epsilon / ( |a| + |b| + 1), 1)$}}
    \step{4}{\pflet{$x, y \in \RR$}}
    \step{5}{\assume{$\rho((a,b),(x,y)) < \delta$}}
    \step{6}{$|ab - xy| < \epsilon$}
    \begin{proof}
        \pf
        \begin{align*}
            |ab - xy| & = |a(b-y) + (a-x)b - (a-x)(b-y)| \\
            & \leq |a| |b-y| + |b| |a-x| + |a-x| |b-y| \\
            & < |a| \delta + |b| \delta + \delta^2 & (\text{\stepref{5}})\\
            & \leq |a| \delta + |b| \delta + \delta & (\text{\stepref{3}})\\
            & \leq \epsilon & (\text{\stepref{3}})
        \end{align*}
    \end{proof}
    \qedstep
    \begin{proof}
        \pf\ Theorem \ref{theorem:continuous_metric}
    \end{proof}
    \qed
\end{proof}

\begin{theorem}
    The function $f : \RR \setminus \{ 0 \} \rightarrow \RR$ defined by $f(x) = \inv{x}$ is continuous.
\end{theorem}

\begin{proof}
    \pf
    \step{1}{For all $a \in \RR$ we have $\inv{f}((a,+\infty))$ is open.}
    \begin{proof}
        \pf\ The set is
        \begin{align*}
            (\inv{a}, + \infty) & \text{if } a > 0 \\
            (0, + \infty) & \text{if } a = 0 \\
            (- \infty, \inv{a}) \cup (0, + \infty) & \text{if } a < 0
        \end{align*}
    \end{proof}
    \step{2}{For all $a \in \RR$ we have $\inv{f}((-\infty,a))$ is open.}
    \begin{proof}
        \pf\ Similar.
    \end{proof}
    \qedstep
    \begin{proof}
        \pf\ By Proposition \ref{proposition:continuous_subbasis} and Lemma
        \ref{lemma:open_rays_subbasis}.
    \end{proof}
    \qed
\end{proof}

\begin{definition}
    For $n \geq 0$, the \emph{unit ball} $B^n$ is the space $\{ x \in \RR^n \mid \| x \| \leq 1 \}$.
\end{definition}

\begin{proposition}
    For all $n \geq 0$, the unit ball $B^n$ is path connected.
\end{proposition}

\begin{proof}
    \pf
    \step{1}{\pflet{$a, b \in B^n$}}
    \step{2}{\pflet{$p : [0,1] \rightarrow B^n$ be the function $p(t) = (1-t)a + tb$}}
    \begin{proof}
        \pf\ We have $p(t) \in B^n$ for all $t$ because
        \begin{align*}
            \| (1-t)a + tb \| & \leq (1-t) \| a \| + t \| b \| \\
            & \leq (1-t) + t \\
            & = 1
        \end{align*}
    \end{proof}
    \step{3}{$p$ is a path from $a$ to $b$.}
    \qed
\end{proof}

\section{The Uniform Topology}

\begin{definition}[Uniform Metric]
    Let $J$ be a set. The \emph{uniform metric} $\overline{\rho}$ on $\RR^J$ is defined by
    \[ \overline{\rho}(a,b) = \sup_{j \in J} \overline{d}(a_j, b_j) \]
    where $\overline{d}$ is the standard bounded metric on $\RR$.

    The \emph{uniform topology} on $\RR^J$ is the topology induced by the uniform metric.
\end{definition}

We prove this is a metric.

\begin{proof}
    \pf
    \step{1}{$\overline{\rho}(a,b) \geq 0$}
    \begin{proof}
        \pf\ Immediate from definitions.
    \end{proof}
    \step{2}{$\overline{\rho}(a,b) = 0$ if and only if $a = b$}
    \begin{proof}
        \pf\ Immediate from definitions.
    \end{proof}
    \step{3}{$\overline{\rho}(a,b) = \overline{\rho}(b,a)$}
    \begin{proof}
        \pf\ Immediate from definitions.
    \end{proof}
    \step{4}{$\overline{\rho}(a,c) \leq \overline{\rho}(a,b) + \overline{\rho}(b,c)$}
    \begin{proof}
        \pf
        \begin{align*}
            \overline{\rho}(a,c) & = \sup_{j \in J} \overline{d}(a_j, c_j) \\
            & \leq \sup_{j \in J} (\overline{d}(a_j, b_j) + \overline{d}(b_j, c_j)) \\ 
            & \leq \sup_{j \in J} \overline{d}(a_j, b_j) + \sup_{j \in J} \overline{d}(b_j, c_j) \\
            & = \overline{\rho}(a,b) + \overline{\rho}(b,c)
        \end{align*}
    \end{proof}
    \qed
\end{proof}

\begin{proposition}
    The uniform topology on $\RR^J$ is finer than the product topology.
\end{proposition}

\begin{proof}
    \pf
    \step{1}{\pflet{$j \in J$ and $U$ be open in $\RR$} \prove{$\inv{\pi_j}(U)$ is open in the uniform topology.}}
    \step{2}{\pflet{$a \in \inv{\pi_j}(U)$}}
    \step{3}{\pick\ $\epsilon > 0$ such that $(a_j - \epsilon, a_j + \epsilon) \subseteq U$}
    \step{4}{$B_{\overline{\rho}}(a, \epsilon) \subseteq \inv{\pi_j}(U)$}
    \qed
\end{proof}

\begin{proposition}
    The uniform topology on $\RR^J$ is coarser than the box topology.
\end{proposition}

\begin{proof}
    \pf
    \step{1}{\pflet{$a \in \RR^J$ and $\epsilon > 0$} \prove{$B(a, \epsilon)$ is open in the box topology.}}
    \step{2}{\pflet{$b \in B(a, \epsilon)$}}
    \step{3}{For $j \in J$ we have $|a_j - b_j| < \epsilon$}
    \step{4}{For $j \in J$, \pflet{$\delta_j = (\epsilon - |a_j - b_j|) / 2$}}
    \step{5}{$\prod_{j \in J} (b_j - \delta_j, b_j + \delta_j) \subseteq B(a, \epsilon)$}
    \qed
\end{proof}

\begin{proposition}
    The uniform topology on $\RR^J$ is strictly finer than the product topology if and only if $J$ is infinite.
\end{proposition}

\begin{proof}
    \pf
    \step{1}{If $J$ is finite then the uniform and product topologies coincide.}
    \begin{proof}
        \pf\ The uniform, box and product topologies are all the same.
    \end{proof}
    \step{2}{If $J$ is infinite then the uniform and product topologies are different.}
    \begin{proof}
        \pf\ The set $B(\vec{0}, 1)$ is open in the uniform topology but not the product topology.
    \end{proof}
    \qed
\end{proof}

\begin{proposition}[DC]
    The uniform topology on $\RR^J$ is strictly coarser than the box topology if and only if $J$ is infinite.
\end{proposition}

\begin{proof}
    \pf
    \step{1}{If $J$ is finite then the uniform and box topologies coincide.}
    \begin{proof}
        \pf\ The uniform, box and product topologies are all the same.
    \end{proof}
    \step{2}{If $J$ is infinite then the uniform and box topologies are different.}
    \begin{proof}
        \pf\ Pick an $\omega$-sequence $(j_1, j_2, \ldots)$ in $J$. Let $U = \prod_{j \in J} U_j$ where $U_{j_i} = (-1/i, 1/i)$ and $U_j = (-1,1)$ for all other $j$.
        Then $\vec{0} \in U$ but there is no $\epsilon > 0$ such that $B(\vec{0}, \epsilon) \subseteq U$.
    \end{proof}
    \qed
\end{proof}

\begin{proposition}
    The closure of $\RR^\infty$ in $\RR^\omega$ under the uniform topology is $\RR^\omega$.
\end{proposition}

\begin{proof}
    \pf\ Given any open ball $B(a, \epsilon)$, pick an integer $N$ such that $1 / \epsilon < N$. Then $B(a, \epsilon)$ includes sequences whose $n$th entry is 0
    for all $n \geq N$. \qed
\end{proof}

\section{Uniform Convergence}

\begin{definition}[Uniform Convergence]
    Let $X$ be a set and $Y$ a metric space. Let $(f_n : X \rightarrow Y)$ be a sequence of functions and $f : X \rightarrow Y$ be a function.
    Then $f_n$ \emph{converges uniformly} to $f$ as $n \rightarrow \infty$ if and only if, for all $\epsilon > 0$, there exists $N$ such that, for all $n \geq N$ and $x \in X$,
    we have $d(f_n(x),f(x)) < \epsilon$.
\end{definition}

\begin{example}
    Define $f_n : [0,1] \rightarrow \RR$ by $f_n(x) = x^n$ for $n \geq 1$, and $f : [0,1] \rightarrow \RR$ by $f(x) = 0$ if $x < 1$, $f(1) = 1$.  Then $f_n$ converges to $f$
    pointwise but not uniformly.
\end{example}

\begin{theorem}[Uniform Limit Theorem]
    Let $X$ be a topological space and $Y$ a metric space.  Let $(f_n : X \rightarrow Y)$ be a sequence of continuous functions and $f : X \rightarrow Y$ be a function.
    If $f_n$ converges uniformly to $f$ as $n \rightarrow \infty$, then $f$ is continuous.
\end{theorem}

\begin{proof}
    \pf
    \step{1}{\pflet{$x \in X$ and $\epsilon > 0$}}
    \step{2}{\pick\ $N$ such that, for all $n \geq N$ and $y \in X$, we have $d(f_n(y), f(y)) < \epsilon / 3$}
    \step{3}{\pick\ a neighbourhood $U$ of $x$ such that $f_N(U) \subseteq B(f_N(x), \epsilon / 3)$ \prove{$f(U) \subseteq B(f(x), \epsilon)$}}
    \step{4}{\pflet{$y \in U$}}
    \step{5}{$d(f(y),f(x)) < \epsilon$}
    \begin{proof}
        \pf
        \begin{align*}
            d(f(y),f(x)) & \leq d(f(y),f_N(y)) + d(f_N(y),f_N(x)) + d(f_N(x),f(x)) & (\text{Triangle Inequality}) \\
            & < \epsilon / 3 + \epsilon / 3 + \epsilon / 3 & (\text{\stepref{2}, \stepref{3}})\\
            & = \epsilon
        \end{align*}
    \end{proof}
    \qed
\end{proof}

\begin{proposition}
    Let $X$ be a topological space and $Y$ a metric space.  Let $(f_n : X \rightarrow Y)$ be a sequence of continuous functions and $f : X \rightarrow Y$ be a function.
    Let $(a_n)$ be a sequence of points in $X$ and $a \in X$. If $f_n$ converges uniformly to $f$ and $a_n$ converges to $a$ in $X$ then $f_n(a_n)$ converges to $f(a)$
    uniformly in $Y$.
\end{proposition}

\begin{proof}
    \pf
    \step{1}{\pflet{$\epsilon > 0$}}
    \step{2}{\pick\ $N_1$ such that, for all $n \geq N_1$ and $x \in X$, we have $d(f_n(x),f(x)) < \epsilon / 2$}
    \step{3}{\pick\ $N_2$ such that, for all $n \geq N_2$, we have $a_n \in \inv{f}(B(a,\epsilon/2))$}
    \begin{proof}
        \pf\ Using the fact that $f$ is continuous from the Uniform Limit Theorem.
    \end{proof}
    \step{4}{\pflet{$N = \max(N_1,N_2)$}}
    \step{5}{\pflet{$n \geq N$}}
    \step{6}{$d(f_n(a_n),f(a)) < \epsilon$}
    \begin{proof}
        \pf
        \begin{align*}
            d(f_n(a_n),f(a)) & \leq d(f_n(a_n),f(a_n)) + d(f(a_n),f(a)) & (\text{Triangle Inequality}) \\
            & < \epsilon / 2 + \epsilon / 2 & (\text{\stepref{2}, \stepref{3}}) \\
            & = \epsilon
        \end{align*}
    \end{proof}
    \qed
\end{proof}

\begin{proposition}
    Let $X$ be a set. Let $(f_n : X \rightarrow \RR)$ be a sequence of functions and $f : X \rightarrow \RR$ be a function.
    Then $f_n$ converges unifomly to $f$ as $n \rightarrow \infty$ if and only if $f_n \rightarrow f$ as $n \rightarrow \infty$
    in $\RR^X$ under the uniform topology.
\end{proposition}

\begin{proof}
    \pf
    \step{1}{If $f_n$ converges uniformly to $f$ then $f_n$ converges to $f$ under the uniform topology.}
    \begin{proof}
        \step{a}{\assume{$f_n$ converges uniformly to $f$}}
        \step{b}{\pflet{$\epsilon > 0$}}
        \step{c}{\pick\ $N$ such that, for all $n \geq N$ and $x \in X$, we have $d(f_n(x),f(x)) < \epsilon / 2$}
        \step{d}{For all $n \geq N$ we have $\overline{\rho}(f_n,f) \leq \epsilon / 2$}
        \step{e}{For all $n \geq N$ we have $\overline{\rho}(f_n,f) < \epsilon$}
    \end{proof}
    \step{2}{If $f_n$ converges to $f$ under the uniform topology then $f_n$ converges uniformly to $f$.}
    \begin{proof}
        \step{a}{\assume{$f_n$ converges to $f$ under the uniform topology.}}
        \step{b}{\pflet{$\epsilon > 0$}}
        \step{c}{\pick\ $N$ such that, for all $n \geq N$, we have $\overline{\rho}(f_n,f) < \min(\epsilon,1/2)$}
        \step{d}{\pflet{$n \geq N$}}
        \step{e}{\pflet{$x \in X$}}
        \step{f}{$\overline{\rho}(f_n,f) < \min(\epsilon,1/2)$}
        \begin{proof}
            \pf\ From \stepref{c}.
        \end{proof}
        \step{f}{$d(f_n(x),f(x)) < \min(\epsilon,1/2)$}
        \step{g}{$d(f_n(x),f(x)) < \epsilon$}
    \end{proof}
    \qed
\end{proof}


\section{Isometric Imbeddings}

\begin{definition}
    Let $X$ and $Y$ be metric spaces. An \emph{isometric imbedding} $f : X \rightarrow Y$ is a function such that, for all $x, y \in X$, we have $d(f(x),f(y)) = d(x,y)$.
\end{definition}

\begin{proposition}
    Every isometric imbedding is an imbedding.
\end{proposition}

\begin{proof}
    \pf
    \step{1}{\pflet{$f : X \rightarrow Y$ be an isometric imbedding.}}
    \step{2}{$f$ is injective.}
    \begin{proof}
        \pf\ If $f(x) = f(y)$ then $d(f(x),f(y)) = 0$ hence $d(x,y) = 0$ hence $x = y$.
    \end{proof}
    \step{3}{$f$ is continuous.}
    \begin{proof}
        \pf\ For all $\epsilon > 0$, if $d(x,y) < \epsilon$ then $d(f(x),f(y)) < \epsilon$.
    \end{proof}
    \step{4}{$f : X \rightarrow f(X)$ is an open map.}
    \begin{proof}
        \pf\ $f(B(a,\epsilon)) = B(f(a),\epsilon) \cap f(X)$.
    \end{proof}
    \qed
\end{proof}