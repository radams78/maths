\chapter{Set Theory}

\section{Membership}

We take as undefined the binary relation of \emph{membership}, $\in$. If $a \in A$ we say $a$ is a \emph{member} or \emph{element} of $A$.
If this does not hold, we write $a \notin A$.

\begin{axiom}[Axiom of Extensionality]
    Two sets with exactly the same elements are equal.
\end{axiom}

\section{Subsets}

\begin{definition}[Subset]
    Let $A$ and $B$ be sets. We say $A$ is a \emph{subset} of $B$, $A \subseteq B$, if and only if every member of $A$ is a member of $B$.    
\end{definition}

\section{Abstraction Notation}

\begin{definitions}[Extensionality]
    Let $P(x)$ be a property. If there is a set whose members are exactly the sets $x$ such that $P(x)$, then we denote this set by $\{ x \mid P(x) \}$.

    It is unique by the Axiom of Extensionality.
\end{definitions}

\section{The Empty Set}

\begin{axiom}[Empty Set Axiom]
    There exists a set with no members.
\end{axiom}

\begin{definition}[Empty Set (Extensionality, Empty Set Axiom)]
    The \emph{empty set} $\emptyset$ is the set with no members $\{ x \mid \bot \}$.
\end{definition}

\section{Pair Sets}

\begin{axiom}[Pairing Axiom]
    For any sets $u$ and $v$, there exists a set having as members just $u$ and $v$.
\end{axiom}

\begin{definition}[Pair Set (Extensionality, Pairing Axiom)]
    For any sets $u$ and $v$, the \emph{pair set} $\{ u, v \}$ is the set $\{ x \mid x = u \vee x = v \}$.   
\end{definition}

\section{Unions}

\begin{axiom}[Union Axiom]
    For any set $A$, there exists a set whose elements are exactly the members of the members of $A$.
\end{axiom}

\begin{definition}[Union (Extensionality, Union)]
    For any set $A$, the \emph{union} $\bigcup A$ is the set $\{ x \mid \exists b \in A. x \in b \}$.
\end{definition}

\begin{definition}[Union (Extensionality, Pair Set, Union)]
    For any sets $a$ and $b$, the \emph{union} $a \cup b$ is the set $\bigcup \{ a, b \}$.
\end{definition}

\section{Power Set}

\begin{axiom}[Power Set Axiom]
    For any set $a$, there is a set whose members are exactly the subsets of $a$.    
\end{axiom}

\begin{definition}[Power Set (Extensionality, Power Set)]
    For any set $a$, the \emph{power set} $\pow a$ is the set $\{ x \mid x \subseteq a \}$.
\end{definition}

\section{Singletons}

\begin{definition}[Singleton (Extensionality, Pair Set)]
    Given any $x$, define the \emph{singleton} $\{ x \}$ to be $\{ x, x \}$.    
\end{definition}

\section{Finite Sets}

\begin{definitions}[Extensionality, Pair Set, Union]
    Given any objects $a_1$, \ldots, $a_n$, define the set $\{ a_1, \ldots, a_n \}$ as follows:
    \[ \{ a_1, \ldots, a_n, a_{n+1} \} = \{ a_1, \ldots, a_n \} \cup \{ a_{n+1} \} \enspace . \]
\end{definitions}

\section{Subset Axioms}

\begin{axioms}[Subset Axioms, Aussonderung Axioms]
    For any property $P(x)$ and any set $B$, there exists a set whose members are exactly the sets $x \in B$ such that $P(x)$.
\end{axioms}

\begin{definitions}[Extensionality, Subset]
    For any property $P(x)$ and any set $B$, we write $\{ x \in B \mid P(x) \}$ for $\{x \mid x \in B \wedge P(x) \}$.
\end{definitions}

\begin{theorem}[Subset]
    There is no set to which every set belongs.
\end{theorem}

\begin{proof}
    \pf
    \step{1}{\pflet{$A$ be a set.} \prove{There exists a set that does not belong to $A$.}}
    \step{2}{\pick\ a set $B$ whose members are exactly the sets $x \in A$ such that $x \notin x$.}
    \begin{proof}
        \pf\ By a Subset Axiom.
    \end{proof}
    \step{3}{If $B \in A$ then we have $B \in B \Leftrightarrow B \notin B$}
    \step{4}{$B \notin A$}
    \qed
\end{proof}

\section{Intersection}

\begin{definition}[Intersection (Extensionality, Subset)]
    For any sets $a$ and $b$, the \emph{intersection} $a \cap b$ is $\{ x \in a \mid x \in b \}$.
\end{definition}

\begin{theorem}[Extensionality, Subset]
    For any nonempty set $A$, there exists a unique set $B$ such that, for any $x$, we have $x \in B$ if and only if $x$ belongs to every member of $A$.
\end{theorem}

\begin{proof}
    \pf
    \step{1}{\pflet{$A$ be a nonempty set.}}
    \step{2}{\pick\ $a \in A$}
    \step{3}{\pflet{$B = \{ x \in a \mid \forall y \in A. x \in y \}$}}
    \step{4}{$B$ is the unique set such that, for any $x$, we have $x \in B$ if and only if $x$ belongs to every member of $A$.}
    \qed
\end{proof}

\begin{definition}[Intersection (Extensionality, Subset)]
    For any nonempty set $A$, the \emph{intersection} $\bigcap A$ is the set whose elements are those sets that belong to every member of $A$.
\end{definition}

\section{Relative Complement}

\begin{definition}[Relative Complement (Extensionality, Subset)]
    For any sets $A$ and $B$, the \emph{relative complement} $A - B$ is $\{ x \in A \mid x \notin B \}$.
\end{definition}

\section{Covers}

\begin{definition}[Cover]
    Let $X$ be a set and $\AA \subseteq \pow X$. Then $\AA$ \emph{covers} $X$,
    or is a \emph{covering} of $X$, if and only if $\bigcup \AA = X$.
\end{definition}

\chapter{Relations}

\section{Ordered Pairs}

\begin{definition}[Ordered Pair (Extensionality, Pairing)]
    For any sets $x$ and $y$, the \emph{ordered pair} $(x,y)$ is defined to be $\{ \{ x \}, \{ x , y \} \}$.    
\end{definition}

\begin{theorem}[Extensionality, Pairing]
    For any sets $u$, $v$, $x$, $y$, we have $(u,v) = (x,y)$ if and only if $u = x$ and $v = y$
\end{theorem}

\begin{proof}
    \pf
    \step{1}{\assume{$\{ \{ u \}, \{ u, v \} \} = \{ \{ x \}, \{ x, y \} \}$}}
    \step{2}{$\{ u \} \in \{ \{ x \}, \{ x, y \} \}$}
    \step{3}{$\{ u, v \} \in \{ \{ x \}, \{ x, y \} \}$}
    \step{4}{$\{ u \} = \{ x \}$ or $\{ u \} = \{ x, y \}$}
    \step{5}{$\{ u, v \} = \{ x \}$ or $\{ u, v \} = \{ x, y \}$}
    \step{6}{\case{$\{ u \} = \{ x, y \}$}}
    \begin{proof}
        \step{a}{$u = x = y$}
        \step{b}{$u = v = x = y$}
        \begin{proof}
            \pf\ From \stepref{5}
        \end{proof}
    \end{proof}
    \step{7}{\case{$\{ u, v \} = \{ x \}$}}
    \begin{proof}
        \pf\ Similar.
    \end{proof}
    \step{8}{\case{$\{ u \} = \{ x \}$ and $\{ u, v \} = \{ x, y \}$}}
    \begin{proof}
        \step{a}{$u = x$}
        \step{b}{$u = y$ or $v = y$}
        \step{c}{\case{$u = y$}}
        \begin{proof}
            \pf\ This case is the case considered in \stepref{6}.
        \end{proof}
        \step{d}{\case{$v = y$}}
        \begin{proof}
            \pf\ We have $u = x$ and $v = y$ as required.
        \end{proof}
    \end{proof}
    \qed
\end{proof}

\section{The Finite Intersection Property}

\begin{definition}[Finite Intersection Property]
    Let $X$ be a set and $\AA \subseteq \pow X$. Then $\AA$ satisfies the \emph{finite intersection property} if and only if every nonempty finite subset of $\AA$
    has nonempty intersection.
\end{definition}

\begin{lemma}
    \label{lemma:maximal_finite_intersection_property}
    Let $X$ be a set. Let $\AA \subseteq \pow X$ have the finite intersection property.
    Then there exists a maximal set $\DD$ such that $\AA \subseteq \DD \subseteq \pow X$
    and $\DD$ has the finite intersection property.
\end{lemma}

\begin{proof}
    \pf
    \step{1}{\pflet{$\mathbb{F} = \{ \DD \mid \AA \subseteq \DD \subseteq \pow X, \DD \text{ has the finite intersection property} \}$}}
    \step{2}{Every chain in $\mathbb{F}$ has an upper bound.}
    \begin{proof}
        \step{a}{\pflet{$\mathbb{C}$ be a chain in $\mathbb{F}$.}}
        \step{b}{\assume{without loss of generality $\mathbb{C} \neq \emptyset$}
        \prove{$\bigcup \mathbb{C} \in \mathbb{F}$}}
        \begin{proof}
            \pf\ If $\mathbb{C} = \emptyset$ then $\AA$ is an upper bound.
        \end{proof}
        \step{b}{$\AA \subseteq \bigcup \mathbb{C} \subseteq \pow X$}
        \step{c}{\pflet{$C_1, \ldots, C_n \in \mathbb{C}$} \prove{$C_1 \cap \cdots \cap C_n \neq \emptyset$}}
        \step{d}{\pick\ $\mathcal{C}_1, \ldots, \mathcal{C}_n \in \mathbb{C}$ such that $C_i \in \mathcal{C}_i$ for all $i$.}
        \step{e}{\assume{without loss of generality $\mathcal{C}_1 \subseteq \cdots \subseteq \mathcal{C}_n$}}
        \step{f}{$C_1, \ldots, C_n \in \mathcal{C}_n$}
        \step{g}{$\mathcal{C}_n$ satisfies the finite intersection property.}
        \step{h}{$C_1 \cap \cdots \cap C_n \neq \emptyset$}
    \end{proof}
    \qedstep
    \begin{proof}
        \pf\ By Zorn's Lemma.
    \end{proof}
    \qed
\end{proof}

\begin{lemma}
    \label{lemma:finite_intersection_maximal}
    Let $X$ be a set. Let $\DD \subseteq \pow X$ be maximal with respect to the finite intersection property.
    Then any finite intersection of elements of $\DD$ is an element of $\DD$.
\end{lemma}

\begin{proof}
    \pf
    \step{1}{\pflet{$D_1, D_2 \in \DD$}}
    \step{2}{$\DD \cup \{ D_1 \cap D_2 \}$ has the finite intersection property.}
    \begin{proof}
        \pf\ Any finite intersection of members of $\DD \cup \{ D_1 \cap D_2 \}$
        is a finite intersection of members of $\DD$.
    \end{proof}
    \step{3}{$\DD = \DD \cup \{ D_1 \cap D_2 \}$}
    \begin{proof}
        \pf\ By maximality of $\DD$.
    \end{proof}
    \step{4}{$D_1 \cap D_2 \in \DD$.}
    \qed
\end{proof}

\begin{lemma}
    \label{lemma:member_maximal_finite_intersection}
    Let $X$ be a set. Let $\DD \subseteq \pow X$ be maximal with respect to the finite intersection property.
    Let $A \subseteq X$. If $A$ intersects every member of $\DD$ then $A \in \DD$.
\end{lemma}

\begin{proof}
    \pf
    \step{1}{$\DD \cup \{ A \}$ has the finite intersection property.}
    \begin{proof}
        \step{a}{\pflet{$D_1, \ldots, D_n \in \DD$} \prove{$D_1 \cap \cdots \cap D_n \cap A \neq \emptyset$}}
        \step{b}{$D_1 \cap \cdots \cap D_n \in \DD$}
        \begin{proof}
            \pf\ Lemma \ref{lemma:finite_intersection_maximal}.
        \end{proof}
        \step{c}{$D_1 \cap \cdots \cap D_n \cap A \neq \emptyset$}
        \begin{proof}
            \pf\ Since $A$ intersects every member of $\DD$.
        \end{proof}
    \end{proof}
    \qedstep
    \begin{proof}
        \pf\ By maximality of $\DD$.
    \end{proof}
    \qed
\end{proof}

\begin{proposition}
    Let $X$ be a set. Let $\DD \subseteq \pow X$ be maximal with respect to the
    finite intersection property. Let $A, D \in \pow X$.
    If $D \in \DD$ and $D \subseteq A$ then $A \in \DD$.
\end{proposition}

\begin{proof}
    \pf
    \step{1}{$\DD \cup \{A\}$ satisfies the finite intersection property.}
    \begin{proof}
        \step{a}{\pflet{$D_1, \ldots, D_n \in \DD$}}
        \step{b}{$D_1 \cap \cdots \cap D_n \cap D \neq \emptyset$}
        \begin{proof}
            \pf\ Since $\DD$ satisfies the finite intersection property.
        \end{proof}
        \step{c}{$D_1 \cap \cdots \cap D_n \cap A \neq \emptyset$}
    \end{proof}
    \step{2}{$\DD = \DD \cup \{A\}$}
    \begin{proof}
        \pf\ By the maximality of $\DD$.
    \end{proof}
    \step{3}{$A \in \DD$}
    \qed
\end{proof}

\begin{definition}[Graph]
    Let $f : A \rightarrow B$. The \emph{graph} of $f$ is the set $\{ (x, f(x)) \mid x \in A \} \subseteq A \times B$.
\end{definition}

\section{Countable Intersection Property}

\begin{definition}[Countable Intersection Property]
    Let $X$ be a set and $\AA \subseteq \pow X$. Then $\AA$ satisfies the \emph{countable intersection property}
    if and only if every countable subset of $\AA$ has nonempty intersection.
\end{definition}

\begin{lemma}
    \label{lemma:countable_intersection_maximal}
    Let $X$ be a set. Let $\DD \subseteq \pow X$ be maximal with respect to the countable intersection property.
    Then any countable intersection of elements of $\DD$ is an element of $\DD$.
\end{lemma}

\begin{proof}
    \pf
    \step{1}{\pflet{$\DD_0 \subseteq \DD$ be countable.}}
    \step{2}{$\DD \cup \{ \bigcap \DD_0 \}$ has the countable intersection property.}
    \begin{proof}
        \pf\ Any countable intersection of members of $\DD \cup \{ \bigcap \DD_0 \}$
        is a finite intersection of members of $\DD$.
    \end{proof}
    \step{3}{$\DD = \DD \cup \{ \bigcap \DD_0 \}$}
    \begin{proof}
        \pf\ By maximality of $\DD$.
    \end{proof}
    \step{4}{$\bigcap \DD_0 \in \DD$.}
    \qed
\end{proof}

\begin{lemma}
    Let $X$ be a set. Let $\DD \subseteq \pow X$ be maximal with respect to the countable intersection property.
    Let $A \subseteq X$. If $A$ intersects every member of $\DD$ then $A \in \DD$.
\end{lemma}

\begin{proof}
    \pf
    \step{1}{$\DD \cup \{ A \}$ has the countable intersection property.}
    \begin{proof}
        \step{a}{\pflet{$\DD_0 \subseteq \DD$ be countable.}
        \prove{$\bigcap \DD_0 \cap A \neq \emptyset$}}
        \step{b}{$\bigcap \DD_0 \in \DD$}
        \begin{proof}
            \pf\ Lemma \ref{lemma:countable_intersection_maximal}.
        \end{proof}
        \step{c}{$\bigcap \DD_0 \cap A \neq \emptyset$}
        \begin{proof}
            \pf\ Since $A$ intersects every member of $\DD$.
        \end{proof}
    \end{proof}
    \qedstep
    \begin{proof}
        \pf\ By maximality of $\DD$.
    \end{proof}
    \qed
\end{proof}

\section{The Axiom of Choice}

\begin{axiom}[Axiom of Choice]
    Let $\AA$ be a set of disjoint nonempty sets. Then there exists a set $C$ consisting of exactly one element from each member of $\AA$.
\end{axiom}

\section{Choice Functions}

\begin{definition}[Choice Function]
    Let $\BB$ be a set of nonempty sets. A \emph{choice function} for $\BB$ is a function $c : \BB \rightarrow \bigcup \BB$ such that, for all $B \in \BB$,
    we have $c(B) \in B$.
\end{definition}

\begin{lemma}[Existence of a Choice Function (AC)]
    \label{lemma:choice_function}
    Every set of nonempty sets has a choice function.
\end{lemma}

\begin{proof}
    \pf
    \step{1}{\pflet{$\BB$ be a set of nonempty sets.}}
    \step{2}{For $B \in \BB$, \pflet{$B' = \{ B \} \times B$}}
    \step{3}{$\{ B' \mid B \in \BB \}$ is a set of disjoint nonempty sets.}
    \step{4}{\pick\ a set $c$ consisting of exactly one element from each $B'$ for $B \in \BB$.}
    \step{5}{$c$ is a choice function for $\BB$.}
    \qed
\end{proof}