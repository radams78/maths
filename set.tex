\chapter{Set Theory}

\section{Covers}

\begin{definition}[Cover]
    Let $X$ be a set and $\AA \subseteq \pow X$. Then $\AA$ \emph{covers} $X$,
    or is a \emph{covering} of $X$, if and only if $\bigcup \AA = X$.
\end{definition}

\section{The Finite Intersection Property}

\begin{definition}[Finite Intersection Property]
    Let $X$ be a set and $\AA \subseteq \pow X$. Then $\AA$ satisfies the \emph{finite intersection property} if and only if every nonempty finite subset of $\AA$
    has nonempty intersection.
\end{definition}

\begin{lemma}
    \label{lemma:maximal_finite_intersection_property}
    Let $X$ be a set. Let $\AA \subseteq \pow X$ have the finite intersection property.
    Then there exists a maximal set $\DD$ such that $\AA \subseteq \DD \subseteq \pow X$
    and $\DD$ has the finite intersection property.
\end{lemma}

\begin{proof}
    \pf
    \step{1}{\pflet{$\mathbb{F} = \{ \DD \mid \AA \subseteq \DD \subseteq \pow X, \DD \text{ has the finite intersection property} \}$}}
    \step{2}{Every chain in $\mathbb{F}$ has an upper bound.}
    \begin{proof}
        \step{a}{\pflet{$\mathbb{C}$ be a chain in $\mathbb{F}$.}}
        \step{b}{\assume{without loss of generality $\mathbb{C} \neq \emptyset$}
        \prove{$\bigcup \mathbb{C} \in \mathbb{F}$}}
        \begin{proof}
            \pf\ If $\mathbb{C} = \emptyset$ then $\AA$ is an upper bound.
        \end{proof}
        \step{b}{$\AA \subseteq \bigcup \mathbb{C} \subseteq \pow X$}
        \step{c}{\pflet{$C_1, \ldots, C_n \in \mathbb{C}$} \prove{$C_1 \cap \cdots \cap C_n \neq \emptyset$}}
        \step{d}{\pick\ $\mathcal{C}_1, \ldots, \mathcal{C}_n \in \mathbb{C}$ such that $C_i \in \mathcal{C}_i$ for all $i$.}
        \step{e}{\assume{without loss of generality $\mathcal{C}_1 \subseteq \cdots \subseteq \mathcal{C}_n$}}
        \step{f}{$C_1, \ldots, C_n \in \mathcal{C}_n$}
        \step{g}{$\mathcal{C}_n$ satisfies the finite intersection property.}
        \step{h}{$C_1 \cap \cdots \cap C_n \neq \emptyset$}
    \end{proof}
    \qedstep
    \begin{proof}
        \pf\ By Zorn's Lemma.
    \end{proof}
    \qed
\end{proof}

\begin{lemma}
    \label{lemma:finite_intersection_maximal}
    Let $X$ be a set. Let $\DD \subseteq \pow X$ be maximal with respect to the finite intersection property.
    Then any finite intersection of elements of $\DD$ is an element of $\DD$.
\end{lemma}

\begin{proof}
    \pf
    \step{1}{\pflet{$D_1, D_2 \in \DD$}}
    \step{2}{$\DD \cup \{ D_1 \cap D_2 \}$ has the finite intersection property.}
    \begin{proof}
        \pf\ Any finite intersection of members of $\DD \cup \{ D_1 \cap D_2 \}$
        is a finite intersection of members of $\DD$.
    \end{proof}
    \step{3}{$\DD = \DD \cup \{ D_1 \cap D_2 \}$}
    \begin{proof}
        \pf\ By maximality of $\DD$.
    \end{proof}
    \step{4}{$D_1 \cap D_2 \in \DD$.}
    \qed
\end{proof}

\begin{lemma}
    \label{lemma:member_maximal_finite_intersection}
    Let $X$ be a set. Let $\DD \subseteq \pow X$ be maximal with respect to the finite intersection property.
    Let $A \subseteq X$. If $A$ intersects every member of $\DD$ then $A \in \DD$.
\end{lemma}

\begin{proof}
    \pf
    \step{1}{$\DD \cup \{ A \}$ has the finite intersection property.}
    \begin{proof}
        \step{a}{\pflet{$D_1, \ldots, D_n \in \DD$} \prove{$D_1 \cap \cdots \cap D_n \cap A \neq \emptyset$}}
        \step{b}{$D_1 \cap \cdots \cap D_n \in \DD$}
        \begin{proof}
            \pf\ Lemma \ref{lemma:finite_intersection_maximal}.
        \end{proof}
        \step{c}{$D_1 \cap \cdots \cap D_n \cap A \neq \emptyset$}
        \begin{proof}
            \pf\ Since $A$ intersects every member of $\DD$.
        \end{proof}
    \end{proof}
    \qedstep
    \begin{proof}
        \pf\ By maximality of $\DD$.
    \end{proof}
    \qed
\end{proof}

\begin{proposition}
    Let $X$ be a set. Let $\DD \subseteq \pow X$ be maximal with respect to the
    finite intersection property. Let $A, D \in \pow X$.
    If $D \in \DD$ and $D \subseteq A$ then $A \in \DD$.
\end{proposition}

\begin{proof}
    \pf
    \step{1}{$\DD \cup \{A\}$ satisfies the finite intersection property.}
    \begin{proof}
        \step{a}{\pflet{$D_1, \ldots, D_n \in \DD$}}
        \step{b}{$D_1 \cap \cdots \cap D_n \cap D \neq \emptyset$}
        \begin{proof}
            \pf\ Since $\DD$ satisfies the finite intersection property.
        \end{proof}
        \step{c}{$D_1 \cap \cdots \cap D_n \cap A \neq \emptyset$}
    \end{proof}
    \step{2}{$\DD = \DD \cup \{A\}$}
    \begin{proof}
        \pf\ By the maximality of $\DD$.
    \end{proof}
    \step{3}{$A \in \DD$}
    \qed
\end{proof}

\begin{definition}[Graph]
    Let $f : A \rightarrow B$. The \emph{graph} of $f$ is the set $\{ (x, f(x)) \mid x \in A \} \subseteq A \times B$.
\end{definition}

\section{Countable Intersection Property}

\begin{definition}[Countable Intersection Property]
    Let $X$ be a set and $\AA \subseteq \pow X$. Then $\AA$ satisfies the \emph{countable intersection property}
    if and only if every countable subset of $\AA$ has nonempty intersection.
\end{definition}

\begin{lemma}
    \label{lemma:countable_intersection_maximal}
    Let $X$ be a set. Let $\DD \subseteq \pow X$ be maximal with respect to the countable intersection property.
    Then any countable intersection of elements of $\DD$ is an element of $\DD$.
\end{lemma}

\begin{proof}
    \pf
    \step{1}{\pflet{$\DD_0 \subseteq \DD$ be countable.}}
    \step{2}{$\DD \cup \{ \bigcap \DD_0 \}$ has the countable intersection property.}
    \begin{proof}
        \pf\ Any countable intersection of members of $\DD \cup \{ \bigcap \DD_0 \}$
        is a finite intersection of members of $\DD$.
    \end{proof}
    \step{3}{$\DD = \DD \cup \{ \bigcap \DD_0 \}$}
    \begin{proof}
        \pf\ By maximality of $\DD$.
    \end{proof}
    \step{4}{$\bigcap \DD_0 \in \DD$.}
    \qed
\end{proof}

\begin{lemma}
    Let $X$ be a set. Let $\DD \subseteq \pow X$ be maximal with respect to the countable intersection property.
    Let $A \subseteq X$. If $A$ intersects every member of $\DD$ then $A \in \DD$.
\end{lemma}

\begin{proof}
    \pf
    \step{1}{$\DD \cup \{ A \}$ has the countable intersection property.}
    \begin{proof}
        \step{a}{\pflet{$\DD_0 \subseteq \DD$ be countable.}
        \prove{$\bigcap \DD_0 \cap A \neq \emptyset$}}
        \step{b}{$\bigcap \DD_0 \in \DD$}
        \begin{proof}
            \pf\ Lemma \ref{lemma:countable_intersection_maximal}.
        \end{proof}
        \step{c}{$\bigcap \DD_0 \cap A \neq \emptyset$}
        \begin{proof}
            \pf\ Since $A$ intersects every member of $\DD$.
        \end{proof}
    \end{proof}
    \qedstep
    \begin{proof}
        \pf\ By maximality of $\DD$.
    \end{proof}
    \qed
\end{proof}

\section{The Axiom of Choice}

\begin{axiom}[Axiom of Choice]
    Let $\AA$ be a set of disjoint nonempty sets. Then there exists a set $C$ consisting of exactly one element from each member of $\AA$.
\end{axiom}

\section{Choice Functions}

\begin{definition}[Choice Function]
    Let $\BB$ be a set of nonempty sets. A \emph{choice function} for $\BB$ is a function $c : \BB \rightarrow \bigcup \BB$ such that, for all $B \in \BB$,
    we have $c(B) \in B$.
\end{definition}

\begin{lemma}[Existence of a Choice Function (AC)]
    \label{lemma:choice_function}
    Every set of nonempty sets has a choice function.
\end{lemma}

\begin{proof}
    \pf
    \step{1}{\pflet{$\BB$ be a set of nonempty sets.}}
    \step{2}{For $B \in \BB$, \pflet{$B' = \{ B \} \times B$}}
    \step{3}{$\{ B' \mid B \in \BB \}$ is a set of disjoint nonempty sets.}
    \step{4}{\pick\ a set $c$ consisting of exactly one element from each $B'$ for $B \in \BB$.}
    \step{5}{$c$ is a choice function for $\BB$.}
    \qed
\end{proof}