\documentclass{article}

\usepackage{amsmath}
\usepackage{amsthm}
\let\proof\relax
\let\endproof\relax
\let\qed\relax
\usepackage{pf2}

\title{Topology}
\author{Robin Adams}

\newtheorem{lemma}{Lemma}
\theoremstyle{definition}
\newtheorem{df}{Definition}

\newcommand{\pow}{\ensuremath{\mathcal{P}}}

\renewcommand{\AA}{\ensuremath{\mathcal{A}}}
\newcommand{\BB}{\ensuremath{\mathcal{B}}}
\newcommand{\CC}{\ensuremath{\mathcal{C}}}
\newcommand{\TT}{\ensuremath{\mathcal{T}}}
\newcommand{\UU}{\ensuremath{\mathcal{U}}}

\begin{document}
\maketitle

\section{Topological Spaces}

\begin{df}[Topology]
    A \emph{topology} on a set $X$ is a set $\TT \subseteq \pow X$ such that:
    \begin{itemize}
        \item $X \in \TT$.
        \item For all $\UU \subseteq \TT$ we have $\bigcup \UU \in \TT$.
        \item For all $U, V \in \TT$ we have $U \cap V \in \TT$.
    \end{itemize}
    We call the elements of $X$ \emph{points} and the elements of $\TT$ \emph{open sets}.
\end{df}

\begin{df}[Topological Space]
    A \emph{topological space} $X$ consists of a set $X$ and a topology on $X$.
\end{df}

\begin{df}[Discrete Space]
    For any set $X$, the \emph{discrete} topology on $X$ is $\pow X$.
\end{df}

\begin{df}[Indiscrete Space]
    For any set $X$, the \emph{indiscrete} or \emph{trivial} topology on $X$ is $\{ \emptyset, X \}$.
\end{df}

\begin{df}[Finite Complement Topology]
    For any set $X$, the \emph{finite complement topology} on $X$ is $\{ U \in \pow X \mid X \setminus U \text{ is finite} \} 
    \cup \{ \emptyset \}$.
\end{df}

\begin{df}[Countable Complement Topology]
    For any set $X$, the \emph{countable complement topology} on $X$ is $\{ U \in \pow X \mid X \setminus U \text{ is countable} \} 
    \cup \{ \emptyset \}$.
\end{df}

\begin{df}[Finer, Coarser]
    Suppose that $\TT$ and $\TT'$ are two topologies on a given set $X$. If $\TT' \supseteq \TT$, we say that $\TT'$ is \emph{finer} than
    $\TT$; if $\TT'$ \emph{properly} contains $\TT$, we say that $\TT'$ is \emph{strictly} finer than $\TT$. We also say that $\TT$ is
    \emph{coarser} than $\TT'$, or \emph{stricly} coarser, in these two respective situations. We say $\TT$ is \emph{comparable} with
    $\TT'$ if either $\TT' \supseteq \TT$ or $\TT \supseteq \TT'$.
\end{df}

\section{Basis for a Topology}

\begin{df}[Basis]
    If $X$ is a set, a \emph{basis} for a topology on $X$ is a set $\BB \subseteq \pow X$ called \emph{basis elements} such that
    \begin{enumerate}
        \item For all $x \in X$, there exists $B \in \BB$ such that $x \in B$.
        \item For all $B_1, B_2 \in \BB$ and $x \in B_1 \cap B_2$, there exists $B_3 \in \BB$ such that
        $x \in B_3 \subseteq B_1 \cap B_2$.
    \end{enumerate}

    If $\BB$ satisfies these two conditions, then we define the topology \emph{generated} by $\BB$ to be
    $\TT = \{ U \in \pow X \mid \forall x \in U. \exists B \in \BB. x \in B \subseteq U \}$.
\end{df}

We prove this is a topology.

\begin{proof}
    \pf
    \step{1}{$X \in \TT$}
    \begin{proof}
        \pf\ For all $x \in X$ there exists $B \in \BB$ such that $x \in B \subseteq X$ by condition 1.
    \end{proof}
    \step{2}{For all $\UU \subseteq \TT$ we have $\bigcup \UU \in \TT$}
    \begin{proof}
        \step{a}{\pflet{$\UU \subseteq \TT$}}
        \step{b}{\pflet{$x \in \bigcup \UU$}}
        \step{c}{\pick\ $U \in \UU$ such that $x \in U$}
        \step{d}{\pick\ $B \in \BB$ such that $x \in B \subseteq U$}
        \begin{proof}
            \pf\ Since $U \in \TT$ by \stepref{a} and \stepref{c}.
        \end{proof}
        \step{e}{$x \in B \subseteq \bigcup \UU$}
    \end{proof}
    \step{3}{For all $U, V \in \TT$ we have $U \cap V \in \TT$}
    \begin{proof}
        \step{a}{\pflet{$U, V \in \TT$}}
        \step{b}{\pflet{$x \in U \cap V$}}
        \step{c}{\pick\ $B_1 \in \BB$ such that $x \in B_1 \subseteq U$}
        \step{d}{\pick\ $B_2 \in \BB$ such that $x \in B_2 \subseteq V$}
        \step{e}{\pick\ $B_3 \in \BB$ such that $x \in B_3 \subseteq B_1 \cap B_2$}
        \begin{proof}
            \pf\ By condition 2.
        \end{proof}
        \step{f}{$x \in B_3 \subseteq U \cap V$}
    \end{proof}
    \qed
\end{proof}

\begin{lemma}
    Let $X$ be a set. Let $\BB$ be a basis for a topology $\TT$ on $X$. Then $\TT$ is the set of all unions of subsets of $\BB$.
\end{lemma}

\begin{proof}
    \pf
    \step{1}{For all $U \in \TT$, there exists $\AA \subseteq \BB$ such that $U = \bigcup \AA$}
    \begin{proof}
        \step{a}{\pflet{$U \in \TT$}}
        \step{b}{\pflet{$\AA = \{ B \in \BB \mid B \subseteq U \}$}}
        \step{c}{$U \subseteq \bigcup \AA$}
        \begin{proof}
            \step{i}{\pflet{$x \in U$}}
            \step{ii}{\pick\ $B \in \BB$ such that $x \in B \subseteq U$}
            \begin{proof}
                \pf\ Since $\BB$ is a basis for $\TT$.
            \end{proof}
            \step{iii}{$x \in B \in \AA$}
        \end{proof}
        \step{d}{$\bigcup \AA \subseteq U$}
        \begin{proof}
            \pf\ From the definition of $\AA$ (\stepref{b}).
        \end{proof}
    \end{proof}
    \step{2}{For all $\AA \subseteq \BB$ we have $\bigcup \AA \in \TT$}
    \begin{proof}
        \step{a}{$\BB \subseteq \TT$}
        \begin{proof}
            \pf\ If $B \in \BB$ and $x \in B$, then there exists $B' \in \BB$ such that $x \in B' \subseteq B$, namely $B' = B$.
        \end{proof}
        \qedstep
        \begin{proof}
            \pf\ Since $\TT$ is closed under union.
        \end{proof}
    \end{proof}
    \qed
\end{proof}

\begin{lemma}
    Let $X$ be a topological space. Suppose that $\CC$ is a set of open sets such that, for every open set $U$ and every point $x \in U$,
    there exists $C \in \CC$ such that $x \in C \subseteq U$. Then $\CC$ is a basis for the topology on $X$.
\end{lemma}

\begin{proof}
    \pf
    \step{1}{For all $x \in X$, there exists $C \in \CC$ such that $x \in C$}
    \begin{proof}
        \pf\ Immediate from hypothesis.
    \end{proof}
    \step{2}{For all $C_1, C_2 \in \CC$ and $x \in C_1 \cap C_2$, there exists $C_3 \in \CC$ such that $x \in C_3 \subseteq C_1 \cap C_2$}
    \begin{proof}
        \pf\ Since $C_1 \cap C_2$ is open.
    \end{proof}
    \step{3}{Every open set is open in the topology generated by $\CC$}
    \begin{proof}
        \pf\ Immediate from hypothesis.
    \end{proof}
    \step{4}{Every union of a subset of $\CC$ is open.}
    \begin{proof}
        \pf\ Since every member of $\CC$ is open.
    \end{proof}
    \qed
\end{proof}
\end{document}