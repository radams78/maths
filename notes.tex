\documentclass{article}

\usepackage{amsmath}
\usepackage{amssymb}
\usepackage{amsthm}
\let\proof\relax
\let\endproof\relax
\let\qed\relax
\usepackage{pf2}

\title{Topology}
\author{Robin Adams}

\newtheorem{lemma}{Lemma}
\newtheorem{corollary}{Corollary}[lemma]
\newtheorem{theorem}[lemma]{Theorem}
\newtheorem{proposition}[lemma]{Proposition}
\theoremstyle{definition}
\newtheorem{definition}[lemma]{Definition}
\newtheorem{example}[lemma]{Example}

\newcommand{\id}[1]{\ensuremath{\mathrm{id}_{#1}}}
\newcommand{\Int}{\ensuremath{\operatorname{Int}}}
\newcommand{\pow}{\ensuremath{\mathcal{P}}}
\newcommand{\inv}[1]{\ensuremath{{#1}^{-1}}}
\newcommand{\diam}{\ensuremath{\operatorname{diam}}}

\renewcommand{\AA}{\ensuremath{\mathcal{A}}}
\newcommand{\BB}{\ensuremath{\mathcal{B}}}
\newcommand{\CC}{\ensuremath{\mathcal{C}}}
\newcommand{\DD}{\ensuremath{\mathcal{D}}}
\newcommand{\NN}{\ensuremath{\mathbb{N}}}
\newcommand{\RR}{\ensuremath{\mathbb{R}}}
\renewcommand{\SS}{\ensuremath{\mathcal{S}}}
\newcommand{\TT}{\ensuremath{\mathcal{T}}}
\newcommand{\UU}{\ensuremath{\mathcal{U}}}
\newcommand{\VV}{\ensuremath{\mathcal{V}}}
\newcommand{\ZZ}{\ensuremath{\mathbb{Z}}}

\begin{document}
\maketitle

\section{Order Theory}

\begin{definition}[Preorder]
    Let $X$ be a set. A \emph{preorder} on $X$ is a binary relation $\leq$ on $X$ such that:
    \begin{description}
        \item[Reflexivity] For all $x \in X$, we have $x \leq x$
        \item[Transitivity] For all $x, y, z \in X$, if $x \leq y$ and $y \leq z$ then $x \leq z$.  
    \end{description}
\end{definition}

\begin{definition}[Preordered Set]
    A \emph{preordered set} consists of a set $X$ and a preorder $\leq$ on $X$.
\end{definition}

\begin{definition}[Interval]
    Let $X$ be a preordered set and $Y \subseteq X$. Then $Y$ is an \emph{interval} if and only if, for
    all $a, b \in Y$ and $c \in X$, if $a \leq c \leq b$ then $c \in Y$.
\end{definition}

\section{Real Analysis}

\begin{definition}
    Let $\RR^\infty$ be the subset of $\RR^\omega$ consisting of all sequences $(a_n)$ such that $a_n = 0$ for all but finitely many $n$.
\end{definition}

\begin{theorem}
    If $(s_n)$ is a bounded sequence of real numbers and $s_n \leq s_{n+1}$ for all $n$ then $(s_n)$ converges to its supremum.
\end{theorem}

\begin{proof}
    \pf\ Let $l = \sup_n s_n$ and let $\epsilon > 0$. Then $l - \epsilon$ is not an upper bound for $(s_n)$. Pick $N$ such that $s_N > l - \epsilon$. Then $|s_n - l| < \epsilon$
    for all $n \geq N$. \qed
\end{proof}

\begin{theorem}
    If $\sum_{i=0}^\infty a_i = s$ and $\sum_{i=0}^\infty b_i = t$ then $\sum_{i=0}^\infty (ca_i + b_i) = cs+t$.
\end{theorem}

\begin{proof}
    \pf $\sum_{i=0}^N (ca_i + b_i) = c \sum_{i=0}^N a_i + \sum_{i=0}^N b_i \rightarrow cs+t$ as $n \rightarrow \infty$. \qed
\end{proof}

\begin{theorem}[Comparison Test]
    If $|a_i| \leq b_i$ for all $i$ and $\sum_{i=0}^\infty b_i$ converges then $\sum_{i=0}^\infty a_i$ converges.
\end{theorem}

\begin{proof}
    \pf
    \step{1}{$\sum_{i=0}^\infty |a_i|$ converges}
    \begin{proof}
        \pf\ The partial sums $\sum_{i=0}^N |a_i|$ form an increasing sequence bounded above by $\sum_{i=0}^\infty b_i$.
    \end{proof}
    \step{2}{\pflet{$c_i = |a_i| + a_i$ for all $i$}}
    \step{3}{$\sum_{i=0}^\infty c_i$ converges}
    \begin{proof}
        \pf\ Each $c_i$ is either $2|a_i|$ or 0. So the partial sums $\sum_{i=0}^N c_i$ form an increasing sequence bounded above by $2 \sum_{i=0}^\infty b_i$.
    \end{proof}
    \qedstep
    \begin{proof}
        \pf\ Since $a_i = c_i - |a_i|$.
    \end{proof}
    \qed
\end{proof}

\begin{corollary}
    If $\sum_{i=0}^\infty |a_i|$ converges then $\sum_{i=0}^\infty a_i$ converges.
\end{corollary}

\begin{theorem}[Weierstrass $M$-test]
    Let $X$ be a set and $(f_n : X \rightarrow \RR)$ be a sequence of functions. Let
    \[ s_n(x) = \sum_{i=0}^n f_i(x) \]
    for all $n$, $x$. Suppose $|f_i(x)| \leq M_i$ for all $i \geq 0$ and $x \in X$.
    If the series $\sum_{i=0}^\infty M_i$ converges, then the sequence $(s_n)$ converges uniformly to
    \[ s(x) = \sum_{i=0}^\infty f_i(x) \enspace . \]
\end{theorem}

\begin{proof}
    \pf
    \step{1}{\pflet{$r_n = \sum_{i=n+1}^\infty M_i$ for all $n$}}
    \step{2}{Given $0 \leq n < k$, we have $|s_k(x) - s_n(x)| \leq r_n$}
    \begin{proof}
        \pf
        \begin{align*}
            |s_k(x) - s_n(x)| & = |\sum_{i=n+1}^k f_i(x)| \\
            & \leq \sum_{i=n+1}^k |f_i(x)| \\
            & \leq \sum_{i=n+1}^k M_i \\
            & \leq r_n
        \end{align*}
    \end{proof}
    \step{3}{Given $n \geq 0$ we have $|s(x) - s_n(x)| \leq r_n$}
    \begin{proof}
        \pf\ By taking the limit $k \rightarrow \infty$ in \stepref{2}.
    \end{proof}
    \qedstep
    \begin{proof}
        \pf\ Since $r_n \rightarrow 0$ as $n \rightarrow \infty$.
    \end{proof}
    \qed
\end{proof}


\section{Topological Spaces}

\begin{definition}[Topology]
    A \emph{topology} on a set $X$ is a set $\TT \subseteq \pow X$ such that:
    \begin{itemize}
        \item $X \in \TT$.
        \item For all $\UU \subseteq \TT$ we have $\bigcup \UU \in \TT$.
        \item For all $U, V \in \TT$ we have $U \cap V \in \TT$.
    \end{itemize}
    We call the elements of $X$ \emph{points} and the elements of $\TT$ \emph{open sets}.
\end{definition}

\begin{definition}[Topological Space]
    A \emph{topological space} $X$ consists of a set $X$ and a topology on $X$.
\end{definition}

\begin{definition}[Discrete Space]
    For any set $X$, the \emph{discrete} topology on $X$ is $\pow X$.
\end{definition}

\begin{definition}[Indiscrete Space]
    For any set $X$, the \emph{indiscrete} or \emph{trivial} topology on $X$ is $\{ \emptyset, X \}$.
\end{definition}

\begin{definition}[Finite Complement Topology]
    For any set $X$, the \emph{finite complement topology} on $X$ is $\{ U \in \pow X \mid X \setminus U \text{ is finite} \} 
    \cup \{ \emptyset \}$.
\end{definition}

\begin{definition}[Countable Complement Topology]
    For any set $X$, the \emph{countable complement topology} on $X$ is $\{ U \in \pow X \mid X \setminus U \text{ is countable} \} 
    \cup \{ \emptyset \}$.
\end{definition}

\begin{definition}[Finer, Coarser]
    Suppose that $\TT$ and $\TT'$ are two topologies on a given set $X$. If $\TT' \supseteq \TT$, we say that $\TT'$ is \emph{finer} than
    $\TT$; if $\TT'$ \emph{properly} contains $\TT$, we say that $\TT'$ is \emph{strictly} finer than $\TT$. We also say that $\TT$ is
    \emph{coarser} than $\TT'$, or \emph{stricly} coarser, in these two respective situations. We say $\TT$ is \emph{comparable} with
    $\TT'$ if either $\TT' \supseteq \TT$ or $\TT \supseteq \TT'$.
\end{definition}

\begin{lemma}
    \label{lemma:open}
    Let $X$ be a topological space and $U \subseteq X$. Then $U$ is open if and only if, for all $x \in U$,
    there exists an open set $V$ such that $x \in V \subseteq U$.
\end{lemma}

\begin{proof}
    \pf
    \step{1}{$\Rightarrow$}
    \begin{proof}
        \pf\ Take $V = U$
    \end{proof}
    \step{2}{$\Leftarrow$}
    \begin{proof}
        \pf\ We have $U = \bigcup \{ V \in \pow X \mid V \subseteq U \}$.
    \end{proof}
    \qed
\end{proof}

\begin{lemma}
    Let $X$ be a set and $\TT$ a nonempty set of topologies on $X$. Then $\bigcap \TT$ is a topology on $X$,
    and is the finest topology that is coarser than every member of $\TT$.
\end{lemma}

\begin{proof}
    \pf
    \step{1}{$X \in \bigcap \TT$}
    \begin{proof}
        \pf\ Since $X$ is in every member of $\TT$.
    \end{proof}
    \step{2}{$\bigcap \TT$ is closed under union.}
    \begin{proof}
        \step{a}{\pflet{$\UU \subseteq \bigcap \TT$}}
        \step{b}{For all $T \in \TT$ we have $\UU \subseteq T$}
        \step{c}{For all $T \in \TT$ we have $\bigcup \UU \in T$}
        \step{d}{$\bigcup \UU \in \bigcap \TT$}
    \end{proof}
    \step{3}{$\bigcap \TT$ is closed under binary intersection.}
    \begin{proof}
        \step{a}{\pflet{$U,V \in \bigcap \TT$}}
        \step{b}{For all $T \in \TT$ we have $U,V \in T$}
        \step{c}{For all $T \in \TT$ we have $U \cap V \in T$}
        \step{d}{$U \cap V \in \bigcap \TT$}
    \end{proof}
    \qed
\end{proof}

\begin{lemma}
    Let $X$ be a set and $\TT$ a set of topologies on $X$. Then there exists a unique coarsest topology
    that is finer than every member of $\TT$.
\end{lemma}

\begin{proof}
    \pf\ The required topology is given by
    \[ \bigcap \{ T \in \pow \pow X \mid T \emph{ is a topology on $X$ that is
    finer than every member of $\TT$} \} \enspace , \] 
    The set is nonempty since it contains the discrete topology. \qed
\end{proof}

\section{Basis for a Topology}

\begin{definition}[Basis]
    If $X$ is a set, a \emph{basis} for a topology on $X$ is a set $\BB \subseteq \pow X$ called \emph{basis elements} such that
    \begin{enumerate}
        \item For all $x \in X$, there exists $B \in \BB$ such that $x \in B$.
        \item For all $B_1, B_2 \in \BB$ and $x \in B_1 \cap B_2$, there exists $B_3 \in \BB$ such that
        $x \in B_3 \subseteq B_1 \cap B_2$.
    \end{enumerate}

    If $\BB$ satisfies these two conditions, then we define the topology \emph{generated} by $\BB$ to be
    $\TT = \{ U \in \pow X \mid \forall x \in U. \exists B \in \BB. x \in B \subseteq U \}$.
\end{definition}

We prove this is a topology.

\begin{proof}
    \pf
    \step{1}{$X \in \TT$}
    \begin{proof}
        \pf\ For all $x \in X$ there exists $B \in \BB$ such that $x \in B \subseteq X$ by condition 1.
    \end{proof}
    \step{2}{For all $\UU \subseteq \TT$ we have $\bigcup \UU \in \TT$}
    \begin{proof}
        \step{a}{\pflet{$\UU \subseteq \TT$}}
        \step{b}{\pflet{$x \in \bigcup \UU$}}
        \step{c}{\pick\ $U \in \UU$ such that $x \in U$}
        \step{d}{\pick\ $B \in \BB$ such that $x \in B \subseteq U$}
        \begin{proof}
            \pf\ Since $U \in \TT$ by \stepref{a} and \stepref{c}.
        \end{proof}
        \step{e}{$x \in B \subseteq \bigcup \UU$}
    \end{proof}
    \step{3}{For all $U, V \in \TT$ we have $U \cap V \in \TT$}
    \begin{proof}
        \step{a}{\pflet{$U, V \in \TT$}}
        \step{b}{\pflet{$x \in U \cap V$}}
        \step{c}{\pick\ $B_1 \in \BB$ such that $x \in B_1 \subseteq U$}
        \step{d}{\pick\ $B_2 \in \BB$ such that $x \in B_2 \subseteq V$}
        \step{e}{\pick\ $B_3 \in \BB$ such that $x \in B_3 \subseteq B_1 \cap B_2$}
        \begin{proof}
            \pf\ By condition 2.
        \end{proof}
        \step{f}{$x \in B_3 \subseteq U \cap V$}
    \end{proof}
    \qed
\end{proof}

\begin{lemma}
    \label{lemma:basis_unions}
    Let $X$ be a set. Let $\BB$ be a basis for a topology $\TT$ on $X$. Then $\TT$ is the set of all unions of 
    subsets of $\BB$.
\end{lemma}

\begin{proof}
    \pf
    \step{1}{For all $U \in \TT$, there exists $\AA \subseteq \BB$ such that $U = \bigcup \AA$}
    \begin{proof}
        \step{a}{\pflet{$U \in \TT$}}
        \step{b}{\pflet{$\AA = \{ B \in \BB \mid B \subseteq U \}$}}
        \step{c}{$U \subseteq \bigcup \AA$}
        \begin{proof}
            \step{i}{\pflet{$x \in U$}}
            \step{ii}{\pick\ $B \in \BB$ such that $x \in B \subseteq U$}
            \begin{proof}
                \pf\ Since $\BB$ is a basis for $\TT$.
            \end{proof}
            \step{iii}{$x \in B \in \AA$}
        \end{proof}
        \step{d}{$\bigcup \AA \subseteq U$}
        \begin{proof}
            \pf\ From the definition of $\AA$ (\stepref{b}).
        \end{proof}
    \end{proof}
    \step{2}{For all $\AA \subseteq \BB$ we have $\bigcup \AA \in \TT$}
    \begin{proof}
        \step{a}{$\BB \subseteq \TT$}
        \begin{proof}
            \pf\ If $B \in \BB$ and $x \in B$, then there exists $B' \in \BB$ such that $x \in B' \subseteq B$, namely $B' = B$.
        \end{proof}
        \qedstep
        \begin{proof}
            \pf\ Since $\TT$ is closed under union.
        \end{proof}
    \end{proof}
    \qed
\end{proof}

\begin{corollary}
    \label{cor:basis_open}
    Let $X$ be a set. Let $\BB$ be a basis for a topology $\TT$ on $X$. Then $\TT$ is the coarsest topology
    that includes $\BB$.
\end{corollary}

\begin{proof}
    \pf\ Since every topology that includes $\BB$ includes all unions of subsets of $\BB$. \qed
\end{proof}

\begin{lemma}
    \label{lemma:basis}
    Let $X$ be a topological space. Suppose that $\CC$ is a set of open sets such that, for every open set $U$ and every point $x \in U$,
    there exists $C \in \CC$ such that $x \in C \subseteq U$. Then $\CC$ is a basis for the topology on $X$.
\end{lemma}

\begin{proof}
    \pf
    \step{1}{For all $x \in X$, there exists $C \in \CC$ such that $x \in C$}
    \begin{proof}
        \pf\ Immediate from hypothesis.
    \end{proof}
    \step{2}{For all $C_1, C_2 \in \CC$ and $x \in C_1 \cap C_2$, there exists $C_3 \in \CC$ such that $x \in C_3 \subseteq C_1 \cap C_2$}
    \begin{proof}
        \pf\ Since $C_1 \cap C_2$ is open.
    \end{proof}
    \step{3}{Every open set is open in the topology generated by $\CC$}
    \begin{proof}
        \pf\ Immediate from hypothesis.
    \end{proof}
    \step{4}{Every union of a subset of $\CC$ is open.}
    \begin{proof}
        \pf\ Since every member of $\CC$ is open.
    \end{proof}
    \qed
\end{proof}

\begin{lemma}
    Let $\BB$ and $\BB'$ be bases for the topologies $\TT$ and $\TT'$ respectively on the set $X$. Then the following are equivalent.
    \begin{enumerate}
        \item $\TT \subseteq \TT'$
        \item For all $B \in \BB$ and $x \in B$, there exists $B' \in \BB'$ such that $x \in B' \subseteq B$.
    \end{enumerate}
\end{lemma}

\begin{proof}
    \pf
    \step{1}{$1 \Rightarrow 2$}
    \begin{proof}
        \step{a}{\assume{$\TT \subseteq \TT'$}}
        \step{b}{\pflet{$B \in \BB$ and $x \in B$}}
        \step{c}{$B \in \TT$}
        \begin{proof}
            \pf\ Corollary \ref{cor:basis_open}.
        \end{proof}
        \step{d}{$B \in \TT'$}
        \begin{proof}
            \pf\ By \stepref{a}
        \end{proof}
        \step{e}{There exists $B' \in \BB'$ such that $x \in B' \subseteq B$}
        \begin{proof}
            \pf\ Since $\BB'$ is a basis for $\TT'$.
        \end{proof}
    \end{proof}
    \step{2}{$2 \Rightarrow 1$}
    \begin{proof}
        \step{a}{\assume{2}}
        \step{b}{\pflet{$U \in \TT$} \prove{$U \in \TT'$}}
        \step{c}{\pflet{$x \in U$} \prove{There exists $B' \in \BB'$ such that $x \in B' \subseteq U$}}
        \step{d}{\pick\ $B \in \BB$ such that $x \in B \subseteq U$}
        \begin{proof}
            \pf\ Since $\BB$ is a basis for $\TT$.
        \end{proof}
        \step{e}{\pick\ $B' \in \BB'$ such that $x \in B' \subseteq B$}
        \begin{proof}
            \pf\ By \stepref{a}.
        \end{proof}
        \step{f}{$x \in B' \subseteq U$}
    \end{proof}
    \qed
\end{proof}

\begin{definition}[Lower Limit Topology on the Real Line]
    The \emph{lower limit topology on the real line} is the topology on $\RR$ generated by the basis consisting of all half-open intervals
    of the form $[a,b)$.

    We write $\RR_l$ for the topological space $\RR$ under the lower limit topology.
\end{definition}

We prove this is a basis for a topology.

\begin{proof}
    \pf
    \step{1}{For all $x \in \RR$ there exists an interval $[a,b)$ such that $x \in [a,b)$.}
    \begin{proof}
        \pf\ Take $[a,b) = [x,x+1)$.
    \end{proof}
    \step{2}{For any open intervals $[a,b)$, $[c,d)$ if $x \in [a,b) \cap [c,d)$, then there exists an interval $[e,f)$ such that
    $x \in [e,f) \subseteq [a,b) \cap [c,d)$}
    \begin{proof}
        \pf\ Take $[e,f) = [\max (a,c),\min (b,d))$.
    \end{proof}
    \qed
\end{proof}

\begin{definition}[$K$-topology on the Real Line]
    Let $K = \{ 1/n \mid n \in \ZZ^+ \}$.

    The \emph{$K$-topology on the real line} is the topology on $\RR$ generated by the basis consisting of all open intervals
    $(a,b)$ and all sets of the form $(a,b) \setminus K$.

    We write $\RR_K$ for the topological space $\RR$ under the $K$-topology.
\end{definition}

We prove this is a basis for a topology.

\begin{proof}
    \pf
    \step{1}{For all $x \in \RR$ there exists an open interval $(a,b)$ such that $x \in (a,b)$.}
    \begin{proof}
        \pf\ Take $(a,b) = (x-1,x+1)$.
    \end{proof}
    \step{2}{For any basic open sets $B_1$, $B_2$ if $x \in B_1 \cap B_2$, then there exists a basic open set $B_3$ such that
    $x \in B_3 \subseteq B_1 \cap B_2$}
    \begin{proof}
        \step{a}{\case{$B_1 = (a,b)$, $B_2 = (c,d)$}}
        \begin{proof}
            \pf\ Take $B_3 = (\max (a,c), \min (b,d))$.
        \end{proof}
        \step{b}{\case{$B_1 = (a,b)$ or $(a,b) \setminus K$, $B_2 = (c,d)$ or $(c,d) \setminus K$, and they are
        not both open intervals.}}
        \begin{proof}
            \pf\ Take $B_3 = (\max (a,c), \min (b,d)) \setminus K$.
        \end{proof}
    \end{proof}
    \qed
\end{proof}

\begin{lemma}
    The lower limit topology and the $K$-topology are incomparable.
\end{lemma}

\begin{proof}
    \pf
    \step{1}{The interval $[10,11)$ is not open in the $K$-topology.}
    \begin{proof}
        \pf\ There is no open interval $(a,b)$ such that $10 \in (a,b) \subseteq [10,11)$ or
        $10 \in (a,b) \setminus K \subseteq [10,11)$.
    \end{proof}
    \step{2}{The set $(-1,1) \setminus K$ is not open in the lower limit topology.}
    \begin{proof}
        \pf\ There is no half-open interval $[a,b)$ such that $0 \in [a,b) \subseteq (-1,1) \setminus K$, since
        there must be a positive integer $n$ with $1/n \in [a,b)$.
    \end{proof}
    \qed
\end{proof}

\begin{definition}[Subbasis]
    A \emph{subbasis} $\mathcal{S}$ for a topology on $X$ is a set $\mathcal{S} \subseteq \pow X$ such that
    $\bigcup \mathcal{S} = X$.

    The topology \emph{generated} by the subbasis $\mathcal{S}$ is the set of all unions of finite
    intersections of elements of $\mathcal{S}$.
\end{definition}

We prove this is a topology.

\begin{proof}
    \pf
    \step{1}{The set $\mathcal{B}$ of all finite intersections of elements of $\mathcal{S}$ forms a basis for a topology
    on $X$.}
    \begin{proof}
        \step{a}{$\bigcup \mathcal{B} = X$}
        \begin{proof}
            \pf\ Since $\mathcal{S} \subseteq \BB$.
        \end{proof}
        \step{b}{$\BB$ is closed under binary intersection.}
        \begin{proof}
            \pf\ By definition.
        \end{proof}
    \end{proof}
    \qedstep
    \begin{proof}
        \pf\ By Lemma \ref{lemma:basis_unions}.
    \end{proof}
    \qed
\end{proof}

We have simultaneously proved:

\begin{proposition}
    \label{proposition:subbasis_basis}
    Let $\SS$ be a subbasis for the topology on $X$.
    Then the set of all finite intersections of elements of $\SS$
    is a basis for the topology on $X$.
\end{proposition}

\begin{proposition}
    \label{proposition:subbasis_coarsest}
    Let $X$ be a set. Let $\mathcal{S}$ be a subbasis for a topology $\TT$ on $X$. Then $\TT$ is the coarsest topology
    that includes $\mathcal{S}$.
\end{proposition}

\begin{proof}
    \pf\ Since every topology that includes $\mathcal{S}$ includes every union of finite intersections of
    elements of $\mathcal{S}$. \qed
\end{proof}

\section{Open Maps}

\begin{definition}[Open Map]
    Let $X$ and $Y$ be topological spaces and $f : X \rightarrow Y$. Then $f$ is an \emph{open map} if and
    only if, for every open set $U$ in $X$, the set $f(U)$ is open in $Y$.
\end{definition}

\begin{lemma}
    \label{lemma:open_map_basis}
    Let $X$ and $Y$ be topological spaces and $f : X \rightarrow Y$. Let $\BB$ be a basis for the topology on $X$.
    If $f(B)$ is open in $Y$ for all $B \in \BB$, then $f$ is an open map.
\end{lemma}

\begin{proof}
    \pf\ From Lemma \ref{lemma:basis_unions}. \qed
\end{proof}

\section{The Order Topology}

\begin{definition}[Order Topology]
    Let $X$ be a linearly ordered set with at least two points. The \emph{order topology} on $X$ is the topology
    generated by the basis $\mathcal{B}$ consisting of:
    \begin{itemize}
        \item all open intervals $(a,b)$;
        \item all intervals of the form $[\bot,b)$ where $\bot$ is least in $X$;
        \item all intervals of the form $(a,\top]$ where $\top$ is greatest in $X$.
    \end{itemize}
\end{definition}

We prove this is a basis for a topology.

\begin{proof}
    \pf
    \step{1}{For all $x \in X$ there exists $B \in \mathcal{B}$ such that $x \in B$.}
    \begin{proof}
        \step{a}{\pflet{$x \in X$}}
        \step{b}{\case{$x$ is greatest in $X$.}}
        \begin{proof}
            \step{i}{\pick\ $y \in X$ with $y \neq x$}
            \step{ii}{$x \in (y,x] \in \BB$}
        \end{proof}
        \step{c}{\case{$x$ is least in $X$.}}
        \begin{proof}
            \step{i}{\pick\ $y \in X$ with $y \neq x$}
            \step{ii}{$x \in [x,y) \in \BB$}
        \end{proof}
        \step{d}{\case{$x$ is neither greatest nor least in $X$.}}
        \begin{proof}
            \step{i}{\pick\ $a, b \in X$ with $a < x$ and $x < b$}
            \step{ii}{$x \in (a,b) \in \BB$}
        \end{proof}
    \end{proof}
    \step{2}{For all $B_1, B_2 \in \mathcal{B}$ and $x \in B_1 \cap B_2$, there exists $B_3 \in \BB$
    such that $x \in B_3 \subseteq B_1 \cap B_2$}
    \begin{proof}
        \step{i}{\pflet{$B_1, B_2 \in \BB$ and $x \in B_1 \cap B_2$}}
        \step{ii}{\case{$B_1 = (a,b)$, $B_2 = (c,d)$}}
        \begin{proof}
            \pf\ Take $B_3 = (\max (a,c), \min (b,d))$.
        \end{proof}
        \step{iii}{\case{$B_1 = (a,b)$, $B_2 = [\bot,d)$}}
        \begin{proof}
            \pf\ Take $B_3 = (a, \min (b,d))$.
        \end{proof}
        \step{iv}{\case{$B_1 = (a,b)$, $B_2 = (c,\top]$}}
        \begin{proof}
            \pf\ Take $B_3 = (\max (a,c), b)$.
        \end{proof}
        \step{v}{\case{$B_1 = [\bot,b)$, $B_2 = [\bot,d)$}}
        \begin{proof}
            \pf\ Take $B_3 = [\bot, \min (b,d))$.
        \end{proof}
        \step{vi}{\case{$B_1 = [\bot, b)$, $B_2 = (c, \top]$}}
        \begin{proof}
            \pf\ Take $B_3 = (c, b)$.
        \end{proof}
    \end{proof}
    \qed
\end{proof}

\begin{lemma}
    \label{lemma:open_rays_subbasis}
    Let $X$ be a linearly ordered set. Then the open rays form a subbasis for the order topology on $X$.
\end{lemma}

\begin{proof}
    \pf
    \step{1}{Every open ray is open.}
    \begin{proof}
        \step{a}{For all $a \in X$, the ray $(- \infty, a)$ is open.}
        \begin{proof}
            \step{i}{\pflet{$x \in (- \infty, a)$}}
            \step{ii}{\case{$x$ is least in $X$}}
            \begin{proof}
                \pf\ $x in [x, a) = (- \infty, a)$.
            \end{proof}
            \step{iii}{\case{$x$ is not least in $X$}}
            \begin{proof}
                \step{one}{\pick\ $y < x$}
                \step{two}{$x \in (y, a) \subseteq (- \infty, a)$}
            \end{proof}
        \end{proof}
        \step{b}{For all $a \in X$, the ray $(a, + \infty)$ is open.}
        \begin{proof}
            \pf\ Similar.
        \end{proof}
    \end{proof}
    \step{2}{Every basic open set is a finite intersection of open rays.}
    \begin{proof}
        \pf\ We have $(a,b) = (a,+\infty) \cap (-\infty, b)$, $[\bot,b) = (- \infty, b)$ and
        $(a, \top] = (a, + \infty)$.
    \end{proof}
    \qed
\end{proof}

\begin{definition}[Standard Topology on the Real Line]
    The \emph{standard topology on the real line} is the order topology on $\RR$ generated by the standard
    order.
\end{definition}

\begin{lemma}
    The lower limit topology is strictly finer than the standard topology on $\RR$.
\end{lemma}

\begin{proof}
    \pf
    \step{1}{Every open interval is open in the lower limit topology.}
    \begin{proof}
        \pf\ If $x \in (a,b)$ then $x \in [x,b) \subseteq (a,b)$.
    \end{proof}
    \step{2}{The half-open interval $[0,1)$ is not open in the standard topology.}
    \begin{proof}
        \pf\ There is no open interval $(a,b)$ such that $0 \in (a,b) \subseteq [0,1)$.
    \end{proof}
    \qed
\end{proof}

\begin{lemma}
    The $K$-topology is strictly finer than the standard topology on $\RR$.
\end{lemma}

\begin{proof}
    \pf
    \step{1}{Every open interval is open in the $K$-topology.}
    \begin{proof}
        \pf\ Corollary \ref{cor:basis_open}.
    \end{proof}
    \step{2}{The set $(-1,1) \setminus K$ is not open in the standard topology.}
    \begin{proof}
        \pf\ There is no open interval $(a,b)$ such that $0 \in (a,b) \subseteq (-1,1) \setminus K$, since
        there must be a positive integer $n$ with $1/n \in (a,b)$.
    \end{proof}
    \qed
\end{proof}

\begin{definition}[Ordered Square]
    The \emph{ordered square} $I_o^2$ is the set $[0,1]^2$ under the order topology generated by the
    dictionary order.
\end{definition}

\section{The Product Topology}

\begin{definition}[Product Topology]
    Let $\{ A_i \}_{i \in I}$ be a family of topological spaces. The \emph{product topology} on $\prod_{i \in I} A_i$
    is the topology generated by the subbasis consisting of the sets of the form $\inv{\pi_i}(U)$ where $i \in I$
    and $U$ is open in $A_i$.
\end{definition}

\begin{proposition}
    \label{proposition:product_basis}
    The product topology on $\prod_{i \in I} A_i$ is generated by the basis consisting of all sets of the form $\prod_{i \in I} U_i$
    where $\{ U_i \}_{i \in I}$ is a family such that each $U_i$ is an open set in $A_i$ and $U_i = A_i$ for all but finitely many $i$.
\end{proposition}

\begin{proof}
    \pf\ From Proposition \ref{proposition:subbasis_basis}. \qed
\end{proof}

\begin{proposition}
    Let $\{ A_i \}_{i \in I}$ be a family of topological spaces. For $i \in I$, let $\BB_i$ be a basis for the topology on $A_i$. Then $\BB = \{ \prod_{i \in I} B_i \mid
    \forall i \in I. B_i \in \BB_i, B_i = A_i \text{ for all but finitely many } i \}$ is a basis for the box topology on $\prod_{i \in I} A_i$.
\end{proposition}

\begin{proof}
    \pf
    \step{1}{Every set in $\BB$ is open.}
    \step{2}{For every point $a \in \prod_{i \in I} A_i$ and every open set $U$ with $a \in U$, there exists $B \in \BB$ such that $a \in B \subseteq U$.}
    \begin{proof}
        \step{a}{\pflet{$U$ be open and $a \in U$}}
        \step{b}{\pick\ a family $\{ U_i \}_{i \in I}$ such that each $U_i$ is open in $A_i$, such that $U_i = A_i$ except for $i = i_1, \ldots, i_n$, and 
        such that $a \in \prod_{i \in I} U_i \subseteq U$.}
        \step{c}{For $j = 1, \ldots, n$, \pick\ $B_{i_j} \in \BB_{i_j}$ such that $a_{i_j} \in B_{i_j} \subseteq U_{i_j}$}
        \step{d}{\pflet{$B = \prod_{i \in I} B_i$ where $B_i = A_i$ for $i \neq i_1, \ldots, i_n$}}
        \step{e}{$B \in \BB$}
        \step{f}{$a \in B \subseteq U$}
    \end{proof}
    \qedstep
    \begin{proof}
        \pf\ Lemma \ref{lemma:basis}.
    \end{proof}
    \qed
\end{proof}

\begin{proposition}
    \label{proposition:open_map_projections}
    Let $\{ A_i \}_{i \in I}$ be a family of topological spaces. Then the projections $\pi_i : \prod_{i \in I} A_i \rightarrow A_i$ are open maps.
\end{proposition}

\begin{proof}
    \pf\ From Lemma \ref{lemma:open_map_basis}. \qed
\end{proof}

\begin{proposition}
    Let $\{ X_i \}_{i \in I}$ be a family of sets. For $i \in I$, let $\TT_i$ and $\UU_i$ be topologies
    on $X_i$. Let $\mathcal{P}$ be the product topology on $\prod_{i \in I} X_i$ generated by
    the topologies $\TT_i$, and $\mathcal{Q}$ the product topology on the same set generated by the topologies
    $\UU_i$. Then $\mathcal{P} \subseteq \mathcal{Q}$ if and only if $\TT_i \subseteq \UU_i$ for all $i$.
\end{proposition}

\begin{proof}
    \pf
    \step{2}{If $\TT_i \subseteq \UU_i$ for all $i$ then $\mathcal{P} \subseteq \mathcal{Q}$}
    \begin{proof}
        \pf\ By Corollary \ref{cor:basis_open}.
    \end{proof}
    \step{3}{If $\mathcal{P} \subseteq \mathcal{Q}$ then $\TT_i \subseteq \UU_i$ for all $i$}
    \begin{proof}
        \step{a}{\assume{$\mathcal{P} \subseteq \mathcal{Q}$}}
        \step{b}{\pflet{$i \in I$}}
        \step{c}{\pflet{$U \in \TT_i$}}
        \step{d}{\pflet{$U_i = U$ and $U_j = X_j$ for $j \neq i$}}
        \step{e}{$\prod_{i \in I} U_i \in \mathcal{P}$}
        \step{f}{$\prod_{i \in I} U_i \in \mathcal{Q}$}
        \step{g}{$U \in \UU_i$}
        \begin{proof}
            \pf\ From Proposition \ref{proposition:open_map_projections}.
        \end{proof}
    \end{proof}
    \qed
\end{proof}

\section{The Subspace Topology}

\begin{definition}[Subspace Topology]
    Let $X$ be a topological space and $Y \subseteq X$. The \emph{subspace topology} on $Y$ is
    $\TT = \{ U \cap Y \mid U \text{ is open in } X \}$.
\end{definition}

We prove this is a topology.

\begin{proof}
    \pf
    \step{1}{$Y \in \TT$}
    \begin{proof}
        \pf\ Since $Y = X \cap Y$
    \end{proof}
    \step{2}{For all $\UU \subseteq \TT$, we have $\bigcup \UU \in \TT$}
    \begin{proof}
        \step{a}{\pflet{$\UU \subseteq \TT$}}
        \step{b}{\pflet{$\VV = \{ V \text{ open in } X \mid V \cap Y \in \UU \}$}}
        \step{c}{$\bigcup \UU = \left( \bigcup \VV  \right) \cap Y$}
    \end{proof}
    \step{3}{For all $U, V \in \TT$, we have $U \cap V \in \TT$}
    \begin{proof}
        \step{a}{\pflet{$U, V \in \TT$}}
        \step{b}{\pick\ $U'$, $V'$ open in $X$ such that $U = U' \cap Y$ and $V = V' \cap Y$}
        \step{c}{$(U \cap V) = (U' \cap V') \cap Y$}
    \end{proof}
    \qed
\end{proof}

\begin{lemma}
    \label{lemma:subspace_basis}
    Let $X$ be a topological space and $Y \subseteq X$. Let $\BB$ be a basis for the topology on $X$.
    Then $\BB' = \{ B \cap Y \mid B \in \BB \}$ is a basis for the subspace topology on $Y$.
\end{lemma}

\begin{proof}
    \pf
    \step{1}{Every element in $\BB'$ is open in $Y$}
    \step{2}{For every open set $U$ in $Y$ and point $y \in U$, there exists $B' \in \BB'$ such that
    $y \in B' \subseteq U$}
    \begin{proof}
        \step{a}{\pflet{$U$ be open in $Y$ and $y \in U$}}
        \step{b}{\pick\ $V$ open in $X$ such that $U = V \cap Y$}
        \step{c}{\pick\ $B \in \BB$ such that $y \in B \subseteq V$}
        \step{d}{\pflet{$B' = B \cap Y$}}
        \step{e}{$B' \in \BB'$}
        \step{f}{$y \in B' \subseteq U$}
    \end{proof}
    \qedstep
    \begin{proof}
        \pf\ By Lemma \ref{lemma:basis}.
    \end{proof}
    \qed
\end{proof}

\begin{lemma}
    \label{lemma:subspace_subbasis}
    Let $X$ be a topological space and $Y \subseteq X$. Let $\SS$ be a basis for the topology on $X$.
    Then $\SS' = \{ S \cap Y \mid S \in \SS \}$ is a subbasis for the subspace topology on $Y$.
\end{lemma}

\begin{proof}
    \pf\ The set $\{ B \cap Y \mid B \text{ is a finite intersection of elements of } \SS \}$ is a basis
    for the subspace topology by Lemma \ref{lemma:subspace_basis}, and this is the set of all finite
    intersections of elements of $\SS'$. \qed
\end{proof}

\begin{lemma}
    \label{lemma:subspace_open}
    Let $Y$ be a subspace of $X$. If $U$ is open in $Y$ and $Y$ is open in $X$ then $U$ is open in $X$.
\end{lemma}

\begin{proof}
    \pf
    \step{1}{\pick\ $V$ open in $X$ such that $U = V \cap Y$}
    \step{2}{$U$ is open in $X$}
    \begin{proof}
        \pf\ Since it is the intersection of two open sets $V$ and $Y$.
    \end{proof}
    \qed
\end{proof}

\begin{theorem}
    Let $\{ X_i \}_{i \in I}$ be a family of topological spaces. Let $A_i$ be a subspace of $X_i$ for all $i \in I$.
    Then the product topology on $\prod_{i \in I} A_i$ is the same as the topology it inherits as a subspace of
    $\prod_{i \in I} X_i$.
\end{theorem}

\begin{proof}
    \pf\ The product topology is generated by the subbasis
    \begin{align*}
        & \{ \inv{\pi_i}(U) \mid i \in I, U \text{ open in } A_i \} \\
        = & \{ \inv{\pi_i}(V) \cap A_i \mid i \in I, V \text{ open in } X_i \} \\
        = & \{ \inv{\pi_i}(V) \mid i \in I, V \text{ open in } X_i \} \cap \prod_{i \in I} A_i
    \end{align*}
    and this is a subbasis for the subspace topology by Lemma \ref{lemma:subspace_subbasis}. \qed
\end{proof}

\begin{theorem}
    Let $X$ be an ordered set in the order topology. Let $Y \subseteq X$ be an interval. Then the order topology
    on $Y$ is the same as the subspace topology on $Y$.
\end{theorem}

\begin{proof}
    \pf
    \step{1}{The order topology is finer than the subspace topology.}
    \begin{proof}
        \step{a}{For every open ray $R$ in $X$, the set $R \cap Y$ is open in the order topology.}
        \begin{proof}
            \step{i}{For all $a \in X$, we have $(-\infty, a) \cap Y$ is open in the order topology.}
            \begin{proof}
                \step{one}{\case{For all $y \in Y$ we have $y < a$}}
                \begin{proof}
                    \pf\ In this case $(-\infty, a) \cap Y = Y$.
                \end{proof}
                \step{two}{\case{For all $y \in Y$ we have $a < y$}}
                \begin{proof}
                    \pf\ In this case $(-\infty, a) \cap Y = \emptyset$.
                \end{proof}
                \step{three}{\case{There exists $y \in Y$ such that $y \leq a$ and $y \in Y$ such that
                $a \leq y$}}
                \begin{proof}
                    \step{aa}{$a \in Y$}
                    \begin{proof}
                        \pf\ Because $Y$ is an interval.
                    \end{proof}
                    \step{bb}{$(-\infty, a) \cap Y = \{ y \in Y \mid y < a \}$}
                \end{proof}
            \end{proof}
            \step{ii}{For all $a \in X$, we have $(a, +\infty) \cap Y$ is open in the order topology.}
            \begin{proof}
                \pf\ Similar.
            \end{proof}
        \end{proof}
        \qedstep
        \begin{proof}
            \pf\ By Lemmas \ref{lemma:open_rays_subbasis} and \ref{lemma:subspace_subbasis} and Proposition
            \ref{proposition:subbasis_coarsest}.
        \end{proof}
    \end{proof}
    \step{2}{The subspace topology is finer than the order topology.}
    \begin{proof}
        \step{a}{Every open ray in $Y$ is open in the subspace topology.}
        \begin{proof}
            \pf\ For any $a \in Y$ we have $(-\infty, a)_Y = (-\infty, a)_X \cap Y$
            and $(a, +\infty)_Y = (a,+\infty)_X \cap Y$.
        \end{proof}
        \qedstep
        \begin{proof}
            \pf\ By Lemma \ref{lemma:open_rays_subbasis} and  Proposition
            \ref{proposition:subbasis_coarsest}
        \end{proof}
    \end{proof}
    \qed
\end{proof}

\begin{proposition}
    The order topology on $I_o^2$ is different from the subspace topology as a subspace of $\RR^2$ under
    the dictionary order topology.
\end{proposition}

\begin{proof}
    \pf\ The set $\{ 1/2 \} \times (1/2, 1)$ is open in the subspace topology but not in the order topology.
    \qed
\end{proof}

\begin{proposition}
    Let $X$ be a topological space, $Y$ a subspace of $X$, and $Z$ a subspace of $Y$. Then the subspace
    topology on $Z$ inherited from $X$ is the same as the subspace topology on $Z$ inherited from $Y$.
\end{proposition}

\begin{proof}
    \pf\ The subspace topology inherited from $Y$ is
    \begin{align*}
        & \{ V \cap Z \mid V \text{ open in } Y \} \\
        = & \{ U \cap Y \cap Z \mid U \text{ open in } X \} \\
        = & \{ U \cap Z \mid U \text{ open in } X \}
    \end{align*}
    which is the subspace topology inherited from $X$. \qed
\end{proof}

\begin{definition}[Unit Circle]
    The \emph{unit circle} $S^1$ is $\{ (x,y) \in \RR^2 \mid x^2 + y^2 = 1 \}$
\end{definition}

\section{Closed Set}

\begin{definition}[Closed Set]
    Let $X$ be a topological space and $A \subseteq X$. Then $A$ is \emph{closed} if and only if $X \setminus
    A$ is open.
\end{definition}

\begin{lemma}
    The empty set is closed.
\end{lemma}

\begin{proof}
    \pf\ Since the whole space $X$ is always open. \qed
\end{proof}

\begin{lemma}
    \label{lemma:closed_whole_set}
    The topological space $X$ is closed.
\end{lemma}

\begin{proof}
    \pf\ Since $\emptyset$ is open. \qed
\end{proof}

\begin{lemma}
    \label{lemma:closed_intersection}
    The intersection of a nonempty set of closed sets is closed.
\end{lemma}

\begin{proof}
    \pf\ Let $\mathcal{C}$ be a nonempty set of closed sets. Then $X \setminus \bigcap \mathcal{C}
    = \bigcup \{ X \setminus C \mid C \in \mathcal{C} \}$ is open. \qed
\end{proof}

\begin{lemma}
    \label{lemma:closed_union}
    The union of two closed sets is closed.
\end{lemma}

\begin{proof}
    \pf\ Let $C$ and $D$ be closed. Then $X \setminus (C \cup D) = (X \setminus C) \cap (X \setminus D)$
    is open. \qed
\end{proof}

\begin{theorem}
    \label{theorem:closed}
    Let $X$ be a topological space and $Y$ a subspace of $X$. Let $A \subseteq Y$. Then $A$ is closed in $Y$
    if and only if there exists a closed set $C$ in $X$ such that $A = C \cap Y$.
\end{theorem}

\begin{proof}
    \pf\ 
    We have
    \begin{align*}
        & A \text{ is closed in } Y \\
        \Leftrightarrow & Y \setminus A \text{ is open in } Y \\
        \Leftrightarrow & \exists U \text{ open in } X. Y \setminus A = Y \cap U \\
        \Leftrightarrow & \exists C \text{ closed in } X. Y \setminus A = Y \cap (X \setminus U) \\
        \Leftrightarrow & \exists C \text{ closed in } X. A = Y \cap U & \qed
    \end{align*}
\end{proof}

\begin{theorem}
    \label{theorem:closed_subspace}
    Let $Y$ be a subspace of $X$ and $A \subseteq Y$. If $A$ is closed in $Y$ and $Y$ is closed in $X$
    then $A$ is closed in $X$.
\end{theorem}

\begin{proof}
    \pf\ Pick a closed set $C$ in $X$ such that $A = C \cap Y$ (Theorem \ref{theorem:closed}). Then
    $A$ is the intersection of two sets closed in $X$, hence $A$ is closed in $X$ (Lemma \ref{lemma:closed_intersection}). \qed
\end{proof}

\begin{proposition}
    Let $X$ be a set and $\mathcal{C} \subseteq \pow X$ a set such that:
    \begin{enumerate}
        \item $\emptyset \in \mathcal{C}$
        \item $X \in \mathcal{C}$
        \item For all $\AA \subseteq \mathcal{C}$ nonempty we have $\bigcap \AA \in \mathcal{C}$
        \item For all $C, D \in \mathcal{C}$ we have $C \cup D \in \mathcal{C}$.
    \end{enumerate}
    Then there exists a unique topology $\TT$ such that $\mathcal{C}$ is the set of closed sets, namely
    \[ \TT = \{ X \setminus C \mid C \in \mathcal{C} \} \]
\end{proposition}

\begin{proof}
    \pf
    \step{1}{\pflet{$\TT = \{ X \setminus C \mid C \in \mathcal{C} \}$}}
    \step{2}{$\TT$ is a topology}
    \begin{proof}
        \step{a}{$X \in \TT$}
        \begin{proof}
            \pf\ Since $\emptyset \in \mathcal{C}$
        \end{proof}
        \step{b}{For all $\UU \subseteq \TT$ we have $\bigcup \UU \in \TT$}
        \begin{proof}
            \step{i}{\pflet{$\UU \subseteq \TT$}}
            \step{ii}{\case{$\UU = \emptyset$}}
            \begin{proof}
                \pf\ In this case $\bigcup \UU = \emptyset \in \TT$ since $X \in \mathcal{C}$
            \end{proof}
            \step{iii}{\case{$\UU \neq \emptyset$}}
            \begin{proof}
                \pf\ In this case $X \setminus \bigcup \UU = \bigcap \{ X \setminus U \mid U \in \UU \} \in
                \mathcal{C}$.
            \end{proof}
        \end{proof}
        \step{c}{For all $U, V \in \TT$ we have $U \cap V \in \TT$}
        \begin{proof}
            \pf\ Since $X \setminus (U \cap V) = (X \setminus U) \cup (X \setminus V) \in \mathcal{C}$.
        \end{proof}
    \end{proof}
    \step{3}{$\mathcal{C}$ is the set of all closed sets in $\TT$}
    \begin{proof}
        \pf
        \begin{align*}
            & C \text{ is closed in } \TT \\
            \Leftrightarrow & X \setminus C \in \TT \\
            \Leftrightarrow & C \in \mathcal{C}
        \end{align*}
    \end{proof}
    \step{4}{If $\TT'$ is a topology and $\mathcal{C}$ is the set of closed sets in $\TT'$ then
    $\TT' = \TT$}
    \begin{proof}
        \pf\ 
        We have
        \begin{align*}
            & U \in \TT \\
            \Leftrightarrow & X \setminus U \in \mathcal{C} \\
            \Leftrightarrow & X \setminus U \text{ is closed in } \TT' \\
            \Leftrightarrow & U \in \TT'
        \end{align*}
    \end{proof}
    \qed
\end{proof}

\begin{proposition}
    \label{proposition:closed_product}
    If $A_i$ is closed in $X_i$ for all $i \in I$ then $\prod_{i \in I} A_i$ is closed in $\prod_{i \in I} X_i$.
\end{proposition}

\begin{proof}
    \pf
    \[ (\prod_{i \in I} X_i) \setminus (\prod_{i \in I} A_i) = \bigcup_{j \in I} \left( \prod_{i \in I} X_i \setminus \inv{\pi_j}(A_j) \right) \qed \]
\end{proof}

\begin{proposition}
    If $U$ is open and $A$ is closed then $U \setminus A$ is open.
\end{proposition}

\begin{proof}
    \pf\ $U \setminus A = U \cap (X \setminus A)$ is the intersection of two open sets. \qed
\end{proof}

\begin{proposition}
    If $U$ is open and $A$ is closed then $A \setminus U$ is closed.
\end{proposition}

\begin{proof}
    \pf\ $A \setminus U = A \cap (X \setminus U)$ is the intersection of two closed sets. \qed
\end{proof}

\section{Interior}

\begin{definition}[Interior]
    Let $X$ be a topological space and $A \subseteq X$. The \emph{interior} of $A$, $\Int A$, is the union
    of all the open subsets of $A$.
\end{definition}

\begin{lemma}
    \label{lemma:interior_open}
    The interior of a set is open.
\end{lemma}

\begin{proof}
    \pf\ It is a union of open sets. \qed
\end{proof}

\begin{lemma}
    \[ \Int A \subseteq A \]
\end{lemma}

\begin{proof}
    \pf\ Immediate from definition. \qed
\end{proof}

\begin{lemma}
    \label{lemma:open_interior}
    A set $A$ is open if and only if $A = \Int A$.
\end{lemma}

\begin{proof}
    \pf\ If $A = \Int A$ then $A$ is open by Lemma \ref{lemma:interior_open}. Conversely if
    $A$ is open then $A \subseteq \Int A$ by the definition of interior and so $A = \Int A$.
\end{proof}

\section{Neighbourhood}

\begin{definition}[Neighbourhood]
    A \emph{neighbourhood} of a point $x$ is an open set that contains $x$.
\end{definition}

\section{Closure}

\begin{definition}[Closure]
    Let $X$ be a topological space and $A \subseteq X$. The \emph{closure} of $A$, $\overline{A}$, is the
    intersection of all the closed sets that include $A$.
\end{definition}

This intersection exists since $X$ is a closed set that includes A (Lemma \ref{lemma:closed_whole_set}).

\begin{lemma}
    The closure of a set is closed.
\end{lemma}

\begin{proof}
    \pf\ Dual to Lemma \ref{lemma:interior_open}. \qed
\end{proof}

\begin{lemma}
    \label{lemma:closure_subset}
    \[ A \subseteq \overline{A} \]
\end{lemma}

\begin{proof}
    \pf\ Immediate from definition. \qed
\end{proof}

\begin{lemma}
    A set $A$ is closed if and only if $A = \overline{A}$.
\end{lemma}

\begin{proof}
    \pf\ Dual to Lemma \ref{lemma:open_interior}. \qed
\end{proof}

\begin{theorem}
    Let $Y$ be a subspace of $X$. Let $A \subseteq Y$. Let $\overline{A}$ be the closure of $A$ in $X$.
    Then the closure of $A$ in $Y$ is $\overline{A} \cap Y$.
\end{theorem}

\begin{proof}
    \pf\ The closure of $A$ in $Y$ is
    \begin{align*}
        & \bigcap \{ C \text{ closed in } Y \mid A \subseteq C \} \\
        = & \bigcap \{ D \cap Y \mid D \text{ closed in } X, A \subseteq D \cap Y \} 
            & (\text{Theorem \ref{theorem:closed}})\\
        = & \bigcap \{ D \mid D \text{ closed in } X, A \subseteq D \} \cap Y \\
        = & \overline{A} \cap Y & \qed
    \end{align*}
\end{proof}

\begin{theorem}
    \label{theorem:closure_neighbourhood}
    Let $X$ be a topological space, $A \subseteq X$ and $x \in X$. Then $x \in \overline{A}$ if and only if
    every neighbourhood of $x$ intersects $A$.
\end{theorem}

\begin{proof}
    \pf\ We have
    \begin{align*}
        & x \in \overline{A} \\
        \Leftrightarrow & \forall C. C \text{ closed } \wedge A \subseteq C \Rightarrow x \in C \\
        \Leftrightarrow & \forall U. U \text{ open } \wedge A \cap U = \emptyset \Rightarrow x \notin U \\
        \Leftrightarrow & \forall U. U \text{ open } \wedge x \in U \Rightarrow U \text{ intersects } A & \qed
    \end{align*}
\end{proof}

\begin{theorem}
    Let $X$ be a topological space, $A \subseteq X$ and $x \in X$. Let $\BB$ be a basis for $X$. Then $x \in 
    \overline{A}$ if and only if, for all $B \in \BB$, if $x \in B$ then $B$ intersects $A$.
\end{theorem}

\begin{proof}
    \pf
    \step{1}{If $x \in \overline{A}$ then, for all $B \in \BB$, if $x \in B$ then $B$ intersects $A$.}
    \begin{proof}
        \pf\ This follows from Theorem \ref{theorem:closure_neighbourhood} since every element of $\BB$ is open
        (Corollary \ref{cor:basis_open}).
    \end{proof}
    \step{2}{Suppose that, for all $B \in \BB$, if $x \in B$ then $B$ intersects $A$. Then $x \in \overline{A}$.}
    \begin{proof}
        \step{a}{\assume{For all $B \in \BB$, if $x \in B$ then $B$ intersects $A$.}}
        \step{b}{\pflet{$U$ be an open set that contains $x$} \prove{$U$ intersects $A$.}}
        \step{c}{\pick\ $B \in \BB$ such that $x \in B \subseteq U$.}
        \step{d}{$B$ intersects $A$.}
        \begin{proof}
            \pf\ From \stepref{a}.
        \end{proof}
        \step{e}{$U$ intersects $A$.}
        \qedstep
        \begin{proof}
            \pf\ By Theorem \ref{theorem:closure_neighbourhood}.
        \end{proof}
    \end{proof}
    \qed
\end{proof}

\begin{proposition}
    \label{proposition:closure_monotone}
    If $A \subseteq B$ then $\overline{A} \subseteq \overline{B}$.
\end{proposition}

\begin{proof}
    \pf\ This holds because $\overline{B}$ is a closed set that includes $A$. \qed
\end{proof}

\begin{proposition}
    \[ \overline{A \cup B} = \overline{A} \cup \overline{B} \]
\end{proposition}

\begin{proof}
    \pf
    \step{1}{$\overline{A} \subseteq \overline{A \cup B}$}
    \begin{proof}
        \pf\ By Proposition \ref{proposition:closure_monotone}.
    \end{proof}
    \step{2}{$\overline{B} \subseteq \overline{A \cup B}$}
    \begin{proof}
        \pf\ By Proposition \ref{proposition:closure_monotone}.
    \end{proof}
    \step{3}{$\overline{A \cup B} \subseteq \overline{A} \cup \overline{B}$}
    \begin{proof}
        \step{a}{\pflet{$x \in \overline{A \cup B}$}}
        \step{b}{\assume{$x \notin \overline{A}$} \prove{$x \in \overline{B}$}}
        \step{c}{\pick\ a neighbourhood $U$ of $x$ that does not intersect $A$}
        \step{c}{\pflet{$V$ be any neighbourhood of $x$}}
        \step{d}{$U \cap V$ is a neighbourhood of $x$}
        \step{e}{$U \cap V$ intersects $A \cup B$}
        \begin{proof}
            \pf\ From \stepref{a} and Theorem \ref{theorem:closure_neighbourhood}.
        \end{proof}
        \step{f}{$U \cap V$ intersects $B$}
        \begin{proof}
            \pf\ From \stepref{c}
        \end{proof}
        \step{g}{$V$ intersects $B$}
        \qedstep
        \begin{proof}
            \pf\ We have $x \in \overline{B}$ from Theorem \ref{theorem:closure_neighbourhood}.
        \end{proof}
    \end{proof}
    \qed
\end{proof}

\begin{proposition}[AC]
    Let $\{ X_i \}_{i \in I}$ be a family of topological spaces. Let $A_i \subseteq X_i$ for all $i \in I$.
    Then
    \[ \prod_{i \in I} \overline{A_i} = \overline{\prod_{i \in I} A_i} \enspace . \]
\end{proposition}

\begin{proof}
    \pf
    \step{1}{$\overline{\prod_{i \in I} A_i} \subseteq \prod_{i \in I} \overline{A_i}$}
    \begin{proof}
        \step{a}{For all $i \in I$ we have $A_i \subseteq \overline{A_i}$}
        \begin{proof}
            \pf\ Lemma \ref{lemma:closure_subset}.
        \end{proof}
        \step{c}{$\prod_{i \in I} A_i \subseteq \prod_{i \in I} \overline{A_i}$}
        \qedstep
        \begin{proof}
            \pf\ Since $\prod_{i \in I} A_i$ is closed by Proposition \ref{proposition:closed_product}.
        \end{proof}
    \end{proof}
    \step{2}{$\prod_{i \in I} \overline{A_i} \subseteq \overline{\prod_{i \in I} A_i}$}
    \begin{proof}
        \step{a}{\pflet{$x \in \prod_{i \in I} \overline{A_i}$}}
        \step{b}{\pflet{$U$ be a neighbourhood of $x$}}
        \step{c}{\pick\ $V_i$ open in $X_i$ such that $x \in \prod_{i \in I} V_i \subseteq U$ with $V_i = X_i$ except for $i = i_1, \ldots, i_n$}
        \step{d}{For $i \in I$, pick $a_i \in V_i \cap A_i$}
        \begin{proof}
            \pf\ By Theorem \ref{theorem:closure_neighbourhood} and \stepref{a} using the Axiom of Choice.
        \end{proof}
        \step{f}{$U$ intersects $\prod_{i \in I} A_i$}
        \qedstep
        \begin{proof}
            \pf\ $a \in U \cap \prod_{i \in I} A_i$
        \end{proof}
    \end{proof}
    \qed
\end{proof}

\begin{example}
    The closure of $\RR^\infty$ in $\RR^\omega$ is $\RR^\omega$
\end{example}

\begin{proof}
    \pf
    \step{1}{\pflet{$a \in \RR^\omega$}}
    \step{2}{\pflet{$U$ be any neighbourhoods of $a$.}}
    \step{3}{\pick\ $U_n$ open in $\RR$ for all $n$ such that $a \in \prod_{n \geq 0} U_n \subseteq U$ and $U_n = \RR$ for all $n$ except $n_1$, \ldots, $n_k$}
    \step{4}{\pflet{$b_n = a_n$ for $n = n_1, \ldots, n_k$ and $b_n = 0$ for all other $n$}}
    \step{5}{$b \in \RR^\infty \cap U$}
    \qedstep
    \begin{proof}
        \pf\ From Theorem \ref{theorem:closure_neighbourhood}.
    \end{proof}
    \qed
\end{proof}

\section{Limit Points}

\begin{definition}[Limit Point]
    Let $X$ be a topological space, $a \in X$ and $A \subseteq X$. Then $a$ is a \emph{limit point},
    \emph{cluster point} or \emph{point of accumulation} for $A$ if and only if every neighbourhood of $a$
    intersects $A$ at a point other than $a$.
\end{definition}

\begin{lemma}
    The point $a$ is an accumulation point for $A$ if and only if $a \in \overline{A \setminus \{a\}}$.
\end{lemma}

\begin{proof}
    \pf\ From Theorem \ref{theorem:closure_neighbourhood}. \qed
\end{proof}

\begin{theorem}
    Let $X$ be a topological space and $A \subseteq X$. Let $A'$ be the set of all limit points of $A$.
    Then $\overline{A} = A \cup A'$.
\end{theorem}

\begin{proof}
    \pf
    \step{1}{For all $x \in \overline{A}$, if $x \notin A$ then $x \in A'$}
    \begin{proof}
        \pf\ From Theorem \ref{theorem:closure_neighbourhood}.
    \end{proof}
    \step{2}{$A \subseteq \overline{A}$}
    \begin{proof}
        \pf\ Lemma \ref{lemma:closure_subset}.
    \end{proof}
    \step{3}{$A' \subseteq \overline{A}$}
    \begin{proof}
        \pf\ From Theorem \ref{theorem:closure_neighbourhood}.
    \end{proof}
    \qed
\end{proof}

\begin{corollary}
    A set is closed if and only if it contains all its limit points.
\end{corollary}

\begin{proposition}
    \label{proposition:indiscrete_limit_point}
    In an indiscrete topology, every point is a limit point of any set that has more than one point.
\end{proposition}

\begin{proof}
    \pf\ Let $X$ be an indiscrete space. Let $A$ be a set with more than one point and $x$ be a point.
    The only neighbourhood of $x$ is $X$, which must intersect $A$ at a point other than $x$. \qed
\end{proof}

\section{$T_1$ Spaces}

\begin{definition}[$T_1$ Space]
    A topological space is $T_1$ if and only if every singleton is closed.
\end{definition}

\begin{lemma}
    A space is $T_1$ if and only if every finite set is closed.
\end{lemma}

\begin{proof}
    \pf\ From Lemma \ref{lemma:closed_union}. \qed
\end{proof}

\begin{theorem}
    In a $T_1$ space, a point $a$ is a limit point of a set $A$ if and only if every neighbourhood of $a$
    contains infinitely many points of $A$.
\end{theorem}

\begin{proof}
    \pf
    \step{1}{If $a$ is a limit point of $A$ then every neighbourhood of $a$ contains infinitely many points
    of $A$.}
    \begin{proof}
        \step{a}{\assume{$a$ is a limit point of $A$.}}
        \step{b}{\pflet{$U$ be a neighbourhood of $a$.}}
        \step{c}{\assume{for a contradiction $U$ contains only finitely many points of $A$.}}
        \step{d}{$(U \cap A) \setminus \{a\}$ is closed.}
        \begin{proof}
            \pf\ By the $T_1$ axiom.
        \end{proof}
        \step{e}{$(U \setminus A) \cup \{a\}$ is open.}
        \begin{proof}
            \pf\ It is $U \setminus ((U \cap A) \setminus \{a\})$.
        \end{proof}
        \step{f}{$(U \setminus A) \cup \{a\}$ intersects $A$ in a point other than $a$.}
        \begin{proof}
            \pf\ From \stepref{a}.
        \end{proof}
        \qedstep
        \qed
    \end{proof}
    \step{2}{If every neighbourhood of $a$ contains infinitely many points of $A$ then $a$ is a limit point
    of $A$.}
    \begin{proof}
        \pf\ Immediate from definitions.
    \end{proof}
    \qed
\end{proof}

(To see this does not hold in every space, see Proposition \ref{proposition:indiscrete_limit_point}.)

\begin{proposition}
    A space is $T_1$ if and only if, for any two distinct points $x$ and $y$, there exist neighbourhoods
    $U$ of $x$ and $V$ of $y$ such that $x \notin V$ and $y \notin U$.
\end{proposition}

\begin{proof}
    \pf
    \step{1}{\pflet{$X$ be a topological space.}}
    \step{2}{If $X$ is $T_1$ then, for any two distinct points $x$ and $y$, there exist neighbourhoods
    $U$ of $x$ and $V$ of $y$ such that $x \notin V$ and $y \notin U$.}
    \begin{proof}
        \pf\ This holds because $\{x\}$ and $\{y\}$ are closed.
    \end{proof}
    \step{3}{Suppose, for any two distinct points $x$ and $y$, there exist neighbourhoods
    $U$ of $x$ and $V$ of $y$ such that $x \notin V$ and $y \notin U$. Then $X$ is $T_1$.}
    \begin{proof}
        \step{a}{\assume{For any two distinct points $x$ and $y$, there exist neighbourhoods
        $U$ of $x$ and $V$ of $y$ such that $x \notin V$ and $y \notin U$.}}
        \step{b}{\pflet{$a \in X$}}
        \step{c}{$\{a\}$ is closed.}
        \begin{proof}
            \pf\ For all $b \neq a$ there exists a neighbourhood $U$ of $b$ such that $U \subseteq X \setminus \{a\}$.
        \end{proof}
    \end{proof}
    \qed
\end{proof}

\section{Hausdorff Spaces}

\begin{definition}[Hausdorff Space]
    A topological space is \emph{Hausdorff} if and only if, for any points $x$, $y$ with $x \neq y$,
    there exist disjoint open sets $U$ and $V$ such that $x \in U$ and $y \in V$.
\end{definition}

\begin{theorem}
    Every Hausdorff space is $T_1$.
\end{theorem}

\begin{proof}
    \pf
    \step{0}{\pflet{$X$ be a Hausdorff space.}}
    \step{1}{\pflet{$b \in X$} \prove{$\overline{\{b\}} = \{b\}$}}
    \step{2}{\assume{$a \in \overline{\{b\}}$ and $a \neq b$}}
    \step{3}{\pick\ disjoint neighbourhoods $U$ of $a$ and $V$ of $b$.}
    \step{4}{$U$ intersects $\{b\}$}
    \begin{proof}
        \pf\ Theorem \ref{theorem:closure_neighbourhood}.
    \end{proof}
    \step{5}{$b \in U$}
    \qedstep
    \begin{proof}
        \pf\ This contradicts the fact that $U$ and $V$ are disjoint (\stepref{3}).
    \end{proof}
    \qed
\end{proof}

\begin{proposition}
    An infinite set under the finite complement topology is $T_1$ but not Hausdorff.
\end{proposition}

\begin{proof}
    \pf
    \step{0}{\pflet{$X$ be an infinite set under the finite complement topology.}}
    \step{1}{Every singleton is closed.}
    \begin{proof}
        \pf\ By definition.
    \end{proof}
    \step{2}{\pick{$a, b \in X$ with $a \neq b$}}
    \step{3}{There are no disjoint neighbourhoods $U$ of $a$ and $V$ of $b$.}
    \begin{proof}
        \step{a}{\pflet{$U$ be a neighbourhood of $a$ and $V$ a neighbourhood of $b$.}}
        \step{b}{$X \setminus U$ and $X \setminus V$ are finite.}
        \step{c}{\pick\ $c \in X$ that is not in $X \setminus U$ or $X \setminus V$.}
        \step{d}{$c \in U \cap V$}
    \end{proof}
    \qed
\end{proof}

\begin{proposition}
    The product of a family of Hausdorff spaces is Hausdorff.
\end{proposition}

\begin{proof}
    \pf
    \step{1}{\pflet{$\{ X_i \}_{i \in I}$ be a family of Hausdorff spaces.}}
    \step{2}{\pflet{$a, b \in \prod_{i \in I} X_i$ with $a \neq b$}}
    \step{3}{\pick\ $i \in I$ such that $a_i \neq b_i$}
    \step{4}{\pick\ $U$, $V$ disjoint open sets in $X_i$ with $a_i \in U$ and $b_i \in V$}
    \step{5}{$\inv{\pi_i}(U)$ and $\inv{\pi_i}(V)$ are disjoint open sets in $\prod_{i \in I} X_i$ with $a \in \inv{\pi_i}(U)$
    and $b \in \inv{\pi_i}(V)$}
    \qed
\end{proof}

\begin{theorem}
    Every linearly ordered set under the order topology is Hausdorff.
\end{theorem}

\begin{proof}
    \pf
    \step{1}{\pflet{$X$ be a linearly ordered set under the order topology.}}
    \step{2}{\pflet{$a, b \in X$ with $a \neq b$}}
    \step{3}{\assume{w.l.o.g.~$a < b$}}
    \step{4}{\case{There exists $c$ such that $a < c < b$}}
    \begin{proof}
        \pf\ The sets $(-\infty,c)$ and $(c,+\infty)$ are disjoint neighbourhoods of $a$ and $b$
        respectively.
    \end{proof}
    \step{5}{\case{There is no $c$ such that $a < c < b$}}
    \begin{proof}
        \pf\ The sets $(-\infty, b)$ and $(a,+\infty)$ are disjoint neighbourhoods of $a$ and $b$
        respectively.
    \end{proof}
    \qed
\end{proof}

\begin{theorem}
    A subspace of a Hausdorff space is Hausdorff.
\end{theorem}

\begin{proof}
    \pf
    \step{1}{\pflet{$X$ be a Hausdorff space and $Y$ a subspace of $X$.}}
    \step{2}{\pflet{$x, y \in Y$ with $x \neq y$}}
    \step{3}{\pick\ disjoint neighbourhoods $U$ of $x$ and $V$ of $y$ in $X$.}
    \step{4}{$U \cap Y$ and $V \cap Y$ are disjoint neighbourhoods of $x$ and $y$ respectively in $Y$.}
    \qed
\end{proof}

\begin{proposition}
    A space $X$ is Hausdorff if and only if the diagonal $\Delta = \{ (x,x) \mid x \in X \}$ is closed in $X^2$.
\end{proposition}

\begin{proof}
    \pf
    \begin{align*}
        & X \text{ is Hausdorff} \\
        \Leftrightarrow & \forall x,y \in X. x \neq y \Rightarrow \exists V, W \text{ open}. x \in V \wedge y \in W \wedge V \cap W = \emptyset \\
        \Leftrightarrow & \forall (x,y) \in X^2 \setminus \Delta. \exists V,W \text{ open}. (x,y) \subseteq V \times W \subseteq X^2 \setminus \Delta \\
        \Leftrightarrow & \Delta \text{ is closed} & \qed
    \end{align*}
\end{proof}

\section{Convergence}

\begin{definition}[Convergence]
    Let $X$ be a topological space. Let $(a_n)_{n \in \NN}$ be a sequence of points in $X$ and $l \in X$.
    Then the sequence $(a_n)_{n \in \NN}$ \emph{converges} to the \emph{limit} $l$, $a_n \rightarrow l$ as $n \rightarrow
    \infty$, if and only if, for every neighbourhood $U$ of $l$, there exists $N$ such that, for all
    $n \geq N$, we have $a_n \in U$.
\end{definition}

\begin{theorem}
    In a Hausdorff space, a sequence has at most one limit.
\end{theorem}

\begin{proof}
    \pf
    \step{1}{\pflet{$X$ be a Hausdorff space.}}
    \step{2}{\assume{for a contradiction $a_n \rightarrow l$ as $n \rightarrow \infty$, $a_n \rightarrow
    m$ as $n \rightarrow \infty$, and $l \neq m$}}
    \step{3}{\pick\ disjoint neighbourhoods $U$ of $l$ and $V$ of $m$}
    \begin{proof}
        \pf\ By the Hausdorff axiom.
    \end{proof}
    \step{4}{\pick\ $M$ and $N$ such that $a_n \in U$ for $n \geq M$ and $a_n \in V$ for $n \geq N$}
    \step{5}{$a_{\max(M,N)} \in U \cap V$}
    \qedstep
    \begin{proof}
        \pf\ This contradicts the fact that $U$ and $V$ are disjoint (\stepref{3}).
    \end{proof}
    \qed
\end{proof}

To see this is not always true in spaces that are $T_1$ but not Hausdorff:

\begin{proposition}
    Let $X$ be an infinite set under the finite complement topology. Let $(a_n)_{n \in \NN}$ be a sequence
    with all points distinct. Then for every $l \in X$ we have $a_n \rightarrow l$ as
    $n \rightarrow \infty$.
\end{proposition}

\begin{proof}
    \pf\ Let $U$ be any neighbourhood of $l$. Since $X \setminus U$ is finite, there must exist $N$
    such that, for all $n \geq N$, we have $a_n \in U$. \qed
\end{proof}

\begin{lemma}[Sequence Lemma]
    Let $X$ be a topological space. Let $A \subseteq X$ and $l \in X$. If there is a sequence of points in $A$ that converges to $l$ then $l \in \overline{A}$.
\end{lemma}

\begin{proof}
    \pf
    \step{1}{\pflet{$(a_n)$ be a sequence of points in $A$ that converges to $l$.}}
    \step{2}{\pflet{$U$ be a neighbourhood of $l$.}}
    \step{3}{\pick\ $N$ such that, for all $n \geq N$, we have $a_n \in U$.}
    \step{4}{$a_N \in U \cap A$}
    \qedstep
    \begin{proof}
        \pf\ Theorem \ref{theorem:closure_neighbourhood}.
    \end{proof}
    \qed
\end{proof}

\begin{proposition}
    \label{proposition:convergence_basis}
    Let $X$ be a topological space. Let $\BB$ be a basis for the topology on $X$.
    Let $(a_n)$ be a sequence in $X$ and $l \in X$. Then $a_n \rightarrow l$ as
    $n \rightarrow \infty$ if and only if, for every $B \in \BB$ with $l \in B$,
    there exists $N$ such that, for all $n \geq N$, we have $a_n \in B$.
\end{proposition}

\begin{proof}
    \pf
    \step{1}{If $a_n \rightarrow l$ as
    $n \rightarrow \infty$ then, for every $B \in \BB$ with $l \in B$,
    there exists $N$ such that, for all $n \geq N$, we have $a_n \in B$.}
    \begin{proof}
        \pf\ Since every element of $\BB$ is open (Corollary \ref{cor:basis_open}).
    \end{proof}
    \step{2}{If, for every $B \in \BB$ with $l \in B$,
    there exists $N$ such that, for all $n \geq N$, we have $a_n \in B$,
    then $a_n \rightarrow l$ as $n \rightarrow \infty$.}
    \begin{proof}
        \step{a}{\assume{for every $B \in \BB$ with $l \in B$, there exists $N$ such that, for all $n \geq N$, we have $a_n \in B$}}
        \step{b}{\pflet{$U$ be a neighbourhood of $l$.}}
        \step{c}{\pick\ $B \in \BB$ such that $l \in B \subseteq U$}
        \step{d}{\pick\ $N$ such that, for all $n \geq N$, we have $a_n \in B$}
        \begin{proof}
            \pf\ From \stepref{a}.
        \end{proof}
        \step{e}{For all $n \geq N$ we have $a_n \in U$}
    \end{proof}
    \qed
\end{proof}

\begin{lemma}
    \label{lemma:converge_constant}
    If a sequence $(a_n)$ is constant with $a_n = l$ for all $n$, then $a_n \rightarrow l$ as $n \rightarrow \infty$.
\end{lemma}

\begin{proof}
    \pf\ Immediate from definitions. \qed
\end{proof}

\begin{theorem}
    Let $X$ be a linearly ordered set. Let $(s_n)$ be a monotone increasing sequence in $X$ with a supremum $s$.
    Then $s_n \rightarrow s$ as $n \rightarrow \infty$.
\end{theorem}

\begin{proof}
    \pf
    \step{1}{\assume{$s$ is not least in $X$.}}
    \begin{proof}
        \pf\ Otherwise $(s_n)$ is the constant sequence $s$ and the result follows from Lemma \ref{lemma:converge_constant}.
    \end{proof}
    \step{2}{\pflet{$U$ be a neighbourhood of $s$.}}
    \step{3}{\pick{$a < s$ such that $(a,s] \subseteq U$}}
    \step{4}{\pick\ $N$ such that $a < a_N$.}
    \step{5}{For all $n \geq N$ we have $a_n \in (a,s]$}
    \step{6}{For all $n \geq N$ we have $a_n \in U$.}
    \qed
\end{proof}

\begin{theorem}
    If $\sum_{i=0}^\infty a_i = s$ and $\sum_{i=0}^\infty b_i = t$ then $\sum_{i=0}^\infty (ca_i + b_i) = cs+t$.
\end{theorem}

\begin{proof}
    \pf $\sum_{i=0}^N (ca_i + b_i) = c \sum_{i=0}^N a_i + \sum_{i=0}^N b_i \rightarrow cs+t$ as $n \rightarrow \infty$. \qed
\end{proof}

\begin{theorem}[Comparison Test]
    If $|a_i| \leq b_i$ for all $i$ and $\sum_{i=0}^\infty b_i$ converges then $\sum_{i=0}^\infty a_i$ converges.
\end{theorem}

\begin{proof}
    \pf
    \step{1}{$\sum_{i=0}^\infty |a_i|$ converges}
    \begin{proof}
        \pf\ The partial sums $\sum_{i=0}^N |a_i|$ form an increasing sequence bounded above by $\sum_{i=0}^\infty b_i$.
    \end{proof}
    \step{2}{\pflet{$c_i = |a_i| + a_i$ for all $i$}}
    \step{3}{$\sum_{i=0}^\infty c_i$ converges}
    \begin{proof}
        \pf\ Each $c_i$ is either $2|a_i|$ or 0. So the partial sums $\sum_{i=0}^N c_i$ form an increasing sequence bounded above by $2 \sum_{i=0}^\infty b_i$.
    \end{proof}
    \qedstep
    \begin{proof}
        \pf\ Since $a_i = c_i - |a_i|$.
    \end{proof}
    \qed
\end{proof}

\begin{corollary}
    If $\sum_{i=0}^\infty |a_i|$ converges then $\sum_{i=0}^\infty a_i$ converges.
\end{corollary}

\begin{theorem}[Weierstrass $M$-test]
    Let $X$ be a set and $(f_n : X \rightarrow \RR)$ be a sequence of functions. Let
    \[ s_n(x) = \sum_{i=0}^n f_i(x) \]
    for all $n$, $x$. Suppose $|f_i(x)| \leq M_i$ for all $i \geq 0$ and $x \in X$.
    If the series $\sum_{i=0}^\infty M_i$ converges, then the sequence $(s_n)$ converges uniformly to
    \[ s(x) = \sum_{i=0}^\infty f_i(x) \enspace . \]
\end{theorem}

\begin{proof}
    \pf
    \step{1}{\pflet{$r_n = \sum_{i=n+1}^\infty M_i$ for all $n$}}
    \step{2}{Given $0 \leq n < k$, we have $|s_k(x) - s_n(x)| \leq r_n$}
    \begin{proof}
        \pf
        \begin{align*}
            |s_k(x) - s_n(x)| & = |\sum_{i=n+1}^k f_i(x)| \\
            & \leq \sum_{i=n+1}^k |f_i(x)| \\
            & \leq \sum_{i=n+1}^k M_i \\
            & \leq r_n
        \end{align*}
    \end{proof}
    \step{3}{Given $n \geq 0$ we have $|s(x) - s_n(x)| \leq r_n$}
    \begin{proof}
        \pf\ By taking the limit $k \rightarrow \infty$ in \stepref{2}.
    \end{proof}
    \qedstep
    \begin{proof}
        \pf\ Since $r_n \rightarrow 0$ as $n \rightarrow \infty$.
    \end{proof}
    \qed
\end{proof}

\section{Boundary}

\begin{definition}[Boundary]
    The \emph{boundary} of a set $A$ is the set $\partial A = \overline{A} \cap \overline{X \setminus A}$.
\end{definition}

\begin{proposition}
    \label{proposition:int_partial_disjoint}
    \[ \Int A \cap \partial A = \emptyset \]
\end{proposition}

\begin{proof}
    \pf\ Since $\overline{X \setminus A} = X \setminus \Int A$. \qed
\end{proof}

\begin{proposition}
    \label{proposition:closure_int_boundary}
    \[ \overline{A} = \Int A \cup \partial A \]
\end{proposition}

\begin{proof}
    \pf
    \begin{align*}
        \Int A \cup \partial A & = \Int A \cup (\overline{A} \cap \overline{X \setminus A}) \\
        & = (\Int A \cup \overline{A}) \cap (\Int A \cup \overline{X \setminus A}) \\
        & = \overline{A} \cap X \\
        & = \overline{A}
    \end{align*}
\end{proof}

\begin{proposition}
    $\partial A = \emptyset$ if and only if $A$ is open and closed.
\end{proposition}

\begin{proof}
    \pf\ If $\partial A = \emptyset$ then $\overline{A} = \Int A$ by Proposition \ref{proposition:closure_int_boundary}.
\end{proof}

\begin{proposition}
    A set $U$ is open if and only if $\partial U = \overline{U} \setminus U$.
\end{proposition}

\begin{proof}
    \pf
    \begin{align*}
        & \partial U = \overline{U} \setminus U \\
        \Leftrightarrow & \overline{U} \setminus \Int U = \overline{U} \setminus U & (\text{Propositions \ref{proposition:int_partial_disjoint}, \ref{proposition:closure_int_boundary}})\\
        \Leftrightarrow & \Int U = U & \qed
    \end{align*}
\end{proof}

\section{Continuous Functions}

\begin{definition}[Continuous]
    Let $X$ and $Y$ be topological spaces. A function $f : X \rightarrow Y$ is \emph{continuous} if and only
    if, for every open set $V$ in $Y$, the set $\inv{f}(V)$ is open in $X$.
\end{definition}

\begin{proposition}
    \label{proposition:continuous_basis}
    Let $X$ and $Y$ be topological spaces and $f : X \rightarrow Y$. Let $\BB$ be a basis for $Y$. Then $f$
    is continuous if and only if, for all $B \in \BB$, we have $\inv{f}(B)$ is open in $X$.
\end{proposition}

\begin{proof}
    \pf
    \step{1}{If $f$ is continuous then, for all $B \in \BB$, we have $\inv{f}(B)$ is open in $X$.}
    \begin{proof}
        \pf\ Since every element of $B$ is open (Lemma \ref{lemma:basis_unions}).
    \end{proof}
    \step{2}{Suppose that, for all $B \in \BB$, we have $\inv{f}(B)$ is open in $X$. Then $f$ is continuous.}
    \begin{proof}
        \step{a}{\assume{For all $B \in \BB$, we have $\inv{f}(B)$ is open in $X$.}}
        \step{b}{\pflet{$V$ be open in $Y$.}}
        \step{c}{\pick\ $\AA \subseteq \BB$ such that $V = \bigcup \AA$}
        \begin{proof}
            \pf\ By Lemma \ref{lemma:basis_unions}.
        \end{proof}
        \step{d}{$\inv{f}(V)$ is open in $X$.}
        \begin{proof}
            \pf
            \begin{align*}
                \inv{f}(V) & = \inv{f} \left( \bigcup \AA \right) \\
                & = \bigcup_{B \in \AA} \inv{f}(B)
            \end{align*}
        \end{proof}
    \end{proof}
    \qed
\end{proof}

\begin{proposition}
    \label{proposition:continuous_subbasis}
    Let $X$ and $Y$ be topological spaces and $f : X \rightarrow Y$. Let $\SS$ be a subbasis for $Y$. Then $f$
    is continuous if and only if, for all $S \in \SS$, we have $\inv{f}(S)$ is open in $X$.
\end{proposition}

\begin{proof}
    \pf
    \step{1}{If $f$ is continuous then, for all $S \in \SS$, we have $\inv{f}(S)$ is open in $X$.}
    \begin{proof}
        \pf\ Since every element of $S$ is open.
    \end{proof}
    \step{2}{Suppose that, for all $S \in \SS$, we have $\inv{f}(S)$ is open in $X$. Then $f$ is continuous.}
    \begin{proof}
        \step{a}{\assume{For all $S \in \SS$, we have $\inv{f}(S)$ is open in $X$.}}
        \step{b}{\pflet{$S_1, \ldots, S_n \in \SS$}}
        \step{c}{$\inv{f}(S_1 \cap \cdots \cap S_n)$ is open in $A$}
        \begin{proof}
            \pf\ Since $\inv{f}(S_1 \cap \cdots \cap S_n) = \inv{f}(S_1) \cap \cdots \cap \inv{f}(S_n)$.
        \end{proof}
        \qedstep
        \begin{proof}
            \pf\ By Propositions \ref{proposition:continuous_basis} and \ref{proposition:subbasis_basis}.
        \end{proof}
    \end{proof}
    \qed
\end{proof}

\begin{proposition}
    Let $X$ and $Y$ be topological spaces and $f : X \rightarrow Y$. Let $\SS$ be a basis for $Y$. Then $f$
    is continuous if and only if, for all $V \in \SS$, we have $\inv{f}(V)$ is open in $X$.
\end{proposition}

\begin{proof}
    \pf
    \step{1}{If $f$ is continuous then, for all $V \in \SS$, we have $\inv{f}(V)$ is open in $X$.}
    \begin{proof}
        \pf\ Since every element of $\SS$ is open.
    \end{proof}
    \step{2}{Suppose that, for all $V \in \SS$, we have $\inv{f}(V)$ is open in $X$. Then $f$ is continuous.}
    \begin{proof}
        \step{a}{\assume{For all $V \in \SS$, we have $\inv{f}(V)$ is open in $X$.}}
        \step{b}{For every set $B$ that is the finite intersection of elemets of $\SS$, we have
        $\inv{f}(B)$ is open in $X$.}
        \begin{proof}
            \pf\ Because $\inv{f}(V_1 \cap \cdots \cap V_n) = \inv{f}(V_1) \cap \cdots \cap \inv{f}(V_n)$.
        \end{proof}
        \qedstep
        \begin{proof}
            \pf\ From Propositions \ref{proposition:subbasis_basis} and \ref{proposition:continuous_basis}.
        \end{proof}
    \end{proof}
    \qed
\end{proof}

\begin{definition}[Continuous at a Point]
    Let $X$ and $Y$ be topological spaces. Let $f : X \rightarrow Y$ and $x \in X$. Then $f$ is
    \emph{continuous at $x$} if and only if, for every neighbourhood $V$ of $f(x)$, there exists a
    neighbourhood $U$ of $x$ such that $f(U) \subseteq V$.
\end{definition}

\begin{theorem}
    \label{theorem:continuous}
    Let $X$ and $Y$ be topological spaces and $f : X \rightarrow Y$. Then the following are equivalent:
    \begin{enumerate}
        \item $f$ is continuous.
        \item For all $A \subseteq X$, we have $f(\overline{A}) \subseteq \overline{f(A)}$
        \item For all $B \subseteq Y$ closed, we have $\inv{f}(B)$ is closed in $X$.
        \item $f$ is continuous at every point of $X$.
    \end{enumerate}
\end{theorem}

\begin{proof}
    \pf
    \step{1}{$1 \Rightarrow 2$}
    \begin{proof}
        \step{a}{\assume{$f$ is continuous.}}
        \step{b}{\pflet{$A \subseteq X$}}
        \step{c}{\pflet{$x \in \overline{A}$} \prove{$f(x) \in \overline{f(A)}$}}
        \step{d}{\pflet{$V$ be a neighbourhood of $f(x)$}}
        \step{e}{$\inv{f}(V)$ is a neighbourhood of $x$}
        \step{f}{\pick\ $y \in A \cap \inv{f}(V)$}
        \begin{proof}
            \pf\ By Theorem \ref{theorem:closure_neighbourhood}.
        \end{proof}
        \step{g}{$f(y) \in V \cap f(A)$}
        \qedstep
        \begin{proof}
            \pf\ By Theorem \ref{theorem:closure_neighbourhood}.
        \end{proof}
    \end{proof}
    \step{2}{$2 \Rightarrow 3$}
    \begin{proof}
        \step{a}{\assume{2}}
        \step{b}{\pflet{$B$ be closed in $Y$}}
        \step{c}{\pflet{$x \in \overline{\inv{f}(B)}$ \prove{$x \in \inv{f}(B)$}}}
        \step{d}{$f(x) \in B$}
        \begin{proof}
            \pf
            \begin{align*}
                f(x) & \in f(\overline{\inv{f}(B)}) \\
                & \subseteq \overline{f(\inv{f}(B))} & (\text{\stepref{a}})\\
                & \subseteq \overline{B} & (Proposition \ref{proposition:closure_monotone}) \\
                & = B
            \end{align*}
        \end{proof}
    \end{proof}
    \step{3}{$3 \Rightarrow 1$}
    \begin{proof}
        \step{a}{\assume{3}}
        \step{b}{\pflet{$V$ be open in $Y$}}
        \step{c}{$Y \setminus V$ is closed in $Y$}
        \step{d}{$\inv{f}(Y \setminus V)$ is closed in $X$}
        \step{e}{$X \setminus \inv{f}(V)$ is closed in $X$}
        \step{f}{$\inv{f}(V)$ is open in $X$}
    \end{proof}
    \step{4}{$1 \Rightarrow 4$}
    \begin{proof}
        \pf\ For any neighbourhood $V$ of $f(x)$, the set $U = \inv{f}(V)$ is a neighbourhood of $x$ such that
        $f(U) \subseteq V$.
    \end{proof}
    \step{5}{$4 \Rightarrow 1$}
    \begin{proof}
        \step{a}{\assume{4}}
        \step{b}{\pflet{$V$ be open in $Y$}}
        \step{c}{\pflet{$x \in \inv{f}(V)$}}
        \step{d}{$V$ is a neighbourhood of $f(x)$}
        \step{e}{\pick\ a neighbourhood $U$ of $x$ such that $f(U) \subseteq V$}
        \step{f}{$x \in U \subseteq \inv{f}(V)$}
        \qedstep
        \begin{proof}
            \pf\ By Lemma \ref{lemma:open}.
        \end{proof}
    \end{proof}
    \qed
\end{proof}

\begin{theorem}
    A constant function is continuous.
\end{theorem}

\begin{proof}
    \pf\ Let $X$ and $Y$ be topological spaces. Let $b \in Y$, and let $f : X \rightarrow Y$ be the constant
    function with value $b$. For any open $V \subseteq Y$, the set $\inv{f}(V)$ is either $X$ (if $b \in V$)
    or $\emptyset$ (if $b \notin V$). \qed
\end{proof}

\begin{theorem}
    If $A$ is a subspace of $X$ then the inclusion $j : A \rightarrow X$ is continuous.
\end{theorem}

\begin{proof}
    \pf\ For any $V$ open in $X$, we have $\inv{j}(V) = V \cap A$ is open in $A$. \qed
\end{proof}

\begin{theorem}
    \label{theorem:composite_continuous}
    The composite of two continuous functions is continuous.
\end{theorem}

\begin{proof}
    \pf\ Let $f : X \rightarrow Y$ and $g : Y \rightarrow Z$ be continuous. For any $V$ open in $Z$,
    we have $\inv{(g \circ f)}(V) = \inv{f}(\inv{g}(V))$ is open in $X$. \qed
\end{proof}

\begin{theorem}
    Let $f : X \rightarrow Y$ be a continuous function and $A$ be a subspace of $X$. Then the restriction
    $f \restriction A : A \rightarrow Y$ is continuous.
\end{theorem}

\begin{proof}
    \pf\ Let $V$ be open in $Y$. Then $\inv{(f \restriction A)}(V) = \inv{f}(V) \cap A$ is open in $A$. \qed
\end{proof}

\begin{theorem}
    Let $f : X \rightarrow Y$ be continuous. Let $Z$ be a subspace of $Y$ such that $f(X) \subseteq Z$. Then
    the corestriction $f : X \rightarrow Z$ is continuous.
\end{theorem}

\begin{proof}
    \pf
    \step{1}{\pflet{$V$ be open in $Z$.}}
    \step{2}{\pick\ $U$ open in $Y$ such that $V = U \cap Z$.}
    \step{3}{$\inv{f}(V) = \inv{f}(U)$}
    \step{4}{$\inv{f}(V)$ is open in $X$.}
    \qed
\end{proof}

\begin{theorem}
    Let $f : X \rightarrow Y$ be continuous. Let $Z$ be a space such that $Y$ is a subspace of $Z$.
    Then the expansion $f : X \rightarrow Z$ is continuous.
\end{theorem}

\begin{proof}
    \pf\ Let $V$ be open in $Z$. Then $\inv{f}(V) = \inv{f}(V \cap Y)$ is open in $X$. \qed
\end{proof}

\begin{theorem}
    Let $X$ and $Y$ be topological spaces. Let $f : X \rightarrow Y$. Suppose $\UU$ is a set of open sets
    in $X$ such that $X = \bigcup \UU$ and, for all $U \in \UU$, we have $f \restriction U : U \rightarrow Y$
    is continuous. Then $f$ is continuous.
\end{theorem}

\begin{proof}
    \pf
    \step{1}{\pflet{$V$ be open in $Y$}}
    \step{2}{$\inv{f}(V) = \bigcup_{U \in \UU} \inv{(f \restriction U)}(V)$}
    \step{3}{For all $U \in \UU$, we have $\inv{(f \restriction U)}(V)$ is open in $U$.}
    \step{4}{For all $U \in \UU$, we have $\inv{(f \restriction U)}(V)$ is open in $X$.}
    \begin{proof}
        \pf\ Lemma \ref{lemma:subspace_open}.
    \end{proof}
    \qed
\end{proof}

\begin{theorem}
    \label{theorem:product_continuous}
        Let $A$ be a topological space and $\{ X_i \}_{i \in I}$ be a family of topological spaces. Let $f : A \rightarrow \prod_{i \in I} X_i$ be a function.
        If $\pi_i \circ f$ is continuous for all $i \in I$ then $f$ is continuous.
\end{theorem}

\begin{proof}
    \pf
    \step{1}{\pflet{$i \in I$ and $U$ be open in $X_i$}}
    \step{2}{$\inv{f}(\inv{\pi_i}(U))$ is open in $A$}
    \qedstep
    \begin{proof}
        \pf\ Proposition \ref{proposition:continuous_subbasis}.
    \end{proof}
    \qed
\end{proof}

\begin{proposition}
    Let $X$ and $X'$ be the same set $X$ under two topologies $\TT$ and $\TT'$. Let $i : X \rightarrow X'$
    be the identity function. Then $i$ is continuous if and only if $\TT' \subseteq \TT$.
\end{proposition}

\begin{proof}
    \pf\ Immediate from definitions. \qed
\end{proof}

\begin{proposition}
    Let $f : \RR \rightarrow \RR$ and $a \in \RR$. Then $f$ is continuous on the right at $a$ if and only if
    $f$ is continuous at $a$ as a function $\RR_l \rightarrow \RR$.
\end{proposition}

\begin{proof}
    \pf
    \step{1}{If $f$ is continuous on the right at $a$ then $f$ is continuous at $a$ as a function $\RR_l
    \rightarrow \RR$.}
    \begin{proof}
        \step{a}{\assume{$f$ is continuous on the right at $a$.}}
        \step{b}{\pflet{$V$ be a neighbourhood of $f(a)$}}
        \step{c}{\pick\ $b$, $c$ such that $f(a) \in (b,c) \subseteq V$.}
        \step{d}{\pflet{$\epsilon = \min(c-f(a),f(a)-b)$}}
        \step{e}{\pick\ $\delta > 0$ such that, for all $x$, if $a < x < a + \delta$ then
        $|f(x)-f(a)|< \epsilon$}
        \step{f}{\pflet{$U = [a,a+\delta)$}}
        \step{g}{$f(U) \subseteq V$}
    \end{proof}
    \step{2}{If $f$ is continuous at $a$ as a function $\RR_l \rightarrow \RR$ then $f$ is continuous on
    the right at $a$.}
    \begin{proof}
        \step{a}{\assume{$f$ is continuous at $a$ as a function $\RR_l \rightarrow \RR$}}
        \step{b}{\pflet{$\epsilon > 0$}}
        \step{c}{\pick\ a neighbourhood $U$ of $a$ such that $f(U) \subseteq (f(a)-\epsilon, f(a)+\epsilon)$}
        \step{d}{\pick\ $b$, $c$ such that $a \in [b,c) \subset U$}
        \step{e}{\pflet{$\delta = c - a$}}
        \step{f}{For all $x$, if $a < x < a + \delta$ then $|f(x)-f(a)| < \epsilon$}
    \end{proof}
    \qed
\end{proof}

\begin{lemma}
    \label{lemma:order_topology_closed}
    Let $X$ be a topological space.
    Let $Y$ be a linearly ordered set in the order topology. Let $f, g : X \rightarrow Y$ be continuous. Then $C = \{ x \in X \mid f(x) \leq g(x) \}$ is closed.
\end{lemma}

\begin{proof}
    \pf
    \step{1}{\pflet{$x \in X \setminus C$}}
    \step{2}{$f(x) > g(x)$ \prove{There exists a neighbourhood $U$ of $x$ such that $U \subseteq X \setminus C$}}
    \step{3}{\case{There exists $y$ such that $g(x) < y < f(x)$}}
    \begin{proof}
        \pf\ Take $U = \inv{g}((-\infty, y)) \cup \inv{f}(y,+\infty)$.
    \end{proof}
    \step{4}{\case{There is no $y$ such that $g(x) < y < f(x)$}}
    \begin{proof}
        \pf\ Take $U = \inv{g}((-\infty, f(x))) \cup \inv{f}(g(x),+\infty)$.
    \end{proof}
    \qed
\end{proof}

\begin{lemma}
    \label{lemma:continuous_open_subspace}
    Let $f : X \rightarrow Y$. Let $Z$ be an open subspace of $X$ and $a \in Z$. If $f \restriction Z$ is continuous at $a$ then $f$ is continuous at $a$.
\end{lemma}

\begin{proof}
    \pf
    \step{1}{\pflet{$V$ be a neighbourhood of $f(x)$}}
    \step{2}{\pick\ a neighbourhood $W$ of $x$ in $Z$ such that $f(W) \subseteq V$}
    \step{3}{$W$ is a neighbourhood of $x$ in $X$ such that $f(W) \subseteq V$}
    \begin{proof}
        \pf\ Lemma \ref{lemma:subspace_open}.
    \end{proof}
    \qed
\end{proof}

\begin{proposition}
    Let $f : A \rightarrow B$ and $g : C \rightarrow D$ be continuous. Define $f \times g : A \times C \rightarrow B \times D$ by
    \[ (f \times g)(a,c) = (f(a), g(c)) \enspace . \]
    Then $f \times g$ is continuous.
\end{proposition}

\begin{proof}
    \pf\ $\pi_1 \circ (f \times g) = f \circ \pi_1$ and $\pi_2 \circ (f \times g) = g \circ \pi_2$ are continuous
    by Theorem \ref{theorem:composite_continuous}. The result follows by Theorem \ref{theorem:product_continuous}.
\end{proof}

\begin{proposition}
    Let $X$ be a topological space. Let $Y$ a Hausdorff space. Let $A \subseteq X$. Let $f, g : \overline{A} \rightarrow Y$ be continuous.
    If $f$ and $g$ agree on $A$ then $f = g$.
\end{proposition}

\begin{proof}
    \pf
    \step{1}{\pflet{$x \in \overline{A}$}}
    \step{2}{\assume{$f(x) \neq g(x)$}}
    \step{3}{\pick\ disjoint neighbourhoods $V$ of $f(x)$ and $W$ of $g(x)$.}
    \step{4}{\pick\ $y \in \inv{f}(V) \cap \inv{g}(W) \cap A$}
    \begin{proof}
        \pf\ Since $\inv{f}(V) \cap \inv{g}(W)$ is a neighbourhood of $x$ and hence intersects $A$.
    \end{proof}
    \step{5}{$f(y) = g(y) \in V \cap W$}
    \qedstep
    \begin{proof}
        \pf\ This contradicts the fact that $V$ and $W$ are disjoint (\stepref{3}).
    \end{proof}
    \qed
\end{proof}

\begin{proposition}
    \label{proposition:converge_continuous}
    Let $X$ and $Y$ be topological spaces and $f : X \rightarrow Y$ be continuous. If $a_n \rightarrow l$ as $n \rightarrow \infty$ in $X$ then
    $f(a_n) \rightarrow f(l)$ as $n \rightarrow \infty$.
\end{proposition}

\begin{proof}
    \pf
    \step{1}{\pflet{$V$ be a neighbourhood of $f(l)$}}
    \step{2}{\pick\ a neighbourhood $U$ of $l$ such that $f(U) \subseteq V$}
    \step{3}{\pick\ $N$ such that, for all $n \geq N$, we have $a_n \in U$}
    \step{4}{For all $n \geq N$ we have $f(n) \in V$}
    \qed
\end{proof}

\begin{proposition}
    \label{proposition:converge_product}
    Let $\{ X_i \}_{i \in I}$ be a family of topological spaces. Let $(a_n)$ be a sequence in $\prod_{i \in I} X_i$ and $l \in \prod_{i \in I} X_i$.
    Then $a_n \rightarrow l$ as $n \rightarrow \infty$ if and only if, for all $i \in I$, we have $\pi_i(a_n) \rightarrow \pi_i(l)$ as $n \rightarrow \infty$.
\end{proposition}

\begin{proof}
    \pf
    \step{1}{If $a_n \rightarrow l$ as $n \rightarrow \infty$ then, for all $i \in I$, we have $\pi_i(a_n) \rightarrow \pi_i(l)$ as $n \rightarrow \infty$}
    \begin{proof}
        \pf\ Proposition \ref{proposition:converge_continuous}.
    \end{proof}
    \step{2}{If, for all $i \in I$, we have $\pi_i(a_n) \rightarrow \pi_i(l)$ as $n \rightarrow \infty$, then $a_n \rightarrow l$ as $n \rightarrow \infty$}
    \begin{proof}
        \step{a}{\assume{For all $i \in I$, we have $\pi_i(a_n) \rightarrow \pi_i(l)$ as $n \rightarrow \infty$}}
        \step{b}{\pflet{$V$ be a neighbourhood of $l$}}
        \step{c}{\pick\ open sets $U_i$ in $X_i$ such that $l \in \prod_{i \in I} U_i \subseteq V$ and $U_i = X_i$ for all $i$ except $i = i_1, \ldots, i_k$}
        \step{d}{For $j = 1, \ldots, k$, \pick\ $N_j$ such that, for all $n \geq N_j$, we have $\pi_{i_j}(a_n) \in U_{i_j}$}
        \step{e}{\pflet{$N = \max(N_1, \ldots, N_k)$}}
        \step{f}{For all $n \geq N$ we have $a_n \in V$}
    \end{proof}
    \qed
\end{proof}

\section{Homeomorphisms}

\begin{definition}[Homeomorphism]
    Let $X$ and $Y$ be topological spaces. A \emph{Homeomorphism} $f$ between $X$ and $Y$, $f : X \cong Y$,
    is a bijection $f : X \rightarrow Y$ such that both $f$ and $\inv{f}$ are continuous.
\end{definition}

\begin{lemma}
    Let $X$ and $Y$ be topological spaces and $f : X \rightarrow Y$ a bijection. Then the following are
    equivalent:
    \begin{enumerate}
        \item $f$ is a homeomorphism.
        \item $f$ is continuous and an open map.
        \item For any $U \subseteq X$, we have $U$ is open if and only if $f(U)$ is open.
    \end{enumerate}
\end{lemma}

\begin{proof}
    \pf\ Immediate from definitions. \qed
\end{proof}

\begin{proposition}
    Let $X$ and $X'$ be the same set $X$ under two topologies $\TT$ and $\TT'$. Let $i : X \rightarrow X'$
    be the identity function. Then $i$ is a homeomorphism if and only if $\TT = \TT'$.
\end{proposition}

\begin{proof}
    \pf\ Immediate from definitions. \qed
\end{proof}

\begin{definition}[Topological Property]
    Let $P$ be a property of topological spaces. Then $P$ is a \emph{topological} property if and only if,
    for any spaces $X$ and $Y$, if $P$ holds of $X$ and $X \cong Y$ then $P$ holds of $Y$.
\end{definition}

\begin{definition}[Topological Imbedding]
    Let $X$ and $Y$ be topological spaces and $f : X \rightarrow Y$. Then $f$ is a \emph{topological
    imbedding} if and only if the corestriction $f : X \rightarrow f(X)$ is a homeomorphism.
\end{definition}

\begin{proposition}
    \label{proposition:imbedding_product}
    Let $X$ and $Y$ be topological spaces and $a \in X$. The function $i : Y \rightarrow X \times Y$ that maps $y$ to $(a,y)$ is an imbedding.
\end{proposition}

\begin{proof}
    \pf
    \step{1}{$i$ is injective}
    \step{2}{$i$ is continuous.}
    \begin{proof}
        \pf\ For $U$ open in $X$ and $V$ open in $Y$, we have $\inv{i}(U \times V)$ is $V$ if $a \in U$, and $\emptyset$ if $a \notin U$.
    \end{proof}
    \step{3}{$i : Y \rightarrow i(Y)$ is an open map.}
    \begin{proof}
        \pf\ For $V$ open in $Y$ we have $i(V) = (X \times V) \cap i(Y)$.
    \end{proof}
    \qed
\end{proof}

\section{Locally Finite Sets}

\begin{definition}[Locally Finite]
    Let $X$ be a topological space and $\{ A_\alpha \}$ a family of subsets of $X$. Then $\AA$ is \emph{locally finite} if and only if every point in $X$ has a neighbourhood that
    intersects $A_\alpha$ for only finitely many $\alpha$.
\end{definition}

\begin{theorem}[Pasting Lemma]
    Let $X$ and $Y$ be topological spaces and $f : X \rightarrow Y$. Let $\{ A_\alpha \}$ be a locally finite family of closed subsets of $X$ that cover $X$. Suppose $f \restriction A_\alpha$
    is continuous for all $\alpha$. Then $f$ is continuous.
\end{theorem}

\begin{proof}
    \pf
    \step{0}{Let $X$ and $Y$ be topological spaces and $f : X \rightarrow Y$. Let $A$ and $B$ be closed subsets of $X$ such that $X = A \cup B$. 
    Suppose $f \restriction A$ and $f \restriction B$ are continuous. Then $f$ is continuous.}
    \begin{proof}
        \step{1}{\pflet{$C \subseteq Y$ be closed.}}
        \step{2}{$\inv{h}(C) = \inv{f}(C) \cup \inv{g}(C)$}
        \step{3}{$\inv{f}(C)$ and $\inv{g}(C)$ are closed in $X$.}
        \begin{proof}
            \pf\ Theorems \ref{theorem:continuous} and \ref{theorem:closed_subspace}.
        \end{proof}
        \step{4}{$\inv{h}(C)$ is closed in $X$.}
        \begin{proof}
            \pf\ Lemma \ref{lemma:closed_union}.
        \end{proof}
        \qedstep
        \begin{proof}
            \pf\ Theorem \ref{theorem:continuous}.
        \end{proof}    
    \end{proof}
    \step{1}{ Let $X$ and $Y$ be topological spaces and $f : X \rightarrow Y$. Let $\{ A_\alpha \}$ be a finite family of closed subsets of $X$ that cover $X$. Suppose $f \restriction A_\alpha$
    is continuous for all $\alpha$. Then $f$ is continuous.}
    \begin{proof}
        \pf\ From \stepref{0} by induction.
    \end{proof}
    \step{2}{ Let $X$ and $Y$ be topological spaces and $f : X \rightarrow Y$. Let $\{ A_\alpha \}$ be a locally finite family of closed subsets of $X$ that cover $X$. Suppose $f \restriction A_\alpha$
    is continuous for all $\alpha$. Then $f$ is continuous.}
    \begin{proof}
        \step{1}{\pflet{$x \in X$} \prove{$f$ is continuous at $x$}}
        \step{2}{\pick\ a neighbourhood $U$ of $x$ that intersects $A_\alpha$ for only finitely many $\alpha$.}
        \step{3}{$f \restriction U$ is continuous}
        \begin{proof}
            \pf\ By \stepref{1}.
        \end{proof}
        \qedstep
        \begin{proof}
            \pf\ Lemma \ref{lemma:continuous_open_subspace}.
        \end{proof}
    \end{proof}
    \qed
\end{proof}

The following example shows that we cannot remove the assumption of local finiteness.

\begin{example}
    Define $f : [-1,1] \rightarrow \RR$ by: $f(x) = 1$ if $x < -1$, $f(x) = 0$ if $x > 1$. Let $C_n = [-1,-1/n]$ for $n \geq 1$, and $D = [0,1]$. Then
    $[-1,1] = \bigcup_{n=1}^\infty C_n \cup D$ and $f$ is continuous on each $C_n$ and each $D$, but $f$ is not continuous on $[-1,1]$.    
\end{example}

\begin{proposition}
    Let $X$ be a topological space.
    Let $Y$ be a linearly ordered set in the order topology. Let $f, g : X \rightarrow Y$ be continuous. Define $h : X \rightarrow Y$ by $h(x) = \min(f(x),g(x))$.
    Then $h$ is continuous.
\end{proposition}

\begin{proof}
    \pf\ By the Pasting Lemma applied to $\{ x \in X \mid f(x) \leq g(x) \}$ and $\{ x \in X \mid g(x) \leq f(x) \}$, which are closed by Lemma \ref{lemma:order_topology_closed}.
\end{proof}

\section{Continuous in Each Variable Separately}

\begin{definition}[Continuous in Each Variable Separately]
    Let $F : X \times Y \rightarrow Z$. Then $F$ is \emph{continuous in each
    variable separately} if and only if:
    \begin{itemize}
        \item for every $a \in X$ the function $\lambda y \in Y. F(a,y)$ is continuous;
        \item for every $b \in Y$ the function $\lambda x \in X. F(x,b)$ is continuous.
    \end{itemize}
\end{definition}

\begin{proposition}
    Let $F : X \times Y \rightarrow Z$. If $F$ is continuous then $F$ is continuous in each
    variable separately.
\end{proposition}

\begin{proof}
    \pf\ For $a \in X$, the function $\lambda y \in Y. F(a,y)$ is $F \circ i$ where $i : Y \rightarrow X \times Y$ maps $y$ to $(a,y)$.
    We have $i$ is continuous by Proposition \ref{proposition:imbedding_product}, hence $F \circ i$ is continuous by Theorem \ref{theorem:composite_continuous}.

    Similarly for $\lambda x \in X. F(x,b)$ for $b \in Y$. \qed
\end{proof}

\begin{example}
    Define $F : \RR \times \RR \rightarrow \RR$ by
    \[ F(x,y) = \begin{cases}
        xy / (x^2 + y^2) & \text{if } (x,y) \neq (0,0) \\
        0 & \text{if } (x,y) = (0,0)
    \end{cases} \]
    Then $F$ is continuous in each variable separately but not continuous.
\end{example}

\section{The Box Topology}

\begin{definition}[Box Topology]
    Let $\{ A_i \}_{i \in I}$ be a family of topological spaces. The \emph{box topology} on $\prod_{i \in I} A_i$ is the topology generated by the set of all sets
    of the form $\prod_{i \in I} U_i$ where $\{ U_i \}_{i \in I}$ is a family such that each $U_i$ is open in $A_i$.
\end{definition}

This is a basis since it covers $\prod_{i \in I} A_i$ and is closed under intersection.

\begin{proposition}
    The box topology is finer than the product topology.
\end{proposition}

\begin{proof}
    \pf\ From Proposition \ref{proposition:product_basis}. \qed
\end{proof}

\begin{corollary}
    \label{corollary:closed_box}
        If $A_i$ is closed in $X_i$ for all $i \in I$ then $\prod_{i \in I} A_i$ is closed in $\prod_{i \in I} X_i$ under the box topology.    
\end{corollary}

\begin{proof}
    \pf\ From Proposition \ref{proposition:closed_product}.
\end{proof}

\begin{proposition}[AC]
    Let $\{ A_i \}_{i \in I}$ be a family of topological spaces. For $i \in I$, let $\BB_i$ be a basis for the topology on $A_i$. Then $\BB = \{ \prod_{i \in I} B_i \mid
    \forall i \in I. B_i \in \BB_i \}$ is a basis for the box topology on $\prod_{i \in I} A_i$.
\end{proposition}

\begin{proof}
    \pf
    \step{1}{Every set of the form $\prod_{i \in I} B_i$ is open.}
    \step{2}{For every point $a \in \prod_{i \in I} A_i$ and every open set $U$ with $a \in U$, there exists $B \in \BB$ such that $a \in B \subseteq U$.}
    \begin{proof}
        \step{a}{\pflet{$U$ be open and $a \in U$}}
        \step{b}{\pick\ a family $\{ U_i \}_{i \in I}$ such that each $U_i$ is open in $A_i$ and $a \in \prod_{i \in I} U_i \subseteq U$.}
        \step{c}{For $i \in I$, \pick\ $B_i \in \BB_i$ such that $a_i \in B_i \subseteq U_i$}
        \begin{proof}
            \pf\ Using the Axiom of Choice.
        \end{proof}
        \step{d}{$a \in \prod_{i \in I} B_i \subseteq U$}
    \end{proof}
    \qedstep
    \begin{proof}
        \pf\ Lemma \ref{lemma:basis}.
    \end{proof}
    \qed
\end{proof}

\begin{theorem}
    Let $\{ X_i \}_{i \in I}$ be a family of topological spaces. Let $A_i$ be a subspace of $X_i$ for all $i \in I$.
    Give $\prod_{i \in I} X_i$ the box topology.
    Then the box topology on $\prod_{i \in I} A_i$ is the same as the topology it inherits as a subspace of
    $\prod_{i \in I} X_i$.
\end{theorem}

\begin{proof}
    \pf\ The box topology is generated by the basis
    \begin{align*}
        & \{ \prod_{i \in I} U_i \mid \forall i \in I, U_i \text{ open in } A_i \} \\
        = & \{ \prod_{i \in I} (V_i \cap A_i) \mid \forall i \in I, V_i \text{ open in } X_i \} \\
        = & \{ \prod_{i \in I} V_i \mid \forall i \in I, V_i \text{ open in } X_i \} \cap \prod_{i \in I} A_i
    \end{align*}
    and this is a basis for the subspace topology by Lemma \ref{lemma:subspace_basis}. \qed
\end{proof}

\begin{proposition}
    Let $\{ X_i \}_{i \in I}$ be a family of Hausdorff spaces. Then $\prod_{i \in I} X_i$ under the box topology is Hausdorff.
\end{proposition}

\begin{proof}
    \pf
    \step{1}{\pflet{$\{ X_i \}_{i \in I}$ be a family of Hausdorff spaces.}}
    \step{2}{\pflet{$a, b \in \prod_{i \in I} X_i$ with $a \neq b$}}
    \step{3}{\pick\ $i \in I$ such that $a_i \neq b_i$}
    \step{4}{\pick\ $U$, $V$ disjoint open sets in $X_i$ with $a_i \in U$ and $b_i \in V$}
    \step{5}{$\inv{\pi_i}(U)$ and $\inv{\pi_i}(V)$ are disjoint open sets in $\prod_{i \in I} X_i$ with $a \in \inv{\pi_i}(U)$
    and $b \in \inv{\pi_i}(V)$}
    \qed
\end{proof}

\begin{proposition}[AC]
    Let $\{ X_i \}_{i \in I}$ be a family of topological spaces. Give $\prod_{i \in I} X_i$ the box topology.
    Let $A_i \subseteq X_i$ for all $i \in I$.
    Then
    \[ \prod_{i \in I} \overline{A_i} = \overline{\prod_{i \in I} A_i} \enspace . \]
\end{proposition}

\begin{proof}
    \pf
    \step{1}{$\overline{\prod_{i \in I} A_i} \subseteq \prod_{i \in I} \overline{A_i}$}
    \begin{proof}
        \step{a}{For all $i \in I$ we have $A_i \subseteq \overline{A_i}$}
        \begin{proof}
            \pf\ Lemma \ref{lemma:closure_subset}.
        \end{proof}
        \step{c}{$\prod_{i \in I} A_i \subseteq \prod_{i \in I} \overline{A_i}$}
        \qedstep
        \begin{proof}
            \pf\ Since $\prod_{i \in I} A_i$ is closed by Corollary \ref{corollary:closed_box}.
        \end{proof}
    \end{proof}
    \step{2}{$\prod_{i \in I} \overline{A_i} \subseteq \overline{\prod_{i \in I} A_i}$}
    \begin{proof}
        \step{a}{\pflet{$x \in \prod_{i \in I} \overline{A_i}$}}
        \step{b}{\pflet{$U$ be a neighbourhood of $x$}}
        \step{c}{\pick\ $V_i$ open in $X_i$ such that $x \in \prod_{i \in I} V_i \subseteq U$}
        \step{d}{For $i \in I$, pick $a_i \in V_i \cap A_i$}
        \begin{proof}
            \pf\ By Theorem \ref{theorem:closure_neighbourhood} and \stepref{a} using the Axiom of Choice.
        \end{proof}
        \step{f}{$U$ intersects $\prod_{i \in I} A_i$}
        \qedstep
        \begin{proof}
            \pf\ $a \in U \cap \prod_{i \in I} A_i$.
        \end{proof}
    \end{proof}
    \qed
\end{proof}

The following example shows that Theorem \ref{theorem:product_continuous} fails in the box topology.

\begin{example}
    Define $f : \RR \rightarrow \RR^\omega$ by $f(t) = (t, t, \ldots)$. Then $\pi_n \circ f = \id{\RR}$ is continuous for all $n$.
    But $f$ is not continuous when $\RR^\omega$ is given the box topology because the inverse image of
    \[ (-1,1) \times (-1/2, 1/2) \times (-1/3,1/3) \times \cdots \]
    is $\{0\}$ which is not open.
\end{example}

The following example shows that Proposition \ref{proposition:converge_product} fails in the box topology.

\begin{example}
    Give $\RR^\omega$ the box topology. Let $a_n = (1/n, 1/n, \ldots)$ for $n \geq 1$ and $l = (0, 0, \ldots)$. Then $\pi_i(a_n) \rightarrow \pi_i(l)$
    as $n \rightarrow \infty$ for all $i$, but $a_n \not\rightarrow l$ as $n \rightarrow \infty$ since the open set
    \[ (-1,1) \times (-1/2, 1/2) \times (-1/3,1/3) \times \cdots \]
    contains $l$ but does not contain any $a_n$.
\end{example}

\begin{example}
    The set $\RR^\infty$ is closed in $\RR^\omega$ under the box topology. For let $(a_n)$ be any sequence not in $\RR^\infty$.
    Let $U_n$ be an open interval around $a_n$ that does not contain $0$ if $a_n \neq 0$, and $U_n = \RR$ if $a_n = 0$.
    Then $\prod_{n \geq 0} U_n$ is a neighbourhood of $(a_n)$ that does not intersect $\RR^\infty$.
\end{example}

\section{The Metric Topology}

\begin{definition}[Metric]
    Let $X$ be a set. A \emph{metric} on $X$ is a function $d : X^2 \rightarrow \RR$ such that:
    \begin{enumerate}
        \item For all $x,y \in X$, $d(x,y) \geq 0$
        \item For all $x,y \in X$, $d(x,y) = 0$ if and only if $x = y$
        \item For all $x,y \in X$, $d(x,y) = d(y,x)$
        \item (\emph{Triangle Inequality}) For all $x, y, z \in X$, $d(x,z) \leq d(x,y) + d(y,z)$
    \end{enumerate}
    We call $d(x,y)$ the \emph{distance} between $x$ and $y$.
\end{definition}

\begin{definition}[Open Ball]
    Let $X$ be a metric space. Let $a \in X$ and $\epsilon > 0$. The \emph{open ball} with \emph{centre} $a$ and \emph{radius} $\epsilon$
    is
    \[ B(a, \epsilon) = \{ x \in X \mid d(a,x) < \epsilon \} \enspace . \]
\end{definition}

\begin{definition}[Metric Topology]
    Let $X$ be a metric space. The \emph{metric topology} on $X$ is the topology generated by the basis consisting of all the open balls.
\end{definition}

We prove this is a basis for a topology.

\begin{proof}
    \pf
    \step{1}{For every point $a$, there exists a ball $B$ such that $a \in B$}
    \begin{proof}
        \pf\ We have $a \in B(a,1)$.
    \end{proof}
    \step{2}{For any balls $B_1$, $B_2$ and point $a \in B_1 \cap B_2$, there exists a ball $B_3$ such that $a \in B_3 \subseteq B_1 \cap B_2$}
    \begin{proof}
        \step{a}{\pflet{$B_1 = B(c_1, \epsilon_1)$ and $B_2 = B(c_2, \epsilon_2)$}}
        \step{b}{\pflet{$\delta = \min(\epsilon_1 - d(c_1,a), \epsilon_2 - d(c_2,a))$} \prove{$B(a,\delta) \subseteq B_1 \cap B_2$}}
        \step{c}{\pflet{$x \in B(a, \delta)$}}
        \step{d}{$x \in B_1$}
        \begin{proof}
            \pf
            \begin{align*}
                d(x,c_1) & = d(x,a) + d(a,c_1) \\
                & < \delta + d(a,c_1) \\
                & \leq \epsilon_1
            \end{align*}
        \end{proof}
        \step{e}{$x \in B_2$}
        \begin{proof}
            \pf\ Similar.
        \end{proof}
    \end{proof}
    \qed
\end{proof}

\begin{proposition}
    \label{proposition:open_in_metric_space}
    Let $X$ be a metric space and $U \subseteq X$. Then $U$ is open if and only if, for every $x \in U$, there exists $\epsilon > 0$ such that $B(x, \epsilon)
    \subseteq U$.
\end{proposition}

\begin{proof}
    \pf
    \step{1}{If $U$ is open then, for all $x \in U$, there exists $\epsilon > 0$ such that $B(x, \epsilon) \subseteq U$.}
    \begin{proof}
        \step{a}{\assume{$U$ is open.}}
        \step{b}{\pflet{$x \in U$}}
        \step{c}{\pick\ $a \in X$ and $\delta > 0$ such that $x \in B(a, \delta) \subseteq U$}
        \step{d}{\pflet{$\epsilon = \delta - d(a,x)$} \prove{$B(x, \epsilon) \subseteq U$}}
        \step{e}{\pflet{$y \in B(x, \epsilon)$}}
        \step{f}{$d(y,a) < \delta$}
        \begin{proof}
            \pf
            \begin{align*}
                d(y,a) & \leq d(a,x) + d(x,y) \\
                & < \delta + d(x,y) \\
                & = \epsilon
            \end{align*}
        \end{proof}
        \step{g}{$y \in U$}
    \end{proof}
    \step{2}{If, for all $x \in U$, there exists $\epsilon > 0$ such that $B(x, \epsilon) \subseteq U$, then $U$ is open.}
    \begin{proof}
        \pf\ Immediate from definitions.
    \end{proof}
    \qed
\end{proof}

\begin{definition}[Discrete Metric]
    Let $X$ be a set. The \emph{discrete metric} on $X$ is defined by
    \[ d(x,y) = \begin{cases}
        0 & \text{if } x = y \\
        1 & \text{if } x \neq y
    \end{cases} \]
\end{definition}

\begin{proposition}
    The discrete metric induces the discrete topology.
\end{proposition}

\begin{proof}
    \pf\ For any (open) set $U$ and point $a \in U$, we have $a \in B(a,1) \subseteq U$. \qed
\end{proof}

\begin{definition}[Standard Metric on $\RR$]
    The \emph{standard metric} on $\RR$ is defined by $d(x,y) = |x-y|$.
\end{definition}

\begin{proposition}
    The standard metric on $\RR$ induces the standard topology on $\RR$.
\end{proposition}

\begin{proof}
    \pf
    \step{1}{Every open ball is open in the standard topology on $\RR$.}
    \begin{proof}
        \pf\ $B(a, \epsilon) = (a - \epsilon, a + \epsilon)$
    \end{proof}
    \step{2}{For every open set $U$ and point $a \in U$, there exists $\epsilon > 0$ such that $B(a, \epsilon) \subseteq U$}
    \begin{proof}
        \step{a}{\pflet{$U$ be an open set and $a \in U$}}
        \step{b}{\pick\ an open interval $b$, $c$ such that $a \in (b,c) \subseteq U$}
        \step{c}{\pflet{$\epsilon = \min(a-b,c-a)$}}
        \step{d}{$B(a, \epsilon) \subseteq U$}
    \end{proof}
    \qed
\end{proof}

\begin{definition}[Metrizable]
    A topological space $X$ is \emph{metrizable} if and only if there exists a metric on $X$ that induces the topology.
\end{definition}

\begin{definition}[Bounded]
    Let $X$ be a metric space and $A \subseteq X$. Then $A$ is \emph{bounded} if and only if there exists $M$ such that,
    for all $x, y \in A$, we have $d(x,y) \leq M$.
\end{definition}

\begin{definition}[Diameter]
    Let $X$ be a metric space and $A \subseteq X$. The \emph{diameter} of $A$ is
    \[ \diam A = \sup_{x,y \in A} d(x,y) \enspace . \]
\end{definition}

\begin{definition}[Standard Bounded Metric]
    Let $d$ be a metric on $X$. The \emph{standard bounded metric} corresponding to $d$ is the metric $\overline{d}$ defined by
    \[ \overline{d}(x,y) = \min(d(x,y),1) \enspace . \]
\end{definition}

We prove this is a metric.

\begin{proof}
    \pf
    \step{1}{$\overline{d}(x,y) \geq 0$}
    \begin{proof}
        \pf\ Since $d(x,y) \geq 0$
    \end{proof}
    \step{2}{$\overline{d}(x,y) = 0$ if and only if $x = y$}
    \begin{proof}
        \pf\ $\overline{d}(x,y) = 0$ if and only if $d(x,y) = 0$ if and only if $x = y$
    \end{proof}
    \step{3}{$\overline{d}(x,y) = \overline{d}(y,x)$}
    \begin{proof}
        \pf\ Since $d(x,y) = d(y,x)$
    \end{proof}
    \step{4}{$\overline{d}(x,z) \leq \overline{d}(x,y) + \overline{d}(y,z)$}
    \begin{proof}
        \pf
        \begin{align*}
            \overline{d}(x,y) + \overline{d}(y,z) & = \min(d(x,y),1) + \min(d(y,z),1) \\
            & = \min(d(x,y) + d(y,z), d(x,y) + 1, d(y,z) + 1, 2) \\
            & \geq \min(d(x,z), 1) \\
            & = \overline{d}(x,z)
        \end{align*}
    \end{proof}
    \qed
\end{proof}

\begin{lemma}
    \label{lemma:basis_radius_less_than_one}
    In any metric space $X$, the set $\BB = \{ B(a, \epsilon) \mid a \in X, \epsilon < 1 \}$ is a basis for the metric topology.
\end{lemma}

\begin{proof}
    \pf
    \step{1}{Every element of $\BB$ is open.}
    \begin{proof}
        \pf\ From Lemma \ref{lemma:basis_unions}.
    \end{proof}
    \step{2}{For every open set $U$ and point $a \in U$, there exists $B \in \BB$ such that $a \in B \subseteq U$}
    \begin{proof}
        \step{a}{\pflet{$U$ be an open set and $a \in U$}}
        \step{b}{\pick $\epsilon > 0$ such that $B(a, \epsilon) \subseteq U$}
        \step{c}{$B(a,\min(\epsilon,1/2)) \subseteq U$}
    \end{proof}
    \qedstep
    \begin{proof}
        \pf\ Lemma \ref{lemma:basis}.
    \end{proof}
    \qed
\end{proof}

\begin{proposition}
    \label{proposition:standard_bounded_metric}
    Let $d$ be a metric on the set $X$. Then the standard bounded metric $\overline{d}$ induces the same metric as $d$.
\end{proposition}

\begin{proof}
    \pf\ This follows from Lemma \ref{lemma:basis_radius_less_than_one} since the open balls with radius $< 1$ are the same under both metrics. \qed
\end{proof}

\begin{lemma}
    \label{lemma:metrics_same_topology}
    Let $d$ and $d'$ be two metrics on the same set $X$. Let $\TT$ and $\TT'$ be the topologies they induce. Then $\TT \subseteq \TT'$ if and only if,
    for all $x \in X$ and $\epsilon > 0$, there exists $\delta > 0$ such that
    \[ B_{d'}(x, \delta) \subseteq B_d(x, \epsilon) \enspace . \]
\end{lemma}

\begin{proof}
    \pf
    \step{1}{If $\TT \subseteq \TT'$ then,
    for all $x \in X$ and $\epsilon > 0$, there exists $\delta > 0$ such that
    $B_{d'}(x, \delta) \subseteq B_d(x, \epsilon)$}
    \begin{proof}
        \pf\ From Proposition \ref{proposition:open_in_metric_space} since $x \in B_d(x, \epsilon) \in \TT'$.
    \end{proof}
    \step{2}{If, for all $x \in X$ and $\epsilon > 0$, there exists $\delta > 0$ such that
    $B_{d'}(x, \delta) \subseteq B_d(x, \epsilon)$, then $\TT \subseteq \TT'$}
    \begin{proof}
        \step{a}{\assume{For all $x \in X$ and $\epsilon > 0$, there exists $\delta > 0$ such that
        $B_{d'}(x, \delta) \subseteq B_d(x, \epsilon)$}}
        \step{b}{\pflet{$U \in \TT$}}
        \step{c}{For all $x \in U$ there exists $\delta > 0$ such that $B_{d'}(x, \delta) \subseteq U$.}
        \begin{proof}
            \step{i}{\pflet{$x \in U$}}
            \step{ii}{\pick\ $\epsilon > 0$ such that $B_d(x, \epsilon) \subseteq U$}
            \begin{proof}
                \pf\ Proposition \ref{proposition:open_in_metric_space}
            \end{proof}
            \step{iii}{\pick\ $\delta > 0$ such that $B_{d'}(x, \delta) \subseteq B_d(x, \epsilon)$}
            \begin{proof}
                \pf\ By \stepref{a}
            \end{proof}
            \step{iv}{$B_{d'}(x, \delta) \subseteq U$}
        \end{proof}
        \step{d}{$U \in \TT'$}
        \begin{proof}
            \pf\ Proposition \ref{proposition:open_in_metric_space}.
        \end{proof}
    \end{proof}
    \qed
\end{proof}

\begin{proposition}
    $\RR^2$ under the dictionary order topology is metrizable.
\end{proposition}

\begin{proof}
    \pf\ Define $d : \RR^2 \rightarrow \RR$ by
    \begin{align*}
        d((x,y),(x,z)) & = \max(|y-z|,1) \\
        d((x,y),(x',y')) & = 1 & \text{if } x \neq x' \qed
    \end{align*}        \step{c}{$x \in \bigcap_{i=1}^N \inv{\pi_i}() \subseteq B_D(a, \epsilon)$}
\end{proof}

\begin{proposition}
    Let $d : X^2 \rightarrow \RR$ be a metric on $X$. Then the metric topology on $X$ is the coarsest topology such that $d$ is continuous.
\end{proposition}

\begin{proof}
    \pf
    \step{1}{$d$ is continuous in each variable separately.}
    \begin{proof}
        \step{a}{\pflet{$a \in X$ and $d_a : X \rightarrow \RR$ be the function $d(a,-)$}}
        \step{b}{\pflet{$b \in X$ and $\epsilon > 0$}}
        \step{c}{For all $x \in X$, if $d(b,x) < \epsilon$ then $|d(a,b) - d(a,x)| < \epsilon$}
    \end{proof}
    \step{2}{If $\TT$ is any topology under which $d$ is continuous then $\TT$ is finer than the metric topology.}
    \begin{proof}
        \pf\ Since $B(a, \epsilon) = \inv{d_a}((-\infty, \epsilon))$
    \end{proof}
    \qed    
\end{proof}

\begin{proposition}
    Let $X$ be a metric space with metric $d$ and $A \subseteq X$. The restriction of $d$ to $A$ is a metric on $A$ that induces the subspace topology.
\end{proposition}

\begin{proof}
    \pf
    \step{1}{The restriction of $d$ to $A$ is a metric on $A$.}
    \step{2}{Every open ball under $d \restriction A$ is open under the subspace topology.}
    \begin{proof}
        \pf\ $B_{d \restriction A}(a, \epsilon) = B_d(a, \epsilon) \cap A$.
    \end{proof}
    \step{3}{If $U$ is open in the subspace topology and $x \in U$, then there exists a $d \restriction A$-ball $B$ such that $x \in B \subseteq U$.}
    \begin{proof}
        \step{a}{\pick\ $V$ open in $X$ such that $U = V \cap A$}
        \step{b}{\pick\ $\epsilon > 0$ such that $B_d(x, \epsilon) \subseteq V$}
        \step{c}{Take $B = B_{d \restriction A}(x, \epsilon)$}
    \end{proof}
    \qed
\end{proof}

\begin{corollary}
    A subspace of a metrizable space is metrizable.
\end{corollary}

\begin{proposition}
    Every metrizable space is Hausdorff.
\end{proposition}

\begin{proof}
    \pf
    \step{1}{\pflet{$X$ be a metric space}}
    \step{2}{\pflet{$a, b \in X$ with $a \neq b$}}
    \step{3}{\pflet{$\epsilon = d(a,b) / 2$}}
    \step{4}{\pflet{$U = B(a, \epsilon)$ and $V = B(b, \epsilon)$}}
    \step{5}{$U$ and $V$ are disjoint neighbourhoods of $a$ and $b$ respectively.}
    \qed
\end{proof}

\begin{proposition}[CC]
    The product of a countable family of metrizable spaces is metrizable.
\end{proposition}

\begin{proof}
    \pf
    \step{0}{\pflet{$(X_n, d_n)$ be a sequence of metric spaces.}}
    \step{0a}{\assume{w.l.o.g.~each $d_n$ is bounded above by 1.}}
    \begin{proof}
        \pf\ By Proposition \ref{proposition:standard_bounded_metric}.
    \end{proof}
    \step{1}{\pflet{$D$ be the metric on $\RR^\omega$ defined by $D(x,y) = \sup_i (d_i(x_i, y_i) / i)$.}}
    \begin{proof}
        \step{a}{$D(x,y) \geq 0$}
        \step{b}{$D(x,y) = 0$ if and only if $x = y$}
        \step{c}{$D(x,y) = D(y,x)$}
        \step{d}{$D(x,z) \leq D(x,y) + D(y,z)$}
        \begin{proof}
            \pf
            \begin{align*}
                D(x,z) & = \sup_i \frac{d_i(x_i, z_i)}{i} \\
                & \leq \sup_i \frac{d_i(x_i, y_i) + d_i(y_i, z_i)}{i} \\
                & \leq \sup_i \frac{d_i(x_i, y_i)}{i} + \sup_i \frac{d_i(y_i, z_i)}{i} \\
                & = D(x,y) + D(y,z)
            \end{align*}
        \end{proof}
    \end{proof}
    \step{2}{Every open ball $B_D(a, \epsilon)$ is open in the product topology.}
    \begin{proof}
        \step{b}{\pick\ $N$ such that $1 / \epsilon < N$}
        \step{c}{$B_D(a,\epsilon) = \prod_{i=1}^\infty U_i$ where $U_i = B(a_i, i \epsilon)$ if $i \leq N$, and $U_i = X_i$ if $i > N$}
    \end{proof}
    \step{3}{For any open set $U$ and $a \in U$, there exists $\epsilon > 0$ such that $B_D(a, \epsilon) \subseteq U$.}
    \begin{proof}
        \step{a}{\pflet{$n \geq 1$, $V$ be an open set in $\RR$ and $a \in \inv{\pi_n}(V)$}}
        \step{b}{\pick\ $\epsilon > 0$ such that $B_{d_n}(a, \epsilon) \subseteq V$}
        \step{c}{$B_D(a, \epsilon / n) \subseteq \inv{\pi_n}(V)$}
    \end{proof}
    \qed
\end{proof}

\begin{theorem}
    Let $X$ and $Y$ be metric spaces and $f : X \rightarrow Y$. Then $f$ is continuous if and only if, for all $x \in X$ and $\epsilon > 0$,
    there exists $\delta > 0$ such that, for all $y \in X$, if $d(x,y) < \delta$ then $d(f(x), f(y)) < \epsilon$.
\end{theorem}

\begin{proof}
    \pf
    \step{1}{If $f$ is continuous then, for all $x \in X$ and $\epsilon > 0$,
    there exists $\delta > 0$ such that, for all $y \in X$, if $d(x,y) < \delta$ then $d(f(x), f(y)) < \epsilon$}
    \begin{proof}
        \step{a}{\assume{$f$ is continuous.}}
        \step{b}{\pflet{$x \in X$ and $\epsilon > 0$}}
        \step{c}{\pick\ a neighbourhood $U$ of $x$ such that $f(U) \subseteq B(f(x), \epsilon)$}
        \begin{proof}
            \pf\ Theorem \ref{theorem:continuous}.
        \end{proof}
        \step{d}{\pick\ $\delta > 0$ such that $B(x, \delta) \subseteq U$}
        \begin{proof}
            \pf\ Proposition \ref{proposition:open_in_metric_space}.
        \end{proof}
        \step{e}{For all $y \in X$, if $d(x,y) < \delta$ then $d(f(x), f(y)) < \epsilon$}
    \end{proof}
    \step{2}{If for all $x \in X$ and $\epsilon > 0$,
    there exists $\delta > 0$ such that, for all $y \in X$, if $d(x,y) < \delta$ then $d(f(x), f(y)) < \epsilon$, then $f$ is continuous.}
    \begin{proof}
        \step{a}{\assume{for all $x \in X$ and $\epsilon > 0$, there exists
        $\delta > 0$ such that, for all $y \in X$, if $d(x,y) < \delta$
        then $d(f(x),f(y)) < \epsilon$}}
        \step{b}{\pflet{$x \in X$ and $V$ be a neighbourhood of $f(x)$}}
        \step{c}{\pick\ $\epsilon > 0$ such that $B(f(x),\epsilon) \subseteq V$}
        \begin{proof}
            \pf\ Proposition \ref{proposition:open_in_metric_space}.
        \end{proof}
        \step{d}{\pick\ $\delta > 0$ such that, for all $y \in X$, if $d(x,y) < \delta$
        then $d(f(x),f(y)) < \epsilon$}
        \begin{proof}
            \pf\ By \stepref{a}
        \end{proof}
        \step{e}{\pflet{$U = B(x,\delta)$}}
        \step{f}{$U$ is a neighbourhood of $x$ with $f(U) \subseteq V$}
        \qedstep
        \begin{proof}
            \pf\ Theorem \ref{theorem:continuous}.
        \end{proof}
    \end{proof}
    \qed
\end{proof}

\begin{proposition}
    \label{proposition:convergence_metric}
    Let $X$ be a metric space. Let $(a_n)$ be a sequence in $X$ and $l \in X$.
    Then $a_n \rightarrow l$ as $n \rightarrow \infty$ if and only if,
    for all $\epsilon > 0$, there exists $N$ such that, for all $n \geq N$,
    we have $d(a_n, l) < \epsilon$.
\end{proposition}

\begin{proof}
    \pf\ From Proposition \ref{proposition:convergence_basis}. \qed
\end{proof}

\begin{lemma}[Sequence Lemma (CC)]
    Let $X$ be a metrizable space. Let $A \subseteq X$ and $l \in \overline{A}$.
    Then there exists a sequence in $A$ that converges to $l$.
\end{lemma}

\begin{proof}
    \pf
    \step{1}{For all $n \geq 1$, \pick\ $a_n \in A \cap B(l, 1/n)$
    \prove{$a_n \rightarrow l$ as $n \rightarrow \infty$}}
    \step{2}{\pflet{$\epsilon > 0$}}
    \step{3}{\pick\ $N$ such that $1 / \epsilon < N$}
    \step{4}{For $n \geq N$ we have $d(a_n, l) < \epsilon$}
    \qedstep
    \begin{proof}
        \pf\ Proposition \ref{proposition:convergence_metric}.
    \end{proof}
    \qed
\end{proof}

\section{Real Linear Algebra}

\begin{definition}[Square Metric]
    The \emph{square metric} $\rho$ on $\RR^n$ is defined by
    \[ \rho(\vec{x}, \vec{y}) = \max(|x_1 - y_1|, \ldots, |x_n - y_n|) \]
\end{definition}

We prove this is a metric.

\begin{proof}
    \pf
    \step{1}{$\rho(\vec{x}, \vec{y}) \geq 0$}
    \begin{proof}
        \pf\ Immediate from definition.
    \end{proof}
    \step{2}{$\rho(\vec{x}, \vec{y}) = 0$ if and only if $\vec{x} = \vec{y}$}
    \begin{proof}
        \pf\ Immediate from definition.
    \end{proof}
    \step{3}{$\rho(\vec{x}, \vec{y}) = \rho(\vec{y}, \vec{x})$}
    \begin{proof}
        \pf\ Immediate from definition.
    \end{proof}
    \step{4}{$\rho(\vec{x}, \vec{z}) \leq \rho(\vec{x}, \vec{y}) + \rho(\vec{y}, \vec{z})$}
    \begin{proof}
        \pf\ Since $|x_i - z_i| \leq |x_i - y_i| + |y_i - z_i|$.
    \end{proof}
    \qed
\end{proof}

\begin{proposition}
    The square metric induces the standard topology on $\RR^n$.
\end{proposition}

\begin{proof}
    \pf
    \step{1}{For every $a \in X$ and $\epsilon > 0$, we have $B_\rho(a, \epsilon)$ is open in the standard product topology.}
    \begin{proof}
        \pf
        \[ B_\rho(a, \epsilon) = (a_1 - \epsilon, a_1 + \epsilon) \times \cdots \times (a_n - \epsilon, a_n + \epsilon) \]
    \end{proof}
    \step{2}{For any open sets $U_1$, \ldots, $U_n$ in $\RR$, we have $U_1 \times \cdots \times U_n$ is open in the square metric topology.}
    \begin{proof}
        \step{a}{\pflet{$\vec{a} \in U_1 \times \cdots \times U_n$}}
        \step{b}{For $i = 1, \ldots, n$, \pick\ $\epsilon_i > 0$ such that $(a_i - \epsilon_i, a_i + \epsilon_i) \subseteq U_i$}
        \step{c}{\pflet{$\epsilon = \min(\epsilon_1, \ldots, \epsilon_n)$}}
        \step{d}{$B_\rho(\vec{a}, \epsilon) \subseteq U$}
    \end{proof}
    \qed
\end{proof}

\begin{definition}
    Given $\vec{x}, \vec{y} \in \RR^n$, define the \emph{sum} $\vec{x} + \vec{y}$ by
    \[ (x_1, \ldots, x_n) + (y_1, \ldots, y_n) = (x_1 + y_1, \ldots, x_n + y_n) \enspace . \]
\end{definition}

\begin{definition}
    Given $\lambda \in \RR$ and $\vec{x} \in \RR^n$, define the \emph{scalar product} $\lambda \vec{x} \in \RR^n$ by
    \[ \lambda (x_1, \ldots, x_n) = (\lambda x_1, \ldots, \lambda x_n) \]
\end{definition}

\begin{definition}[Inner Product]
    Given $\vec{x}, \vec{y} \in \RR^n$, define the \emph{inner product} $\vec{x} \cdot \vec{y} \in \RR$ by
    \[ (x_1, \ldots, x_n) \cdot (y_1, \ldots, y_n) = x_1 y_1 + \cdots + x_n y_n \enspace . \]
    We write $\vec{x}^2$ for $\vec{x} \cdot \vec{x}$.
\end{definition}

\begin{definition}[Norm]
    Let $n \geq 1$. The \emph{norm} on $\RR^n$ is the function $\| \ \| : \RR^n \rightarrow \RR$ defined by
    \[ \| (x_1, \ldots, x_n) \| = \sqrt{x_1^2 + \cdots + x_n^2} \]
\end{definition}

\begin{lemma}
    \[ \| \vec{x} \|^2 = \vec{x}^2 \]
\end{lemma}

\begin{proof}
    \pf\ Immediate from definitions. \qed
\end{proof}

\begin{lemma}
    \[ \vec{x} \cdot (\vec{y} + \vec{z}) = \vec{x} \cdot \vec{y} + \vec{x} \cdot \vec{z} \]
\end{lemma}

\begin{proof}
    \pf\ Each is equal to $(x_1 y_1 + x_1 z_1, \ldots, x_n y_n + x_n z_n)$. \qed
\end{proof}

\begin{lemma}
    \label{lemma:Cauchy-Schwarz}
    \[ |\vec{x} \cdot \vec{y}| \leq \| \vec{x} \| \| \vec{y} \| \]
\end{lemma}

\begin{proof}
    \pf
    \step{1}{\assume{$\vec{x} \neq \vec{0} \neq \vec{y}$}}
    \begin{proof}
        \pf\ Otherwise both sides are 0.
    \end{proof}
    \step{2}{\pflet{$a = 1 / \| \vec{x} \|$}}
    \step{3}{\pflet{$b = 1 / \| \vec{y} \|$}}
    \step{5}{$(a \vec{x} + b \vec{y})^2 \geq 0$ and $(a \vec{x} - b \vec{y})^2 \geq 0$}
    \step{6}{$a^2 \| \vec{x} \|^2 + 2 a b \vec{x} \cdot \vec{y} + b^2 \| \vec{y} \|^2 \geq 0$ and $a^2 \| \vec{x} \|^2 - 2 a b \vec{x} \cdot \vec{y} + b^2 \| \vec{y} \|^2 \geq 0$}
    \step{7}{$2ab \vec{x} \cdot \vec{y} + 2 \geq 0$ and $-2ab \vec{x} \cdot \vec{y} + 2 \geq 0$}
    \step{8}{$\vec{x} \cdot \vec{y} \geq - 1/ab$ and $\vec{x} \cdot \vec{y} \leq 1/ab$}
    \step{9}{$\vec{x} \cdot \vec{y} \geq - \| \vec{x} \| \| \vec{y} \|$ and $\vec{x} \cdot \vec{y} \leq \| \vec{x} \| \| \vec{y} \|$}
    \qed
\end{proof}

\begin{lemma}[Triangle Inequality]
    \label{lemma:triangle_inequality}
    \[ \| \vec{x} + \vec{y} \| \leq \| \vec{x} \| + \| \vec{y} \| \]
\end{lemma}

\begin{proof}
    \pf
    \begin{align*}
        \| \vec{x} + \vec{y} \|^2 & = \| \vec{x} \|^2 + 2 \vec{x} \cdot \vec{y} + \| \vec{y} \|^2 \\
        & \leq \| \vec{x} \|^2 + 2 \| \vec{x} \| \| \vec{y} \| + \| \vec{y} \|^2 & (\text{Lemma \ref{lemma:Cauchy-Schwarz}}) \\
        & = (\| \vec{x} \| + \| \vec{y} \|)^2 & \qed
    \end{align*}
\end{proof}

\begin{definition}[Euclidean Metric]
    Let $n \geq 1$. The \emph{Euclidean metric} on $\RR^n$ is defined by
    \[ d(\vec{x}, \vec{y}) = \| \vec{x} - \vec{y} \| \enspace . \]
\end{definition}

We prove this is a metric.

\begin{proof}
    \step{1}{$d(\vec{x}, \vec{y}) \geq 0$}
    \begin{proof}
        \pf\ Immediate from definition.
    \end{proof}
    \step{2}{$d(\vec{x}, \vec{y}) = 0$ if and only if $\vec{x} = \vec{y}$}
    \begin{proof}
        \pf\ $d(\vec{x}, \vec{y}) = 0$ if and only if $\vec{x} - \vec{y} = \vec{0}$.
    \end{proof}
    \step{3}{$d(\vec{x}, \vec{y}) = d(\vec{y}, \vec{x})$}
    \begin{proof}
        \pf\ Immediate from definition.
    \end{proof}
    \step{4}{$d(\vec{x}, \vec{z}) \leq d(\vec{x}, \vec{y}) + d(\vec{y}, \vec{z})$}
    \begin{proof}
        \pf
        \begin{align*}
            \| \vec{x} - \vec{z} \| & = \| (\vec{x} - \vec{y}) + (\vec{y} - \vec{z}) \| \\
            & \leq \| \vec{x} - \vec{y} \| + \| \vec{y} - \vec{z} \| & (\text{Lemma \ref{lemma:triangle_inequality}})
        \end{align*}
    \end{proof}
    \qed
\end{proof}

\begin{proposition}
    The Euclidean metric induces the standard topology on $\RR^n$.
\end{proposition}

\begin{proof}
    \pf
    \step{1}{\pflet{$\rho$ be the square metric.}}
    \step{2}{For all $\vec{a} \in \RR^n$ and $\epsilon > 0$, we have $B_d(\vec{a}, \epsilon) \subseteq B_\rho(\vec{a}, \epsilon)$}
    \begin{proof}
        \step{a}{\pflet{$\vec{x} \in B_d(\vec{a}, \epsilon)$}}
        \step{b}{$\sqrt{(x_1 - a_1)^2 + \cdots + (x_n - a_n)^2} < \epsilon$}
        \step{c}{$(x_1 - a_1)^2 + \cdots + (x_n - a_n)^2 < \epsilon^2$}
        \step{d}{For all $i$ we have $(x_i - a_i)^2 < \epsilon^2$}
        \step{e}{For all $i$ we have $|x_i - a_i| < \epsilon$}
        \step{f}{$\rho(\vec{x}, \vec{a}) < \epsilon$}
    \end{proof}
    \step{3}{For all $\vec{a} \in \RR^n$ and $\epsilon > 0$, we have $B_\rho(\vec{a}, \epsilon / \sqrt{n}) \subseteq B_d(\vec{a}, \epsilon)$}
    \begin{proof}
        \step{a}{\pflet{$\vec{x} \in B_\rho(\vec{a}, \epsilon / \sqrt{n})$}}
        \step{b}{$\rho(\vec{x}, \vec{a}) < \epsilon / \sqrt{n}$}
        \step{c}{For all $i$ we have $|x_i - x_a| < \epsilon / \sqrt{n}$}
        \step{d}{For all $i$ we have $(x_i - x_a)^2 < \epsilon^2 / n$}
        \step{e}{$d(\vec{x}, \vec{a}) < \epsilon$}
    \end{proof}
    \qedstep
    \begin{proof}
        \pf\ By Lemma \ref{lemma:metrics_same_topology}.
    \end{proof}
    \qed
\end{proof}

\begin{lemma}
    If $\sum_{i=0}^\infty x_i^2$ and $\sum_{i=0}^\infty y_i^2$ converge then $\sum_{i=0}^\infty |x_i y_i|$ converges.
\end{lemma}

\begin{proof}
    \pf
    \step{1}{For all $N \geq 0$ we have
        $\sum_{i=0}^N |x_i y_i| \leq \sqrt{\sum_{i=0}^N |x_i|^2} \sqrt{\sum_{i=0}^N |y_i|^2}$}
    \begin{proof}
        \pf\ By the Cauchy-Schwarz inequality
    \end{proof}
    \qedstep
    \begin{proof}
        \pf\ Since $\sum_{i=0}^N |x_i y_i|$ is an increasing sequence bounded above by \\ $(\sum_{i=0}^\infty x_i^2) (\sum_{i=0}^\infty y_i^2)$.
    \end{proof}
    \qed
\end{proof}

\begin{corollary}
    \label{corollary:l2_sum_converge}
    If $\sum_{i=0}^\infty x_i^2$ and $\sum_{i=0}^\infty y_i^2$ converge then $\sum_{i=0}^\infty (x_i + y_i)^2$ converges.
\end{corollary}

\begin{proof}
    \pf\ Since $\sum_{i=0}^\infty x_i^ 2$, $\sum_{i=0}^\infty y_i^2$ and $2 \sum_{i=0}^\infty x_i y_i$ all converge.
\end{proof}

\begin{definition}[$l^2$-metric]
    The \emph{$l^2$-metric} on 
    \[ \left\{ (x_n) \in \RR^\omega \mid \sum_{n=0}^\infty x_n^2 \text{ converges} \right\} \] is defined by
    \[ d(x,y) = \left( \sum_{n=0}^\infty (x_n - y_n)^2 \right)^{1/2} \]
\end{definition}

We prove this is a metric.

\begin{proof}
    \pf
    \step{1}{$d$ is well-defined.}
    \begin{proof}
        \pf\ By Corollary \ref{corollary:l2_sum_converge}.
    \end{proof}
    \step{2}{$d(x,y) \geq 0$}
    \step{3}{$d(x,y) = 0$ if and only if $x = y$}
    \step{4}{$d(x,y) = d(y,x)$}
    \step{5}{$d(x,z) \leq d(x,y) + d(y,z)$}
    \begin{proof}
        \pf\ By Lemma \ref{lemma:triangle_inequality}.
    \end{proof}
    \qed
\end{proof}

\section{The Uniform Topology}

\begin{definition}[Uniform Metric]
    Let $J$ be a set. The \emph{uniform metric} $\overline{\rho}$ on $\RR^J$ is defined by
    \[ \overline{\rho}(a,b) = \sup_{j \in J} \overline{d}(a_j, b_j) \]
    where $\overline{d}$ is the standard bounded metric on $\RR$.

    The \emph{uniform topology} on $\RR^J$ is the topology induced by the uniform metric.
\end{definition}

We prove this is a metric.

\begin{proof}
    \pf
    \step{1}{$\overline{\rho}(a,b) \geq 0$}
    \begin{proof}
        \pf\ Immediate from definitions.
    \end{proof}
    \step{2}{$\overline{\rho}(a,b) = 0$ if and only if $a = b$}
    \begin{proof}
        \pf\ Immediate from definitions.
    \end{proof}
    \step{3}{$\overline{\rho}(a,b) = \overline{\rho}(b,a)$}
    \begin{proof}
        \pf\ Immediate from definitions.
    \end{proof}
    \step{4}{$\overline{\rho}(a,c) \leq \overline{\rho}(a,b) + \overline{\rho}(b,c)$}
    \begin{proof}
        \pf
        \begin{align*}
            \overline{\rho}(a,c) & = \sup_{j \in J} \overline{d}(a_j, c_j) \\
            & \leq \sup_{j \in J} (\overline{d}(a_j, b_j) + \overline{d}(b_j, c_j)) \\ 
            & \leq \sup_{j \in J} \overline{d}(a_j, b_j) + \sup_{j \in J} \overline{d}(b_j, c_j) \\
            & = \overline{\rho}(a,b) + \overline{\rho}(b,c)
        \end{align*}
    \end{proof}
    \qed
\end{proof}

\begin{proposition}
    The uniform topology on $\RR^J$ is finer than the product topology.
\end{proposition}

\begin{proof}
    \pf
    \step{1}{\pflet{$j \in J$ and $U$ be open in $\RR$} \prove{$\inv{\pi_j}(U)$ is open in the uniform topology.}}
    \step{2}{\pflet{$a \in \inv{\pi_j}(U)$}}
    \step{3}{\pick\ $\epsilon > 0$ such that $(a_j - \epsilon, a_j + \epsilon) \subseteq U$}
    \step{4}{$B_{\overline{\rho}}(a, \epsilon) \subseteq \inv{\pi_j}(U)$}
    \qed
\end{proof}

\begin{proposition}
    The uniform topology on $\RR^J$ is coarser than the box topology.
\end{proposition}

\begin{proof}
    \pf
    \step{1}{\pflet{$a \in \RR^J$ and $\epsilon > 0$} \prove{$B(a, \epsilon)$ is open in the box topology.}}
    \step{2}{\pflet{$b \in B(a, \epsilon)$}}
    \step{3}{For $j \in J$ we have $|a_j - b_j| < \epsilon$}
    \step{4}{For $j \in J$, \pflet{$\delta_j = (\epsilon - |a_j - b_j|) / 2$}}
    \step{5}{$\prod_{j \in J} (b_j - \delta_j, b_j + \delta_j) \subseteq B(a, \epsilon)$}
    \qed
\end{proof}

\begin{proposition}
    The uniform topology on $\RR^J$ is strictly finer than the product topology if and only if $J$ is infinite.
\end{proposition}

\begin{proof}
    \pf
    \step{1}{If $J$ is finite then the uniform and product topologies coincide.}
    \begin{proof}
        \pf\ The uniform, box and product topologies are all the same.
    \end{proof}
    \step{2}{If $J$ is infinite then the uniform and product topologies are different.}
    \begin{proof}
        \pf\ The set $B(\vec{0}, 1)$ is open in the uniform topology but not the product topology.
    \end{proof}
    \qed
\end{proof}

\begin{proposition}[DC]
    The uniform topology on $\RR^J$ is strictly coarser than the box topology if and only if $J$ is infinite.
\end{proposition}

\begin{proof}
    \pf
    \step{1}{If $J$ is finite then the uniform and box topologies coincide.}
    \begin{proof}
        \pf\ The uniform, box and product topologies are all the same.
    \end{proof}
    \step{2}{If $J$ is infinite then the uniform and box topologies are different.}
    \begin{proof}
        \pf\ Pick an $\omega$-sequence $(j_1, j_2, \ldots)$ in $J$. Let $U = \prod_{j \in J} U_j$ where $U_{j_i} = (-1/i, 1/i)$ and $U_j = (-1,1)$ for all other $j$.
        Then $\vec{0} \in U$ but there is no $\epsilon > 0$ such that $B(\vec{0}, \epsilon) \subseteq U$.
    \end{proof}
    \qed
\end{proof}

\begin{proposition}
    The closure of $\RR^\infty$ in $\RR^\omega$ under the uniform topology is $\RR^\omega$.
\end{proposition}

\begin{proof}
    \pf\ Given any open ball $B(a, \epsilon)$, pick an integer $N$ such that $1 / \epsilon < N$. Then $B(a, \epsilon)$ includes sequences whose $n$th entry is 0
    for all $n \geq N$. \qed
\end{proof}

\end{document}