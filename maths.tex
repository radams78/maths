\documentclass{book}

\usepackage{amsmath}
\usepackage{amssymb}
\usepackage{amsthm}
\usepackage{hyperref}
\let\proof\relax
\let\endproof\relax
\let\qed\relax
\usepackage{pf2}

\newtheorem{lemma}{Lemma}[chapter]
\newtheorem{corollary}{Corollary}[lemma]
\newtheorem{theorem}[lemma]{Theorem}
\newtheorem{theorems}[lemma]{Theorem Schema}
\newtheorem{proposition}[lemma]{Proposition}
\newtheorem{axiom}[lemma]{Axiom}
\newtheorem{axioms}[lemma]{Axiom Schema}
\theoremstyle{definition}
\newtheorem{definition}[lemma]{Definition}
\newtheorem{definitions}[lemma]{Definition Schema}
\newtheorem{example}[lemma]{Example}

\title{The Universe}
\author{Robin Adams}

\newcommand{\BB}{\ensuremath{\mathcal{B}}}
\newcommand{\CC}{\ensuremath{\mathcal{C}}}
\newcommand{\DD}{\ensuremath{\mathcal{D}}}
\newcommand{\PP}{\ensuremath{\mathcal{P}}}
\newcommand{\TT}{\ensuremath{\mathcal{T}}}

\newcommand{\id}[1]{\ensuremath{\mathrm{id}_{#1}}}
\newcommand{\inv}[1]{\ensuremath{{#1}^{-1}}}
\newcommand{\op}[1]{\ensuremath{[#1]^{\mathrm{op}}}}
\newcommand{\pow}[1]{\ensuremath{\PP {#1}}}
\newcommand{\Real}{\ensuremath{\mathbb{R}}}
\newcommand{\Set}{\ensuremath{\mathbf{Set}}}

\begin{document}

\maketitle
\tableofcontents

\chapter{Category Theory}

\section{Categories}

\begin{definition}[Category]
    A \emph{category} $\CC$ consists of:
    \begin{itemize}
        \item a class $|\CC|$ of \emph{objects};
        \item for any objects $X, Y \in \CC$, a set $\CC[X,Y]$ of \emph{morphisms}. We write $f : X \rightarrow Y$
        for $f \in \CC[X,Y]$
        \item for any object $X \in \CC$, an \emph{identity} morphism $\id{X} : X \rightarrow X$
        \item for any morphisms $f : X \rightarrow Y$ and $g : Y \rightarrow Z$, a morphism $g \circ f : X \rightarrow Z$,
        the \emph{composite} of $f$ and $g$
    \end{itemize}
    such that:
    \begin{description}
        \item[Unit Laws] For any $f : X \rightarrow Y$ we have $f = \id{Y} \circ f = f \circ \id{X}$
        \item[Associativity] For any $f : X \rightarrow Y$, $g : Y \rightarrow Z$ and $h : Z \rightarrow W$, we have
        \[ h \circ (g \circ f) = (h \circ g) \circ f \]
    \end{description}
\end{definition}

\begin{definition}[Category of Sets]
    The \emph{category of sets} $\Set$ is the category with objects all sets and morphisms all functions.
\end{definition}

\begin{definition}[Category of Pointed Sets]
    The \emph{category of pointed sets} $\Set_*$ is the category with objects all pairs $(A,a)$ such that $A$ is a set and $a \in A$;
    and morphisms $f : (A,a) \rightarrow (B,b)$ all functions $f : A \rightarrow B$ such that $f(a) = b$.
\end{definition}

\begin{definition}[Opposite Category]
    Let $\CC$ be a category. The \emph{opposite} category $\op{\CC}$ is the category with
    $|\op{\CC}| = |\CC|$ and $\op{\CC}[X,Y] = \CC[Y,X]$.
\end{definition}

\begin{definition}[Section, Retraction]
    Let $\CC$ be a category.
    Let $r : A \rightarrow B$ and $s : B \rightarrow A$ in $\CC$. Then $r$ is a \emph{retraction} of $s$,
    and $s$ is a \emph{section} of $r$, if and only if $r \circ s = \id{B}$.
\end{definition}

\begin{definition}[Isomorphism]
    Let $\CC$ be a category. A morphism $f : A \rightarrow B$ is an \emph{isomorphism} in $\CC$, $f : A \cong B$,
    if and only if there exists a morphism $\inv{f} : B \rightarrow A$, the \emph{inverse} of $f$, that is both
    a section and a retraction of $f$.

    Two objects $A$ and $B$ are \emph{isomorphic}, $A \cong B$, if and only if there exists an isomorphism between them.
\end{definition}

\begin{proposition}
    The inverse of an isomorphism is unique.
\end{proposition}

\begin{proposition}
    A morphism is an isomorphism if and only if it is both a section and a retraction.
\end{proposition}

\begin{proposition}
    For any object $X$, we have $\id{X} : X \cong X$ and $\inv{\id{X}} = \id{X}$.
\end{proposition}

\begin{proposition}
    For any isomorphism $f : X \cong Y$ we have $\inv{f} : Y \cong X$ and $\inv{(\inv{f})} = f$.
\end{proposition}

\begin{proposition}
    For any isomorphisms $f : X \cong Y$ and $g : Y \cong Z$ we have $g \circ f : X \cong Z$
    and $\inv{(g \circ f)} = \inv{f} \circ \inv{g}$.
\end{proposition}

\begin{proposition}
    A function is an isomorphism in $\Set$ if and only if it is bijective.
\end{proposition}

\begin{theorem}
    Let $\CC$ be a category. Let $f : A \rightarrow B$ in $\CC$. Then the following are equivalent.
    \begin{enumerate}
        \item $f$ is an isomorphism.
        \item For all $X$, the function $\CC[f,-] : \CC[B,X] \rightarrow \CC[A,X]$ is a bijection.
        \item For all $X$, the function $\CC[-,f] : \CC[X,A] \rightarrow \CC[X,B]$ is a bijection.
    \end{enumerate}
\end{theorem}

\section{Functors}

\begin{definition}[Functor]
    Let $\CC$ and $\DD$ be categories. A \emph{functor} $F : \CC \rightarrow \DD$ consists of:
    \begin{itemize}
        \item for every object $A \in \CC$, an object $FA \in \DD$
        \item for every morphism $f : A \rightarrow B : \CC$, a morphism $Ff : FA \rightarrow FB : \DD$
    \end{itemize}
    such that
    \begin{itemize}
        \item for every object $A \in \CC$, we have $F \id{A} = \id{FA}$
        \item for any morphisms $f : A \rightarrow B$ and $g : B \rightarrow C$ in $\CC$, we have $F(g \circ f) = Fg \circ Ff$.
    \end{itemize}
\end{definition}

\begin{definition}[Hom-Set Functor]
    Let $\CC$ be a category. The \emph{hom-set functor} $\CC[-,-] : \op{\CC} \times \CC \rightarrow \Set$ is the functor defined by:
    \begin{itemize}
        \item Given objects $A, B \in \CC$, we have $\CC[A,B]$ is the set of all morphisms from $A$ to $B$.
        \item Given morphisms $f : A \rightarrow A'$ and $g : B \rightarrow B'$, then $\CC[f,g] : \CC[A',B] \rightarrow \CC[A,B']$
        is defined by
        \[ \CC[f,g](h) = g \circ h \circ f \]
    \end{itemize}
\end{definition}

\begin{definition}[Faithful]
    A functor $F : \CC \rightarrow \DD$ is \emph{faithful} if and only if, for any objects $A, B \in \CC$ and morphisms $f, g : A \rightarrow B$,
    if $Ff = Fg$ then $f = g$.
\end{definition}

\begin{definition}[Full]
    A functor $F : \CC \rightarrow \DD$ is \emph{full} if and only if, for any objects $A, B \in \CC$ and morphism $g : FA \rightarrow FB$,
    there exists $f : A \rightarrow B$ such that $Ff = g$.
\end{definition}

\begin{definition}[Fully Faithful]
    A functor is \emph{fully faithful} if and only if it is full and faithful.
\end{definition}

\begin{definition}[Full Embedding (Evil)]
    A functor is a \emph{full embedding} if and only if it is fully faithful and injective on objects.
\end{definition}

\begin{proposition}
    Functors preserve isomorphisms.

    That is, let $F : \CC \rightarrow \DD$ and let $f : A \cong B$ be an isomorphism in $\CC$. Then $Ff : FA \cong FB$ and
    $\inv{(Ff)} = F \inv{f}$.
\end{proposition}

\section{Natural Transformations}

\begin{definition}[Natural Transformation]
    Let $F, G : \CC \rightarrow \DD$. A \emph{natural transformation} $\eta : F \rightarrow G$ consists of:
    \begin{itemize}
        \item for every object $X \in \CC$, a morphism $\eta_X : FX \rightarrow GX$
    \end{itemize}
    such that
    \begin{itemize}
        \item for every morphism $f : A \rightarrow B$ in $\CC$, the following di
    \end{itemize}
\end{definition}

\chapter{Topology}

\section{Topologies and Topological Spaces}

\begin{definition}[Topology]
    Let $X$ be a set. A \emph{topology} on $X$ is a set $\TT \subseteq \pow{X}$
    such that:
    \begin{enumerate}
        \item $X \in \TT$
        \item $\forall \mathcal{U} \subseteq \TT. \bigcup \mathcal{U} \in \TT$
        \item $\forall U,V \in \TT. U \cap V \in \TT$
    \end{enumerate}
\end{definition}

\begin{definition}[Topological Space]
    A \emph{topological space} $X$ consists of a set $X$ and a topology $\TT$ on $X$.
    We call the elements of $X$ \emph{points} and the elements of $\TT$ \emph{open sets}.
\end{definition}

\begin{definition}[Discrete Topology]
    Let $X$ be a set. The \emph{discrete} topology on $X$ is $\pow{X}$.
\end{definition}

\begin{definition}[Indiscrete Topology]
    Let $X$ be a set. The \emph{indiscrete} or \emph{trivial} topology on $X$ is $\{ \emptyset, X \}$.
\end{definition}

\begin{definition}[Open Neighbourhood]
    Let $X$ be a topological space. Let $x \in X$ and $U \subseteq X$. Then $U$ is an \emph{open Neighbourhood}
    of $x$ if and only if $x \in U$ and $U$ is open.
\end{definition}

\begin{definition}[Coarser, Finer]
    Let $\TT$ and $\TT'$ be two topologies on the same set $X$. Then $\TT$ is \emph{coarser}, \emph{smaller} or \emph{weaker} than $\TT'$,
    and $\TT'$ is \emph{finer}, \emph{larger} or \emph{stronger} than $\TT$, if and only if $\TT \subseteq \TT'$.
\end{definition}

\begin{proposition}
    Let $X$ be a set. The intersection of a set of topologies on $X$ is a topology on $X$.
\end{proposition}

\begin{corollary}
    Let $X$ be a set. The poset of topologies on $X$ is a complete lattice.
\end{corollary}

\section{Closed Sets}

\begin{definition}[Closed Set]
    Let $X$ be a topological space and $C \subseteq X$. Then $C$ is \emph{closed} if and only if $X - C$ is open.
\end{definition}

\section{Basis for a Topology}

\begin{definition}[Basis for a Topology]
    Let $X$ be a set. A \emph{basis} for a topology on $X$ is a set $\BB \subseteq \pow{X}$ such that:
    \begin{enumerate}
        \item $\bigcup \BB = X$
        \item $\forall B_1, B_2 \in \BB. \forall x \in B_1 \cap B_2. \exists B_3 \in \BB. x \in B_3 \subseteq B_1 \cap B_2$
    \end{enumerate}

    The topology \emph{generated} by $\BB$ is then the coarsest topology that includes $\BB$.

    Given $x \in X$, a \emph{basic open neighbourhood} of $x$ is a set $B \in \BB$ such that $x \in B$.
\end{definition}

\begin{proposition}
    Let $X$ be a set and $\BB$ be a basis for a topology $\TT$ on $X$. Let $U \subseteq X$.
    Then $U \in \TT$ if and only if $\forall x \in U. \exists B \in \BB. x \in B \subseteq U$.
\end{proposition}

\section{Continuous Functions}

\begin{definition}[Continuous]
    Let $X$ and $Y$ be topological spaces and $f : X \rightarrow Y$. Then $f$ is \emph{continuous} if and only if, for any open set $V$
    in $Y$, we have $\inv{f}(V)$ is open in $X$.
\end{definition}

\begin{proposition}
    For any topological space $X$, the identity function on $X$ is continuous.
\end{proposition}

\begin{proposition}
    Let $X$, $Y$ and $Z$ be topological spaces. Let $f : X \rightarrow Y$ and $g : Y \rightarrow Z$ be continuous functions.
    Then $g \circ f$ is continuous.
\end{proposition}

\begin{definition}[Category of Topological Spaces]
    The \emph{category of topological spaces} $\mathbf{Top}$ is the category with objects all topological spaces and morphisms all
    continuous functions.
\end{definition}

\begin{definition}[Category of Pointed Topological Spaces]
    The \emph{category of pointed topological spaces} $\mathbf{Top}_*$ is the category with objects all pairs
    $(X,x)$ where $X$ is a topological space and $x \in X$; and morphisms $f : (X,x) \rightarrow (Y,y)$ all
    continuous functions $f : X \rightarrow Y$ such that $f(x) = y$.
\end{definition}

\begin{definition}[Homeomorphism]
    A \emph{homeomorphism} is an isomorphism in $\mathbf{Top}$.

    Topological spaces are \emph{homeomorphic} if and only if they are isomorphic in $\mathbf{Top}$.
\end{definition}

\section{Homotopy Theory}

\begin{definition}
    Let $\mathbf{hTop}$ be the category with objects all topological spaces and morphisms $X \rightarrow Y$
    all continuous functions $X \rightarrow Y$ quotiented by homotopy.
\end{definition}

\begin{definition}[Homotopy Equivalence]
    A \emph{homotopy equivalence} is an isomorphism in $\mathbf{hTop}$.

    Topological spaces are \emph{homotopic} if and only if they are isomorphic in $\mathbf{hTop}$.
\end{definition}

\chapter{Metric Spaces}

\section{Metrics}

\begin{definition}[Metric, Metric Space]
    Let $X$ be a set. A \emph{metric} on $X$ is a function $d : X^2 \rightarrow \Real$ such that:
    \begin{enumerate}
        \item $\forall x,y \in X. d(x,y) \geq 0$
        \item $\forall x,y \in X. d(x,y) = 0 \Leftrightarrow x = y$
        \item $\forall x,y \in X. d(x,y) = d(y,x)$
        \item $\forall x,y,z \in X. d(x,z) \leq d(x,y) + d(y,z)$
    \end{enumerate}

    A \emph{metric space} $X$ consists of a set $X$ and a metric on $X$.
\end{definition}

\begin{definition}[Open Ball]
    Let $X$ be a metric space. Let $x \in X$ and $\epsilon > 0$. The \emph{open ball} with \emph{center} $x$
    and \emph{radius} $\epsilon$ is $B(x, \epsilon) = \{ y \in X \mid d(x,y) < \epsilon \}$.
\end{definition}

\begin{definition}[Metric Topology]
    On any metric space, the \emph{metric topology} is the topology generated by the basis consisting of the open balls.
\end{definition}

\begin{definition}[Metrizable]
    A topological space $X$ is \emph{metrizable} if and only if there exists a metric $d$ on $X$ such that the topology on $X$
    is the metric topology induced by $d$.
\end{definition}

\begin{definition}[Euclidean Metric]
    The \emph{Euclidean metric} on $\Real^n$ is defined by
    \[ d((x_1, \ldots, x_n),(y_1, \ldots, y_n)) = \sqrt{(x_1 - y_1)^2 + \cdots + (x_n - y_n)^2} \enspace . \]
\end{definition}

We write just $\Real^n$ for the metric space $\Real^n$ under the Euclidean metric.

\section{Subspaces}

\begin{proposition}
    Let $X$ be a set and $Y \subseteq X$. Let $d$ be a metric on $X$. Then $d \restriction Y^2$ is a metric on $Y$.
\end{proposition}

Given a metric space $(X,d)$ and a set $Y \subseteq X$, we will write just $Y$ for the metric space $(Y, d \restriction Y^2)$.

\begin{definition}[Interval]
    The \emph{interval} $I$ is the metric space $I = [0,1]$ as a subspace of $\Real$.
\end{definition}

\begin{definition}[Disk]
    Let $n \in \mathbb{Z}^+$. The \emph{$n$-disk} $D^n$ is the metric space
    \[ D^n = \{ x \in \Real^n \mid d(x,0) \leq 1 \} \]
    as a subspace of $\Real^n$.
\end{definition}

\begin{definition}[Sphere]
    Let $n \in \mathbb{Z}^+$. The \emph{$n$-sphere} $S^n$ is the metric space
    \[ D^n = \{ x \in \Real^{n+1} \mid d(x,0) = 1 \} \]
    as a subspace of $\Real^{n+1}$.
\end{definition}

\begin{proposition}
    Boundedness is not a topological property. That is, there exist homeomorphic metric spaces such that one is bounded and the other is
    not.
\end{proposition}

\begin{proof}
    \pf\ We have $\Real$ is complete but $(-1,1)$ is not. \qed
\end{proof}

\section{Complete Metric Spaces}

\begin{definition}[Complete]
    A metric space is \emph{complete} if and only if every Cauchy sequence converges.
\end{definition}

\begin{proposition}
    Completeness is not a topological property. That is, there exist homeomorphic metric spaces such that one is complete and the other is
    not.
\end{proposition}

\begin{proof}
    \pf\ We have $\Real$ is complete but $(-1,1)$ is not. \qed
\end{proof}

\chapter{Group Theory}

\begin{definition}[Category of Groups]
    The \emph{category of groups} $\mathbf{Grp}$ is the category with objects all groups and morphisms all group homomorphisms.
\end{definition}

\begin{definition}
    We identify any group $G$ with the category with one object $\bullet$ such that $G[\bullet,\bullet]$ is the set of elements
    of $G$ and composition is the group multiplication.
\end{definition}

\begin{definition}[Forgetful Functor]
    The \emph{forgetful functor} $U : \mathbf{Grp} \rightarrow \Set$ is the functor that maps a group $G$ to the set $G$, and a group
    homomorphism $f$ to $f$.
\end{definition}

\chapter{Ring Theory}

\section{Modules}

\begin{definition}[Category of Modules]
    Let $R$ be a ring. The \emph{category of $R$-modules} $R\mathbf{-Mod}$ is the category with objects the modules over $R$
    and morphisms the $R$-linear maps.
\end{definition}

\begin{definition}[Homology Theory]
    A \emph{homology theory} is a functor $\mathbf{Top} \rightarrow R \mathbf{-Mod}$.
\end{definition}

\chapter{Linear Algebra}

\section{Vector Spaces}

\begin{definition}[Category of Vector Spaces]
    Let $K$ be a field. The \emph{category of vector spaces} over $K$, $\mathbf{Vect}_K$,
    is the category with objects all vector spaces over $K$ and morphisms all linear transformations.
\end{definition}

\end{document}