\documentclass{book}

\usepackage{amsmath}
\usepackage{amssymb}
\usepackage{amsthm}
\usepackage{hyperref}
\let\proof\relax
\let\endproof\relax
\let\qed\relax
\usepackage{pf2}

\newtheorem{lemma}{Lemma}[chapter]
\newtheorem{corollary}{Corollary}[lemma]
\newtheorem{theorem}[lemma]{Theorem}
\newtheorem{theorems}[lemma]{Theorem Schema}
\newtheorem{proposition}[lemma]{Proposition}
\newtheorem{axiom}[lemma]{Axiom}
\newtheorem{axioms}[lemma]{Axiom Schema}
\theoremstyle{definition}
\newtheorem{definition}[lemma]{Definition}
\newtheorem{definitions}[lemma]{Definition Schema}
\newtheorem{example}[lemma]{Example}

\title{The Universe}
\author{Robin Adams}

\newcommand{\BB}{\ensuremath{\mathcal{B}}}
\newcommand{\PP}{\ensuremath{\mathcal{P}}}
\newcommand{\TT}{\ensuremath{\mathcal{T}}}

\newcommand{\inv}[1]{\ensuremath{{#1}^{-1}}}
\newcommand{\pow}[1]{\ensuremath{\PP {#1}}}
\newcommand{\Real}{\ensuremath{\mathbb{R}}}

\begin{document}

\maketitle
\tableofcontents

\chapter{Topology}

\section{Topologies and Topological Spaces}

\begin{definition}[Topology]
    Let $X$ be a set. A \emph{topology} on $X$ is a set $\TT \subseteq \pow{X}$
    such that:
    \begin{enumerate}
        \item $X \in \TT$
        \item $\forall \mathcal{U} \subseteq \TT. \bigcup \mathcal{U} \in \TT$
        \item $\forall U,V \in \TT. U \cap V \in \TT$
    \end{enumerate}
\end{definition}

\begin{definition}[Topological Space]
    A \emph{topological space} $X$ consists of a set $X$ and a topology $\TT$ on $X$.
    We call the elements of $X$ \emph{points} and the elements of $\TT$ \emph{open sets}.
\end{definition}

\begin{definition}[Discrete Topology]
    Let $X$ be a set. The \emph{discrete} topology on $X$ is $\pow{X}$.
\end{definition}

\begin{definition}[Indiscrete Topology]
    Let $X$ be a set. The \emph{indiscrete} or \emph{trivial} topology on $X$ is $\{ \emptyset, X \}$.
\end{definition}

\begin{definition}[Open Neighbourhood]
    Let $X$ be a topological space. Let $x \in X$ and $U \subseteq X$. Then $U$ is an \emph{open Neighbourhood}
    of $x$ if and only if $x \in U$ and $U$ is open.
\end{definition}

\begin{definition}[Coarser, Finer]
    Let $\TT$ and $\TT'$ be two topologies on the same set $X$. Then $\TT$ is \emph{coarser}, \emph{smaller} or \emph{weaker} than $\TT'$,
    and $\TT'$ is \emph{finer}, \emph{larger} or \emph{stronger} than $\TT$, if and only if $\TT \subseteq \TT'$.
\end{definition}

\begin{proposition}
    Let $X$ be a set. The intersection of a set of topologies on $X$ is a topology on $X$.
\end{proposition}

\begin{corollary}
    Let $X$ be a set. The poset of topologies on $X$ is a complete lattice.
\end{corollary}

\section{Closed Sets}

\begin{definition}[Closed Set]
    Let $X$ be a topological space and $C \subseteq X$. Then $C$ is \emph{closed} if and only if $X - C$ is open.
\end{definition}

\section{Basis for a Topology}

\begin{definition}[Basis for a Topology]
    Let $X$ be a set. A \emph{basis} for a topology on $X$ is a set $\BB \subseteq \pow{X}$ such that:
    \begin{enumerate}
        \item $\bigcup \BB = X$
        \item $\forall B_1, B_2 \in \BB. \forall x \in B_1 \cap B_2. \exists B_3 \in \BB. x \in B_3 \subseteq B_1 \cap B_2$
    \end{enumerate}

    The topology \emph{generated} by $\BB$ is then the coarsest topology that includes $\BB$.

    Given $x \in X$, a \emph{basic open neighbourhood} of $x$ is a set $B \in \BB$ such that $x \in B$.
\end{definition}

\begin{proposition}
    Let $X$ be a set and $\BB$ be a basis for a topology $\TT$ on $X$. Let $U \subseteq X$.
    Then $U \in \TT$ if and only if $\forall x \in U. \exists B \in \BB. x \in B \subseteq U$.
\end{proposition}

\section{Continuous Functions}

\begin{definition}[Continuous]
    Let $X$ and $Y$ be topological spaces and $f : X \rightarrow Y$. Then $f$ is \emph{continuous} if and only if, for any open set $V$
    in $Y$, we have $\inv{f}(V)$ is open in $X$.
\end{definition}

\begin{proposition}
    For any topological space $X$, the identity function on $X$ is continuous.
\end{proposition}

\begin{proposition}
    Let $X$, $Y$ and $Z$ be topological spaces. Let $f : X \rightarrow Y$ and $g : Y \rightarrow Z$ be continuous functions.
    Then $g \circ f$ is continuous.
\end{proposition}

\chapter{Metric Spaces}

\section{Metrics}

\begin{definition}[Metric, Metric Space]
    Let $X$ be a set. A \emph{metric} on $X$ is a function $d : X^2 \rightarrow \Real$ such that:
    \begin{enumerate}
        \item $\forall x,y \in X. d(x,y) \geq 0$
        \item $\forall x,y \in X. d(x,y) = 0 \Leftrightarrow x = y$
        \item $\forall x,y \in X. d(x,y) = d(y,x)$
        \item $\forall x,y,z \in X. d(x,z) \leq d(x,y) + d(y,z)$
    \end{enumerate}

    A \emph{metric space} $X$ consists of a set $X$ and a metric on $X$.
\end{definition}

\begin{definition}[Open Ball]
    Let $X$ be a metric space. Let $x \in X$ and $\epsilon > 0$. The \emph{open ball} with \emph{center} $x$
    and \emph{radius} $\epsilon$ is $B(x, \epsilon) = \{ y \in X \mid d(x,y) < \epsilon \}$.
\end{definition}

\begin{definition}[Metric Topology]
    On any metric space, the \emph{metric topology} is the topology generated by the basis consisting of the open balls.
\end{definition}

\begin{definition}[Metrizable]
    A topological space $X$ is \emph{metrizable} if and only if there exists a metric $d$ on $X$ such that the topology on $X$
    is the metric topology induced by $d$.
\end{definition}

\begin{definition}[Euclidean Metric]
    The \emph{Euclidean metric} on $\Real^n$ is defined by
    \[ d((x_1, \ldots, x_n),(y_1, \ldots, y_n)) = \sqrt{(x_1 - y_1)^2 + \cdots + (x_n - y_n)^2} \enspace . \]
\end{definition}

We write just $\Real^n$ for the metric space $\Real^n$ under the Euclidean metric.

\section{Subspaces}

\begin{proposition}
    Let $X$ be a set and $Y \subseteq X$. Let $d$ be a metric on $X$. Then $d \restriction Y^2$ is a metric on $Y$.
\end{proposition}

Given a metric space $(X,d)$ and a set $Y \subseteq X$, we will write just $Y$ for the metric space $(Y, d \restriction Y^2)$.

\begin{definition}[Interval]
    The \emph{interval} $I$ is the metric space $I = [0,1]$ as a subspace of $\Real$.
\end{definition}

\begin{definition}[Disk]
    Let $n \in \mathbb{Z}^+$. The \emph{$n$-disk} $D^n$ is the metric space
    \[ D^n = \{ x \in \Real^n \mid d(x,0) \leq 1 \} \]
    as a subspace of $\Real^n$.
\end{definition}

\begin{definition}[Sphere]
    Let $n \in \mathbb{Z}^+$. The \emph{$n$-sphere} $S^n$ is the metric space
    \[ D^n = \{ x \in \Real^{n+1} \mid d(x,0) = 1 \} \]
    as a subspace of $\Real^{n+1}$.
\end{definition}

\end{document}